A sketch of how we verify C programs. Note where we 
integrate with other projects.

\vspace{0.5em}

\begin{tikzpicture}[on grid, cmd/.style={rectangle, fill=yellow!50!white, minimum width=2em, minimum height=1em, draw=black, rounded corners=5pt},
  ourlib/.style={rounded corners=8pt, line width=1pt},
  test/.style={rounded corners=8pt, draw=red},
  file/.style={rounded corners=1pt, line width=1pt, fill=yellow!50!white},
  ->/.style={-Stealth, line width=1.5pt}, rotate=90,transform shape, x=1.6cm, y=1.6cm]
  %% \draw [test] (-1.5, -0.5) rectangle (1.5, 2.5);
  %% \draw [test] (-1.5, 3.5) rectangle (1.5, 6.5);
  %% \draw [test] (-1.1, 5.5) rectangle (1.1, 8.5);
  %% \draw [test] (-1.5, 7.5) rectangle (1.5, 10.5);
  \draw [ourlib, fill=maincolor] (-1.5, -0.75) -- ++(3, 0) -- ++(0, 2.1) -- ++(1, 0) -- ++(0, -2.1) -- ++(1.4, 0) -- ++(0, 16.5) -- ++(-1.4, 0) --
  ++(0, -2.1) -- ++(-1, 0) -- ++(0, 2.1) -- ++(-3, 0) -- ++(0, -4.5) -- ++(3, 0) -- ++(0, 2.1) -- ++(1, 0) -- ++(0, -2.7) -- ++(-1.4, 0) -- ++(0, 2.1) --
  ++(-2.2, 0) -- ++(0, -4.5) -- ++(2.2, 0) -- ++(0, 2.1) -- ++(1.4, 0) -- ++(0, -2.7) -- ++(-1, 0) -- ++(0, 2.1) -- ++(-3, 0) -- ++(0, -4.5) -- ++(3, 0)
  -- ++(0, 2.1) -- ++(1, 0) -- ++(0, -5.7) -- ++(-1, 0) -- ++(0, 2.1) -- ++(-3, 0) -- cycle;
  \node (Pn) at (0, 0) {\rotatebox{-90}{\color{white}$\{P_n\}$}};
  \node (Pn1) [above=2.9 of Pn]  {\rotatebox{-90}{\color{white}$\{P_{n-1}\}$}};
  \node (etc) [above=1.6 of Pn1] {\vdots};
  \node (P3) [above=1.5 of etc]  {\rotatebox{-90}{\color{white}$\{P_3\}$}};
  \node (P2) [above=3 of P3]  {\rotatebox{-90}{\color{white}$\{P_2\}$}};
  \node (P1) [above=3 of P2]  {\rotatebox{-90}{\color{white}$\{P_1\}$}};
  \node (P0) [above=3 of P1]  {\rotatebox{-90}{\color{white}$\{P_0\}$}};
  \node (Cn) [above=1.5 of Pn] [cmd] {\rotatebox{-90}{$C_n$}};
  \node (C3) [above=1.5 of P3] [cmd] {\rotatebox{-90}{$C_3$}};
  \node (C2) [above=3 of C3] [cmd] {\rotatebox{-90}{$C_2$}};
  \node (C1) [above=3 of C2] [cmd] {\rotatebox{-90}{$C_1$}};

  % If I uncomment this line I get the "file", but I also get a yellow triangle below the figure.
  % I have tried various tricks to get it to go away but I can't seem to figure it out.
  % \draw [file] (-4.5, 13.95) -- ++(2, 0) -- ++ (0, 1.5) -- +(-0.5, 0.5) -- +(-2, 0.5) -- cycle ++(2, 2.3) -- ++(-0.5, 0) -- ++(0, 0.5);

  \draw [file] (-4.5, 05.30) -- ++(2, 0) -- ++ (0, 1.8) -- +(-0.5, 0.5) -- +(-2, 0.5) -- cycle ++(2, 2.3) -- ++(-0.5, 0) -- ++(0, 0.5);
  
  % \draw [file] (-4.5, -0.75) -- ++(2, 0) -- ++ (0, 1.5) -- +(-0.5, 0.5) -- +(-2, 0.5) -- cycle ++(2, 2.3) -- ++(-0.5, 0) -- ++(0, 0.5);
  
  \node at (-3.5, 14.6) {\rotatebox{-90}{C}};
  \node at (-3.5, 6.5) {\rotatebox{-90}{Clight}};
  \node at (-3.5, 0.2) {\rotatebox{-90}{Exec}};
  \draw [->, dashed] (-2.5, 6.2) .. controls ++(0.5, 0.5) and (-2.5, 13.5) .. (C1.west);
  \draw [->, dashed] (-2.5, 6.2) .. controls ++(1, 0) and (-2.5, 10.5) .. (C2.west);
  \draw [->, dashed] (-2.5, 6.2) .. controls ++(1, 0) and (-2.5, 8.5) .. (C3.west);
  \draw [->, dashed] (-2.5, 6.2) .. controls ++(0.5, 0) and (-2.5, 1.5) .. (Cn.west);
  \draw [->] (-3.5, 13.95) -- (-3.5, 13.0);
  \draw [->] (-3.5, 12) -- (-3.5, 7.6);
  \draw [->] (-3.5, 5.3) -- (-3.5, 4.6);
  \draw [->] (-3.5, 4.5) -- (-3.5, 1.25);
  %% \draw [test] (-6.9, -0.75) rectangle (-5.5, 15.75);
  %% \draw [test] (-4.2, 9.9) rectangle (-2.8, 11.95);
  %% \draw [test] (-4.2, 3.05) rectangle (-2.8, 5.1);
  \draw [ourlib, fill=accentcolor] (-6.9, -0.75) -- (-5.5, -0.75) -- (-5.5, 1.8) -- (-2.5, 1.8) -- (-2.5, 4.6) -- (-5.5, 4.6) -- (-5.5, 8.1) --
  (-2.5, 8.1) -- (-2.5, 13.0) -- (-5.5, 13) -- (-5.5, 15.75) -- (-6.9, 15.75) -- cycle;
  \node at (-4, 10.55) {\rotatebox{-90}{\begin{tabular}{c} Parser\\ Type Checker \\ Simplifier \end{tabular}}};
  \node at (-4, 3.2) {\rotatebox{-90}{\begin{tabular}{c} Verified \\ Compiler \end{tabular}}};
  \node at (-6.2, 7.5) {\rotatebox{-90}{The CompCert Project}};
  \draw [ourlib, fill=accentcolor] (-6.9, -3.75) rectangle (6.3, -1.75);
  \node at (-0.3, -2.75) {Verified Software Toolchain};
  \draw [ourlib, fill=maincolor] (-6.9, 16.75) rectangle (6.3, 18.75);
  \node at (-0.3, 17.75) {\color{white}Mathematical Graph Library};
  \node at (3.2, 7.5) {\rotatebox{-90}{\color{white}Verification of a Graph-Manipulating Function}};
  \draw [ourlib, fill=maincolor] (4.9, -0.75) rectangle (6.3, 15.75);
  \node at (5.6, 7.5) {\rotatebox{-90}{\color{white}Spatial Graph Library}};

  %% down interface
  \draw [line width=3pt] (5.6, -0.35) -- +(0, -0.7) +(0, -1.05) -- +(0, -1.8) +(0.3, -1) arc [start angle=0, end angle=180, radius=0.3];
  \fill (5.6, -1.35) circle [radius=0.15];
  \draw [line width=3pt] (3.2, -0.35) -- +(0, -0.7) +(0, -1.05) -- +(0, -1.8) +(0.3, -1) arc [start angle=0, end angle=180, radius=0.3];
  \fill (3.2, -1.35) circle [radius=0.15];

  %% up interface
  \draw [line width=3pt] (-6.2, -2.15) -- +(0, 0.7) +(0, 1.05) -- +(0, 1.8) +(0.3, 1) arc [start angle=0, end angle=-180, radius=0.3];
  \fill (-6.2, -1.15) circle [radius=0.15];
  \draw [line width=3pt] (3.2, 15.35) -- +(0, 0.7) +(0, 1.05) -- +(0, 1.8) +(0.3, 1) arc [start angle=0, end angle=-180, radius=0.3];
  \fill (3.2, 16.35) circle [radius=0.15];
  \draw [line width=3pt] (5.6, 15.35) -- +(0, 0.7) +(0, 1.05) -- +(0, 1.8) +(0.3, 1) arc [start angle=0, end angle=-180, radius=0.3];
  \fill (5.6, 16.35) circle [radius=0.15];

  %% right interface
  \draw [line width=3pt] (3.5, 7.5) -- +(0.7, 0) +(1.05, 0) -- +(1.8, 0) +(1, 0.3) arc [start angle=90, end angle=270, radius=0.3];
  \fill (4.5, 7.5) circle [radius=0.15];
  \draw [line width=3pt] (3.5, 11.5) -- +(0.7, 0) +(1.05, 0) -- +(1.8, 0) +(1, 0.3) arc [start angle=90, end angle=270, radius=0.3];
  \fill (4.5, 11.5) circle [radius=0.15];  
  \draw [line width=3pt] (3.5, 3.5) -- +(0.7, 0) +(1.05, 0) -- +(1.8, 0) +(1, 0.3) arc [start angle=90, end angle=270, radius=0.3];
  \fill (4.5, 3.5) circle [radius=0.15];
\end{tikzpicture}
