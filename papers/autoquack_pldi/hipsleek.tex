HIP/SLEEK is a toolset for verifying programs using separation logic~\cite{chin:hipsleek}.  As compared to VST, H/S has a heavier focus on automation: \emph{e.g.}, users need only specify loop invariants and the pre/postconditions of methods, rather than describing each program point.  H/S has two interlocking components.  HIP applies Hoare rules using forward reasoning to verify programs, \emph{i.e.} each entailment is of the form $P |- Q * R$, where the heap from the antecedent~$P$ is matched with the consequent $Q$ and a frame/residue $R$. To check these separation logic entailments and calculate~$R$, HIP calls SLEEK.  SLEEK handles spatial operators such as~$*$ and $|->$ before handing any remaining pure entailments to a variety of external solvers such as Z3.  One of H/S's distinguishing features is support for user-defined recursive predicates.  HIP/SLEEK has been used to verify programs manipulating data structures like lists, arrays, and trees.

\begin{figure}[t]
\lstset{otherkeywords={data,relation,rlemma,abstract,axiom,self,null,or}}
  \begin{lstlisting}
$\label{code:hipdatanode}$data node { int val; node left; node right; }

$\label{code:hipstartrelation}$relation lookup(abstract G, node x,
  int d, node l, node r).
$\label{code:hipmark1relation}$relation update(abstract G, node x, int d,
  abstract G1).
$\label{code:hipmarkrelation}$relation mark(abstract G, node x, abstract G1).
relation
  subset_reach(abstract G, node x, abstract G1).
relation
$\label{code:hipendrelation}$ eq_notreach(abstract G, node x, abstract G1).

$\label{code:hipaxiom}$axiom lookup(G,x,1,l,r) ==> mark(G,x,G).
$\label{code:hipmarkramconnect}$axiom mark(G,x,G1) & lookup(G,y,v,l,r) ==>
  subset_reach(G,x,G1) & eq_notreach(G,x,G1) &
$\label{code:hipmarkramconnectend}$  lookup(G1,y,_,l,r).
// ... other $\color{red!80!black}\tx{axioms elided ...}$

$\label{code:hipbegingraphdef}$graph<G> == self = null or
 self::node<v,l,r> U* l::graph<G> U* r::graph<G>
$\label{code:hipendgraphdef}$ & lookup(G,self,v,l,r);

$\label{code:hipbeginlemma}$rlemma "subgraphupdate_l" l::graph<G1> *
 (l::graph<G> --@ (x::node<v,l,r> U*
 (l::graph<G> U* r::graph<G>))) &
 subset_reach(G,l,G1) & eq_notreach(G,l,G1)
 & lookup(G,x,v,l,r) & lookup(G1,x,v1,l,r)
  -> x::node<v1,l,r> U*
$\label{code:hipendlemma}$   (l::graph<G1> U* r::graph<G1>);
// ... other ramification lemmas elided ...

void mark(node x)
requires x::graph<G> $\label{code:hipmarkstart}$
ensures x::graph<G1> & mark(G,x,G1); { $\label{code:hipmarkend}$
  node l, r;
  if (x == null) return;
  else {
    if (x.val == 1) return; $\label{ifcondmark}$
    l = x.left;
    r = x.right;
    x.val = 1;
    mark(l);$\label{beforemarkleft}$
    mark(r);
} } $\label{code:hipmarkstop}$
\end{lstlisting}
%% \vspace{-8pt}
\vspace{-2ex}
\caption{Bigraph marking in HIP/SLEEK}
\vspace{-2ex}
\label{fig:hipmarkgraph}
\lstset{deletekeywords={data,relation,rlemma,abstract,axiom,self,null,or}}
\end{figure}

Figure~\ref{fig:hipmarkgraph} gives most of the HIP/SLEEK file used to verify \li{mark} (we have elided about 15 lines to save space).  We specify the pre- and postcondition of \li{mark} in line~\ref{code:hipmarkstart} and~\ref{code:hipmarkend}) respectively; notice that this is the same specification we verified in VST in Figure~\ref{fig:markgraph}.  It is quite unusual for H/S to verify the full functional correctness of an algorithm since its focus on heavier automation tends to trade off proving very exact specifications (\emph{e.g.} H/S proofs about lists usually treat their values as multisets instead of sequences).  We do not give H/S any ``hints'' at intermediate program points.

Line~\ref{code:hipdatanode} defines the recursive data structure \li{node} and lines~\ref{code:hipbegingraphdef}--\ref{code:hipendgraphdef} use H/S's normal user-defined recursive predicate feature to define the \li{graph} predicate.
H/S uses the keyword \li{self} to refer to the object being defined; in other words H/S's definition is very close to the fold/unfold relationship given as equation~\eqref{eqn:bigraphintrofoldunfold} in Figure~\ref{fig:markgraph} and we can justify its soundness as in \S\ref{sec:goodgraph}.  H/S uses the \li{::} operator for $|->$ and to provide the root pointer for a recursive predicate, \li{U*} for $**$ and automatically quantifies free variables existentially.  H/S does not need alignment restrictions because it enjoys a Java-like memory model with objects and fields: \emph{i.e.} it is not possible to have a pointer pointing into the ``middle'' of a record.  The last line of the \li{graph} predicate definition~(\ref{code:hipendgraphdef}) includes the \li{lookup} abstract relation.  In Figure~\ref{fig:markgraph}, this would be written $\gamma(\li{self})= (v,l,r)$.

To implement ramifications in HIP/SLEEK we extended its lemma system~\cite{NguyenC08}.
Normal lemmas in H/S are user defined, automatically checked, and automatically
applied in program verifications.  In contrast, our new lemmas are still user defined and still automatically applied in program verifications, but \textbf{not} automatically checked, so we call them \emph{externally verified lemmas}; their key advantage is that they can be much more complex.

The first step to adding such lemmas is to add abstract relations; the relations used by the \li{mark} verification are in lines~\ref{code:hipstartrelation}--\ref{code:hipendrelation}.  We have already met \li{lookup}; lines~\ref{code:hipmark1relation} and~\ref{code:hipmarkrelation} are how H/S models \m{mark1} and \m{mark}, respectively.  Users do not provide any definitions for these relations, but we do give H/S some axioms for how they behave: \emph{e.g.} line~\ref{code:hipaxiom} tells H/S that if the root of a graph is marked then we can consider the whole graph marked.  This axiom is used on line~\ref{ifcondmark} to safely \li{$\tx{return}$} once we encounter an already-marked \li{node}.  Users also provide spatial ramification rules as in lines~\ref{code:hipbeginlemma}--\ref{code:hipendlemma}.  %H/S only supports first order quantification so these are a little unpleasant to state, being essentially very concrete versions of the ramification library~(\S\ref{sec:ramifylib}).

\begin{figure}[t]
\lstset{deletekeywords={union},otherkeywords={Module,Type,Prop,Parameter,Axiom,End,forall},numbers=none}
  \begin{lstlisting}
Module Type Mgraphmark.
 ...
 Parameter G : Type.
 Parameter node : Type.
 Parameter graph : node -> G -> formula.
 Parameter mark : G -> node -> G -> formula.
 ...
 Axiom axiom_3 : forall l r x G,
   valid (imp (lookup G x true l r) (mark G x G)).
 ...
 Axiom subgraphupdate_l : forall G v G1 x v1 l r,
  valid (imp (and (star (graph l G1)
  (mwand (graph l G) (union (ptto_node x v l r)
  (union (graph l G) (graph r G)))))
  (and (subset_reach G l G1) (and
  (eq_notreach G l G1) (and (lookup G x v l r)
  (lookup G1 x v1 l r)))))
  (union (ptto_node x v1 l r) (union (graph l G1)
  (graph r G1)))).
 ...
End Mgraphmark.
\end{lstlisting}
%% \vspace{-8pt}
\lstset{deletekeywords={Module,Type,Prop,Parameter,Axiom,End,forall}}
\vspace{-2ex}
\caption{Coq \li{$\tx{Module Type}$} generated by HIP/SLEEK}
\vspace{-2ex}
\label{fig:hipcoqfile}
\end{figure}

The lemmas and axioms that HIP/SLEEK has used in the verification still need to be checked.  Accordingly, HIP/SLEEK generates a Coq file with a \li{Module Type} specifying them.  The \li{Module Type} generated for \li{mark} is given in Figure~\ref{fig:hipcoqfile} to illustrate this process.  To obtain a complete verification for \li{mark}, users must build a Coq \li{Module} that satisfies \li{Mgraphmark} using definitions for the abstract relations that they think are reasonable (in particular, for the \li{mark} relation, since it is used in the specification of the algorithm).

\subsection{Automatic localizations in HIP/SLEEK}

HIP/SLEEK reasons about graph algorithms using three key ideas.  First, H/S uses fold/unfold relationships whenever it needs to reason about a recursive predicate during the entailment checking that occurs during forward reasoning. For example, to check the dereference in line \ref{ifcondmark}, the \li{x::graph<G>} predicate is unfolded to reach\footnote{We will write predicates in a more conventional mathematical style rather than the ASCII format used by H/S.}
\begin{align*}
\exists m,l,r.~ & \gamma(\li{x}) \! = \! (m,l,r) /| \null \\
& \li{x} \! |-> \! m,l,r ** \p{graph}(l, \gamma) ** \p{graph}(r, \gamma)
\end{align*}

% (\emph{e.g.} $\li{x} \mapsto m,l,r$)
Second, once HIP/SLEEK realizes that its forward reasoning needs to isolate a predicate that is $**$-shared, it uses the existential wand ``septraction'' operator~$\septraction$~(defined in Figure~\ref{fig:seplogsem}) to reach the strongest post condition.  The existential wand $--o$ is simpler to introduce than the universal one $--*$ during forward reasoning because H/S already knows $G_1$ and $L_1$ when it needs to calculate a residual frame~$R$.  For example, the precondition at line~\ref{beforemarkleft} is
\[
\begin{array}{l}
\gamma(\li{x}) = (0,\li{l},\li{r}) /| \m{mark1}(\gamma,\li{x},\gamma') /| \null \\
\quad \li{x} |-> 1,\li{l},\li{r} ** \p{graph}(\li{l}, \gamma') ** \p{graph}(\li{r}, \gamma')
\end{array}
\]
and since H/S knows the call to \li{mark(l)} requires $\p{graph}(\li{l}, \gamma')$, it can calculate the frame residue $R$ as $Q --o P$ since\footnote{We omit pure conjuncts \emph{e.g.} $\m{mark1}(\gamma,\li{x},\gamma')$ from both $P$ and $R$.}
\begin{gather*}
\overbrace{\li{x} |-> 1,\li{l},\li{r} ** \p{graph}(\li{l}, \gamma') ** \p{graph}(\li{r}, \gamma')}^P |- \overbrace{\p{graph}(\li{l}, \gamma')}^Q * \null \\
\underbrace{\Big(\!\p{graph}(\li{l}, \gamma') \! --o \! \big(\li{x} \! |-> \! 1,\!\li{l},\!\li{r}  **  \p{graph}(\li{l},\! \gamma')  **  \p{graph}(\li{r}, \!\gamma')\big)\!\Big)}_{R}
\end{gather*}
Note that $P |- Q * (Q --o P)$ is \textbf{not} a tautology since not every $P$ has a $Q$ ``hiding inside it''.  However, H/S's unfold of \p{graph} makes it clear that $\p{graph}(l, \gamma)$ \textbf{is} inside the premise of the entailment $P$, so H/S is justified calculating this $R$. %in calculating the frame $R$.

To calculate the postcondition of line~\ref{beforemarkleft}, H/S $*$-combines the residue~$R$ with the postcondition of \li{mark(l)} to reach: %, \emph{i.e.}: % to reach% $\p{graph}(\li{l}, \gamma') /| \m{mark}(\gamma,\li{l},\gamma')$, to reach
\begin{gather*}
\big(\p{graph}(\li{l}, \gamma'')  /|  \m{mark}(\gamma',\li{l},\gamma'')\big)  *  \gamma(\li{x})  =  (0,\li{l},\li{r})  /| \null\\
\m{mark1}(\gamma,\li{x},\gamma') /| \Big(\p{graph}(\li{l}, \gamma') --o \\ \big(x |-> 1,\li{l},\li{r} ** \p{graph}(\li{l}, \gamma') ** \p{graph}(\li{r}, \gamma')\big)\Big)
\end{gather*}
%
%, where $\gamma'$ is the mathematical graph after the \texttt{mark(l)} recursive call.

The final step is to use lemmas to eliminate $\septraction$. After line~\ref{beforemarkleft} H/S uses the lemma from lines~\ref{code:hipbeginlemma}--\ref{code:hipendlemma}:
\[
\begin{array}{@{}l@{}}
\bigg(\!\Big(\!\p{graph}(l, \gamma) --o \big(x |-> m,l,r ** \p{graph}(l, \gamma) \ocon \p{graph}(r, \gamma)\big)\!\Big) \\
~ \null * \p{graph}(l, \gamma') \! /| \! \m{subset\_reach}(\gamma,l,\!\gamma') \! /| \! \m{eqnot\_reach}(\gamma,l,\!\gamma') \\
~ \null /| \gamma(x)=(m,l,r) /| \gamma'(x)=(m1,l,r)\bigg) => \\
\quad \qquad \big(x \mapsto m1,l,r \ocon \p{graph}(l, \gamma') \ocon \p{graph}(r, \gamma')\big)
\end{array}
\]
This lemma does not mention $\m{mark}$ explicitly, allowing it to be reused in other algorithms.  The connection between the \m{mark}, \m{subset\_reach}, and \m{eqnot\_reach} relations is given by the axiom on lines~\ref{code:hipmarkramconnect}--\ref{code:hipmarkramconnectend}; other (elided) lemmas connect \m{mark1} as well.  H/S must apply all of them, guided by structural matching, to eliminate the $--o$ to reach the following relatively pleasant-looking postcondition of line~\ref{beforemarkleft}
\[
\begin{array}{l}
\gamma(\li{x}) = (0,\li{l},\li{r}) /| \m{mark1}(\gamma,\li{x},\gamma') /| \m{mark}(\gamma',\li{l},\gamma'') /| \null \\
\quad \li{x} |-> 1,\li{l},\li{r} ** \p{graph}(\li{l}, \gamma'') ** \p{graph}(\li{r}, \gamma'')
\end{array}
\]
H/S now continues the verification with \li{mark(r)}.

\paragraph{Other examples in H/S.} In the future we would like to get other examples such as \li{span} and \li{copy} to go through.  The largest issue is the requirement in H/S that each predicate have a ``root pointer,'' (\li{self} in lines~\ref{code:hipbegingraphdef}--\ref{code:hipendgraphdef}).  This prevents us from defining notions such ``set of spatial graphs'', which have sets of root pointers rather than a single such pointer.  Without such notions it is quite difficult to find a proof for our other examples, even with only pen and paper.  Unfortunately, this requirement is deeply ``baked in'' to H/S and will require substantial effort to modify; see \S\ref{sec:future}.
