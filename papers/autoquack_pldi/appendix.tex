
\appendix

\section{Spanning and Copying}
\label{apx:spanning}

\begin{figure}[htbp]
  \begin{lstlisting}
struct Node {
    int m;
    struct Node * l;
    struct Node * r; };

// We use $R$ to represent $\p{reachable}(\gamma,\tx x)$

void spanning(struct Node * x) { 
// $\{\p{graph}(\tx{x},\gamma)/|\gamma(\tx{x}).1=0\}$ 
    struct Node * l, * r; int root_mark;
// $\{\p{graph}(\tx x,\gamma) /| \exists l,r.~ \gamma(\tx{x}) = (0,l,r)\}$
// $\{\p{graph}(\tx x,\gamma) /| \gamma(\tx{x}) = (0,l,r)\}$
// $\{\p{vertices\_at}(\p{reachable}(\gamma,\tx x), \gamma) /| \gamma(\tx{x}) = (0,l,r)\}$
// $\{\p{vertices\_at}(R, \gamma) /| \gamma(\tx{x}) = (0,l,r)\}$
// $\searrow \{\tx x|-> 0,l,r /| \gamma(\tx{x}) = (0,l,r)\}$
    l = x -> l;
    r = x -> r;
    x -> m = 1;
// $\swarrow \{\tx x|-> 1,\tx{l},\tx{r} /| \gamma(\tx{x}) = (0,\tx{l},\tx{r}) /| \exists \gamma_1.~ \m{mark1}(\gamma, \tx{x}, \gamma_1)\}$
// $\{\exists \gamma_1.~\p{vertices\_at}(R, \gamma_1) /| \gamma(\tx{x}) = (0,\tx{l},\tx{r}) /| \m{mark1}(\gamma, \tx{x}, \gamma_1)\}$
// $\{\p{vertices\_at}(R, \gamma_1) /| \gamma(\tx{x}) = (0,\tx{l},\tx{r}) /| \m{mark1}(\gamma, \tx{x}, \gamma_1)\}$
    if (l) {
        root_mark = l -> m;
        if (root_mark == 0) {
            spanning(l);
        } else { x -> l = 0; } }
// $\left\{\!\!\!\begin{array}{l@{}}\exists\gamma_2. ~\p{vertices\_at}(R,\gamma_2)/| \gamma(\tx{x}) = (0,\tx{l},\tx{r})  /| \null \\ \m{mark1}(\gamma, \tx{x}, \gamma_1) /| \m{e\_span}(\gamma_1,\tx{x}.\text{L},\gamma_2)\end{array}\right\}$
// $\left\{\!\!\!\begin{array}{l@{}}\p{vertices\_at}(R,\gamma_2)/| \gamma(\tx{x}) = (0,\tx{l},\tx{r})  /| \null \\ \m{mark1}(\gamma, \tx{x}, \gamma_1) /|  \m{e\_span}(\gamma_1,\tx{x}.\text{L},\gamma_2)\end{array}\right\}$
    if (r) {
        root_mark = r -> m;
        if (root_mark == 0) {
           spanning(r);
        } else { x -> r = 0; } }
// $\left\{\!\!\!\begin{array}{l@{}}\exists\gamma_3.~\p{vertices\_at}(R,\gamma_3)/| \gamma(\tx{x}) = (0,\tx{l},\tx{r})  /| \null \\ \m{mark1}(\gamma, \tx{x}, \gamma_1) /| \m{e\_span}(\gamma_1,\tx{x}.\text{L},\gamma_2) /| \m{e\_span}(\gamma_2,\tx{x}.\text{R},\gamma_3)\end{array}\right\}$
} // $\{\exists \gamma_3.~\p{vertex\_at}(\p{reachable}(\gamma, \tx{x}), \gamma_3)/|\m{span}(\gamma,\tx{x},\gamma_3)\}$
  \end{lstlisting}
  \small
\begin{gather*}
  \p{vertices\_at}(\p{reachable}(\gamma_1, \tx x), \gamma_2)\defeq \underset{v\in\p{reachable}(\gamma_1, \tx x)}{\bigstar}v \mapsto \gamma_2(v)\\
  \begin{split}
  \m{span}(\gamma_1, \tx x, \gamma_2)\defeq &\m{mark}(\gamma_1,\tx x, \gamma_2) /| \gamma_1\!\uparrow(\lambda v. x\mathrel{{\stackrel{\gamma_1~}{\leadsto^{\star}_{0}}}} v) \text{ is a tree} /| \null\\
  & \gamma_1\!\uparrow\!(\lambda v.\neg x\mathrel{{\stackrel{\gamma_1~}{\leadsto^{\star}_{0}}}} v) = \gamma_2\!\uparrow\!(\lambda v.\neg x\mathrel{{\stackrel{\gamma_1~}{\leadsto^{\star}_{0}}}} v) /| \null\\
  & (\forall v.~x\mathrel{{\stackrel{\gamma_1~}{\leadsto^{\star}_{0}}}} v => \gamma_2\models x \leadsto v) /| \null\\
  & (\forall a,b.~x\mathrel{{\stackrel{\gamma_1~}{\leadsto^{\star}_{0}}}} a => \neg x\mathrel{{\stackrel{\gamma_1~}{\leadsto^{\star}_{0}}}} b => \neg \gamma_2\models a \leadsto b)\\
  \end{split}\\
  \m{e\_span}(\gamma_1, e, \gamma_2)\defeq
  \begin{cases}
    \gamma_1 - e = \gamma_2  & t(\gamma_1,e)=1\\
    \m{span}(\gamma_1, t(\gamma_1,e), \gamma_2) & t(\gamma_1,e)=0\\
  \end{cases}
\end{gather*}
\caption{Clight code and proof sketch for bigraph spanning tree.}
\label{fig:spanning}

\end{figure}


In Figure~\ref{fig:spanning} we show a simplified proof script for
the spanning tree algorithm.  Unlike graph marking, the spanning tree algorithm changes the
structure of the graph, leading to a more complicated specification,
in both the pure part and the spatial part. Observe that the $\m{span}$ relation is
rather long; the $\m{e\_span}$ handles the case of either calling spanning tree or deleting an edge.

We put the proof sketch of the graph copying algorithm in
Figure~\ref{fig:copy-part1} and Figure~\ref{fig:copy-part2}. Just like
other parts of the paper, both algorithms have been machine verified.

\begin{figure}[htbp]
  \begin{lstlisting}
struct Node {
    int m;
    struct Node * l;
    struct Node * r; };

// We use $x \xleftrightarrow{\gamma} x'$ to represent $x = x' = 0 \vee \gamma(x) = (x', \_, \_)$

struct Node * copy(struct Node * x) { 
    struct Node * l, * r, * x0, * l0, * r0;
// $\{\p{graph}(\tx{x},\gamma)\}$
      if (x == 0)
        return 0;
// $\{\p{graph}(\tx{x},\gamma)/| x\neq 0\}$
// $\{\p{graph}(\tx x,\gamma) /| \exists x_0,l,r.~ \gamma(\tx{x}) = (x_0,l,r)\}$
// $\{\p{graph}(\tx x,\gamma) /| \gamma(\tx{x}) = (x_0,l,r)\}$
// $\searrow \{\tx x|-> x_0,l,r /| \gamma(\tx{x}) = (x_0,l,r)\}$
      x0 = x -> m;
// $\swarrow \{\tx x|-> x_0,l,r /| \gamma(\tx{x}) = (x_0,l,r) /| \tx{x0} = x_0\}$
// $\{\p{graph}(\tx x,\gamma) /| \gamma(\tx{x}) = (\tx{x0},l,r)\}$
      if (x0 != 0)
        return x0;
// $\{\p{graph}(\tx x,\gamma) /| \gamma(\tx{x}) = (0,l,r) \}$
      x0 = (struct Node *) mallocN (sizeof (struct Node));
// $\{\p{graph}(\tx x,\gamma) * \tx{x0} |-> \_, \_, \_ /| \gamma(\tx{x}) = (0,l,r) \}$
// $\searrow \{\tx x|-> 0,l,r * \tx{x0} |-> \_, \_, \_ /| \gamma(\tx{x}) = (0,l,r)\}$
      l = x -> l;
      r = x -> r;
      x -> m = x0;
      x0 -> m = 0;
// $\swarrow \left\{\!\!\!\begin{array}{l@{}} \tx x|-> \tx{x0},\tx{l},\tx{r} * \tx{x0} |-> 0, \_, \_ /| \\ \gamma(\tx{x}) = (0,\tx{l},\tx{r}) /|  \exists \gamma_1 \gamma_1'. \m{v\_copy1}(\gamma, \tx{x}, \gamma_1, \gamma_1') \end{array}\right\}$
// $\left\{\!\!\!\begin{array}{l@{}}\exists \gamma_1 \gamma_1'. \p{graph}(\tx x,\gamma_1) * \tx{x0} |-> 0, \_, \_ /| \\ \gamma(\tx{x}) = (0,\tx{l},\tx{r}) /| \m{v\_copy1}(\gamma, \tx{x}, \gamma_1, \gamma_1')\end{array}\right\}$
// $\left\{\!\!\!\begin{array}{l@{}} \p{graph}(\tx x,\gamma_1) * \tx{x0} |-> 0, \_, \_ /| \\ \gamma(\tx{x}) = (0,\tx{l},\tx{r}) /| \m{v\_copy1}(\gamma, \tx{x}, \gamma_1, \gamma_1')\end{array}\right\}$
// $\left\{\!\!\!\begin{array}{l@{}} \p{graph}(\tx x,\gamma_1) * \tx{x0} |-> 0, \_, \_ * \p{holegraph}(\tx{x0}, \gamma_1') /| \\ \gamma(\tx{x}) = (0,\tx{l},\tx{r}) /| \m{v\_copy1}(\gamma, \tx{x}, \gamma_1, \gamma_1')\end{array}\right\}$
// $\searrow \{\p{graph}(\tx l,\gamma_1)\}$
      l0 = copy(l);
// $\swarrow \left\{\!\!\!\begin{array}{l@{}} \exists \gamma_2 \gamma_2''. \p{graph}(\tx{l},\gamma_2) * \p{graph}(\tx{l0}, \gamma_2'') /| \\ \m{copy}(\gamma_1, \tx{l}, \gamma_2, \gamma_2'') /| \tx{l} \xleftrightarrow{\gamma_2} \tx{l0} \end{array}\right\}$
// $\left\{\!\!\!\begin{array}{l@{}} \exists \gamma_2 \gamma_2''. \p{graph}(\tx{x},\gamma_2)  * \tx{x0} |-> 0, \_, \_ * \p{holegraph}(\tx{x0}, \gamma_1') * \\ \p{graph}(\tx{l0}, \gamma_2'') /|\gamma(\tx{x}) = (0,\tx{l},\tx{r}) /| \m{v\_copy1}(\gamma, \tx{x}, \gamma_1, \gamma_1')/| \\ \m{copy}(\gamma_1, \tx{l}, \gamma_2, \gamma_2'') /| \tx{l} \xleftrightarrow{\gamma_2} \tx{l0}\end{array}\right\}$
// $\left\{\!\!\!\begin{array}{l@{}} \exists \gamma_2 \gamma_2'. \p{graph}(\tx{x},\gamma_2)  * \tx{x0} |-> 0, \_, \_ * \p{holegraph}(\tx{x0}, \gamma_2') /| \\ \gamma(\tx{x}) = (0,\tx{l},\tx{r}) /| \m{v\_copy1}(\gamma, \tx{x}, \gamma_1, \gamma_1') /| \\ \m{e\_copy}(\gamma_1, \gamma_1', \tx{x}.L, \gamma_2, \gamma_2') /| \tx{l} \xleftrightarrow{\gamma_2} \tx{l0} \end{array}\right\}$
// $\left\{\!\!\!\begin{array}{l@{}}\p{graph}(\tx{x},\gamma_2)  * \tx{x0} |-> 0, \_, \_ * \p{holegraph}(\tx{x0}, \gamma_2') /| \\ \gamma(\tx{x}) = (0,\tx{l},\tx{r}) /| \m{v\_copy1}(\gamma, \tx{x}, \gamma_1, \gamma_1') /| \\ \m{e\_copy}(\gamma_1, \gamma_1', \tx{x}.L, \gamma_2, \gamma_2') /| \tx{l} \xleftrightarrow{\gamma_2} \tx{l0} \end{array}\right\}$
      x0 -> l = l0;
// $\left\{\!\!\!\begin{array}{l@{}} \p{graph}(\tx{x},\gamma_2)  * \tx{x0} |-> 0, \tx{l0}, \_ * \p{holegraph}(\tx{x0}, \gamma_2') /| \\ \gamma(\tx{x}) = (0,\tx{l},\tx{r}) /| \m{v\_copy1}(\gamma, \tx{x}, \gamma_1, \gamma_1') /| \\ \m{e\_copy}(\gamma_1, \gamma_1', \tx{x}.L, \gamma_2, \gamma_2') /| \tx{l} \xleftrightarrow{\gamma_2} \tx{l0}\end{array}\right\}$
// $\searrow \{\p{graph}(\tx r,\gamma_2)\}$
      r0 = copy(r);
// $\swarrow \left\{\!\!\!\begin{array}{l@{}} \exists \gamma_3 \gamma_3''. \p{graph}(\tx{r},\gamma_3) * \p{graph}(\tx{r0}, \gamma_3'') /| \\ \m{copy}(\gamma_2, \tx{r}, \gamma_3, \gamma_3'') /| \tx{r} \xleftrightarrow{\gamma_3} \tx{r0}\end{array}\right\}$
// $\left\{\!\!\!\begin{array}{l@{}} \exists \gamma_3 \gamma_3''. \p{graph}(\tx{x},\gamma_3) * \tx{x0} |-> 0, \tx{l0}, \_ * \p{holegraph}(\tx{x0}, \gamma_3') * \\ \p{graph}(\tx{r0}, \gamma_3'')  /| \gamma(\tx{x}) = (0,\tx{l},\tx{r}) /| \m{v\_copy1}(\gamma, \tx{x}, \gamma_1, \gamma_1') /| \\ \m{e\_copy}(\gamma_1, \gamma_1', \tx{x}.L, \gamma_2, \gamma_2') /| \m{copy}(\gamma_2, \tx{r}, \gamma_3, \gamma_3'') /| \\ \tx{l} \xleftrightarrow{\gamma_2}  \tx{l0} /| \tx{r} \xleftrightarrow{\gamma_3} \tx{r0} \end{array}\right\}$
// $\left\{\!\!\!\begin{array}{l@{}} \exists \gamma_3 \gamma_3'. \p{graph}(\tx{x},\gamma_3) * \tx{x0} |-> 0, \tx{l0}, \_ * \p{holegraph}(\tx{x0}, \gamma_3')  /| \\ \gamma(\tx{x}) = (0,\tx{l},\tx{r}) /| \m{v\_copy1}(\gamma, \tx{x}, \gamma_1, \gamma_1') /| \\ \m{e\_copy}(\gamma_1, \gamma_1', \tx{x}.L, \gamma_2, \gamma_2') /| \m{e\_copy}(\gamma_2, \gamma_2', \tx{x}.R, \gamma_3, \gamma_3') /| \\ \tx{l} \xleftrightarrow{\gamma_2} \tx{l0} /| \tx{r} \xleftrightarrow{\gamma_3} \tx{r0} \end{array}\right\}$
  \end{lstlisting}
\caption{Proof sketch for bigraph copy - part 1}
\label{fig:copy-part1}
\end{figure}

\newpage 

\begin{figure}[htbp]
  \begin{lstlisting}
// $\left\{\!\!\!\begin{array}{l@{}} \p{graph}(\tx{x},\gamma_3) * \tx{x0} |-> 0, \tx{l0}, \_ * \p{holegraph}(\tx{x0}, \gamma_3')  /| \\ \gamma(\tx{x}) = (0,\tx{l},\tx{r}) /| \m{v\_copy1}(\gamma, \tx{x}, \gamma_1, \gamma_1') /| \\ \m{e\_copy}(\gamma_1, \gamma_1', \tx{x}.L, \gamma_2, \gamma_2') /| \m{e\_copy}(\gamma_2, \gamma_2', \tx{x}.R, \gamma_3, \gamma_3') /| \\ \tx{l} \xleftrightarrow{\gamma_2} \tx{l0} /| \tx{r} \xleftrightarrow{\gamma_3} \tx{r0} \end{array}\right\}$
      x0 -> r = r0;
// $\left\{\!\!\!\begin{array}{l@{}} \p{graph}(\tx{x},\gamma_3) * \tx{x0} |-> 0, \tx{l0}, \tx{r0} * \p{holegraph}(\tx{x0}, \gamma_3')  /| \\ \gamma(\tx{x}) = (0,\tx{l},\tx{r}) /| \m{v\_copy1}(\gamma, \tx{x}, \gamma_1, \gamma_1') /| \\ \m{e\_copy}(\gamma_1, \gamma_1', \tx{x}.L, \gamma_2, \gamma_2') /| \m{e\_copy}(\gamma_2, \gamma_2', \tx{x}.R, \gamma_3, \gamma_3') /| \\ \tx{l} \xleftrightarrow{\gamma_2} \tx{l0} /| \tx{r} \xleftrightarrow{\gamma_3} \tx{r0} \end{array}\right\}$
// $\left\{\p{graph}(\tx{x},\gamma_3) * \p{graph}(\tx{x0}, \gamma_3') /| \m{copy}(\gamma, \tx{x}, \gamma_3, \gamma_3') /| \tx{x} \xleftrightarrow{\gamma_3} \tx{x0}\right\}$
  \end{lstlisting}
  \small
\begin{gather*}
  \p{holegraph}(x, \gamma)\defeq \underset{v\in\p{reachable}(\gamma, \tx x) - \{x\}}{\bigstar}v \mapsto \gamma(v)\\
\, \\
  \begin{split}
  \m{iso}(f_V, f_E, \gamma_1, \gamma_2) \defeq 
& f_V \text{ is a bijection between $\phi_V(\gamma_1)$ and $\phi_V(\gamma_2)$} /| \\
& f_E \text{ is a bijection between $\phi_E(\gamma_1)$ and $\phi_E(\gamma_2)$}  /| \\
& \forall e, f_V(s(\gamma_1, e)) = s(\gamma_2, f_E(e)) /| \\
& \forall e, f_V(d(\gamma_1, e)) = d(\gamma_2, f_E(e)) 
  \end{split} \\
  \begin{split}
  \m{v\_copy1}(\gamma_1, x, \gamma_2, \gamma_2')\defeq  \exists x'. & x \neq 0 /| \m{mark1}(\gamma_1,x, \gamma_2) /| \\
    & x \xleftrightarrow{\gamma_2} x' /| \gamma_2' = \{x_0\}
  \end{split}\\
  \begin{split}
  \m{copy}(\gamma_1, x, \gamma_2, \gamma_2')\defeq &\m{mark}(\gamma_1,x, \gamma_2) /| \\
   & \exists f_V f_E. \m{iso}(f_V, f_E,  \gamma_2\!\uparrow(\lambda v. x\mathrel{{\stackrel{\gamma_2~}{\leadsto^{\star}_{0}}}} v), \gamma_2') /| \\
   & \forall x x'. f_V(x) = x' \Leftrightarrow x \xleftrightarrow{\gamma_2} x'
  \end{split}\\
  \begin{split}
  \m{e\_copy}(\gamma_1, \gamma_1', e, \gamma_2, \gamma_2')\defeq & \exists \gamma_2''. \gamma_2' = \gamma_1' + \gamma_2'' /| \\
  & \m{mark}(\gamma_1,x, \gamma_2) /| \exists f_V f_E.  \\
   & \m{iso}(f_V, f_E,  \{e\} +\gamma_2\!\uparrow(\lambda v. x\mathrel{{\stackrel{\gamma_2~}{\leadsto^{\star}_{0}}}} v), \gamma_2'') /| \\
   & \forall x x'. f_V(x) = x' \Leftrightarrow x \xleftrightarrow{\gamma_2} x'
  \end{split}\\
  \begin{split}
&\text{Here, when we mention \m{mark} and \m{mark1}, the value 1s in the} \\
&\text{original definition are changed to non-zero values.}
  \end{split}\\
\end{gather*}
\caption{Proof sketch for bigraph copy - part 2}
\label{fig:copy-part2}
\end{figure}



\section{Proof of \infrulestyle{Ramify-P} and \infrulestyle{Ramify-PQ}}
\label{apx:ruleproofs}

\begin{figure*}
Proof of \infrulestyle{Ramify-P} from \infrulestyle{Frame} and \infrulestyle{Consequence}:
\vspace{-3em}
\[
\begin{array}{c}
\infrule{}{
  G_1 |- L_1 * \pguards{c}(L_2 --* G_2) \\
  \infrule{}{\{L_1\}~c~\{L_2\}}
            {\{L_1 * \pguards{c}(L_2 --* G_2)\}~c~\{L_2 * \pguards{c}(L_2 --* G_2)\}}{(1)} \\
  \infrule{}{
            \infrule{}{\stackrel{\langle c \rangle}{\cong} \text{ is reflexive}}{\pguards{c}(L_2 --* G_2) |- L_2 --* G_2}{(2)}}
            {L_2 * \pguards{c}(L_2 --* G_2) |- G_2}{(3)}}
{\{G_1\}~c~\{G_2\}}
{} \\
[5pt]
(1)~ \forall P.~ \pguards{c}P \text{ ignores } \FV(c) \qquad (2)~ \text{axiom T of modal logic} \qquad (3)~ (P * Q |- R) <=> (P |- Q --* R)
\end{array}
\]

Proof of \infrulestyle{Ramify-PQ} from \infrulestyle{Ramify-P}:
\vspace{-4em}
\[
\begin{array}{c}
\infrule{}
{
  \{L_1\}~c~\{\exists x.~ L_2\} \hspace{-0.5em} \\
  \infrule{}
  {
    G_1 |- L_1 * \pguards{c}\big(\forall x.~ (L_2 --* G_2)\big) \hspace{-0.5em} \\
    \infrule{}{
      \infrule{}{
        \infrule{}{
          \vdots
        } {
          \forall x.~ (L_2 --* G_2) |- (\exists x.~ L_2) --* (\exists x.~ G_2)
        } {(1)}
      } {
        \pguards{c}\big(\forall x.~ (L_2 --* G_2)\big) |- \pguards{c}\big((\exists x.~ L_2) --* (\exists x.~ G_2)\big)
      } {(2)}
    } {
      L_1 * \pguards{c}\big(\forall x.~ (L_2 --* G_2)\big) |- L_1 * \pguards{c}\big((\exists x.~ L_2) --* (\exists x.~ G_2)\big)
    } {}
  } {
    G_1 |- L_1 * \pguards{c}\big((\exists x.~ L_2) --* (\exists x.~ G_2)\big)
  } {}
} {
  \{G_1\}~c~\{\exists x.~ G_2\}
} {}
\\
[5pt]
(1)~ \text{tautology using $(P * Q |- R) <=> (P |- Q --* R)$} \qquad (2)~ \text{reduction using modal axioms K and N} %\qquad (3)~ (P |- Q) => (P * F |- Q * F)
\end{array}
\]
\caption{Proofs of \infrulestyle{Ramify-P} and \infrulestyle{Ramify-PQ}}
\label{fig:rampqproofs}
\end{figure*}

See Figure~\ref{fig:rampqproofs} for the proofs of \infrulestyle{Ramify-P} and \infrulestyle{Ramify-PQ}. 

\section{Difficulty using $\graphkt$}
\label{apx:problemrecgraph}

\begin{figure*}
\[
\infrule{}
{\infrule{}
  {100 |-> 42,100,0 ~ |- ~ 100 |-> 42,100,0 ** \graphkt(100,\hat{\gamma})}
  {100 |-> 42,100,0 ~ |- ~ \hat{\gamma}(100) = (42,100,0) ~ /| ~ 100 |-> 42,100,0 ** \graphkt(100,\hat{\gamma}) ** \graphkt(0,\hat{\gamma})}
  {(2)}
}
{100 |-> 42,100,0 ~ |- ~ \graphkt(100,\hat{\gamma})}
{(1)}
\]
(1) Unfold $\graphkt$, dismiss first disjunct (contradiction), introduce existentials (which must be 42,100,0) \\
(2) simplify using $P * \p{emp} -|- P$ and remove pure conjunct

\caption{An attempt to prove a ``simple'' entailment}
\label{fig:badcycle}
\end{figure*}

See Figure \ref{fig:badcycle} for an attempt to prove the entailment $100 |-> 42,100,0 ~ |- ~ \graphkt(100,\hat{\gamma})$.  Part of the problem is that the recursive structure interacts very badly with $**$: if the recursion involved $*$ then it \textbf{would} be provable, by induction on the finite memory (each ``recursive call'' would be on a strictly smaller subheap).  This is why Knaster-Tarski works so well with list, tree, and DAG predicates in separation logic.

\section{Problem with Appel and McAllester's fixpoint}
\label{apx:appelfixpiont}

Appel and McAllester proposed another fixpoint $\mu_{\mathsf{A}}$
that is sometimes used to define recursive predicates in separation
logic \cite{appel:fixpoint}.  This time the functional $F_P$ needs to be
\emph{contractive}, which to a first order of approximation means that
all recursion needs to be guarded by the ``approximation
modality''~$\rhd$~\cite{appel:vmm}, \emph{i.e.} our graph predicate would
look like
\begin{align*}
\grapham(x, \gamma) ~ &\stackrel{\Delta}{=}\\
 (x = 0 /| \p{emp}) & |/ \exists m,l,r.~ \gamma(x)=(m,l,r) /| \null \\
 x |-> m,l,r & ** \rhd \grapham(l, \gamma) ** \rhd \grapham(r, \gamma)
\end{align*}

Unfortunately, $\rhd P$ is not precise for all $P$, so $\grapham$ is not precise either.  The approximation modality's universal imprecision has never been noticed before.
