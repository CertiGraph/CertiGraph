
\appendix

\section{More examples}

\begin{figure}[htbp]
  \begin{lstlisting}
struct Node {
    int m;
    struct Node * l;
    struct Node * r; };

// We use $R$ to represent $\p{reachable}(\gamma,\tx x)$

void spanning(struct Node * x) { 
// $\{\p{graph}(\tx{x},\gamma)/|\gamma(\tx{x}).1=0\}$ 
    struct Node * l, * r; int root_mark;
// $\{\p{graph}(\tx x,\gamma) /| \exists l,r.~ \gamma(\tx{x}) = (0,l,r)\}$
// $\{\p{graph}(\tx x,\gamma) /| \gamma(\tx{x}) = (0,l,r)\}$
// $\{\p{vertices\_at}(\p{reachable}(\gamma,\tx x), \gamma) /| \gamma(\tx{x}) = (0,l,r)\}$
// $\{\p{vertices\_at}(R, \gamma) /| \gamma(\tx{x}) = (0,l,r)\}$
// $\searrow \{\tx x|-> 0,l,r /| \gamma(\tx{x}) = (0,l,r)\}$
    l = x -> l;
    r = x -> r;
    x -> m = 1;
// $\swarrow \{\tx x|-> 1,\tx{l},\tx{r} /| \gamma(\tx{x}) = (0,\tx{l},\tx{r}) /| \exists \gamma_1.~ \m{mark1}(\gamma, \tx{x}, \gamma_1)\}$
// $\{\exists \gamma_1.~\p{vertices\_at}(R, \gamma_1) /| \gamma(\tx{x}) = (0,\tx{l},\tx{r}) /| \m{mark1}(\gamma, \tx{x}, \gamma_1)\}$
// $\{\p{vertices\_at}(R, \gamma_1) /| \gamma(\tx{x}) = (0,\tx{l},\tx{r}) /| \m{mark1}(\gamma, \tx{x}, \gamma_1)\}$
    if (l) {
        root_mark = l -> m;
        if (root_mark == 0) {
            spanning(l);
        } else { x -> l = 0; } }
// $\left\{\!\!\!\begin{array}{l@{}}\exists\gamma_2. ~\p{vertices\_at}(R,\gamma_2)/| \gamma(\tx{x}) = (0,\tx{l},\tx{r})  /| \null \\ \m{mark1}(\gamma, \tx{x}, \gamma_1) /| \m{e\_span}(\gamma_1,\tx{x}.\text{L},\gamma_2)\end{array}\right\}$
// $\left\{\!\!\!\begin{array}{l@{}}\p{vertices\_at}(R,\gamma_2)/| \gamma(\tx{x}) = (0,\tx{l},\tx{r})  /| \null \\ \m{mark1}(\gamma, \tx{x}, \gamma_1) /|  \m{e\_span}(\gamma_1,\tx{x}.\text{L},\gamma_2)\end{array}\right\}$
    if (r) {
        root_mark = r -> m;
        if (root_mark == 0) {
           spanning(r);
        } else { x -> r = 0; } }
// $\left\{\!\!\!\begin{array}{l@{}}\exists\gamma_3.~\p{vertices\_at}(R,\gamma_3)/| \gamma(\tx{x}) = (0,\tx{l},\tx{r})  /| \null \\ \m{mark1}(\gamma, \tx{x}, \gamma_1) /| \m{e\_span}(\gamma_1,\tx{x}.\text{L},\gamma_2) /| \m{e\_span}(\gamma_2,\tx{x}.\text{R},\gamma_3)\end{array}\right\}$
} // $\{\exists \gamma_3.~\p{vertex\_at}(\p{reachable}(\gamma, \tx{x}), \gamma_3)/|\m{span}(\gamma,\tx{x},\gamma_3)\}$
  \end{lstlisting}
  \small
\begin{gather*}
  \p{vertices\_at}(\p{reachable}(\gamma_1, \tx x), \gamma_2)\defeq \underset{v\in\p{reachable}(\gamma_1, \tx x)}{\bigstar}v \mapsto \gamma_2(v)\\
  \begin{split}
  \m{span}(\gamma_1, \tx x, \gamma_2)\defeq &\m{mark}(\gamma_1,\tx x, \gamma_2) /| \gamma_1\!\uparrow(\lambda v. x\mathrel{{\stackrel{\gamma_1~}{\leadsto^{\star}_{0}}}} v) \text{ is a tree} /| \null\\
  & \gamma_1\!\uparrow\!(\lambda v.\neg x\mathrel{{\stackrel{\gamma_1~}{\leadsto^{\star}_{0}}}} v) = \gamma_2\!\uparrow\!(\lambda v.\neg x\mathrel{{\stackrel{\gamma_1~}{\leadsto^{\star}_{0}}}} v) /| \null\\
  & (\forall v.~x\mathrel{{\stackrel{\gamma_1~}{\leadsto^{\star}_{0}}}} v => \gamma_2\models x \leadsto v) /| \null\\
  & (\forall a,b.~x\mathrel{{\stackrel{\gamma_1~}{\leadsto^{\star}_{0}}}} a => \neg x\mathrel{{\stackrel{\gamma_1~}{\leadsto^{\star}_{0}}}} b => \neg \gamma_2\models a \leadsto b)\\
  \end{split}\\
  \m{e\_span}(\gamma_1, e, \gamma_2)\defeq
  \begin{cases}
    \gamma_1 - e = \gamma_2  & t(\gamma_1,e)=1\\
    \m{span}(\gamma_1, t(\gamma_1,e), \gamma_2) & t(\gamma_1,e)=0\\
  \end{cases}
\end{gather*}
\caption{Clight code and proof sketch for bigraph spanning tree.}
\label{fig:spanning}

\end{figure}


In Figure~\ref{fig:spanning} we show a simplified proof script for
the spanning tree algorithm.

\section{Proof of \infrulestyle{Ramify-P} and \infrulestyle{Ramify-PQ}}
\label{apx:ruleproofs}

\begin{figure*}
Proof of \infrulestyle{Ramify-P} from \infrulestyle{Frame} and \infrulestyle{Consequence}:
\vspace{-3em}
\[
\begin{array}{c}
\infrule{}{
  G_1 |- L_1 * \pguards{c}(L_2 --* G_2) \\
  \infrule{}{\{L_1\}~c~\{L_2\}}
            {\{L_1 * \pguards{c}(L_2 --* G_2)\}~c~\{L_2 * \pguards{c}(L_2 --* G_2)\}}{(1)} \\
  \infrule{}{
            \infrule{}{\stackrel{\langle c \rangle}{\cong} \text{ is reflexive}}{\pguards{c}(L_2 --* G_2) |- L_2 --* G_2}{(2)}}
            {L_2 * \pguards{c}(L_2 --* G_2) |- G_2}{(3)}}
{\{G_1\}~c~\{G_2\}}
{} \\
[5pt]
(1)~ \forall P.~ \pguards{c}P \text{ ignores } \FV(c) \qquad (2)~ \text{axiom T of modal logic} \qquad (3)~ (P * Q |- R) <=> (P |- Q --* R)
\end{array}
\]

Proof of \infrulestyle{Ramify-PQ} from \infrulestyle{Ramify-P}:
\vspace{-4em}
\[
\begin{array}{c}
\infrule{}
{
  \{L_1\}~c~\{\exists x.~ L_2\} \hspace{-0.5em} \\
  \infrule{}
  {
    G_1 |- L_1 * \pguards{c}\big(\forall x.~ (L_2 --* G_2)\big) \hspace{-0.5em} \\
    \infrule{}{
      \infrule{}{
        \infrule{}{
          \vdots
        } {
          \forall x.~ (L_2 --* G_2) |- (\exists x.~ L_2) --* (\exists x.~ G_2)
        } {(1)}
      } {
        \pguards{c}\big(\forall x.~ (L_2 --* G_2)\big) |- \pguards{c}\big((\exists x.~ L_2) --* (\exists x.~ G_2)\big)
      } {(2)}
    } {
      L_1 * \pguards{c}\big(\forall x.~ (L_2 --* G_2)\big) |- L_1 * \pguards{c}\big((\exists x.~ L_2) --* (\exists x.~ G_2)\big)
    } {}
  } {
    G_1 |- L_1 * \pguards{c}\big((\exists x.~ L_2) --* (\exists x.~ G_2)\big)
  } {}
} {
  \{G_1\}~c~\{\exists x.~ G_2\}
} {}
\\
[5pt]
(1)~ \text{tautology using $(P * Q |- R) <=> (P |- Q --* R)$} \qquad (2)~ \text{reduction using modal axioms K and N} %\qquad (3)~ (P |- Q) => (P * F |- Q * F)
\end{array}
\]
\caption{Proofs of \infrulestyle{Ramify-P} and \infrulestyle{Ramify-PQ}}
\label{fig:rampqproofs}
\end{figure*}

\section{Difficulty using $\graphkt$}
\label{apx:problemrecgraph}

\begin{figure*}
\[
\infrule{}
{\infrule{}
  {100 |-> 42,100,0 ~ |- ~ 100 |-> 42,100,0 ** \graphkt(100,\hat{\gamma})}
  {100 |-> 42,100,0 ~ |- ~ \hat{\gamma}(100) = (42,100,0) ~ /| ~ 100 |-> 42,100,0 ** \graphkt(100,\hat{\gamma}) ** \graphkt(0,\hat{\gamma})}
  {(2)}
}
{100 |-> 42,100,0 ~ |- ~ \graphkt(100,\hat{\gamma})}
{(1)}
\]
(1) Unfold $\graphkt$, dismiss first disjunct (contradiction), introduce existentials (which must be 42,100,0) \\
(2) simplify using $P * \p{emp} -|- P$ and remove pure conjunct

\caption{An honest academic tries to prove a ``simple'' entailment}
\label{fig:badcycle}
\end{figure*}
