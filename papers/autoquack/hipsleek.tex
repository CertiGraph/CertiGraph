* the connection to HIP/SLEEK

In the H/S section we talk about the engineering inside H/S, the module type/module interface, forward ramify, etc.

\subsection{Background on HIP/SLEEK}
HIP is an automated verifier for separation logic, it supports arbitrary user defined predicates and lemmas. While, SLEEK
is an entailment checker for separation logic, it contains a sound  procedure to decide entailments in separation logic and uses existing solvers like Z3 to reason over pure facts. Together, the HIP/SLEEK verification system \cite{chin:hipsleek} has been used to verify programs manipulating data structures like lists, arrays, trees etc. 

In HIP/SLEEK, the user only needs to specify the pre and post condition for each method (and loop invariants) and the verifier automatically uses forward reasoning to generate necessary entailments that are checked by SLEEK.  $P \vdash Q$

\subsection{Automating Ramifications with Lemmas}