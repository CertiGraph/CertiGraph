%
% LaTeX template for prepartion of submissions to PLDI'15
%
% Requires temporary version of sigplanconf style file provided on
% PLDI'15 web site.
%
\documentclass[pldi]{sigplanconf-pldi15}

%
% the following standard packages may be helpful, but are not required
%

\usepackage[utf8]{inputenc}
\usepackage[T1]{fontenc}
\usepackage[tbtags]{amsmath} %\usepackage{amsmath}
\usepackage{mathtools}
\usepackage{mathpartir}
\usepackage{listings}          % format code
\usepackage{stmaryrd}
\usepackage{relsize}
\usepackage{setspace}
\usepackage[small,compact]{titlesec}

\usepackage{amssymb}
\usepackage{mathabx}
\usepackage{wasysym}
%% \usepackage{SIunits}            % typset units correctly
\usepackage{courier}            % standard fixed width font
\usepackage[scaled]{helvet} % see www.ctan.org/get/macros/latex/required/psnfss/psnfss2e.pdf
\usepackage{url}                  % format URLs
\usepackage{enumitem}      % adjust spacing in enums
\usepackage[colorlinks=true,allcolors=blue,breaklinks,draft=false]{hyperref}   % hyperlinks, including DOIs and URLs in bibliography
% known bug: http://tex.stackexchange.com/questions/1522/pdfendlink-ended-up-in-different-nesting-level-than-pdfstartlink
\usepackage{tikz}
%\usetikzlibrary{arrows.meta}
\usepackage{fancybox}
\usepackage{multicol}
\usepackage{semantic}
\newcommand{\triple}[3]{\{#1\}\,#2\,\{#3\}}
\newcommand{\doi}[1]{doi:~\href{http://dx.doi.org/#1}{\Hurl{#1}}}   % print a hyperlinked DOI
\usepackage{mathtools}

%\input mathlig

%\usepackage{common}
%\usepackage{seplog}

\lstset{%
  language=C,
  morecomment=[n][{\color{red!80!black}}]{/*}{*/},
  morecomment=[l][{\color{red!80!black}}]{//},
  sensitive=true,
  mathescape=true,
  showlines=true,
  basicstyle=\normalfont\smaller\tt,
  keywordstyle=\color{blue},
  numbers=left,
  numberstyle=\tiny,
  numbersep=5pt,
  boxpos=t,
}


%\lstset{language=C,basicstyle=\small,mathescape=true,columns=fullflexible}

\newcommand{\scon}{\mathbin{\varstar}}
\newcommand{\ocon}{%
  \mathbin{\mbox{$\mathrlap{\cup}\hspace*{.15em}
      \raisebox{.01em}[0ex][0ex]{$\scon$}$\hspace*{.07em}}}}
\newcommand{\wand}{%
 \mathrel{\mbox{$\hspace*{-0.03em}\mathord{-}\hspace*{-0.66em}
     \mathord{-}\hspace*{-0.36em}\mathord{\scon}$\hspace*{-0.005em}}}}
\newcommand{\defeq}{\mathbin{\stackrel{\Delta}{=}}}

\begin{document}

%
% any author declaration will be ignored  when using 'plid' option (for double blind review)
%

%\title{The Ramifications of Mechanizing Verification}
\title{The Ramifications of Mechanized Localizations within Data Structures}
\authorinfo{Shengyi Wang$^{*}$ \qquad Qinxiang Cao$^{+}$ \qquad Asankhaya Sharma$^{*}$ \qquad Aquinas Hobor$^{\dagger,*}$}
{}
{School of Computing$^{*}$ and Yale-NUS College$^{\dagger}$, National University of Singapore; Princeton University$^{+}$}

\maketitle

\begin{abstract}
We show how to mechanically verify programs manipulating data structures with intrinsic sharing such as heap-represented graphs.  We upgrade the theory of ramification to better support modified program variables and existential quantifiers in assertions.  We develop a modular and general setup for reasoning about mathematical graphs and show
how to connect this setup to a general theory for graphs in separation logic.  We connect our theories to two verification tools with different levels of automation and use them to verify several canonical graph algorithms. Our proofs are entirely machine-checked in Coq.
\end{abstract}

\newcommand\hide[1]{}

\section{Introduction}
Over the last fifteen years great strides have been made in automating verifications of programs that manipulate
tree-like data structures using separation logic CITE CITE CITE.  Unfortunately, verifying programs that manipulate
graph-like data structures (e.g. structures with \emph{intrinsic sharing}) has been far more challenging.
Indeed, verifying such programs was formidable enough that a number of the early landmark results in separation logic
devoted substantial efforts to verify single examples such as Schorr-Waite~\cite{hongseok:phd} and XXX CITE with pen and
paper---avoiding entirely the additional challenges inherent in machine-assisted reasoning.

In recent years, Hobor and Villard introduced the concept of \emph{ramification} as a kind of proof pattern or framework
to verify graph-manipulating programs with pen and paper~\cite{hobor:ramification}, but left open the question of how such proofs could
be incorporated in a machine-assisted setting.  In this paper, we show how this can be done, and demonstrate the
value of our approach by adding ramification to two rather sizeable---albeit quite differently flavored---separation logic-based
verification tools: the Coq-based tactic system of the Verified Software Toolchain CITE and the more highly-automated HIP/SLEEK
program verifier~\cite{chin:hipsleek}.  Despite the substantial differences between these systems, the vast majority of our infrastructure is
shared between them, and since many of the other computer-assisted verification tools under development
today CITE CITE CITE CITE have much in common with at least one of these tools, we believe that our techniques will be
applicable to many other systems as well.

Along the way we develop an improved proof rule for ramification that supports existential variables and enjoys a smoother interaction with


with modified

 references to modified local program variables more generally.

 smoothly



a smoother and more uniform support for modified program variables.

along with a more

make a number upgrades to the theory of ramification and present

, with a better treatment of modified variables, existentials,

Along the way we discover---and show how to fix---a rather subtle error in Hobor and Villard's presentation: neither the
Knaster-Tarski \cite{tarski:fixpoint} nor the Appel-McAllester \cite{appel:fixpoint} method for solving recursive fixpoints is suitable for defining
recursive graph predicates in separation logic.

  We also develop a general framework for defining and reasoning about mathematical
graphs and

different kinds of mathematical
graphs can be implemented in separation logic in a uniform way;

generalize their setting
so that it can handle a wider variety of data structures;

We use both systems to verify a number of different programs utilizing graph-manipulating structures,
letting us understand the advantages and disadvantages of both.

\paragraph{Contributions and structure of the remainder of this paper.}
\begin{itemize}
\item Example \S\ref{sec:example}.  Contributions: new ramify rules, new notation, everything machine-checked, multiple tools sharing mathematical infrastructure.
\item Mathematical graphs.  Contributions: computable, compositional, \& general graph library in Coq.  Treatment of null.
\item Spatial graphs.  Contributions: correct general graph predicate.  Problem with fixed point.  Problem with ``later'' not being precise.  Fold/unfold, precise, etc.
\item Integrating ramification into verification tools.  Contributions: VST (localize/unlocalize). H/S (external axioms).  Additional examples.  New proof of ``copy'' that does not use regions.
\item Related work, future work, and conclusion
\end{itemize}

\section{Generalizing localizations}
\label{sec:orientation}

\paragraph{Mark example.} In Qinxiang's new format.

\newcommand{\tx}[1]{\text{#1}}
\newcommand{\p}[1]{\ensuremath{\mathsf{#1}}} % predicate font
\newcommand{\m}[1]{\ensuremath{\mathit{#1}}} % math font
\let\ramify\lightning

\begin{figure}
  \begin{lstlisting}
struct Node {
  int  _Alignas(16) m;
  struct Node * _Alignas(8) l;
  struct Node * r; };

void mark(struct Node * x) { // $\{\p{graph}(\tx{x},\gamma)\}$
  struct Node * l, * r; int root_mark;
  if (x == 0) return;
// $\{\p{graph}(\tx x,\gamma) /| \exists m,l,r.~ \gamma(\tx{x}) = (m,l,r)\}$
// $\{\p{graph}(\tx x,\gamma) /| \gamma(\tx{x}) = (m,l,r)\}$
// $\searrow \{\tx x|-> m,l,r \}$
      root_mark = x -> m;
// $\swarrow \{\tx x|-> m,l,r /| m = \tx{root\_mark} \}$
// $\{\p{graph}(\tx x,\gamma) /| \gamma(\tx{x}) = (m,l,r) /| m = \tx{root\_mark}\}$
  if (root_mark == 1) return;
// $\{\p{graph}(\tx x,\gamma) /| \gamma(\tx{x}) = (0,l,r) \}$
// $\searrow \{\tx x|-> 0,l,r /| \gamma(\tx{x}) = (0,l,r)\}$
      l = x -> l;
      r = x -> r;
      x -> m = 1;
// $\swarrow \{\tx x|-> 1,\tx{l},\tx{r} /| \gamma(\tx{x}) = (0,\tx{l},\tx{r}) /| \exists \gamma'.~ \m{mark1}(\gamma, \tx{x}, \gamma')\}$
// $\{\exists \gamma'.~ \p{graph}(\tx x,\gamma') /| \gamma(\tx{x}) = (0,\tx{l},\tx{r}) /| \m{mark1}(\gamma, \tx{x}, \gamma')\}$
// $\{\p{graph}(\tx x,\gamma') /| \gamma(\tx{x}) = (0,\tx{l},\tx{r}) /| \m{mark1}(\gamma, \tx{x}, \gamma')\}$
// $\searrow \{\p{graph}(\tx l, \gamma')\}$
      mark(l);
// $\swarrow \{\exists \gamma''.~ \p{graph}(\tx l, \gamma'') /| \m{mark}(\gamma', \tx{l}, \gamma'')\}$
// $\left\{\!\!\!\begin{array}{l@{}}\exists \gamma''.~ \p{graph}(\tx x,\gamma'') /| \gamma(\tx{x}) = (0,\tx{l},\tx{r}) /| \null \\ \m{mark1}(\gamma, \tx{x}, \gamma') /| \m{mark}(\gamma', \tx{l}, \gamma'')\end{array}\right\}$
// $\left\{\!\!\!\begin{array}{l@{}}\p{graph}(\tx x,\gamma'') /| \gamma(\tx{x}) = (0,\tx{l},\tx{r}) /| \null \\ \m{mark1}(\gamma, \tx{x}, \gamma') /| \m{mark}(\gamma', \tx{l}, \gamma'')\end{array}\right\}$
// $\searrow \{\p{graph}(\tx r, \gamma'')\}$
      mark(r);
// $\swarrow \{\exists \gamma'''.~ \p{graph}(\tx r, \gamma''') /| \m{mark}(\gamma'', \tx{r}, \gamma''')\}$
// $\left\{\!\!\!\begin{array}{l@{}}\exists \gamma'''.~ \p{graph}(\tx x,\gamma''') /| \gamma(\tx{x}) = (0,\tx{l},\tx{r}) /| \null \\ \m{mark1}(\gamma, \tx{x}, \gamma') /| \m{mark}(\gamma', \tx{l}, \gamma'') /| \m{mark}(\gamma'', \tx{r}, \gamma''')\end{array}\right\}$
} // $\{\exists \gamma'''.~ \p{graph}(\tx x,\gamma''') /| \m{mark}(\gamma, \tx{x}, \gamma''')\}$
\end{lstlisting}
%% \vspace{-8pt}
\caption{Clight code and proof sketch for bigraph mark. {\color{magenta} The steps that induce
  ramifications are indicated with $\ramify_i$, where the associated ramification entailment is equation number $i$.}} %whose numbers point with their associated ramification entailment reference.}
%\vspace{-19pt}
\label{fig:markgraph}
\end{figure}


\section{A framework for graph theory}
In order to verify the functional correctness of graph
algorithms, we need to first reason about mathematical graphs.
Our graph library is powerful and expressive, allowing 
us to verify realistic algorithms that work in an end-to-end
system. One of its strengths is its modularity, 
which allows us to intuitively reuse and compose our proofs when
mechanising our verifications. In this section, we present our 
mathematical graph framework with an emphasis on this modularity. 
{\color{magenta} We continue to use Union-Find from 
\S\ref{sec:orientation} as our motivating example.}

% saving old version just in case...
\hide{As will be shown in \S\ref{sec:development}, our mathematical
graph constructions comprise a considerable fraction of our
codebase. Indeed, as discussed in \S\ref{sec:related},
25 years of research into mechanized graph theory can
be summarized as ``it is a little tricky''. 
First, as demonstrated in \S\ref{sec:orientation},
our development is expressive and powerful enough to verify realistic
algorithms---that is, it actually works in an end-to-end system.
Second, we have taken considerable care to develop a modular and
general-purpose framework for such mathematical graphs to allow
such verifications to be mechanized without undue pain.
Accordingly, in this section we will present our framework
at a high level to communicate the overall architecture rather
than focusing on the nitty-gritty details.} % end hide

\subsection{Structure of the mathematical graph framework}\label{sec:mathinfra}

\begin{figure}[t]
\centering
%\beginpgfgraphicnamed{variousgraph}
\begin{tikzpicture}
[->/.style={thick,arrows={-Stealth}},
-->/.style={thick,arrows={-Stealth}, decorate, decoration={snake, amplitude=.4mm,segment length=2mm,post length=2mm}},
   realG/.style={shape=rectangle, rounded corners=4pt, draw, fill=gray!40},
   propG/.style={shape=rectangle, rounded corners=4pt, draw}]
\node[realG] (PG) at (0, 0) {\small PreGraph};
\node[realG] (LG) [right=0.8 of PG] {\small LabeledGraph};
\node[realG] (GG) [right=2 of LG] {\small GeneralGraph};
\draw [double, ->] (PG) -- (LG) node [pos=0.5, above] {\small Label} ;
\draw [double, ->] (LG) -- (GG) node (SC) [pos=0.5, above, align=center]
{\small Soundness \\ \small Condition};
\node[propG] (Prop) [below=0.6 of SC] {\small Property};
\node[propG] (PropL) [below=0.4 of Prop] {\small Property Lemmas};
\node[propG] (PGL) [below=2 of PG, align=center] {\small PreGraph \\\small Lemmas};
\node[propG] (LGL) [below=2 of LG, align=center] {\small LabeledGraph \\\small Lemmas};
\node[propG] (GGL) [below=2 of GG, align=center] {\small GeneralGraph \\\small Lemmas};
\draw [double, ->] (PGL) to (LGL);
%% \draw [double, ->] (LGL) to (GGL);
\draw [->] (PG) to (PGL);
\draw [->] (Prop) to (PropL);
\draw [-->] (Prop) to (SC);
\coordinate [left=0.2 of LG.south] (LGs1);
\coordinate [left=0.2 of LGL.north] (LGLn1);
\draw [->] (LGs1) to (LGLn1);
\coordinate [right=0.2 of LG.south] (LGs2);
\coordinate [right=0.2 of LGL.north] (LGLn2);
\draw [->] (LGs2) |- (Prop);
\draw [double, ->] (LGLn2) |- (PropL);
\coordinate [right=0.2 of GG.south] (GGs);
\coordinate [left=0.2 of GGL.north] (GGLn1);
\coordinate [right=0.2 of GGL.north] (GGLn2);
\draw [double, ->] (PropL) -| (GGLn1);
\draw [->] (GGs) to (GGLn2);
\node [draw, thick, rectangle, dashed, fit=(Prop) (PropL)] {};
\node (legend1) [below right=0.2 and -0.3 of PGL] {\small Depends};
\coordinate[left=0.8 of legend1]  (l1);
\draw [->] (l1) to (legend1);
\node (legend2) [right=1 of legend1] {\small Inherits};
\coordinate[left=0.8 of legend2]  (l2);
\draw [double, ->] (l2) to (legend2);
\node (legend3) [right=1 of legend2] {\small Instantializes};
\coordinate[left=0.8 of legend3]  (l3);
\draw [-->] (l3) to (legend3);
\end{tikzpicture}
%\endpgfgraphicnamed
\vspace{1ex}
\caption{Structure of the Mathematical Graph Library}\label{fig:graphs}
\end{figure}


Figure~\ref{fig:graphs} shows the architecture of our mathematical graph library. 
\hide{The most basic kind of graph is PreGraph, out of which we build 
LabeledGraph, and which in turn are used
to build GeneralGraphs.  Each kind has some lemmas and also inherits the lemmas of the 
previous kind.  The dashed box represents a ``plugin'' system for attaching arbitrary 
properties to LabeledGraphs (\ref{subsec:graphplugins}). %and will be discussed later. 
%We will consider each in turn.
} % end hide

\begin{figure}[t]
\centering
\beginpgfgraphicnamed{pregraphexp}
\begin{tikzpicture}
[vad/.style={circle, fill=black, inner sep=0pt, minimum size=4pt},
 inv/.style={circle, draw=black, thick, inner sep=0pt, minimum size=4pt},
 ->/.style={thick, arrows={-Stealth}}]
\node[vad] (n1) at (0, 0) {};
\node[vad] (n2) at (1, 1) {};
\node[inv] (n3) at (1, -1) {};
\node[vad] (n4) at (-2,2) {};
\node[vad] (n5) at (-2,-2) {};
\node[vad] (n6) at (-3,0) {};
\node[vad] (n7) at (3,1.5) {};
\node[inv] (n8) at (3,-1.5) {};
\node[vad] (n9) at (3.5, 1) {};
\node[inv] (n10) at (3.5, 0.5) {};
\node at (3.5, 1) [right=1.5pt] {\small Valid node};
\node at (3.5, 0.5) [right=1.5pt] {\small Invalid node};
\node at (3.5, 0) [right=1.5pt] {\small Valid edge};
\node at (3.5, -0.5) [right=1.5pt] {\small Invalid edge};
\draw[->] (n1) to (n2);
\draw[->,dashed] (n1) to (n3);
\draw[->,dashed] (n3) to (n5);
\draw[->] (n2) to (n3);
\draw[->] (n2) to (n7);
\draw[->] (n2) to (n8);
\draw[->,dashed] (n3) to (n8);
\draw[->] (n4) to (n1);
\draw[->] (n1) to (n5);
\draw[->] (n2) to (n4);
\draw[->] (n1) to [bend left=20] (n6);
\draw[->] (n6) to (n5);
\draw[->] (n8) to [bend left=20] (n7);
\draw[->] (n4) to [bend right=35] (n1);
\draw[->] (3.0, 0) -- (3.6, 0);
\draw[->,dashed] (3.0, -0.5) -- (3.6, -0.5);
\end{tikzpicture}
\endpgfgraphicnamed
\vspace{1ex}
\caption{A PreGraph with valid and invalid vertices and edges.}\label{fig:pregraph}
\end{figure}
% move to its own file?

\vspace{-0.75ex}
\iftrue
\paragraph{PreGraph.} A PreGraph is a hextuple $(VT, ET, V, E, s, d)$, where $VT$ 
and $ET$ are the underlying carrier sets of vertices and edges, and $V$ and $E$, 
subsets $VT$ and $ET$ respectively, introduce the notion of \emph{validity} in the 
graph. In Figure \ref{fig:pregraph}, valid vertices are in $V$ and 
invalid vertices are in $VE \smallsetminus V$. Importantly, 
both kinds of vertices are legally part 
of the PreGraph. Finally, $s$ and $d$ are functions that map 
an edge to its source and destination respectively; {\color{magenta}this model means 
that PreGraphs are directed rather than undirected.} 
With an eye to flexibility, we make no further 
requirements of a legal PreGraph, not even a specific notion 
of how the four sets are related.
Indeed, the PreGraph in Figure \ref{fig:pregraph} contains invalid 
vertices and edges in an arbitrary configuration.

Many graph concepts such as \emph{path}, \emph{reachability}, and \emph{subgraph} are 
defined on PreGraphs. In \S\ref{fig:find} we saw \emph{reachable}, written 
$\m{a} \mathrel{{\stackrel{\gamma~}{\leadsto^{1}}}} \m{b}$. It is 
defined as 

$$\m{a} \mathrel{{\stackrel{\gamma~}{\leadsto^{1}}}} \m{b} \defeq a, b \in V(\gamma) /| \exists e.~ e \in E(\gamma) /| s (e,\gamma) = a /| 
d (e,\gamma) = b.$$ 
The reflexive, transitive closure on \emph{reachable} is written 
$\m{a} \mathrel{{\stackrel{\gamma~}{\leadsto^{\star}}}} \m{b}$, and 
$\neg (\m{a} \mathrel{{\stackrel{\gamma~}{\leadsto^{\star}}}} \m{b})$ 
is written $\m{a} \mathrel{{\stackrel{\gamma~}{\not\leadsto^{\star}}}} \m{b}$.
{\color{blue} I think the definition is rather simple and doens't deserve so much space.
How about just ``...written $\m{a} \mathrel{{\stackrel{\gamma~}{\leadsto^{1}}}} \m{b}$.
It means that a and b are in $V(\gamma)$ and that there exists a edge (in $E(\gamma)$)
that goes from a to b.''}

PreGraph's ability to reason about missing vertices and edges is convenient when 
verifying real programs. Suppose some $\gamma$ satisfied some stronger notion of
``well-formed'', in the sense that valid vertices have only valid edges and 
vice versa. Could we then subtract some vertices and edges from it and reason about the 
resulting structure? This is precisely what we needed to do in \ref{fig:find}, where
we argued for a condition of congruence on 
$\gamma \smallsetminus (v \in \gamma \mid \m{x} 
\mathrel{{\stackrel{\gamma~}{\leadsto^{\star}}}} \m{v})$. 
A strong notion of well-formedness may have stopped us short at this point, 
declaring the structure ill-formed because of the dangling edges 
pointing to recently-removed vertices. 
A PreGraph is more accommodating, since
it produces a fresh PreGraph after this selective subtraction 
and then allows us to go ahead and reason about congruence as we need to.

\hide{
For example, consider the difference of two graphs, $\gamma_1
- \gamma_2$.  Even if both of these graphs are ``well-formed'' to begin with, in the 
sense that valid vertices have only valid edges and vice versa, their difference 
may not be since there may be dangling edges pointing to the 
now-removed vertices of $\gamma_2$.} % end hide

\hide
{In \S\ref{sec:spacegraph} we will tie a mathematical graph $\gamma$ to 
a spatial graph predicate
$\p{graph}(x, \gamma)$.   As we will see, a $\p{graph}$ ``owns'' only the
spatial portion of $\gamma$ that is reachable
from $x$ even though $\gamma$ may have other valid vertices.
} % end hide
\fi
%%%%%%%%%%%%%%%%%%%%%%%%%%%%%%%%%%%%%%%%%%%%%%%%%%
%%%%%% Edit 1: Qinxiang's proposal ends
%%%%%%%%%%%%%%%%%%%%%%%%%%%%%%%%%%%%%%%%%%%%%%%%%%
%%%%%% Edit 1: Original version starts
%%%%%%%%%%%%%%%%%%%%%%%%%%%%%%%%%%%%%%%%%%%%%%%%%%
\iffalse
\paragraph{Pregraphs.} A PreGraph is a hextuple $(V, E, \phi_V, \phi_E, s, d)$,
where $V$ and $E$ are the underlying carrier set of vertices and edges.  
Not every $v \in V$ or $e \in E$ is actually ``in'' the graph, so we provide 
the predicates $\phi_V$ and $\phi_E$ to classify vertices and edges as 
\emph{valid} (in) or not (out).  Finally, $s$ and $d : E -> V$ are functions that 
map an edges to their source and destination respectively; this model means that 
PreGraphs are directed rather than undirected.  By design, there are no requirements 
for \emph{e.g.} how the validities of edges and vertices relate.  As shown in 
Figure \ref{fig:pregraph}, a PreGraph can contain invalid nodes and edges in an 
arbitrary configuration.

Many graph concepts such as \emph{path}, \emph{reachability}, and \emph{subgraph} are 
defined on PreGraphs.  Wefor write $\gamma\models n_1 \xrightarrow{P} n_2$ to mean 
that there is a valid path from $n_1$ to $n_2$ such that each vertex in the path 
satisfies the predicate $P$.  The set of all reachable vertices from $v$, written $\p{
reachable}(\gamma,v)$, is then just $\{v' ~|~ \gamma\models v \xrightarrow{\top} v'\}$.
In \S\ref{sec:spacegraph} we will tie mathematical graphs $\gamma$ to a spatial graph 
predicate
$\p{graph}(x, \gamma)$.   As we will see, $\p{graph}$ ``owns'' only the
spatial portion of $\gamma$ that is reachable
from $x$ even though $\gamma$ may have other valid vertices.

The advantage of designing a graph type that can reason about missing vertices and 
edges is because concepts necessary to verify real programs require such flexibility.  
For example, consider the difference of two graphs, $\gamma_1 - \gamma_2$.  Even if 
both of these graphs are ``well-formed'' to begin with, in the sense that valid nodes 
have only valid edges and vice versa, their difference may not since there may be 
dangling edges pointing to the now-removed vertices of $\gamma_2$.
\fi
%%%%%%%%%%%%%%%%%%%%%%%%%%%%%%%%%%%%%%%%%%%%%%%%%%
%%%%%% Edit 1: Original version ends
%%%%%%%%%%%%%%%%%%%%%%%%%%%%%%%%%%%%%%%%%%%%%%%%%%


\vspace{-0.75ex}
%% Anshuman's proposal 
\iftrue
\paragraph{LabeledGraph.}
A LabeledGraph is a PreGraph with the addition of \emph{labels} on 
vertices, edges, and/or the graph as a whole. The need for such labels
is fairly clear; the bare structure of a graph can only 
contain so much information, and many classic graph problems 
such as graph coloring, shortest path, and network flow rely on 
additional information in the form of labels. In our architecture, a
LabeledGraph inherits any lemmas proved about its associated PreGraph. 
In addition, we can define additional lemmas that use labels, 
\emph{e.g.} the union-find graph has an integer label denoting \emph{rank}.
We could prove a lemma that running \texttt{find} does not alter
any vertex's rank. 
\hide{add string labels to edges and reason about a trie.}
\fi
% Anshuman's proposal ends

%%%%%%%%%%%%%%%%%%%%%%%%%%%%%%%%%%%%%%%%%%%%%%%%%%
%%%%%% Edit 2
%%%%%%%%%%%%%%%%%%%%%%%%%%%%%%%%%%%%%%%%%%%%%%%%%%
%%%%%% Edit 2: Qinxiang's proposal starts
%%%%%%%%%%%%%%%%%%%%%%%%%%%%%%%%%%%%%%%%%%%%%%%%%%
\iffalse
\paragraph{LabeledGraph.}
A LabeledGraph is a PreGraph with labels, e.g. the ``mark bit'' used in Figure~\ref{fig:markgraph} are labels.
\fi
%%%%%%%%%%%%%%%%%%%%%%%%%%%%%%%%%%%%%%%%%%%%%%%%%%
%%%%%% Edit 2: Qinxiang's proposal ends
%%%%%%%%%%%%%%%%%%%%%%%%%%%%%%%%%%%%%%%%%%%%%%%%%%
%%%%%% Edit 2: Original version starts
%%%%%%%%%%%%%%%%%%%%%%%%%%%%%%%%%%%%%%%%%%%%%%%%%%
\iffalse
\paragraph{LabeledGraph.}
Although many basic lemmas can be proved about PreGraphs, they are inadequate for real program verification.
When reasoning about the concrete graphs manipulated by various algorithms,
we usually need to add a notion of \emph{labels} on vertices and/or edges, such as
the ``mark bit'' used in Figure~\ref{fig:markgraph}, letting us define notions like ``the vertices reachable via an unmarked path''
on LabeledGraphs.
\fi
%%%%%%%%%%%%%%%%%%%%%%%%%%%%%%%%%%%%%%%%%%%%%%%%%%
%%%%%% Edit 2: Original version ends
%%%%%%%%%%%%%%%%%%%%%%%%%%%%%%%%%%%%%%%%%%%%%%%%%%

\vspace{-0.75ex}
\paragraph{GeneralGraph.}
PreGraphs and LabeledGraphs are quite universal, and allow us 
to state and prove useful lemmas that are true by virtue of the 
way the graphs were set up. However, when proving the correctness
of graph algorithms, we often need more specificity in our mathematical graphs
so that we may model the real program's restrictions more closely. 
For example, the graph used in Find earlier had the restriction 
that each vertex have exactly one out-edge. 
We achieve this using GeneralGraphs. 
A GeneralGraph augments a LabeledGraph by adding a 
``soundness condition'' plugin, indicated in 
Figure~\ref{fig:graphs} by a dashed border. 
This soundness condition can be arbitrarily complex, 
and can thus specify arbitrary restrictions on the graph. 
This is what makes GeneralGraphs truly versatile: 
it allows us to state algorithm-specific properties 
of any complexity on the graph, and then derive lemmas based on those properties.
Thankfully, we do not need to state these complicated soundness 
conditions afresh for each program that we verify. In the next section, 
we explain how to compose these out of smaller, reusable pieces.

\subsection{Composing Graph plugins}
\label{subsec:graphplugins}

We use Coq's typeclass system to manage our soundness plugins smoothly, 
benefiting from the \emph{compositionality} of the system. 
If we have two soundness properties, each with its associated proved lemmas, 
we can combine them, 
prove lemmas about their combination using known facts about 
the separate pieces,
and then treat the combination as a new plugin. 

Consider the following oft-used graph properties:
\begin{itemize}
\vspace{-1ex}
\item \p{FiniteGraph}: The validity sets $V$ and $E$ are finite.
%\vspace{-1ex}
\item \p{MathGraph}: invalid nodes are allowed to be destinations
of valid edges, thus allowing null values to represent unused nodes.
\hide{More subtly, consider that many real data structures use special null values to 
represent unused nodes.  The  property introduces this concept---
\emph{i.e.} some special invalid nodes are allowed to appear as 
destinations for valid edges.} % end hide
%\vspace{-1ex}
\item \p{BiGraph}: there are exactly two outgoing edges per node. 
\item \p{LstGraph}: the graph is list-like; loops are forbidden and 
nodes have one out-edge each.
\end{itemize}

\hide{
\begin{figure}[t]
\centering
\beginpgfgraphicnamed{graphproperty}
\begin{tikzpicture}
[->/.style={thick,arrows={-Stealth}},
   group/.style={shape=rectangle, draw, thick, dashed},
   propG/.style={shape=rectangle, rounded corners=4pt, draw}]
\node[propG] (PL12) at (0, 0) {\footnotesize Lemmas of Property 1 and 2};
\coordinate [left=1 of PL12.north] (PL12n1);
\coordinate [right=1 of PL12.north] (PL12n2);
\node[propG] (PL1) [above=0.5 of PL12n1, align=center] {\footnotesize Property 1 \\\footnotesize Lemmas};
\node[propG] (PL2) [above=0.5 of PL12n2, align=center] {\footnotesize Property 2 \\\footnotesize Lemmas};
\node[propG] (P1) [above=0.5 of PL1] {\footnotesize Property 1};
\node[propG] (P2) [above=0.5 of PL2] {\footnotesize Property 2};
\draw [->] (P1) to (PL1);
\draw [->] (P2) to (PL2);
\draw [double, ->] (PL1) to (PL12n1);
\draw [double, ->] (PL2) to (PL12n2);
\draw [->] (P1.west) to [bend right=45] (PL12.west);
\draw [->] (P2.east) to [bend left=45] (PL12.east);
\node (R1) [group, fit=(P1) (PL1)] {};
\node (R2) [group, fit=(P2) (PL2)] {};
\node (R3) [group, fit=(current bounding box)] {};
\node [propG] (P1P2) [above right=-1.1 and 1.2 of R3, align=left] {\footnotesize Property 1$/|$ \\\footnotesize Property 2};
\node [propG] (PL1PL2) [below=0.5 of P1P2, align=center] {\footnotesize Prop. 1 Lemmas \\\footnotesize Prop. 2 Lemmas \\\footnotesize Prop. 1$/|$2 Lemmas};
\draw [->] (P1P2) to (PL1PL2);
\node (R4) [group, fit=(P1P2) (PL1PL2)] {};
\node (EQ) [right=0 of R3] {\bf \Large $\mapsto$};
\end{tikzpicture}
\endpgfgraphicnamed
\vspace{1ex}
\caption{Combining plugins{\color{blue} [cut this?]}}\label{fig:properties}
\end{figure}
} % end hide

As a first step, we can prove many general, reusable lemmas
about these properties. However, these properties are still 
too general to model a real program. The next step is compose 
these plugins to arrive at a more specific set of restrictions 
that more closely models our particular graph. 
For instance, we can compose 
\p{LstGraph}, \p{MathGraph}, and \p{FiniteGraph} 
together into a new plugin called \p{LiMaFin}, which, incidentally, is the 
soundness condition used to verify Find in Figure~\ref{fig:find}.
In our verification of Mark~\ref{fig:markgraph}, we use a similar soundness condition
\p{BiMaFin}, which uses \p{BiGraph} instead of \p{LstGraph}. 
The commonalities and differences between \p{LiMaFin} 
and \p{BiMaFin} are readily apparent from their construction.
Composing plugins in this way is much more elegant than creating a
custom plugin for each new algorithm as it promotes the natural reuse of graph 
properties across algorithms.

{\color{blue} Maybe move this somewhere.} 
Coq also handles our notion of inherited 
lemmas seamlessly: in our verfication of Find, we 
work directly with a \p{LiMaFin} GeneralGraph, but, as 
we saw, we still use properties such as reachability 
and operations such as selective subtraction, which are defined on the 
embedded PreGraph, not the GeneralGraph. 
Coq handles the appropriate coercions with 
remarkable elegance.

%% We can apply our framework to define related structures such as DAGs and trees.
%% For example, a DAG has the additional property that for any $x$ and $y$,
%% if $x$ is reachable from $y$, then $x = y$ or $y$ is not reachable
%% from $x$.
%  Similarly, we define tree by saying that for any reachable node $n$ there is
%a unique path from the root to $n$.

\subsection{Reasoning about relations between graphs} %Using our framework to reasonApplication of the framework}

{\color{blue}Needs revised examples based on new Orientation.}

In Figure~\ref{fig:markgraph} we defined the relation 
$\m{mark}(\gamma, \tx x, \gamma')$
for the graph marking algorithm.  Similarly, we define $\m{span}$ for the 
spanning tree program
and $\m{copy}$ for the graph copy program.
These relations all capture how the graph has changed from before to after the program
execution.  By specifying $\m{copy}$ relationally
rather than functionally we avoid explicitly modeling how the memory 
allocator works, a major advantage.

As previously mentioned, we reuse $\m{mark}$ and its
related lemmas to prove facts about spanning tree and graph copy
because the latter two programs mark nodes as they work.
Accordingly, we can reuse facts such as the following:
%%%%%%%%%%%%%%%%%%%%%%%%%%%%%%%%%%%%%%%%%%%%%%%%%%
%%%%%% Edit 4
%%%%%%%%%%%%%%%%%%%%%%%%%%%%%%%%%%%%%%%%%%%%%%%%%%
%%%%%%%%%%%%%%%%%%%%%%%%%%%%%%%%%%%%%%%%%%%%%%%%%%
%%%%%% Edit 4: Qinxiang's proposal starts
%%%%%%%%%%%%%%%%%%%%%%%%%%%%%%%%%%%%%%%%%%%%%%%%%%
\[
\begin{array}{@{}l@{}}
\text{if } \gamma(x)=(0, v_1, \dots,v_n), \m{mark1}(\gamma, x, \gamma_1),
\text{ and } \forall i, \m{mark}(\gamma_i,v_i,\gamma_{i+1}) \text{ then } \m{mark}(\gamma,x,\gamma_{n+1}).
\end{array}
\]
%We prove this theorem sound for any LabeledGraph, not just
%\p{BiGraph}s (\emph{i.e.}, we do not assume only two neighbors).
%%%%%%%%%%%%%%%%%%%%%%%%%%%%%%%%%%%%%%%%%%%%%%%%%%
%%%%%% Edit 4: Qinxiang's proposal ends
%%%%%%%%%%%%%%%%%%%%%%%%%%%%%%%%%%%%%%%%%%%%%%%%%%



\marginpar{\tiny \color{blue} We are interested in cutting \S\ref{sec:fixpointfail}. Needs massaging.}
To prove the functional correctness of graph-manipulating algorithms implemented in a real language, we need to connect the heap representation of graphs, the memory model of the programming language, and the mathematical properties of graphs from \S\ref{sec:mathgraph}.  The first of these turns out to be surprisingly subtle as we shall see in \S\ref{sec:fixpointfail} and \S\ref{sec:goodgraph}.  The main challenge for the others is to engineer a framework that is generic enough and modular enough to be useful in practice in a variety of settings; we cover it in~\S\ref{sec:ramifylib}.

\subsection{Recursive definitions yield poor \p{graph} predicates}\label{sec:fixpointfail}

\newcommand{\graphkt}{\p{graph}_T}
\newcommand{\grapham}{\p{graph}_A}


\marginpar{\tiny \color{blue} Crunch. Not our main point, just a lesson for the reader 
so we can make our point.}
{\color{magenta}Recursive predicates are ubiquitous in separation logic---so
much so that when one writes the definition of a predicate as
\mbox{$P$ ``$\defeq$'' $\ldots P \! \ldots$}, no one raises an eyebrow despite the
dangers of circularity in mathematics. Indeed, the vast majority of the time there
is no danger thanks to the magic of the Knaster-Tarski fixpoint
$\mu_{\mathsf{T}}$ \cite{tarski:fixpoint}.  Formally, one does not define $P$ directly, 
but rather defines a functional
\mbox{$F_P \defeq \lambda P.~ \ldots P \! \ldots$} and then defines $P$ itself as
\mbox{$P \defeq \mu_{\mathsf{T}} \, F_P$}.  
Assuming {\color{magenta}(as one typically does without comment)} 
that $F_P$ is \emph{covariant}, i.e. $(P \vdash Q)
\Rightarrow (F \, P \vdash F \, Q)$, one then enjoys the fixpoint
equation $P \Leftrightarrow \ldots P \ldots$, formally justifying
the typically-written pseudodefinition (``$\defeq$'').}

%Definition corec {B A: Type}  (F: (B -> pred A) -> (B -> pred A)) : B -> pred A :=
%fun x w => forall P: B -> pred A, (forall x, F P x |-- P x) -> P x w.

Suppose we define a graph predicate $\graphkt$ this way, \emph{e.g.} along the lines of the fold/unfold definition in Figure~\ref{fig:markgraph}: %, \emph{i.e.}
\vspace{-1ex}
\[
\begin{array}{@{}l@{}l@{}}
\graphkt(x, \gamma) \stackrel{\Delta}{=} \; &(x = 0 /| \p{emp}) |/ \null \\
& \exists m,l,r.~ \gamma(x)=(m,l,r) /| \null \\
& ~~  x |-> m,l,r ** \graphkt(l, \gamma) ** \graphkt(r, \gamma)
\end{array}
\vspace{-1ex}
\]
%We have removed the alignment-related portions of equation~\eqref{eqn:bigraphintrofoldunfold} to focus on a more serious issue, even though as we will explain in~\S\ref{sec:goodgraph} alignment concerns are also necessary for the fold/unfold relationship to hold in C-like memory models.
Although we can apply Knaster-Tarski (because the functional needed to 
define $\graphkt$ is covariant), the result is hard to use. 
Consider the following memory $m$ for a toy machine:

\begin{minipage}{.24\textwidth}
\qquad \[
\begin{array}{l|l}
\textrm{address} & \textrm{value} \\
\hline
102 & 0 \\
101 & 100 \\
100 & 42 \\
\end{array}
\]
\end{minipage}
\begin{minipage}{.19\textwidth}
\centering
\beginpgfgraphicnamed{selfref}
\begin{tikzpicture}
[->/.style={thick,arrows={-Stealth}},
   propG/.style={shape=circle, draw}]
   \path[use as bounding box] (-1, -1) rectangle (0.5, 0.5);
   \node[propG] (P) at (0, 0) {42};
   \draw[->] (P.west) .. controls (-1.5, 0) and (0, -1.5) .. (P.south);
\end{tikzpicture}
\endpgfgraphicnamed
\end{minipage}
\vspace{0.75ex}

\noindent Clearly $m |= 100 |-> 42,100,0$.  But it seems also clear that this memory represents a one-cell cyclic graph as illustrated in the accompanying diagram, \emph{i.e.} we want $m |= \graphkt(100,\hat{\gamma})$, where $\hat{\gamma}(100) = (42,100,0)$.  This is equivalent to wanting to be able to prove $100 |-> 42,100,0 |- \graphkt(100,\hat{\gamma})$.  Unfortunately, as explained in Appendix~C\hide{\ref{apx:problemrecgraph}}, this is rather difficult to do so since applying the natural proof techniques actually strengthens the goal. In fact we do not know if this entailment is provable, but the difficulties encountered in proving what ``should be'' straightforward suggest that Knaster-Tarski should be treated with caution when defining spatial predicates for graphs.

The other direction, \mbox{$\graphkt(100,\hat{\gamma}) |- 100 |-> 42,100,0$},
\textbf{is} true but is not easy to prove, relying on the constructions in \S\ref{sec:goodgraph} and the fact that $\mu_{\mathsf{T}}$ constructs the least fixpoint.  In contrast, $\graphkt(100,\hat{\gamma}) |- 100 |-> 42,100,0 * \top$ is easy. % to prove. % via fold/unfold.

%As explained in~\S\ref{sec:foldunfold}, this alignment is necessary in C-like memory models to prove fold-unfold \eqref{eqn:bigraphintrofoldunfold}, which is why \eqref{eqn:bigraphintrofoldunfold} includes an alignment restriction $x~\mathsf{mod}~16 = 0$ and an existentially-quantified ``blank'' second field for the root $x \mapsto m,-,l,r$.

%As shown in (\ref{eqn:bigraphintrofoldunfold}) of Section
%\ref{sec:orientation}, Hobor and Villard\cite{hobor:ramification}
%defined the separation logic graph predicate
%$\mathsf{graph}(x,\gamma)$ in direct analogy to the standard
%separation logic definition of a tree. Note that the two-neighborhood
%means $\gamma$ is a BiGraph. However, it is peculiarly challenging in
%rigorously formalizing $\p{graph}$.

%* Showing that neither traditional fixpoint method works

%Appel and McAllester proposed another fixpoint $\mu_{\mathsf{A}}$
%that is sometimes used to define recursive predicates in separation
%logic \cite{appel:fixpoint}.  This time the functional $F_P$ needs to be
%\emph{contractive}, which to a first order of approximation means that
%all recursion needs to be guarded by the ``approximation
%modality''~$\rhd$~\cite{appel:vmm}, \emph{i.e.} our graph predicate would
%look like
%\begin{align*}
%\grapham(x, \gamma) ~ &\stackrel{\Delta}{=}\\
% (x = 0 /| \p{emp}) & |/ \exists m,l,r.~ \gamma(x)=(m,l,r) /| \null \\
% x |-> m,l,r & ** \rhd \grapham(l, \gamma) ** \rhd \grapham(r, \gamma)
%\end{align*}
%
%Unfortunately, $\rhd P$ is not precise for all $P$, so $\grapham$ is not precise either.  The approximation modality's universal imprecision has never been noticed before. % in the literature.  %We must do better.

%%%%%%%%%%%%%%%%%%%%%%%%%%%%%%%%%%%%%%%%%%%%%%%%%%
%%%%%% Edit0: cut down contractive recursion
%%%%%%%%%%%%%%%%%%%%%%%%%%%%%%%%%%%%%%%%%%%%%%%%%%


\subsection{Defining a good \p{graph} predicate}\label{sec:goodgraph}

Rather than trying to define \p{graph} as a recursive fixpoint, 
we will instead give it a flat structure.  Graphs in separation
logic have been defined in similar ways before~\cite{ilya-graphs};
our innovation is that we prove---with the amount of precision 
required to convince Coq---that we can still enjoy fold/unfold 
with our flat definition.  Our path starts with the iterated 
separating conjunction or ``big star'', which is first defined over 
lists and then extended to sets as follows:

%%%%%%%%%%%%%%%%%%%%%%%%%%%%%%%%%%%%%%%%%%%%%%%%%%
%%%%%% Edit1
%%%%%%%%%%%%%%%%%%%%%%%%%%%%%%%%%%%%%%%%%%%%%%%%%%
%%%%%% Edit1: Qinxiang's proposal starts
%%%%%%%%%%%%%%%%%%%%%%%%%%%%%%%%%%%%%%%%%%%%%%%%%%
\iftrue
\[
\begin{array}{@{}l@{}}
\underset{\{l_1, l_2,\dots,l_n\}}{\bigstar}P ~~ \defeq ~~ P(l_1) *
  P(l_2) * \dots * P(l_n) \\
\underset{S}{\bigstar} P ~~ \defeq ~~ \exists L.~ (\p{NoDup}\ L) /| (\forall x.~ x\ \p{in}\ L <=> x \in S) /| \underset{L}{\bigstar}P
\end{array}
\]
We are now ready to define a good \p{graph} predicate:
\fi
%%%%%%%%%%%%%%%%%%%%%%%%%%%%%%%%%%%%%%%%%%%%%%%%%%
%%%%%% Edit1: Qinxiang's proposal ends
%%%%%%%%%%%%%%%%%%%%%%%%%%%%%%%%%%%%%%%%%%%%%%%%%%
%%%%%% Edit1: Original version starts
%%%%%%%%%%%%%%%%%%%%%%%%%%%%%%%%%%%%%%%%%%%%%%%%%%
\iffalse
\begin{equation*}
  \underset{\{l_1, l_2,\dots,l_n\}}{\bigstar}P ~~ \defeq ~~ P(l_1) *
  P(l_2) * \dots * P(l_n).
\end{equation*}
Formally $\bigstar$ is defined over a list rather than a set and is parameterized by a predicate $P$.  It is natural to extend $\bigstar$ to a set $S$ with an existentially-quantified duplicate-free list~$L$:
\[
\underset{S}{\bigstar} P ~~ \defeq ~~ \exists L.~ (\p{NoDup}\ L) /| (\forall x.~ x\ \p{in}\ L <=> x \in S) /| \underset{L}{\bigstar}P
\]
We use the same $\bigstar$ notation since the concepts are similar, but the existential adds a little pain since we need to prove that all choices of $L$ yield equivalent predicates.

We are now ready to give a good \p{graph} predicate:
\fi
%%%%%%%%%%%%%%%%%%%%%%%%%%%%%%%%%%%%%%%%%%%%%%%%%%
%%%%%% Edit1: Original version ends
%%%%%%%%%%%%%%%%%%%%%%%%%%%%%%%%%%%%%%%%%%%%%%%%%%
\vspace{-1.5ex}
\begin{equation}\label{eqn:iter_def}
  \p{graph}(x, \gamma) ~~ \defeq ~~ \underset{v \in \mathit{reach}(\gamma, x)}{\bigstar} v\mapsto\gamma(v)
\vspace{-1.5ex}
\end{equation}
$\gamma$ is a GeneralGraph and ``$x |-> \gamma(x)$'' is a predicate that says how a single node fits in memory. In Figure~\ref{fig:markgraph} it was:
\[
\exists m,l,r.~\gamma(x) = (m,l,r) /| x |-> m,-,l,r /| x\ \p{mod}\ 16 = 0
\]
{\color{magenta}$\gamma$ need not be a bigraph, but \emph{e.g.} can have many edges.}

Our definition of \p{graph} is flat in the sense that there is no obvious way to follow the link structure recursively.  Happily, we can recover a general recursive fold/unfold (if $x |-> \gamma(x)$ and the GeneralGraph has the necessary properties in its soundness condition):
\vspace{-1ex}
\begin{equation}
\label{eqn:unfold_graph}
\hspace{-1em}\begin{array}{@{}lc@{\hspace{1pt}}c@{\hspace{1pt}}l@{}}
\p{graph}(x,\gamma)  <=>  x |-> \gamma(x) ** \big(\!\!\!\!\!\!\!\!\!\!\!\!\!\underset{\color{magenta}n \in \p{neighbors}(\gamma,x)}{\raisebox{-0.3ex}{\resizebox{0.75em}{!}{$\scon$}} \hspace{-2.18ex} \bigcup}\!\!\!\!\!\!\!\!\!\!\!\! \p{graph}(\gamma,n) \big) \\
[2pt]
\text{~~~ where ~~ }\underset{l_1,\dots,l_n}{\raisebox{-0.3ex}{\resizebox{0.75em}{!}{$\scon$}}\hspace{-2.18ex} \bigcup} \! \! P  \defeq  P(l_1) ** \ldots ** P(l_n) \end{array}
\vspace{-1ex}
\end{equation}

The proof of the $<=$ direction requires care. The difficulty is that if two nodes $x |-> \gamma(x)$ and $x' |-> \gamma(x')$ are \emph{skewed}, \emph{i.e.} ``partially overlapping'' with some---but not all---of $x$'s memory cells shared with $x'$, then the $\bigstar$ on the left hand side cannot separate them.  To avoid skewing we require that $x |-> \gamma(x)$ be \emph{alignable}. A predicate $P$ is alignable when
\[
\forall x,y.~ \Big(P(x) ** P(y) |- \big(P(x) /| x = y\big) |/ \big(P(x) * P(y)\big)\Big)
\]
That is, they overlap either completely or not at all. In a Java-like memory model this property is automatic because pointers in such a model always point to the root/beginning of an object.  In contrast, in a C-like memory model such as in VST/CompCert, this property is not automatic because pointers can point anywhere.  In such a model, alignment is most easily enforced by storing graph nodes at addresses that are multiples of an appropriate size (16 in Figure~\ref{fig:markgraph}).

{\color{magenta}Some of our VST proofs do not use fold/unfold, instead preferring to use the lemmas in~\S\ref{sec:ramifylib} directly.  On the other hand, for HIP/SLEEK fold/unfold is vital, and it is heartening to know that the recursive relationship holds.  We also prove fold/unfold lemma for DAGs in which we get a $*$ between the root and its $**$-joined neighbors. % rather than the $**$ present in \eqref{eqn:unfold_graph}.
}

\subsection{Ramification Libraries}\label{sec:ramifylib}

\begin{figure}[t]
\centering
\beginpgfgraphicnamed{infrastructure}
\begin{tikzpicture}[
->/.style={thick, arrows={-Stealth}},
ent/.style={shape=rectangle, rounded corners=4pt, draw, on grid}]
\node[ent] (SM) at (0, 0) {\small Step-Indexed Model};
\node[ent] (DM) [right=4.4 of SM] {\small Direct Model};
\node[ent] (CL) [above=1 of SM] {\small Core Logic};
\node[ent] (SL) [above=1 of DM] {\small Supplementary Logic};
\node[ent] (LF) [above left=1 and 1.2 of CL] {\small Logic Facts};
\node[ent] (RF) [above right=1 and 1.2 of CL] {\small Basic Ramification};
\node[ent] (BF) [above=1 of LF] {\small $\bigstar$ Facts};
\node[ent] (BR) [above=1 of RF] {\small $\bigstar$ Ramification};
\node[ent] (GF) [above=1 of BF] {\small Graph Facts};
\node[ent] (GR) [above=1 of BR] {\small Graph Ramification};
\node[ent] (SLF) [above=1 of SL] {\small Supplementary Logic Facts};
\node[ent] (SBF) [above=1 of SLF] {\small Supplementary $\bigstar$ Facts};
\node[ent] (SGF) [above=1 of SBF] {\small Supplementary Graph Facts};
\draw [double, ->] (SM) to (CL);
\draw [double, ->] (SM) to (SL);
\draw [double, ->] (DM) to (CL);
\draw [double, ->] (DM) to (SL);
\draw [->] (CL) to (LF);
\draw [->] (CL) to (RF);
\draw [->] (CL) to (SL);
\draw [->] (SL) to (SLF);
\draw [->] (SLF) to (SBF);
\draw [->] (SBF) to (SGF);
\draw [->] (LF) to (RF);
\draw [->] (LF) to (BF);
\draw [->] (RF) to (SLF);
\draw [->] (RF) to (BR);
\draw [->] (BF) to (BR);
\draw [->] (BF) to (GF);
\draw [->] (GF) to (GR);
\draw [->] (GR) to (SGF);
\draw [->] (BR) to (GR);
\draw [->] (BR) to (SBF);
\node (legend1) [below right=0.2 and -1.2 of SM] {\small Dependence};
\coordinate[left=0.8 of legend1]  (l1);
\draw [->] (l1) to (legend1);
\node (legend2) [right=1 of legend1] {\small Instantialization Choices};
\coordinate[left=0.8 of legend2]  (l2);
\draw [double, ->] (l2) to (legend2);
\end{tikzpicture}
\endpgfgraphicnamed
\vspace{1ex}
\caption{Infrastructure of ramification library}\label{fig:infra}
\end{figure}

We provide the architecture of our spatial development in Figure~\ref{fig:infra}.  Starting from the bottom, notice that there are two underlying heap models: the Step-Indexed Model, which is the main heap model used in VST, and a much simpler Direct Model, {\color{magenta} which is used by HIP/SLEEK among others}. The Step-Indexed model is much fancier, but none of our development depends on its bells and whistles.

To isolate our development from these unnecessary complications, 
{\color{magenta}and to ensure that HIP/SLEEK can reuse our spatial 
reasoning, we use two interfaces: Core Logic and Supplementary 
Logic.  Both models can instantiate both interfaces, but generally 
speaking our VST proofs only need the Core properties to prove 
our examples, whereas HIP/SLEEK uses both Core and Supplemental.} 
Each interface defines some operators of separation logic and 
provides some axioms about how they work.  For example, $*$ and 
$--*$ are in Core Logic, along with the axiom 
$(P |- Q --* R) <=> (P * Q |- R)$.  On the other hand, 
the $**$ and $--o$ operators are in Supplementary Logic, 
along with rules like $P |- P ** P$.

Above the Logic layer we have three towers, each three levels high.  The leftmost tower contains basic lemmas about Logic, $\bigstar$, and \p{graph}.  In the $\bigstar$ Facts box we prove \emph{e.g.}:
\[
\infrule{}
{A \cap B = \emptyset}
{\underset{x\in A}{\bigstar} P(x) \; * \underset{x\in B}{\bigstar} P(x) \;\, \Leftrightarrow \underset{x\in A \cup B}{\bigstar} P(x)}{}
\]

The middle tower is more interesting in that it is entirely focused on ramification entailments.  A robust library of ramification entailments is essential to make ramification work smoothly in practice.  The lowest level contains lemmas like:
\[
%\infrule{Ramify-Q-SPLIT}
\infrule{}
{G_1 \vdash L_1 * \forall x.~ (L_2 --* G_2) \\
 G'_1 \vdash L'_1 * \forall x.~ (L_2' --* G'_2)}
{G_1 * G'_1 \vdash (L_1 * L'_1) * \forall x.~ \big((L_2 * L'_2)--* (G_2 * G'_2)\big)}{}
\]
We use this lemma to break large ramification entailments into more manageable pieces in a compositional way. % compositionally.

The middle level contains $\bigstar$ ramification lemmas, \emph{e.g.}:
\begin{equation}
\label{ramify:bigstar}
\infrule{}
{A \cap B = \emptyset  \qquad  A' \cap B = \emptyset}
{\underset{x\in A\cup B}{\bigstar} P(x) \vdash \! \underset{x\in A}{\bigstar} \! P(x) * \Big( \underset{x\in A'}{\bigstar} \! P(x) \! \wand \! \! \! \! \underset{x\in A' \cup B}{\bigstar} \! \! P(x)\Big)}{}
\end{equation}

The top level is focused on graph ramifications, such as the ``update one node'' lemma:
\begin{equation}
\label{lem:updategraphnode}
%{\gamma(x_0) = \gamma'(x_0) \text{ for any } x_0 \neq x }
\infrule{}
{\forall x_0 \neq x.~ \gamma(x_0) = \gamma'(x_0) \\ \p{neighbors}(\gamma,x)=\p{neighbors}(\gamma',x)}
{\p{graph}(x, \gamma) \! \vdash \! x \! \mapsto \! \gamma(x) \! * \! \big(x \! \mapsto \! \gamma'(x) \! \wand \! \p{graph}(x, \gamma')\big)}{}
\end{equation}
This lemma was used on line~\ref{code:markram2} in Figure~\ref{fig:markgraph}.

This layered structure enables proof reuse. All of the theorems for $\p{graph}$ are proved from the properties of iterated separating conjunction, but having a modular library allows $\bigstar$ to be reused in other structures smoothly.

Also, all of our verifications of different graph algorithms use the proof rules of $\p{graph}$ at the top level in the library. Taking the marking algorithm we introduced in \S\ref{sec:orientation} as an example, we prove the following theorem from the library:
\begin{equation}
\label{lem:updatesubgraph}
\frac
{n \in \p{neighbors}(\gamma,x)}
{
\mbox{
$\begin{array}{@{}l@{}l@{}}
\p{graph}(x, \gamma) \vdash \\
\p{graph}(n, \gamma) \! * \!
\big(\forall \gamma'. \m{mark}(\gamma, n, \gamma') \! /| \! \p{graph}(n, \gamma') \! --*
\m{mark}(\gamma, n, \gamma') \! /| \! \p{graph}(x, \gamma')\big)
\end{array}$
}
}
\end{equation}

The Supplementary tower contains properties not used by most of the VST examples.  This includes the fold/unfold relationship from \S\ref{sec:goodgraph}, facts about precision, \emph{etc}. Some of these properties are needed by HIP/SLEEK, while others are mostly included for completeness.

% are there just because we felt they might be useful in the future.

%One benefit of the definition in (\ref{eqn:iter_def}) is that the pure
%mathematical graph $\gamma$ in $\mathtt{graph}$ is not necessarily a
%BiGraph. (\ref{eqn:iter_def}) can represent a general graph with
%variant number of neighors as long as extending the definition of
%$\gamma(x)$ to data mapped by the label function and every neighbor of
%node $x$.
%
%Moreover, it turns out that the $\bigstar$ notation is a more useful
%and fundamental concept than $\mathtt{graph}$. There are two parts of
%the $\bigstar$ in (\ref{eqn:iter_def}): one is the predicate $\mapsto$
%and the other is the node set which the $\mapsto$ iterates on. They
%both bind to $\gamma$ in (\ref{eqn:iter_def}) for $\p{graph}(x,
%\gamma)$, which is a special case. In section \ref{sec:applicable}, we
%will see the specification of a spanning tree algorithm which uses
%$\bigstar$ directly instead of $\p{graph}$ because in that
%specification, the predicate $\mapsto$ and the node set bind to
%different mathematical graphs. Furthermore, we generalize the
%ramification rules for $\p{graph}$ in \cite{hobor:ramification}, which
%uses $\bigstar$ so as to be applied in all verification examples.

%% 1.2. \texttt{Iter\_sepcon} and \texttt{pred\_sepcon} are defined. And related ramification rules are proved.
%% 1.3. The most general graph-spatial-predicate \texttt{vertices\_at} are defined (for all possible styles of graphs). Related ramification rules are proved. Graph and graphs are defined as special cases of vertices at.

%% 2. A minor implementation trick. There are many tactics defined in \texttt{msl\_ext/ramify\_tactics.v}, which can manipulate low level heaps efficiently.

%% * Separating the material into the general vs. tool-specific part.  Measurements of etc.


\section{Verifying graph-manipulating programs}

4. General Strategy for Verifying Programs
4.1. Using relation. (Better for proof's code reuse). For example, spanning tree is mark together with some structural requirement.
4.2. Using existential quantifier in post condition.

%[5:19:46 AM] Aquinas Hobor: the localize/unlocalize tactics or the semantic model for semax_ram
%[5:19:46 AM] caoqinxiang: And we will discuss the trick in "\item Integrating ramification into verification tools" section right?
%[5:20:00 AM] Aquinas Hobor: Yes, and also the model for semax_ram
%[5:20:06 AM] Aquinas Hobor: And also the other examples
%[5:20:12 AM] caoqinxiang: Aha, yes.
%[5:20:23 AM] Aquinas Hobor: And also point out which parts are easier to automate, etc.
%[5:20:49 AM] Aquinas Hobor: For example, how does the "close w.r.t. modified variables" work?  Is that calculated automagically?
%[5:21:06 AM] caoqinxiang: yes.
%[5:21:22 AM | Edited 5:21:37 AM] caoqinxiang: Box_c(P) is closed w.r.t. to modvar(c)
%[5:21:34 AM] Aquinas Hobor: Right, so we point that out.  What about how the existentials etc. work.  Is that automagic also or does the user have to specify them?
%[5:22:20 AM] caoqinxiang: User specify which variables are dependent, and the tactic makes them quantified.
%[5:22:29 AM] Aquinas Hobor: So "semiautomatic"?
%[5:22:44 AM] caoqinxiang: Right, user should give hints.
%[5:22:52 AM] caoqinxiang: And tactic do the rest.
%[5:23:06 AM] Aquinas Hobor: Right.  Anyhow, that goes there, along with the extra VST examples.

DO WE NEED RAMIFICATION BEFORE THIS SECTION?

VST is a correctness-certified tool to prove functional correctness of C programs CITE. All Hoare rules are proved sound and users can use them to build modularized proof. At the same time, users can have all the convenience offered by separation logic. For example, frame rule is already proved sound as well. VST is fully developed in Coq and it uses the C semantics offered by ComCert. CITE

AQUINAS, IS IT CORRECT TO USE PRESENT TENSE TO TALK ABOUT VST? SHALL I USE PAST TENSE?

$$ \frac{\{ P \} c \{Q \} \quad FreeVar(F) \cap ModVar(c) = \emptyset } {\{P * F \} c \{ Q * F \}} $$

\[
\frac{
\begin{array}{c}
\{ L \} c \{L' \} \quad FreeVar(L' -* G') \cap ModVar(c) = \emptyset \\
G \vdash L * (L' -* G')
\end{array}
}
{\{G \} c \{ G' \}}
\]

EDITING FORMAT

In this work we present, we establish the soundness of ramification rule based on frame rule. Noticing that local variables are stored in stack in C, expressing the property of a local variable's value is a pure fact rather than a spatial fact. So, we prove a pure-facts-related rule besides the primary one such that ramification is much easier to apply in VST.

\[
\frac
{
\begin{array}{c}
\{ L \} c \{PureLocal \wedge L' \} \\ 
FreeVar(PureFrame \wedge (L' -* G')) \cap ModVar(c) = \emptyset \\
G \vdash L * (PureFrame \wedge (L' -* G'))
\end{array}
} {\{G \} c \{ PureFrame \wedge PureLocal \wedge G' \}}
\]

EDITING FORMAT


MORE TO WRITE: divide pure facts into frame and local based on VST's canonical form. Based on localize/unlocalize, compared with frame rule, ramification rule have better use on VST's automatic symbolic execution system.





2.4. VST instance of \texttt{pSpatialGraph\_Graph\_Bi} and \texttt{sSpatialGraph\_Graph\_Bi} are constructed in "\texttt{spatial\_graph\_aligned\_bi\_VST.v}" and "\texttt{spatial\_graph\_unaligned\_bi\_VST.v}".

3. Embed ramification into VST.
3.1. Ramification rule are proved sound in VST.
3.2. A special ramification rule for VST's Sep-Local-Prop style pre/post condition is prove. The point is traditional ramification rule require the whole frame-like-wand-expression to be closed w.r.t. the modified variables. This special rule split closed and unclosed away.
3.3. Localize and unlocalize are defined.
3.3.1. Localize/unlocalize offer a user-friendly way of using ramification rule.
3.3.2. Unlocalize tactic need "Grab Existential Variables" afterwards. It is not nice.
3.3.3. Writing Ocaml plugin is one solution. But we need to develop for both mac and windows.
3.3.4. Or we can see whether Coq's next version offers more tactics for existential variables.


\section{Enabling externally-verified lemmas in HIP/SLEEK}

* the connection to HIP/SLEEK

In the H/S section we talk about the engineering inside H/S, the module type/module interface, forward ramify, etc.

\section{Applying ramification}

5. Mark algorithm
5.1. For Ramification-Paper-style proof
5.1.1. Math land theorems for marking algorithm (general situation and bi-graph situation) are all proved. Mainly in "\texttt{marked\_graph.v}" and "\texttt{spatial\_graph\_mark\_bi.v}".
5.1.2. Ramification rule for marking algorithm (bi-graph situation) are all proved in "\texttt{spatial\_graph\_mark\_bi.v}".
5.1.3. Combining 2.4 and 3.1.1 and 3.1.2, we have a end-to-end proof for marking-graph in VST.
5.1.4. We have an end-to-end proof for marking-dag, but not defining dag predicate as a whole.
5.1.5. The module type which will be generated by HIP/SLEEK should be instantiated by 2.2 and 2.5.

6. Spanning tree algorithm
6.1. We divide the spanning tree relation into structural part and marking part. They are both defined properly.
6.2. Important pure facts and ramification rules are not proved yet.
6.3. Shengyi has already known how to use VST to handle the C program of bigraph spanning tree.

\paragraph{Comparison with~\citet{hobor:ramification}.}
Our work builds on the theory of ramification by Hobor and Villard,
who verified graph algorithms on pen-and-paper using their \infrulestyle{Ramify} rule:
\begin{equation*}
%\label{eq:ramify}
\inferrule[Ramify]
{\{ L_1 \} ~ c ~ \{ L_2 \} \\
G_1 |- L_1 * (L_2 --* G_2)}
{\{ G_1 \} ~ c ~ \{ G_2 \}} \qquad \mathit{freevars}(L_2 --* G_2) \cap \MV(c) = \emptyset
\end{equation*}
Our \textsc{Localize} rule upgrades \textsc{Ramify} to better handle modified program
variables (note the side condition and recall the discussion in \S\ref{sec:localizations})
and existential quantifiers in postconditions.  Hobor and Villard avoided these challenges
by proposing a unwieldy variant of \infrulestyle{Ramify} called \infrulestyle{RamifyAssign}, which
could reason about the special case of a single assignment $\li{x=}f(\ldots)$, assuming
the verifier can make the local program translation to $\li{x'=}f(\ldots)\li{; x=x'}$,
where \li{x'} is fresh.  This is nontrivial in large existing formal
developments, such as VST, that do not have any way to prove programs equivalent.
Hobor and Villard could not verify unmodified program code, modify program variables
inside nested localization blocks, or handle multiple assignments in a single block as
in lines~\ref{code:markbeforetripleramify}--\ref{code:markaftertripleramify} of
Figure~\ref{fig:markgraph}.  They avoided existentials in localized
postconditions by defining all mathematical operations (\emph{e.g.} $\m{mark}$) as
functions rather than as relations; this is fine for pen-and-paper, but painful in
a mechanized setting wherein functions must be proven to terminate.

Hobor and Villard treated mathematical graphs as triples $(V,E,L)$ of
vertices, edges, and a vertex labeling function, where vertices had no more than two
neighbors. Our mathematical graph framework~(\S\ref{sec:mathgraph}) is more
modular and versatile, and ships with hundreds of reusable definitions and theorems. Further, our library has been tuned to work smoothly in a mechanized context.


Hobor and Villard erroneously defined spatial graphs
recursively. Unfortunately, other members of the research
community (\emph{e.g.}~\citet{raadvg15}) followed their lead.  We expose this
error~(\S\ref{sec:fixpointfail}) and provide a sound and rather general definition for
\p{graph} that recovers fold/unfold reasoning~(\S\ref{sec:goodgraph}).  We develop a
much more general and more modular set of related lemmas and connected our spatial
reasoning to the verification framework of CompCert/VST~(\S\ref{sec:vst}).
Our development is entirely
machine-checked~(\S\ref{sec:development}) whereas they used only pen and paper.

\paragraph{Other pen-and-paper verification of graph algorithms and/or $**$.}

\citet{hongseok:phd} verified the Schorr-Waite algorithm, and this 
is widely considered a landmark in the early separation logic literature. 
\citet{bornat:aliasing04}~gave an early attempt to reason about graph algorithms 
in separation logic in a more general way. 
\citet{neelthesis}~provided the first separation logic proof of union-find.

\citet{rey-slnotes}~was the first to document the overlapping 
conjunction $**$, albeit without any strategy to reason about it using Hoare rules. 
\citet{gardnerms12}~were the first to reason about a program using $**$ in 
Javascript. 
\citet{raadvg15}~used $**$ within their CoLoSL program logic to reason about 
a concurrent spanning algorithm using a kind of ``concurrent localization''.

%\paragraph{Alternative fixpoint constructions.}
%Appel and McAllester defined an alternative ``contractive'' fixpoint that is sometimes used to define recursive predicates in separation
%logic~\cite{appel:fixpoint}.  In Appendix~\ref{apx:appelfixpiont} we explain why Appel and McAllester's  is also unsuitable to define graphs.

\paragraph{Machine-checked verification of graph algorithms.}
A decade after Yang verified Schorr-Waite on paper, \citet{leino10} automated 
its verification in Dafny. 
\citet{ilya-graphs}~verified a concurrent spanning tree algorithm, and 
moreover developed mechanized Coq proofs. Their algorithm was written in FCSL, 
a monadic DSL that combines effectful operations with pure Coq expressions; 
FSCL cannot be executed. 
\citet{chen18}~compared how three provers (Coq, Isabelle, and Why3) can 
verify Tarjan’s strongly-connected component algorithm written in the native 
language of each of the tools. Because these are written in the native languages 
of a proof assistant, they avoid “real-world” language concerns such as 
memory models and overflow.

\citet{lamneu15}~extended the Isabelle Refinement Framework to verify a range of 
DFS algorithms via stepwise refinement.
Their framework allows the reuse of previously-proved DFS 
invariants by establishing an inductive
``most specific invariant'' and deriving other inductive invariants from it.
\citet{lamsef19}~extended this further and presented verifications of 
the correctness and time complexity of the Edmonds-Karp and push-relabel
algorithms. Lammich et al. produced very readable proofs of classic 
textbook algorithms by using the Isar language atop their Isabelle proofs.
They used Isabelle's code generator to export efficient executable code, 
but with the caveat that the code comes with a guarantee of only 
partial correctness semantics.

\citet{char11}~used his CFML tool to Coq-verify an OCaml implementation of 
Dijkstra. 
\citet{gueneauetal19}~extended CFML and verified the correctness 
and time complexity of a modified version of the BFGT cycle-detection algorithm.
The graph algorithms verified in CFML tend to be ``graph theory'' in flavour, 
whereas the algorithms we have verified tend to have more of a
``systems'' flavor. This difference is partially explained by the fact that 
code written in ML can take advantage of its high-level design, whereas
code written in C is often interested in handling grungy systems tasks. For 
example, references in ML cannot be null and do not support pointer arithmetic; 
of course both are possible---and lead to nontrivial complications---in 
C. Accordingly, the CFML proofs benefit from ML’s cleaner computational model. 
Our verifications are in C so we must contend with C’s memory model, pointer 
arithmetic, significant scope for undefined behavior, and so forth.

\citet{charpott15, charpott19}~used CFML to verify the correctness and 
time complexity of union-find. Their work is an interesting counterpoint to 
ours because, while it maintains an abstraction between the client and the 
internal mathematical/spatial facts that the client need not know, it does 
not maintain a separation between the mathematical and spatial 
facts themselves, as we do in~\S\ref{sec:mathgraph} and ~\S\ref{sec:spacegraph}.
This separation is worthwhile: our modular method let us verify an alternate 
version of union-find that uses an array of vertices rather than individually 
heap-allocated nodes. This 
secondary verification then used \emph{exactly the same} mathematical proof of 
functional correctness despite the radically different layout of spatial 
memory.
Our work does not verify the time complexity of union-find. When 
we attempted to prove the necessary amortisation bounds we ran into an 
overflow issue: it was impossible to prove that the rank would not exceed 
\li{max\_int} because the CompCert memory model does not place a bound on the 
total number of allocations. Informally, this overflow is impossible in 
practice because no computer has $2^{2^{\tiny 64}}$ bytes of memory, which would 
be required for this overflow to occur, but Coq remains unconvinced. 
Chargu{\'{e}}raud and Pottier acknowledged and sidestepped this issue by 
representing rank using the Coq type \li{Z}, which was not an option for us 
given the end-to-end nature of the VST+CompCert toolchain.

\paragraph{Verification tools in Coq.}
Our work interacts with the Floyd verification module within the Verified 
Software Toolchain (VST)~\cite{appel:programlogics}. The Floyd module uses 
tactics to enable the separation-logic verification of CompCert C programs. 
VST connects to the CompCert certified C compiler~\cite{leroy:compcert}, and 
thus has no gaps or admits between the verified source code and the eventual
assembly code~\cite{appelvst}.

Charge! likewise uses Coq tactics to work with a shallow embedding of higher 
order separation logic, but focuses on OO programs written in 
Java/C\#~~\cite{bengtson:charge}. Iris Proof Mode provides a similar framework 
for higher-order concurrent reasoning in Coq~\cite{krebbers:iris}.

CFML enables the verification of OCaml programs by reasoning about their
``characteristic formulae'' in separation logic using Coq~\cite{char10, char11}. 
CFML has been used to verify a range of functional and imperative programs,
including some graph-related algorithms as discussed 
above. \citet{charpott15, charpott19}~extended CFML to reason about time 
credits. The work of \citet{gueneau17}~indicates that CFML is exploring a connection 
with the certified CakeML compiler~\cite{cakeml}.

While the tools above require substantial human guidance, 
Bedrock~\cite{chlipala:bedrock} is a more automated approach to the verification of 
low level programs using separation logic in Coq. 
Bedrock leverages the fact that phrasing function 
specifications in a \emph{computational} style 
(in this case, inspired by functional programming) 
leads to separation logic proof obligations that are quite automatable.
It simplifies these obligations into pure mathematics using a 
custom workhorse tactic, and then discharges those 
obligations using standard Coq automation.

\paragraph{Other verification tools.} 

Many more-automated verification tools also use separation logic in a forward
reasoning style. Smallfoot~\cite{berdine:smallfoot}, jStar~~\cite{distefanop08}, 
HIP/SLEEK~\cite{chin:hipsleek}, and Verifast~\cite{jacobs:verifast} are landmarks
at various points on the expressibility-automatability spectrum. 
KeY~\cite{beckert:2007} and Dafny~\cite{leino10} are verifiers that are not 
based on separation logic. KeY uses an interactive verifier while Dafny pursues
 automation with Z3~\cite{moura2008}.

\paragraph{Mechanized mathematical graph theory.}
There is a long history, going back at least 28 years, of mechanized 
reasoning about mathematical graphs~\cite{wong1991}. 
The most famous mechanically verified “graph theorem” is the Four Color 
Theorem~\cite{gonthier2005computer}; however the development actually uses
hypermaps instead of graphs. In general most “mathematical graph” frameworks in 
the literature~\cite{wong1991, chou1994, yamamoto1995formalization, rwpgt1998, yamamoto1998formalization, tamai2000formal, duprat2001coq, ridge2005graphs, nipkow2016, dijkstra_shortest_path-afp} were not used to verify real 
code, for which they seem unsuitable. Verifying real code requires delicate concepts such as removing a subgraph, null nodes, and parallel edges, and one of our contributions is that our framework is general enough to support such verification. 
\citet{noschinski2015}~built a graph library in Isabelle/HOL whose formalization 
is the closest to ours, 
\emph{e.g.} supporting graphs with labeled and parallel arcs. 
Beyond being in Coq, our setup supports at least three features beyond 
Noschinski’s: reasoning about incomplete graphs (as discussed 
in~\S\ref{sec:mathinfra} using figure~\ref{fig:pregraph}), labeling the graph 
as a whole (used, for example, in the garbage collector to store 
metainformation about the number and location of the generations), and our 
modular typeclass-supported “graphs with properties” setup in General Graph 
(as described in~\S\ref{subsec:graphplugins}). 
\citet{dubois2015graphes} and \citet{noschinski2015formalizing}~used proof assistants to 
design verifiable checkers for solutions to graph problems. 
\citet{bauer20025} and \citet{yamamoto1995formalization}~used an inductive encoding of graphs to formalize planar graph theory.


\paragraph{Verification of garbage collection algorithms.}
Schism \cite{gcexample4,gcexample4a} is a certified concurrent
collector built in a Java VM that services multi-core architectures with weak memory consistency.
\citet{gcexample5, gcexample3} introduced GCminor, which is
a certified translation step added to CompCert's translation from Clight to assembly.
GCminor makes explicit the specific invariants that the garbage collector
relies upon, thus minimising errors due to the violation of invariants
between the garbage collector and the mutator.
\citet{gcexample2} annotated x86 code
for two GCs by hand, and then used Boogie and the Z3 automated theorem prover
to verify their correctness automatically.

The closest piece of work to our certified GC is probably the excellent certified GC
for the Cake ML project~\cite{cakemlgc}, since both integrate a certified GC into 
a certified compiler for a functional language.  Their GC is written closer to assembly 
than C, which is both a positive---in that they avoid undefined behaviors---and a negative, 
in that their GC is harder to understand and upgrade and cannot take advantage of the
mature CompCert compiler.  Their GC lacks some of our optimisations (\emph{e.g.} they have 
only three generations), but on the other hand handles mutation in the GC heap.  The largest 
difference, however, is that we present an integrated graph framework suitable for reasoning 
about many graph algorithms, of which our GC is merely the flagship.  In contrast, they focus 
much more narrowly on the problem of certified GCs.


\section{Conclusion}

\bibliographystyle{abbrvnat}
%\bibliographystyle{tropbien_nopages}
\bibliography{autoquack}


\appendix

{\color{magenta}
\section{Simplifying ramification entailments}
After applying \infrulestyle{Solve Ramify-PQ}, it is often desirable to break the ramification entailment into smaller disjoint pieces before trying to solve it directly.
One common case is to ``frame out'' an unneeded part of the global state:
\[
\infrule{Frame Ramify-Q}
{G_1 \vdash L_1 * \forall x.~ (L_2 --* G_2)}  
{G_1 * F \vdash L_1 * \forall x.~ \big(L_2 --* (G_2 * F)\big) }
{\begin{array}{c}F \text{ ignores} \\ \MV(c) \cup \{x\} \end{array}} \qquad \qquad \qquad
\]
In fact \infrulestyle{Frame Ramify-Q} is a consequence of the more general \[
\infrule{Split Ramify-Q}
{G_1 \vdash L_1 * \big(\forall x.~ (L_2 --* G_2)\big) \! \! \! \! \\
 G_1' \vdash L_1' * \big(\forall x.~ (L_2' --* G_2')\big) }
{G_1 * G_1' \vdash L_1 * L_1' * \Big(\forall x.~ \big((L_2 * L_2') --* (G_2 * G_2')\big)\Big)} {}
\]
In general the strategy is to apply \infrulestyle{Frame Ramify-P} and \infrulestyle{Split Ramify-P} until the ramification entailments are as small as they can be (while remaining true!) before using \infrulestyle{Solve Ramify-P} on the remaining ``atoms''.

The situation is unfortunately a little messier when the postconditions contain existential quantifiers.

\[\text{UNSOUND-RAM-Q-SPLIT}\]
\Rule{}
{G_1 \vdash L_1 * (\exists x, L_1' (x) --* \exists x, G_1'(x)) \\
G_2 \vdash L_2 * (\exists x, L_2' (x) --* \exists x, G_2'(x)) \\}
{G_1 * G_2 \vdash L_1 * L_2 * (\exists x, L_1'(x) * L_2'(x) --* \exists x, G_1'(x) * G_2'(x)) }


\[
\infrule{Ramify-Q}
{\{ L \} ~ c ~ \{\exists x.~ L' \} \\
 G \vdash L * \big(\forall x.~ (L' --* G')\big)}
{\{ G \} ~ c ~ \{ \exists x.~G' \}}{}
\]

}


\section{Junk}
{\color{magenta} Universally-quantified metavariables can appear free in the predicates to make further connections.
Assuming that the abstracted pre- and postconditions $A$, $B$, $C$, and $D$ above all use \li{x}, we proceed
as follows.  First we introduce a new fresh metavariable $x$ whose value will be equal to \li{x} after the localization, and then choose $F \stackrel{\Delta}{=} [\li{x} |-> x] (C -* D)$, that is we substitute the program
variable \li{x} for the metavariable $x$.  Since we have substituted away \li{x}, $F$ ignores it and so we satisfy the side condition on \infrulestyle{Solve Ramify-P}.  We then must strengthen $C$ into $C' \stackrel{\Delta}{=} C /| \li{x} = x$ to make the connection at the appropriate program point.  Now we are left with the entailments
\[
\begin{array}{lcl}
\li{x} = 5 /| A & |- & (\li{x} = 5 /| B) * F \\
F & |- & (\li{x} = 6 /| C') -* (x = 6 /| D)
\end{array}
\]
To further relate the earlier and later values of \li{x} in $F$ we can introduce a second fresh $x'$ and use $B' \stackrel{\Delta}{=} B /| \li{x} = x'$.
}

The \infrulestyle{Ramify} rule is sound but interacts poorly with modified program variables (as in lines~\ref{code:markbeforetripleramify}--\ref{code:markaftertripleramify} of Figure~\ref{fig:markgraph}) {\color{magenta} and
localized existentials (as in lines~\ref{code:beforemarkl}--\ref{code:aftermarkl})}.  Both of these limitations are annoying enough in paper proofs and graduate to major headaches in mechanized ones.  Happily, we show how to overcome both limitations in \S\ref{sec:freevars} and \S\ref{sec:existentials}, respectively, by presenting new variants of \infrulestyle{Ramify}.  Our notation carries over without significant change: just use the new rules to enable the more general ramification entailments they permit.
%When in doubt the most general rule, \infrulestyle{Ramify-PQ} from \S\ref{sec:existentials}, implies all of the others.

\section{More junk}
\hide{
\section{Ramification Rules}


\Rule{Frame  }
{\{ P \} c \{Q \} \\
  F \text{ is stable w.r.t. } \MV(c)\\}
 {\{P * F \} c \{ Q * F \}}

\Rule{Ramification   }
{\{ L \} c \{L' \} \\
 G \vdash L * (L' -* G') \\
 (L' -* G') \text{ is stable w.r.t. } \MV(c)\\}
{\{ G \} c \{ G' \}}

\Rule{Ramification-P }
{\{ L \} c \{L' \} \\
 G \vdash L * \Box^{\llbracket c \rrbracket} (L' -* G') \\}
{\{ G \} c \{ G' \}}

\subsection{P for Pure Facts}

Separation logic has been mechanized by many projects CITE CITE CITE.
In many of them, like VST and Charge!, expressing the value of a local
variable (a variable stored in stack) is a pure fact rather than a
spatial fact. Because the side condition of ramification rule requires $(L' -* G')$ to be stable w.r.t. modified local variables in $c$\footnote{In previous papers, the side conditions of Frame rule and ramification rule are usually expressed as ``$\FV(F) \cap \MV(c) = \emptyset$'' and ``$\FV(L' -* G') \cap \MV(c) = \emptyset$''. The side conditions used in this paper are equivalent with typical ones if the semantic interpretation of $\FV$ is used. All the previous mentioned projects takes semantic interpretation instead of syntactical interpretation.}, it is almost impossible to apply ramification rule in any practical situations in these systems. In this paper, we present a pure-facts-related rule (we call it ramification-P rule, or just P rule, in the rest of this paper) such that it is sound and practical in the most general setting of separation logics.

The primary ramification rule is essentially an application of the frame rule using $(L' -* G')$ as frame.
Thus, the key point of handling pure facts is to find a legal frame even if $(L' -* G')$ is not stable w.r.t. $\MV(c)$. This frame is $\Box^{\llbracket c \rrbracket} (L' -* G')$ in ramification-P rule.
\begin{eqnarray*}
m \models \Box^R P &  \Leftrightarrow  & \forall m', \text{ if } m\xrightarrow{R}m' \text{ then } m' \models P \\
m \xrightarrow{\llbracket c \rrbracket} m' & \Leftrightarrow &   \text{$m$ and $m'$ coincide everywhere} \\
&& \text{except $\MV(c)$} \\
P \text{ is stable} &  \Leftrightarrow  & \forall m \ m',  \text{if $m$ and $m'$ coincide everywhere} \\
\text{w.r.t. $S$} && \text{except $S$, then $m \models P$ iff $m' \models P$}
\end{eqnarray*}

Here, $\Box$ represents the necessity modal operator. The formula $\Box^{\llbracket c \rrbracket} (L' -* G')$ says, it is true on a state $m$ if and only if for any state $m'$, if $m$ and $m'$
coincide everywhere except on the variables modified by $c$, then $(L' -* G')$ is true on $m'$.

Based on the combination frame rule, consequence rule and three basic facts below, we can immediate prove ramification-P rule.
\begin{quotation}
(a) $\Box^{\llbracket c \rrbracket} (L' -* G')$ is stable w.r.t. $\MV(c).$\footnote{This can be proved directly from the definition of $\llbracket c \rrbracket$ and stability, and the fact that $\llbracket c \rrbracket$ is an equivalence relation.}

(b) $G \vdash L * \Box^{\llbracket c \rrbracket} (L' -* G')$. (Assumption)

(c) $L' *  \Box^{\llbracket c \rrbracket} (L' -* G') \vdash G'$. \footnote{When $R$ is reflexive, T-Axiom of modal logic is sound, i.e. for any $P$, $\Box^R P \vdash P$. As $\llbracket c \rrbracket$ is reflexive, we know the fact that $\Box^{\llbracket c \rrbracket} (L' -* G') \vdash L' -* G'$, which is immediate followed by $L' *  \Box^{\llbracket c \rrbracket (L' -* G')} \vdash G'$.}
\end{quotation}

\subsection{Establish the Assumption Entailment of P Rule}

It is well-known that the proof theory with magic wand is already complicated, so generally speaking, it will not be a easy task to prove an entailment with magic wand together with modality. However, people need to prove an entailment with form
\begin{equation}G \vdash  L * \Box^R (L' -* G') \label{eqn:Passu} \end{equation}
at first when applying ramification-P rule. Luckily, this special form makes the task simpler.

First of all, SOLVE-RAM-P rule can turn the proof goal into two wand-free and modality-free entailments. Specifically, people only need to find an $R$-stable predicate $F$, such that $G \vdash L * F$ and $F * L' \vdash G'$ are both true.

SOLVE-RAM-P alone is not a satisfactory proof theory because in that case using P rule would have no different from using frame rule directly. The key point here is that, an entailment with form \ref{Passu} can be proved in a modularized way. For primary ramification rule, CITE proposed two proof rule, RAM-FRAME and RAM-SPLIT\footnote{\Rule{RAM-FRAME }
{G \vdash L * (L' -* G') \\
F \text{ is stable w.r.t. } \MV(c) \\}
{G * F \vdash L * (L' -* G' * F) }

\Rule{RAM-SPLIT }
{G_1 \vdash L_1 * (L_1' -* G_1') \\
G_2 \vdash L_2 * (L_2' -* G_2') \\}
{G_1 * G_2 \vdash L_1 * L_2 * (L_1' * L_2' -* G_1' * G_2') }
}, to divide an entailment with form $G \vdash L * (L' -* G')$ into small pieces. When it comes to ramification-P rule, two corresponding proof rules, RAM-P-FRAME and RAM-P-SPLIT are still sound.

\Rule{SOLVE-RAM-P }
{G \vdash L * F\\
F * L' \vdash G' \\
F \text{ is stable w.r.t. } \MV(c) \\}
{G \vdash L * \Box^{\llbracket c \rrbracket} (L' -* G') }

\Rule{RAM-P-FRAME }
{G \vdash L * \Box^{\llbracket c \rrbracket} (L' -* G') \\
F \text{ is stable w.r.t. } \MV(c) \\}
{G * F \vdash L * \Box^{\llbracket c \rrbracket} (L' -* G' * F) }

\Rule{RAM-P-SPLIT }
{G_1 \vdash L_1 * \Box^{\llbracket c \rrbracket} (L_1' -* G_1') \\
G_2 \vdash L_2 * \Box^{\llbracket c \rrbracket} (L_2' -* G_2') \\}
{G_1 * G_2 \vdash L_1 * L_2 * \Box^{\llbracket c \rrbracket} (L_1' * L_2' -* G_1' * G_2') }

To conclude, if $L'$ and $G'$ are two separating conjunctions of a bunch of atomic predicates, RAM-P-FRAME and RAM-P-SPLIT can establish \ref{Passu} from entailments with the same form but smaller size. Atomic sized entailments can be proved using SOLVE-RAM-P. They are usually general purposed entailments and do not need to be proved for every single program. In section \ref{vst}, we will see examples of this approach for real programs.

\subsection{Q for Quantifiers}

In secion ???, we have already seen that it is a practical approach writing pre/postconditions as a separating conjunction of a list of atomic predicates (which makes RAM-P-FRAME and RAM-P-SPLIT useful). But unfortunately, an existential in post condition (also very common as we have seen in section ???) will prevent us from using these two rules. Now, one natural solution is to find other proof rules, like the following one, to deal with existential quantifiers.
\[\text{UNSOUND-RAM-Q-SPLIT}\]
\Rule{}
{G_1 \vdash L_1 * (\exists x, L_1' (x) -* \exists x, G_1'(x)) \\
G_2 \vdash L_2 * (\exists x, L_2' (x) -* \exists x, G_2'(x)) \\}
{G_1 * G_2 \vdash L_1 * L_2 * (\exists x, L_1'(x) * L_2'(x) -* \exists x, G_1'(x) * G_2'(x)) }

But this rule is NOT sound (even though we have not add $\Box$ operator to deal with local variable related stuff). The reason is that, given the local piece of memory satisfies $L_1'(x) * L_2'(x)$ for some specific $x$, we know that it can be split into two small piece of memory and they satisfies $L_1'(x)$ and $L_2'(x)$ respectively. Then the assumption tell us that the global piece can be split into two corresponding piece, $G_1'(x_1)$ and $G_2'(x_2)$ are true on them for some specific $x_1$ and $x_2$. Now the problem comes. Only if we could prove $x_1 = x_2$, we could prove the conclusion. But we cannot.

The key point of the failure above is that the frame, $\exists x, L' (x) -* \exists x, G'(x)$, says if $L'(x)$ is true on local then there is another (might be same one) $x_0$ such that $G'(x_0)$ is true on global. This is too weak for modularity. In many practical cases, we can in fact prove that $G'(x)$ should be true for the exact same $x$. This observation brings us to the ramification-PQ rule here.
\Rule{Ramification-PQ}
{\{ L \} c \{ \exists x, L' (x) \} \\
 G \vdash L * \Box^{\llbracket c \rrbracket} (\forall x, L' (x) -* G' (x)) \\}
{\{ G \} c \{ \exists x, G' (x)\}}

PQ rule can be directly derived from P rule by using the following theorem from separation logic\footnote{
$$\frac{\frac{\frac{\forall x, (L' (x) -* G' (x)) \vdash L' (x_0) -* G' (x_0)}{\forall x, (L' (x) -* G' (x)) * L' (x_0) \vdash G' (x_0)}}
{\forall x, (L' (x) -* G' (x)) * \exists x, L' (x) \vdash \exists x, G' (x)}}
{\forall x, (L' (x) -* G' (x)) \vdash \exists x, L' (x) -* \exists x, G' (x)}$$
}.
$$\forall x, (L' (x) -* G' (x)) \vdash \exists x, L' (x) -* \exists x, G' (x)$$
Like what we do to P rule, three corresponding rules, SOLVE-RAM-PQ, RAM-PQ-FRAME and RAM-PQ-SPLIT, are proved sound and can be used to establish the assumption of PQ rule in a modularized way. For those who do not care about local variable related issue, a ramification-Q rule can be used to deal with existentials. For the sake for space here, we omit them in this paper.

\subsection{Ramification in Decorated Programs}

One nice thing about Hoare logic is that it enables people to write combinational proofs. Moreover, such kind of proofs can be written in a nice printed form, decorated programs. %For example,
% \begin{figure}[h]
%\begin{tabular}{c | c}
%\begin{lstlisting}
%$\{\ \ \ P_1 \ \ \ \}$
%  c1;
%$\{\ \ \ P_2 \ \ \ \}$
%$\{\ \ \ P_3 \ \ \ \}$
%  c2;
%$\{\ \ \ P_4 \ \ \ \}$
%$\{\ \ \ P_5 \ \ \ \}$
%\end{lstlisting}
%&
%$$
%\inference[]
%{\triple{P_1}{c1}{P_2} &
%\inference[]
%{P_2 \vdash P_3 \\ P_4 \vdash P_5 \\ \triple{P_3}{c2}{P_4} }
%{\triple{P_2}{c2}{P_5}}
%}
%{\triple{P_1}{c1;c2}{P_5}}
%$$
% \\
%%TODO: fix format
%\end{tabular}
%\end{figure}

%The decorated program on the left is actually representing the Hoare logic proof on the right side.
By adding a new pattern, we call it localized and unlocalize, ramification proofs can also be presented in a decorated programs.

\begin{figure}[h]
\begin{tabular}{c | c}
\begin{lstlisting}
$\{\ \ \ G_1 \ \ \ \}$
$\searrow \{\ \ \ L_1 \ \ \ \}$
      c1;
      ...
      c5;
$\swarrow \{\ \ \ L_2 \ \ \ \}$
$\{\ \ \ G_2 \ \ \ \}$
\end{lstlisting}
&
$$
\inference[]
{\triple{L_1}{c1;...;c5}{L_2} \\
G_1 \vdash L_1 * (L_2 -* G_2)
}
{\triple{G_1}{c1;...;c5}{G_2}}
$$
 \\
%TODO: fix format
\end{tabular}
\caption{Localize and unlocalize in decorated programs}
\label{figure:lul}
\end{figure}

Figure \ref{figure:lul} shows such a decorated program. We call the action in line 2 \emph{localize} and call the action in line 6 \emph{unlocalize}. A Hoare logic proof using ramification rule can always be written as a decorated program with localize and unlocalize, as long as wherever we write do unlocalize action, we should prove a side condition, e.g. $G_1 \vdash L_1 * (L_2 -* G_2)$ in this example.
}

\section{Remaining proof of \infrulestyle{Ramify-PQ}}
\label{apx}

See figure \ref{fig:remainrampq}.

\begin{figure*}[t]
\[
\infrule{}
{
  L_1 |- L_1 \\
  \infrule{}
  {
    \infrule{}
    {
      \infrule{}
      {
        \infrule{}
        {
          \infrule{}
          {
            \infrule{}
            {
              \infrule{}
              {
                \infrule{}
                {
                  \infrule{}
                  {
                    \infrule{}
                    {
                      [x |-> x_0] (L_2 -* G_2) |- [x |-> x_0](L_2 -* G_2)
                    } {
                      \forall x.~ (L_2 -* G_2) |- [x |-> x_0](L_2 -* G_2)
                    } {\forall \mathsf{e}}
                  } {
                    \forall x.~ (L_2 -* G_2) |- ([x |-> x_0]L_2) -* ([x |-> x_0]G_2)
                  } {\textrm{substitute}}
                } {
                  \big(\forall x.~ (L_2 -* G_2)\big) * [x |-> x_0]L_2 |- [x |-> x_0]G_2
                } {(3)}
              } {
                \big(\forall x.~ (L_2 -* G_2)\big) * [x |-> x_0]L_2 |- \exists x.~ G_2
              } {\exists \mathsf{i}}
            } {
            \big(\forall x.~ (L_2 -* G_2)\big) * (\exists x.~ L_2) |- \exists x.~ G_2
            } {\exists \mathsf{e}}
          } {
            \forall x.~ (L_2 -* G_2) |- (\exists x.~ L_2) -* (\exists x.~ G_2)
          } {(3)}
        } {
          |- \big(\forall x.~ (L_2 -* G_2)\big) => \big((\exists x.~ L_2) -* (\exists x.~ G_2)\big)
        } {=> \mathsf{i}}
      } {
        |- \pguards{c}\Big(\big(\forall x.~ (L_2 -* G_2)\big) => \big((\exists x.~ L_2) -* (\exists x.~ G_2)\big)\Big)
      } {\mathsf{N}}
    } {
      |- \Big(\pguards{c}\big(\forall x.~ (L_2 -* G_2)\big) \Big) => \Big( \pguards{c}\big((\exists x.~ L_2) -* (\exists x.~ G_2)\big) \Big)
    } {\mathsf{K}}
  } {
    \pguards{c}\big(\forall x.~ (L_2 -* G_2)\big) |- \pguards{c}\big((\exists x.~ L_2) -* (\exists x.~ G_2)\big)
  } {\mathsf{i} =>}
} {
  L_1 * \pguards{c}\big(\forall x.~ (L_2 -* G_2)\big) |- L_1 * \pguards{c}\big((\exists x.~ L_2) -* (\exists x.~ G_2)\big)
} {* \textrm{ split} }
\]
\caption{Remaining proof of \infrulestyle{Ramify-PQ}}
\label{fig:remainrampq}
\end{figure*}

\subsection{More junk}
 precision helps enable the forward style
of reasoning used by HIP/SLEEK.  To use $\mu_{\mathsf{A}}$, our graph predicate
would have $\rhd$, \emph{i.e.}

%forall w w1 w2 : A, P w1 -> P w2 -> join_sub w1 w -> join_sub w2 w -> w1 = w2

$\rhd P$ is not , for any $P$.

  We do not believe
that this point has been observed before in the literature.  Precision is a standard technical
property that separation logic predicates can have

. Informally, a contractive
function is one such that if $\tau$ is approximately equal to
$\sigma$, then $F_p(\tau)$ is more accurately equal to
$F_p(\sigma)$.


  whose
mechanically verified its soundness. People can still define recursive
predicate $P$ through $F_p$ and $\mu_{\mathsf{R}}$, but this time the
$F_p$ needs to be

 The approximate equality is achieved by a data type as
a sequence of accurate approximations taken successively. This idea is
called step-indexing.

We attempted to formulate $\mathtt{graph}$ through fixed-point
functions $\mu_{\mathsf{T}}$ and $\mu_{\mathsf{R}}$. The contractive
functor $\mathtt{graphF}$ is defined as follows:
\[\label{eqn:graphFcotr}
  \begin{split}
  & \mathtt{graphF}(Q, x, \gamma)\defeq (x = 0 \wedge \mathtt{emp})
    \vee \\ & \exists d,l,r . \gamma(x)=(d,l,r) \wedge x \mapsto
    d,l,r\, \ocon \triangleright Q(l, \gamma) \ocon \triangleright
    Q(r, \gamma)
  \end{split}
\]
where $\triangleright$ is is the ``later'' operator which implements
the machinery of step-indexing. Note that $\mathtt{graphF}$ is a
normal predicate without recursion. $\mathtt{graph}$ is defined as
$\mu_{\mathsf{R}}\,\mathtt{graphF}$. One advantage of this definition
of $\mathtt{graph}$ is that proof by induction is possible because the
step-index can be seen as the inductive number. Unfortunately
$\mathtt{graph}$ is not \emph{precise} under this definition. For any
spatial predicate $P$, $\text{precise}(P)$ means whenever $P$ is
satisfied on a sub-state, that sub-state must be unique. Being precise
is a crucial requirement of $\mathtt{graph}$ for key theorems in our
framework. Further-more, it can be proved that for any predicate $P$,
$\triangleright P$ is not precise. So this defintion is abandoned.

Similarly we can define a covariant functor $\mathtt{graphQ}$ as
follows:
\[\label{eqn:graphFco}
  \begin{split}
  & \mathtt{graphQ}(Q, x, \gamma)\defeq (x = 0 \wedge
  \mathtt{emp}) \vee \\ & \exists d,l,r . \gamma(x)=(d,l,r) \wedge  x
  \mapsto d,l,r\, \ocon Q(l, \gamma) \ocon Q(r, \gamma)
  \end{split}
\]
The only difference between $\mathtt{graphQ}$ and $\mathtt{graphF}$ is
that $\mathtt{graphQ}$ does not have the $\triangleright$
operator. With this definition $\mathtt{graph}$ can be defined as
$\mu_{\mathsf{T}}\,\mathtt{graphQ}$. Again we need to prove the
preciseness of $\mathtt{graph}$. Since there is no induction principle
for this definition, we tried to prove it through the following lemma:
\begin{equation}\label{eqn:graph_iter}
\mathtt{graph}(x, \gamma) \dashv\vdash
\underset{v\in\mathit{reach}(\gamma, x)}{\bigstar} v\mapsto\gamma(v)
\end{equation}
where $\mathit{reach}(\gamma, x)$ is the set of nodes reachable from
$x$ in $\gamma$ and 




\begin{figure*}
  \begin{lstlisting}
struct Node {
    int m;
    struct Node * l;
    struct Node * r; };

// We use $R$ to represent $\p{reachable}(\gamma,\tx x)$

void spanning(struct Node * x) { // $\{\p{graph}(\tx{x},\gamma)/|\gamma(\tx{x}).1=0\}$
    struct Node * l, * r; int root_mark;
// $\{\p{graph}(\tx x,\gamma) /| \exists l,r.~ \gamma(\tx{x}) = (0,l,r)\}$
// $\{\p{graph}(\tx x,\gamma) /| \gamma(\tx{x}) = (0,l,r)\}$
// $\{\p{vertices\_at}(\p{reachable}(\gamma,\tx x), \gamma) /| \gamma(\tx{x}) = (0,l,r)\}$
// $\{\p{vertices\_at}(R, \gamma) /| \gamma(\tx{x}) = (0,l,r)\}$
// $\searrow \{\tx x|-> 0,l,r /| \gamma(\tx{x}) = (0,l,r)\}$
    l = x -> l;
    r = x -> r;
    x -> m = 1;
// $\swarrow \{\tx x|-> 1,\tx{l},\tx{r} /| \gamma(\tx{x}) = (0,\tx{l},\tx{r}) /| \exists \gamma_1.~ \m{mark1}(\gamma, \tx{x}, \gamma_1)\}$
// $\{\exists \gamma_1.~\p{vertices\_at}(R, \gamma_1) /| \gamma(\tx{x}) = (0,\tx{l},\tx{r}) /| \m{mark1}(\gamma, \tx{x}, \gamma_1)\}$
// $\{\p{vertices\_at}(R, \gamma_1) /| \gamma(\tx{x}) = (0,\tx{l},\tx{r}) /| \m{mark1}(\gamma, \tx{x}, \gamma_1)\}$
    if (l) {
//   $\{\p{vertices\_at}(R, \gamma_1) /| \gamma(\tx{x}) = (0,\tx{l},\tx{r}) /| \exists m_2, l_2, r_2.~\gamma_1(\tx{l})=(m_2, l_2, r_2) /| \m{mark1}(\gamma, \tx{x}, \gamma_1)\}$
//   $\{\p{vertices\_at}(R, \gamma_1) /| \gamma(\tx{x}) = (0,\tx{l},\tx{r}) /| \gamma_1(\tx{l})=(m_2, l_2, r_2) /| \m{mark1}(\gamma, \tx{x}, \gamma_1)\}$
//   $\searrow\{\tx{l} |-> m_2, -, l_2, r_2\}$
        root_mark = l -> m;
//   $\swarrow\{\tx{l} |-> m_2, -, l_2, r_2/|m_2 = \tx{root\_mark}\}$
//   $\{\p{vertices\_at}(R, \gamma_1)/| \gamma(\tx{x}) = (0,\tx{l},\tx{r})/|\gamma_1(\tx{l})=(m_2, l_2, r_2)/|m_2 = \tx{root\_mark} /| \m{mark1}(\gamma, \tx{x}, \gamma_1)\}$
        if (root_mark == 0) {
//     $\{\p{vertices\_at}(R, \gamma_1)/|\gamma(\tx{x}) = (0,\tx{l},\tx{r})/|\gamma_1(\tx{l})=(0, l_2, r_2) /| \m{mark1}(\gamma, \tx{x}, \gamma_1)\}$
//     $\searrow\{\p{graph}(\tx{l}, \gamma_1)/|\gamma_1(\tx{l})=(0, l_2, r_2)\}$
            spanning(l);
//     $\swarrow\{\exists \gamma_2. ~\p{vertices\_at}(\p{reachable}(\gamma_1,\tx l), \gamma_2)/|\gamma_1(\tx{l})=(0, l_2, r_2)/|\m{span}(\gamma_1,\tx{l},\gamma_2)\}$
//     $\{\exists \gamma_2. ~\p{vertices\_at}(R, \gamma_2)/|\gamma(\tx{x}) = (0,\tx{l},\tx{r}) /|\gamma_1(\tx{l})=(0, l_2, r_2) /| \m{mark1}(\gamma, \tx{x}, \gamma_1) /|\m{span}(\gamma_1,\tx{l},\gamma_2)\}$
        } else {
//     $\{\p{vertices\_at}(R, \gamma_1)/| \gamma(\tx{x}) = (0,\tx{l},\tx{r})/|\gamma_1(\tx{l})=(1, l_2, r_2)  /| \m{mark1}(\gamma, \tx{x}, \gamma_1)\}$
//     $\searrow \{\tx x|-> 0,\tx{l},\tx{r} /| \gamma(\tx{x}) = (0,\tx{l},\tx{r})\}$
            x -> l = 0;
//     $\swarrow \{\tx x|-> 0,0,\tx{r} /| \gamma(\tx{x}) = (0,\tx{l},\tx{r})\}$
//     $\{\exists \gamma_2. ~\p{vertices\_at}(R, \gamma_2)/|\gamma(\tx{x}) = (0,\tx{l},\tx{r})/|\gamma_1(\tx{l})=(1, l_2, r_2) /| \m{mark1}(\gamma, \tx{x}, \gamma_1) /| \m{e\_rm}(\gamma_1, \tx{x}.\text{L}, \gamma_2)\}$
        }
//   $\{\exists\gamma_2.~\p{vertices\_at}(R,\gamma_2)/| \gamma(\tx{x}) = (0,\tx{l},\tx{r})  /| \m{mark1}(\gamma, \tx{x}, \gamma_1) /| \m{e\_span}(\gamma_1,\tx{x}.\text{L},\gamma_2)\}$
    }
    else {
//   $\{\p{vertices\_at}(R, \gamma_1) /| \gamma(\tx{x}) = (0,\tx{l},\tx{r}) /| \tx{l}= 0  /| \m{mark1}(\gamma, \tx{x}, \gamma_1)\}$
        skip;
//   $\{\exists\gamma_2. ~\p{vertices\_at}(R,\gamma_2)/| \gamma(\tx{x}) = (0,\tx{l},\tx{r})  /| \m{mark1}(\gamma, \tx{x}, \gamma_1) /| \m{e\_span}(\gamma_1,\tx{x}.\text{L},\gamma_2)\}$
    }
// $\{\exists\gamma_2. ~\p{vertices\_at}(R,\gamma_2)/| \gamma(\tx{x}) = (0,\tx{l},\tx{r})  /| \m{mark1}(\gamma, \tx{x}, \gamma_1) /| \m{e\_span}(\gamma_1,\tx{x}.\text{L},\gamma_2)\}$
// $\{\p{vertices\_at}(R,\gamma_2)/| \gamma(\tx{x}) = (0,\tx{l},\tx{r})  /| \m{mark1}(\gamma, \tx{x}, \gamma_1) /|  \m{e\_span}(\gamma_1,\tx{x}.\text{L},\gamma_2)\}$
    if (r) {
//   $\{\p{vertices\_at}(R, \gamma_2) /| \gamma(\tx{x}) = (0,\tx{l},\tx{r}) /| \exists m_2, l_2, r_2.~\gamma_1(\tx{l})=(m_2, l_2, r_2) /| \m{mark1}(\gamma, \tx{x}, \gamma_1)/| \m{e\_span}(\gamma_1,\tx{x}.\text{L},\gamma_2)\}$
//   $\{\p{vertices\_at}(R, \gamma_2) /| \gamma(\tx{x}) = (0,\tx{l},\tx{r}) /| \gamma_2(\tx{r})=(m_2, l_2, r_2) /| \m{mark1}(\gamma, \tx{x}, \gamma_1)/| \m{e\_span}(\gamma_1,\tx{x}.\text{L},\gamma_2)\}$
//   $\searrow\{\tx{r} |-> m_2, -, l_2, r_2\}$
        root_mark = r -> m;
//   $\swarrow\{\tx{r} |-> m_2, -, l_2, r_2/|m_2 = \tx{root\_mark}\}$
//   $\{\p{vertices\_at}(R, \gamma_2)/| \gamma(\tx{x}) = (0,\tx{l},\tx{r})/|\gamma_2(\tx{r})=(m_2, l_2, r_2)/|m_2 = \tx{root\_mark} /| \m{mark1}(\gamma, \tx{x}, \gamma_1)/| \m{e\_span}(\gamma_1,\tx{x}.\text{L},\gamma_2)\}$
        if (root_mark == 0) {
//     $\{\p{vertices\_at}(R, \gamma_2)/|\gamma(\tx{x}) = (0,\tx{l},\tx{r})/|\gamma_2(\tx{r})=(0, l_2, r_2) /| \m{mark1}(\gamma, \tx{x}, \gamma_1)/| \m{e\_span}(\gamma_1,\tx{x}.\text{L},\gamma_2)\}$
//     $\searrow\{\p{graph}(\tx{r}, \gamma_2)/|\gamma_2(\tx{r})=(0, l_2, r_2)\}$
           spanning(r);
//     $\swarrow\{\exists \gamma_3. ~\p{vertices\_at}(\p{reachable}(\gamma_2,\tx r), \gamma_3)/|\gamma_2(\tx{r})=(0, l_2, r_2)/|\m{span}(\gamma_2,\tx{r},\gamma_3)\}$
//     $\{\exists \gamma_3. ~\p{vertices\_at}(R, \gamma_3)/|\gamma(\tx{x}) = (0,\tx{l},\tx{r}) /|\gamma_2(\tx{r})=(0, l_2, r_2) /| \m{mark1}(\gamma, \tx{x}, \gamma_1) /|\m{span}(\gamma_1,\tx{l},\gamma_2)/|\m{span}(\gamma_2,\tx{r},\gamma_3)\}$
        } else {
//     $\{\p{vertices\_at}(R, \gamma_2)/| \gamma(\tx{x}) = (0,\tx{l},\tx{r})/|\gamma_2(\tx{r})=(1, l_2, r_2)  /| \m{mark1}(\gamma, \tx{x}, \gamma_1) /| \m{e\_span}(\gamma_1,\tx{x}.\text{L},\gamma_2)\}$
//     $\searrow \{\tx x|-> 0,?l,\tx{r} /| \gamma(\tx{x}) = (0,?l,\tx{r})\}$
            x -> r = 0;
//     $\swarrow \{\tx x|-> 0,?l,\tx{r} /| \gamma(\tx{x}) = (0,?l,\tx{r})\}$
//     $\{\exists \gamma_3. ~\p{vertices\_at}(R, \gamma_3)/|\gamma(\tx{x}) = (0,\tx{l},\tx{r})/|\gamma_2(\tx{r})=(1, l_2, r_2) /| \m{mark1}(\gamma, \tx{x}, \gamma_1) /| \m{e\_span}(\gamma_1,\tx{x}.\text{L},\gamma_2) /| \m{e\_rm}(\gamma_2, \tx{x}.\text{R}, \gamma_3)\}$
        }
//   $\{\exists\gamma_3.~\p{vertices\_at}(R,\gamma_3)/| \gamma(\tx{x}) = (0,\tx{l},\tx{r})  /| \m{mark1}(\gamma, \tx{x}, \gamma_1) /| \m{e\_span}(\gamma_1,\tx{x}.\text{L},\gamma_2) /| \m{e\_span}(\gamma_2,\tx{x}.\text{R},\gamma_3)\}$
    }
    else {
//   $\{\p{vertices\_at}(R, \gamma_2) /| \gamma(\tx{x}) = (0,\tx{l},\tx{r}) /| \tx{r}= 0  /| \m{mark1}(\gamma, \tx{x}, \gamma_1)  /| \m{e\_span}(\gamma_1,\tx{x}.\text{L},\gamma_2)\}$
        skip;
//   $\{\exists\gamma_3.~\p{vertices\_at}(R,\gamma_3)/| \gamma(\tx{x}) = (0,\tx{l},\tx{r})  /| \m{mark1}(\gamma, \tx{x}, \gamma_1) /| \m{e\_span}(\gamma_1,\tx{x}.\text{L},\gamma_2) /| \m{e\_span}(\gamma_2,\tx{x}.\text{R},\gamma_3)\}$
    }
//   $\{\exists\gamma_3.~\p{vertices\_at}(R,\gamma_3)/| \gamma(\tx{x}) = (0,\tx{l},\tx{r})  /| \m{mark1}(\gamma, \tx{x}, \gamma_1) /| \m{e\_span}(\gamma_1,\tx{x}.\text{L},\gamma_2) /| \m{e\_span}(\gamma_2,\tx{x}.\text{R},\gamma_3)\}$
} // $\{\exists \gamma_3.~\p{vertex\_at}(\p{reachable}(\gamma, \tx{x}), \gamma_3)/|\m{span}(\gamma,\tx{x},\gamma_3)\}$
\end{lstlisting}

\caption{Clight code and proof sketch for bigraph spanning tree.}
\label{fig:spanning-full}

\end{figure*}

\end{document}
