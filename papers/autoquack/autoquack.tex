%
% LaTeX template for prepartion of submissions to PLDI'15
%
% Requires temporary version of sigplanconf style file provided on
% PLDI'15 web site.
%
\documentclass[pldi]{sigplanconf-pldi15}

%
% the following standard packages may be helpful, but are not required
%

\usepackage{mathpartir}
\usepackage{amsmath}
\usepackage{ amssymb }
\usepackage{wasysym}
\usepackage{stmaryrd}
\usepackage{SIunits}            % typset units correctly
\usepackage{courier}            % standard fixed width font
\usepackage[scaled]{helvet} % see www.ctan.org/get/macros/latex/required/psnfss/psnfss2e.pdf
\usepackage{url}                  % format URLs
\usepackage{listings}          % format code
\usepackage{enumitem}      % adjust spacing in enums
\usepackage[colorlinks=true,allcolors=blue,breaklinks,draft=false]{hyperref}   % hyperlinks, including DOIs and URLs in bibliography
% known bug: http://tex.stackexchange.com/questions/1522/pdfendlink-ended-up-in-different-nesting-level-than-pdfstartlink
\usepackage{tikz}
\usepackage{fancybox}
\usepackage{multicol}
\usetikzlibrary{arrows.meta}
\usepackage{semantic}
\newcommand{\triple}[3]{\{#1\}\,#2\,\{#3\}}
\newcommand{\doi}[1]{doi:~\href{http://dx.doi.org/#1}{\Hurl{#1}}}   % print a hyperlinked DOI

\lstset{language=C,basicstyle=\small,mathescape=true,columns=fullflexible}

\begin{document}

%
% any author declaration will be ignored  when using 'plid' option (for double blind review)
%

\title{The Ramifications of Mechanizing Verification}
\authorinfo{Shengyi Wang$^{*}$ \qquad Qinxiang Cao$^{+}$ \qquad Asankhaya Sharma$^{*}$ \qquad Aquinas Hobor$^{\dagger,*}$}
{}
{School of Computing$^{*}$ and Yale-NUS College$^{\dagger}$, National University of Singapore; Princeton University$^{+}$}

\maketitle
\begin{abstract}
Blah blah blah
\end{abstract}

\newcommand\hide[1]{}

\section{Introduction}
Over the last fifteen years great strides have been made in automating verifications of programs that manipulate
tree-like data structures using separation logic CITE CITE CITE.  Unfortunately, verifying programs that manipulate
graph-like data structures (e.g. structures with \emph{intrinsic sharing}) has been far more challenging.
Indeed, verifying such programs was formidable enough that a number of the early landmark results in separation logic
devoted substantial efforts to verify single examples such as Schorr-Waite \cite{hongseok:phd} and XXX CITE with pen and
paper---avoiding entirely the additional challenges inherent in machine-assisted reasoning.

In recent years, Hobor and Villard introduced the concept of \emph{ramification} as a kind of proof pattern or framework
to verify graph-manipulating programs with pen and paper \cite{hobor:ramification}, but left open the question of how such proofs could
be incorporated in a machine-assisted setting.  In this paper, we show how this can be done, and demonstrate the
value of our approach by adding ramification to two rather sizeable---albeit quite differently flavored---separation logic-based
verification tools: the Coq-based tactic system of the Verified Software Toolchain CITE and the more highly-automated HIP/SLEEK
program verifier \cite{chin:hipsleek}.  Despite the substantial differences between these systems, the vast majority of our infrastructure is
shared between them, and since many of the other computer-assisted verification tools under development
today CITE CITE CITE CITE have much in common with at least one of these tools, we believe that our techniques will be
applicable to many other systems as well.



Along the way we discover---and show how to fix---a rather subtle error in Hobor and Villard's presentation: neither the
Knaster-Tarski \cite{tarski:fixpoint} nor the Appel-McAllester \cite{appel:fixpoint} method for solving recursive fixpoints is suitable for defining
recursive graph predicates in separation logic.

  We also develop a general framework for defining and reasoning about mathematical
graphs and

different kinds of mathematical
graphs can be implemented in separation logic in a uniform way;

generalize their setting
so that it can handle a wider variety of data structures;

We use both systems to verify a number of different programs utilizing graph-manipulating structures,
letting us understand the advantages and disadvantages of both.

\section{A framework for graph theory}
To enable the verification of full functional correctness of graph
algorithms we need a way to reason about mathematical graphs.
To allow such verifications to be mechanized without undue pain
we must take care to develop a modular and general-purpose 
framework for such mathematical graphs.

\subsection{Structure of the mathematical graph framework}\label{sec:mathinfra}

\begin{figure}[t]
\centering
\beginpgfgraphicnamed{variousgraph}
\begin{tikzpicture}
[->/.style={thick,arrows={-Stealth}},
-->/.style={thick,arrows={-Stealth}, decorate, decoration={snake, amplitude=.4mm,segment length=2mm,post length=2mm}},
   realG/.style={shape=rectangle, rounded corners=4pt, draw, fill=gray!40},
   propG/.style={shape=rectangle, rounded corners=4pt, draw}]
\node[realG] (PG) at (0, 0) {\small PreGraph};
\node[realG] (LG) [right=0.8 of PG] {\small LabeledGraph};
\node[realG] (GG) [right=2 of LG] {\small GeneralGraph};
\draw [double, ->] (PG) -- (LG) node [pos=0.5, above] {\small Label} ;
\draw [double, ->] (LG) -- (GG) node (SC) [pos=0.5, above, align=center]
{\small Soundness \\ \small Condition};
\node[propG] (Prop) [below=0.6 of SC] {\small Property};
\node[propG] (PropL) [below=0.4 of Prop] {\small Property Lemmas};
\node[propG] (PGL) [below=2 of PG, align=center] {\small PreGraph \\\small Lemmas};
\node[propG] (LGL) [below=2 of LG, align=center] {\small LabeledGraph \\\small Lemmas};
\node[propG] (GGL) [below=2 of GG, align=center] {\small GeneralGraph \\\small Lemmas};
\draw [double, ->] (PGL) to (LGL);
%% \draw [double, ->] (LGL) to (GGL);
\draw [->] (PG) to (PGL);
\draw [->] (Prop) to (PropL);
\draw [-->] (Prop) to (SC);
\coordinate [left=0.2 of LG.south] (LGs1);
\coordinate [left=0.2 of LGL.north] (LGLn1);
\draw [->] (LGs1) to (LGLn1);
\coordinate [right=0.2 of LG.south] (LGs2);
\coordinate [right=0.2 of LGL.north] (LGLn2);
\draw [->] (LGs2) |- (Prop);
\draw [double, ->] (LGLn2) |- (PropL);
\coordinate [right=0.2 of GG.south] (GGs);
\coordinate [left=0.2 of GGL.north] (GGLn1);
\coordinate [right=0.2 of GGL.north] (GGLn2);
\draw [double, ->] (PropL) -| (GGLn1);
\draw [->] (GGs) to (GGLn2);
\node [draw, thick, rectangle, dashed, fit=(Prop) (PropL)] {};
\node (legend1) [below right=0.2 and -0.3 of PGL] {\small Dependence};
\coordinate[left=0.8 of legend1]  (l1);
\draw [->] (l1) to (legend1);
\node (legend2) [right=1 of legend1] {\small Inheritance};
\coordinate[left=0.8 of legend2]  (l2);
\draw [double, ->] (l2) to (legend2);
\node (legend3) [right=1 of legend2] {\small Instantialize};
\coordinate[left=0.8 of legend3]  (l3);
\draw [-->] (l3) to (legend3);
\end{tikzpicture}
\endpgfgraphicnamed
\vspace{1ex}
\caption{Structure of the Mathematical Graph Library}\label{fig:graphs}
\end{figure}

\begin{figure}[t]
\centering
\beginpgfgraphicnamed{pregraphexp}
\begin{tikzpicture}
[vad/.style={circle, fill=black, inner sep=0pt, minimum size=4pt},
 inv/.style={circle, draw=black, thick, inner sep=0pt, minimum size=4pt},
 ->/.style={thick, arrows={-Stealth}}]
\node[vad] (n1) at (0, 0) {};
\node[vad] (n2) at (1, 1) {};
\node[inv] (n3) at (1, -1) {};
\node[vad] (n4) at (-2,2) {};
\node[vad] (n5) at (-2,-2) {};
\node[vad] (n6) at (-3,0) {};
\node[vad] (n7) at (3,1.5) {};
\node[inv] (n8) at (3,-1.5) {};
\node[vad] (n9) at (3.5, 1) {};
\node[inv] (n10) at (3.5, 0.5) {};
\node at (3.5, 1) [right=1.5pt] {\small Valid node};
\node at (3.5, 0.5) [right=1.5pt] {\small Invalid node};
\node at (3.5, 0) [right=1.5pt] {\small Valid edge};
\node at (3.5, -0.5) [right=1.5pt] {\small Invalid edge};
\draw[->] (n1) to (n2);
\draw[->,dashed] (n1) to (n3);
\draw[->,dashed] (n3) to (n5);
\draw[->] (n2) to (n3);
\draw[->] (n2) to (n7);
\draw[->] (n2) to (n8);
\draw[->,dashed] (n3) to (n8);
\draw[->] (n4) to (n1);
\draw[->] (n1) to (n5);
\draw[->] (n2) to (n4);
\draw[->] (n1) to [bend left=20] (n6);
\draw[->] (n6) to (n5);
\draw[->] (n8) to [bend left=20] (n7);
\draw[->] (n4) to [bend right=35] (n1);
\draw[->] (3.0, 0) -- (3.6, 0);
\draw[->,dashed] (3.0, -0.5) -- (3.6, -0.5);
\end{tikzpicture}
\endpgfgraphicnamed
\vspace{1ex}
\caption{A PreGraph with invalid nodes and edges.}\label{fig:pregraph}
\end{figure}

Figure~\ref{fig:graphs} gives the overall architecture of how our graphs are constructed.
The most basic kind of graph is PreGraph, out of which we build LabeledGraphs, and which in turn are used
to build GeneralGraphs.  Each kind of graph has some associated lemmas, and each kind inherits the lemmas of the previous kind.  The dashed box represents a ``plugin'' system for attaching arbitrary properties to LabeledGraphs and will be discussed more later. %We will consider each in turn.

\paragraph{Pregraphs.} A PreGraph is a hextuple $(V, E, \phi_V, \phi_E, s, t)$,
where $V$ and $E$ are the underlying carrier set of vertices and edges.  Not every $v \in V$ or $e \in E$ is actually ``in'' the graph, so we provide the predicates $\phi_V$ and $\phi_E$ to classify vertices and edges as \emph{valid} (in) or not (out).  Finally, $s$ and $e : E -> V$ are functions that map an edges to their source and destination respectively; this model means that PreGraphs are directed rather than undirected.  By design, there are no requirements for \emph{e.g.} how the validities of edges and vertices relate.  As shown in Figure \ref{fig:pregraph}, a PreGraph can contain invalid nodes and edges in an arbitrary configuration.

The advantage of designing a graph type that can reason about missing vertices and edges is because some of our later definitions need such flexibility.  Consider the difference of two graphs, $\gamma_1 - \gamma_2$.  Even if both of these graphs are ``well-formed'' to begin with, in the sense that valid nodes have only valid edges and vice versa, their difference may not since there may be dangling edges pointing to the now-removed vertices of $\gamma_2$.

Many basic graph concepts such as \emph{path}, \emph{reachability}, and \emph{subgraph} are defined on PreGraphs.
Informally a path is a list of nodes connected by edges.  Formally it is more convenient to
define a path as an ordered pair $(n,l)$ where $n$ is a node and $l$ is a list of edges.
A valid path requires $n$ to be the source of the first edge of $l$ (if one exists) and moreover
requires $l$ to be ``well chained''.  That is, the destination of one edge in $l$ must be the 
source of the next edge.  The list $l$ can be null to represent an empty path starting and ending
at the node $n$.  We prefer this encoding as opposed to some others (\emph{e.g.} a list of edges)
because the definitions of important concepts like reachability are cleaner.

The most important definition on PreGraph is the concept of reachability.
We use the notation $\gamma \models L_{n_2}^{n_1}(P)$ to mean that we can reach $n_2$ from $n_1$ via the path $L$ in PreGraph $\gamma$, and moreover that every node in $L$ satisfies predicate $P$.  This notation, along
with some derived ones such as
\begin{equation*}
\begin{split}
\gamma\models n_1 \xrightarrow{P} n_2 &\defeq \exists L, \gamma \models L_{n_2}^{n_1}(P),\\
\gamma\models n_1 \leadsto n_2 &\defeq \exists L, \gamma \models L_{n_2}^{n_1}(True)
\end{split}
\end{equation*}
form the bedrock of nearly every nontrivial predicate about and relation between
graphs.
We write $\p{reachable}(\gamma,v)$ to mean the set of vertices reachable in $\gamma$ from a given vertex $v$.

In \S\ref{sec:spacegraph} we will tie mathematical graphs $\gamma$ to a spatial graph predicate
$\p{graph}(x, \gamma)$.   As we will see, $\p{graph}$ ``owns'' only the
spatial portion of $\gamma$ that is reachable
from $x$ even though $\gamma$ may contain other nodes.  Accordingly, when we reason about
$\p{graph}(x, \gamma)$, it is natural to want to describe the
reachable portion of $\gamma$.  In fact we generalize this
idea into two concepts: the subgraph of $\gamma$ satisfying
an arbitrary predicate $P$, written $\gamma \!\downarrow\! P$, and the \emph{relaxed subgraph}, written $\gamma \!\uparrow\! P$, which contains the all of the subgraph plus some additional edges.  In particular, $\gamma \!\downarrow\! P$ contains exactly the vertices satisfying $P$ and only the edges whose source and destination both satisfy $P$.  The relaxed subgraph $\gamma \!\uparrow\! P$ adds the additional edges whose \textbf{source} satisfies $P$, even though their destination may not.
We can use these definitions to, for example, extract the subgraph or relaxed subgraph reachable from a vertex $v$ by writing \emph{e.g.}
\[
\gamma \!\downarrow\! (\lambda v'. \gamma\models v \leadsto v')
\]

\paragraph{LabeledGraph.} 
PreGraph and its derived properties (reachability, subgraph, etc) are
inadequate for real program verification, even though many basic lemmas can already be proved about them.
However, when reasoning about the concrete graphs manipulated by various algorithms, 
we usually need to add a notion of \emph{labels} on vertices and/or edges, such as
the ``mark bit'' used in Figure~\ref{fig:markgraph}.

\paragraph{GeneralGraph.}
Much more interesting is the concept of GeneralGraph, which augments a LabeledGraph by adding a user-specified soundness condition.  In Figure~\ref{fig:graphs} this soundness condition is highlighted by a dashed border.  These ``plugins'' can specify many different kinds of properties.  Each property, in turn, can be used to prove many property-specific lemmas, all of which then apply to the instantiating GeneralGraph.

\subsection{Graph plugins}

Many theorems about graphs require certain properties, such as:
\begin{itemize}
\item A graph may be finite (\p{FiniteGraph}), meaning that both the set of valid vertices and the set of valid edges are finite.
\item A less restrictive property is \emph{locally finite} (\p{LocalFiniteGraph}), in which each vertex has a finite number of neighbors.  Locally finite graphs are useful, for example, when we wish to reason about algorithms that process vertices by cycling through neighbors, such as breadth-first search.
\item More subtly, consider that many real data structures use special null values to represent unused edges.  The \p{MathGraph} property introduces this concept---\emph{i.e.} some invalid nodes that are none-the-less allowed to appear as destinations for valid edges.
\item Many of our verified algorithms have only two outgoing edges per node.  The \p{BiGraph} property lets us reason about this common special case in a convenient manner.
\end{itemize}
We have a number of other properties in our codebase, but these four are the most important basic ones.

\begin{figure}[t]
\centering
\beginpgfgraphicnamed{graphproperty}
\begin{tikzpicture}
[->/.style={thick,arrows={-Stealth}},
   group/.style={shape=rectangle, draw, thick, dashed},
   propG/.style={shape=rectangle, rounded corners=4pt, draw}]
\node[propG] (PL12) at (0, 0) {\small Lemmas of Property 1 and 2};
\coordinate [left=1 of PL12.north] (PL12n1);
\coordinate [right=1 of PL12.north] (PL12n2);
\node[propG] (PL1) [above=0.5 of PL12n1, align=center] {\small Property 1 \\\small Lemmas};
\node[propG] (PL2) [above=0.5 of PL12n2, align=center] {\small Property 2 \\\small Lemms};
\node[propG] (P1) [above=0.5 of PL1] {\small Property 1};
\node[propG] (P2) [above=0.5 of PL2] {\small Property 2};
\draw [->] (P1) to (PL1);
\draw [->] (P2) to (PL2);
\draw [double, ->] (PL1) to (PL12n1);
\draw [double, ->] (PL2) to (PL12n2);
\draw [->] (P1.west) to [bend right=45] (PL12.west);
\draw [->] (P2.east) to [bend left=45] (PL12.east);
\node (R1) [group, fit=(P1) (PL1)] {};
\node (R2) [group, fit=(P2) (PL2)] {};
\node (R3) [group, fit=(current bounding box)] {};
\node [propG] (P1P2) [above right=-1.1 and 1.2 of R3, align=left] {\small Property 1$/|$ \\\small Property 2};
\node [propG] (PL1PL2) [below=0.5 of P1P2, align=center] {\small Prop. 1 Lemmas \\\small Prop. 2 Lemmas \\\small Prop. 1$/|$2 Lemmas};
\draw [->] (P1P2) to (PL1PL2);
\node (R4) [group, fit=(P1P2) (PL1PL2)] {};
\node (EQ) [right=0 of R3] {\bf \Large $\mapsto$};
\end{tikzpicture}
\endpgfgraphicnamed
\vspace{1ex}
\caption{Combining plugins}\label{fig:properties}
\end{figure}

We use Coq's typeclass system to manage our plugins in a smooth manner.  Essentially the typeclass system enables the diagram in Figure~\ref{fig:properties}.  The idea is that if we have two properties, each of which come with some already-proved lemmas, we can combine these plugins to prove the emergent lemmas that result from the combination, and then treat the new combination as a new plugin.  Since the system is compositional, we can easily mix many different properties together.  For example, we compose \p{BiGraph}, \p{MathGraph}, and \p{FiniteGraph} together into a new plugin we call \p{BiMaFin}.  \p{BiMaFin} is the actual soundness condition used to verify the program in Figure~\ref{fig:markgraph}.

One interesting example of this process is the following lemma:
\newtheorem{mylem}{Lemma}
\begin{mylem}[Computable reachability]\label{lem:computereach}
\[
\begin{split}
\forall \gamma, x.~&\p{MathGraph}\ \gamma => \p{LocalFiniteGraph}\ \gamma =>\\
        & \p{reachable}(\gamma, x)\textrm{ is finite} => \\
        & \text{we can \textbf{compute} a set }S \text{ s.t.\ } \\
        & \quad \forall v.~v \in S <=> \gamma \models x \leadsto v.
\end{split}
\]
\end{mylem}
The proof of this lemma is a bit subtle.  To compute the reachable set
we need to design a decision procedure that explores the (potentially cyclic)
graph.  Accordingly we implement a flavor of breadth-first search, also keeping
track of the nodes that we have explored previously.  Since nodes are explored
in BFS order while reachability is in some sense defined in DFS order, when
we reach a node $n$ we must take some care to reconstruct a path from the origin to $n$
to satisfy the reachability property.  The definition is also
painful because of Coq's insistence that all computations terminate; we 
utilized the techniques outlined by Chlipala to pacify Coq's termination checker~\cite{chlipala:cpdt}.
 
\subsection{Reasoning about relations between graphs} %Using our framework to reasonApplication of the framework}

We apply our framework to define other structures and relations
between these structures (including graph), and then use them to prove
certain pure facts proposed by real verifications.

For example, we define DAG (directed acyclic graph) as a PreGraph with
an additional property: forall any $x$ and $y$, if $x$ is reachable
from $y$, then $x = y$ or $y$ is not reachable from $x$.  Similarly, we define tree 
by saying that for any reachable node $n$ there is a unique path from the root
to $n$.

We already defined the relation $\m{mark}(\gamma, \tx x, \gamma')$, used in 
the graph marking algorithm, in Figure~\ref{fig:markgraph}.  
Similarly, we define $\m{span}$ for the spanning tree program
and $\m{copy}$ for the graph copy program.  These relations all
capture how the graph has changed from before to after the program 
execution.  As previously mentioned, we reuse $\m{mark}$ and its
related lemmas to prove facts about spanning tree and graph copy 
because the latter two programs mark nodes as they work. 
Accordingly, we can reuse the following fact:
\begin{equation*}
\begin{split}
\forall \gamma, x, n.~&\gamma(x)=(0, v_1, v_2,\dots,v_n) => \m{mark1}(\gamma, x, \gamma_1) =>\\
        & \m{mark}(\gamma_1,v_1,\gamma_2) => \m{mark}(\gamma_2,v_2,\gamma_3) => \dots =>\\
        & \m{mark}(\gamma_n,v_n,\gamma_{n+1}) => \m{mark}(\gamma,x,\gamma_{n+1}).
\end{split}
\end{equation*}
This is a general theorem for any \p{LocalFiniteGraph}, not just
\p{BiGraph}. In our framework, we aspire to prove theorems as generally as possible.



\section{Defining and reasoning about spatial graphs}

2.3. \texttt{sSpatialGraph\_Graph\_Bi} defined in "\texttt{spatial\_graph\_bi.v}" is the premise type class for defining how a BiMaFin graph is stored in memory.

2.5. Heap-Model-Direct instances are broken now. They are not hard to get fixed. Previous proofs are in "\texttt{spatial\_graph\_HMD.v}".

\subsection{Traditional fixpoints fail}

Hobor and Villard defined the separation logic graph predicate $\mathsf{graph}(x,\gamma)$ in direct analogy to the standard
separation logic definition of a tree as follows:
\[
\begin{array}{l}
\mathsf{graph}(x,\gamma) ~~ \stackrel{\Delta}{=} ~~ ({x = 0} \wedge \mathsf{emp}) \vee \null \\
~~~~ (\exists d,l,r.~ x \mapsto d,l,r, ** \mathsf{graph}(l,\gamma) ** \mathsf{graph}(r,\gamma))
\end{array}
\]

%* Showing that neither traditional fixpoint method works

Recursive/inductive predicates are ubiquitous in separation logic---so much so that when a person writes the definition of a
predicate as $P \stackrel{\textrm{``}\Delta\textrm{''}}{=} \ldots P \ldots$ no one raises an eyebrow, despite the dangers of circularity in
mathematics.  Indeed, 95\% of the time there is no danger thanks to the magic of the Knaster/Tarski fixpoint $\mu_{\mathsf{T}}$ \cite{tarski:fixpoint}.
Formally what is going on is instead of defining $P$ directly, one defines a functional
$F_P \stackrel{\Delta}{=} \lambda P.~ \ldots P \ldots$ and then defines $P$ itself as $P \stackrel{\Delta}{=} \mu_{\mathsf{T}} \, F_P$.
Assuming (as one typically does without comment) that $F_P$ is \emph{covariant}, i.e. $(P \vdash Q) \Rightarrow (F \, P \vdash F \, Q)$,
one then enjoys the fixpoint equation
$P \Leftrightarrow \ldots P \ldots$, formally justifying typically written pseudodefinition ($\stackrel{\textrm{``}\Delta\textrm{''}}{=}$).





Appel and McAllester developed an additional fixpoint \cite{appel:fixpoint} whose \cite{appel:vmm}

\subsection{The iterated separating conjunction}
1.2. \texttt{Iter\_sepcon} and \texttt{pred\_sepcon} are defined. And related ramification rules are proved.
1.3. The most general graph-spatial-predicate \texttt{vertices\_at} are defined (for all possible styles of graphs). Related ramification rules are proved. Graph and graphs are defined as special cases of vertices at.

2. A minor implementation trick. There are many tactics defined in \texttt{msl\_ext/ramify\_tactics.v}, which can manipulate low level heaps efficiently.

* Separating the material into the general vs. tool-specific part.  Measurements of etc.

\section{Verifying graph-manipulating programs}

4. General Strategy for Verifying Programs
4.1. Using relation. (Better for proof's code reuse). For example, spanning tree is mark together with some structural requirement.
4.2. Using existential quantifier in post condition.

\subsection{Speed of loop invariants}
The recommended tactic to solve invariants and ENTAIL goals in VST is \textit{entailer!}. However, we observed that our invariants contained a significiant number of mathematical statements, encoded by VST as PROPs, that were difficult to solve with VST's \textit{entailer!} tactic. As a result, \textit{entailer!} often took a long time to solve or reduce the invariant. To overcome this, we had to pre-solve every PROP in the invariant and preformat the SEP clauses. In the most significant case, an \textit{entailer!} call that took 446 seconds and still returned a large number of unsolved PROPs, was reduced to 36 seconds at the fastest recorded. (Tested on a VirtualBox Lubuntu 18.04 machine assigned 8GB RAM and 2 processors)

That the tactic could not solve many of our invariant \texttt{PROP}s by itself was expected and reasonable, as they were often complex properties of graphs. However, we were surprised that several solved \texttt{PROP}s took a significant amount of time despite being trivial at first glance. This is especially noticed in PROPs with preconditions of elements in empty lists. Thus, an immediate suggestion to the \textit{entailer!} tactic is to test "\textit{try contradiction}" as early as possible.
\begin{lstlisting}
assert (Hinv_10: forall u v : V,
	In u (nil (A:=V)) -> In v (nil (A:=V)) ->
	connected g u v <-> connected edgeless_graph' u v). {
		intros. contradiction.
}
\end{lstlisting}
The above Coq assertion in our loop precondition, which was easily solved by two basic tactics, cost an additional 50s when left to \textit{entailer!} to prove.
\newline\newline
Given the possibility that invariants carry PROPs on abstract graph properties that we do not expect \textit{entailer!} to solve, a thought is to provide a reduced version of \textit{entailer!} that does not reason about PROPs at all, but leaves them to the user to solve, and focuses on more VST-specific issues such as SEP and LOCAL clauses. This will be useful for verification of functions that implement abstract models.

\subsection{Shared clightgen issue with libraries.} It is obvious that one must include the dependencies of the program when compiling C code. However, we observe an issue of the opposite nature in VST and CompCert - that when running \textit{clightgen} for a program, one must include all programs using the same dependency to properly regenerate the AST file of said dependency. Failure to do so results in the proof of the other program failing to run.

For instance, in the above figure, both Dijkstra's and Prim's implementations are dependent on the same verified priority queue. When running clightgen for Prim's, we include the priority queue's C implementation to regenerate the AST tree. However, without including Dijkstra's in the same command, the existing verification of Dijkstra's experiences an error over the priority queue's newly regenerated AST file.

This is a hindrance to the modular design we aspire to, and we believe is worth looking into, to see if the error is caused by a trivial matter in CompCert's clightgen or worse.

\section{Enabling externally-verified lemmas in HIP/SLEEK}

* the connection to HIP/SLEEK

\section{Applying ramification}

5. Mark algorithm
5.1. For Ramification-Paper-style proof
5.1.1. Math land theorems for marking algorithm (general situation and bi-graph situation) are all proved. Mainly in "\texttt{marked\_graph.v}" and "\texttt{spatial\_graph\_mark\_bi.v}".
5.1.2. Ramification rule for marking algorithm (bi-graph situation) are all proved in "\texttt{spatial\_graph\_mark\_bi.v}".
5.1.3. Combining 2.4 and 3.1.1 and 3.1.2, we have a end-to-end proof for marking-graph in VST.
5.1.4. We have an end-to-end proof for marking-dag, but not defining dag predicate as a whole.
5.1.5. The module type which will be generated by HIP/SLEEK should be instantiated by 2.2 and 2.5.

6. Spanning tree algorithm
6.1. We divide the spanning tree relation into structural part and marking part. They are both defined properly.
6.2. Important pure facts and ramification rules are not proved yet.
6.3. Shengyi has already known how to use VST to handle the C program of bigraph spanning tree.

\section{Related and future work}

5.2. An alternative way of verifying marking program is reasoning about the whole history of marking operations. The disadvantage of it is that it currently needs more work in a Hoare logic framework. The advantage of it is that its reasoning structure are more similar with the way we understand it in our first algorithm class.
5.3. I take some effort on garbage collector like graph structural. Though it is only connecting this special structural with 5.1.1 and 1.3, it takes much time and it is not finished yet.

\section{Conclusion}

\bibliographystyle{abbrvnat}
\bibliography{autoquack}

\end{document}
