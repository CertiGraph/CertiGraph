Over the last fifteen years great strides have been made in automating verifications of programs that manipulate
tree-like data structures using separation logic
\cite{berdine:smallfoot,chin:hipsleek,jacobs:verifast,chlipala:bedrock,bengtson:charge,appel:programlogics}.  Unfortunately, verifying programs that manipulate graph-like data structures (i.e. structures with \emph{intrinsic sharing}) has been more challenging.  Indeed, verifying such programs was formidable enough that a number of the early landmark results in separation logic devoted substantial effort to verify single examples such as Schorr-Waite~\cite{hongseok:phd} with pen and paper---avoiding the additional challenges inherent in mechanized reasoning.

In recent years, Hobor and Villard introduced the concept of \emph{ramification} as a kind of proof pattern or framework to verify graph-manipulating programs on pen and paper~\cite{hobor:ramification}.  The major focus of this paper is to develop methods to verify realistic graph programs in a mechanized context.  We do so by upgrading the theory of ramification and by developing a general and modular library for graph-related reasoning in separation logic.  We incorporate our approach into two sizeable separation logic-based verification tools: the Floyd system of the Verified Software Toolchain (VST)~\cite{appel:programlogics} and the HIP/SLEEK program verifier~\cite{chin:hipsleek}.  VST and HIP/SLEEK inhabit quite different points in the design space for verification tools, with VST primarily focused on heavily human-guided verifications with an emphasis on end-to-end machine-checked proofs, and HIP/SLEEK focusing on more automation.  Despite these differences, the vast majority of our Coq code base is shared between them,
giving us hope that our work will be applicable to other verification tools.

%% \marginpar{\color{magenta} computable mathgraphs, null, pregraphs Problem with ``later'' not being precise.}
The structure of our paper is as follows:
\begin{itemize}
\item[\S\ref{sec:orientation}] We verify a graph marking algorithm and explain why such algorithms are easier to verify using relations instead of functions.  We introduce \emph{localization blocks} as a new notation for ramification.  We upgrade Hobor and Villard's \infrulestyle{Ramify} rule to handle both modified program variables and existential quantifiers more gracefully.
\item[\S\ref{sec:mathgraph}] We develop a general mechanization of mathematical graphs powerful enough to support realistic verification. %{\color{magenta} What else can we say here?}
\item[\S\ref{sec:spacegraph}] We suggest that the standard Knaster-Tarski fixpoint~\cite{tarski:fixpoint} cannot define a usable separation logic graph predicate.  We propose a better definition for general spatial graphs that still enjoys a ``recursive'' fold/unfold.  We prove general theorems about spatial graphs in a way that can be utilized in multiple flavors of separation logic, such as the logics contained in VST and HIP/SLEEK. %{\color{magenta} Line about modularity?}
\item[\S\ref{sec:vst}] We explain how we integrated ramification into VST by developing two new Floyd tactics, \li{localize} and \li{unlocalize}.  We discuss three additional VST-certified examples: marking a DAG, pruning a graph into a spanning tree (\emph{e.g.} for disposal), and making a structure-preserving copy of a graph.
\item[\S\ref{sec:hipsleek}] We explain how we modified HIP/SLEEK to introduce ramifications when programs modify data structures with intrinsic sharing and to automatically discharge the associated obligations using Coq-verified external lemmas.
\item[\S\ref{sec:development}] We document some statistics related to our development.
\item[\S\ref{sec:related}] We discuss related work.
\item[\S\ref{sec:conclusion}] We discuss directions for future work and conclude.
\end{itemize}
All of our results are machine checked.

