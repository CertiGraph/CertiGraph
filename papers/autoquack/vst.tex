The Verified Software Toolchain is a series of machine-checked modules written in Coq whose focus is reasoning about C programs~\cite{appel:programlogics}, especially those programs that can be compiled with the CompCert compiler~\cite{leroy:compcert}.  One of VST's modules, Floyd, is a separation-logic based engine to help users
verify concrete programs.  The modules interlock so there are no ``gaps'' in the end-to-end certified results; accordingly all of the rules employed by Floyd have been proved sound with respect to the underlying semantics used by CompCert.  Floyd is written in a combination of Ltac and Gallina and is designed to help users verify the full functional correctness of their programs.  Although Floyd devotes considerable effort to make this task simpler, it prefers expressibility and completeness to more highly-automated tools like HIP/SLEEK.

Floyd presents users with a pleasant ``decorated program'' visualization for Hoare proofs, in which users work from the top of the program to the bottom even though the formal proof is maintained as applications of inference rules.  For example, suppose the proof goal is $\triple{P_1}{c_1\li{;}c_2}{P_5}$ and VST's user tells Floyd to apply a Hoare rule for $c_1$, \emph{e.g.}~$\triple{P_1}{c_1}{P_2}$.  Floyd will then automatically apply the \infrulestyle{Sequence} rule and show the user $\triple{P_2}{c_2}{P_5}$ as the remaining goal.
When the user is in the middle of a verification, the decorated program is partially done (\emph{i.e.} the proof is finished from the top to ``the current program point'') and the inference tree is also partially done (\emph{i.e.} with holes that are represented by the remaining proof goals in Coq).

\subsection{The \li{localize} and \li{unlocalize} tactics}

We wish to preserve this ``decorated program'' view while extending Floyd to support ramification.  Our task therefore is to construct a proof in Coq's underlying logic that allows a localization block to be constructed in this manner---that is, we wish to enter a localization block without requiring the user to specify the ``exit point'' in advance.  The engineering is tricky because the proof Floyd is constructing (\emph{i.e.} applications of inference rules) has holes in places where the user's ``top to bottom'' view of things has not yet arrived.

\begin{figure}[t]
\begin{minipage}{.1\textwidth}
\begin{lstlisting}
$\{\ \ P_1 \ \ \}$
    c1
$\{\ \ P_2 \ \ \}\label{code:localglobalin}$
$\searrow \{\ \ P_3 \ \ \}\label{code:locallocalin}$
      c2;
$~~\,\{\ \ P_4 \ \ \}\label{code:localsndcmd}$

    ...



\end{lstlisting}
\end{minipage} \vline ~
\begin{minipage}{.11\textwidth}
\begin{lstlisting}[numbers=none]
$\{\ \ P_1 \ \ \}$
    c1
$\{\ \ P_2 \ \ \}$
$\{\ \ ?F * P_3 \ \}$
    c2;
$\{\ \ ?F * P_4 \ \}$

  ...



\end{lstlisting}
\end{minipage} \vline ~
\begin{minipage}{.11\textwidth}
\begin{lstlisting}[numbers=none]
$\{\ \ P_1 \ \ \}$
    c1
$\{\ \ P_2 \ \ \}$
$\searrow \{\ \ P_3 \ \ \}$
      c2;
  $\,\{\ \ P_4 \ \ \}$
      c3;
$\swarrow\{\ \ P_5 \ \ \}$
$\{\ \ P_6 \ \ \}$

    ...
\end{lstlisting}
\end{minipage} \vline ~
\begin{minipage}{.11\textwidth}
\begin{lstlisting}[numbers=none]
$\{\ \ P_1 \ \ \}$
    c1
$\{\ \ P_2 \ \ \}$
$\{\ \ ?F * P_3 \ \}$
    c2;
$\{\ \ ?F * P_4 \ \}$
    c3;
$\{\ \ ?F * P_5 \ \}$
$\{\ \ P_6 \ \ \}$

  ...
\end{lstlisting}
\end{minipage}
\caption{Front and back ends of \li{localize} and \li{unlocalize}}
\label{figure:backend}
\vspace{-2ex}
\end{figure}

Figure~\ref{figure:backend} has four partially-decorated ``proofs in progress'', from both the user's (front end) and Floyd's (back end) points of view.  In the first column, from the user's point of view, they saw the assertion $P_2$ (line~\ref{code:localglobalin}) and decided to use the \li{localize} tactic to zoom into $P_3$ (line~\ref{code:locallocalin}).  They then applied some proof rules to move past $c_2$ to reach the assertion $P_4$ (line~\ref{code:localsndcmd}).  At this point, Floyd does not know when the corresponding \li{unlocalize} tactic will execute, so it does not know how which commands will be inside the block or what the final local and global postconditions will be.

Accordingly, the \li{localize} tactic builds an incremental proof in the underlying program logic by applying \infrulestyle{Frame} with an uninstantiated metavariable.  The second column of Figure~\ref{figure:backend} shows the back end with the unknown frame $?F$, which will eventually be instantiated by \li{unlocalize}.

In the third column, the user has advanced past $c_3$ to reach the local postcondition $P_5$ and now wishes to \li{unlocalize} to~$P_6$.  Afterwards, the internal state looks like the fourth column, and so to a first approximation, \li{unlocalize} can instantiate $?F$ with $\pguards{c_2\li{;} c_3}(P_5 \wand P_6)$.  In fact, as discussed in \S\ref{sec:existentials}, \li{unlocalize} replaces $P_5$ and $P_6$ with equivalent $P_5'$ and $P_6'$ that use existentials to pack up modified program variables, and then instantiates $?F$ with a version of $\pguards{c_2\li{;} c_3}(P_5 \wand P_6)$ in which those same modified program variables have been replaced with universally-quantified metavariables.  Reformulating assertions to isolate program variables is aided by Floyd's use of a canonical form for assertions that explicitly separates assertions containing program variables from spatial assertions.
%original tactic system can help its users to write their assertions in such a isolated style.

As indicated in \S\ref{sec:existentials}, this leaves two proof goals.  The first, $P_2 |- ?F * P_3$, is simplified to remove the $\pguards{c_2\li{;} c_3}$ and referred to the user as a new obligation; most often the user solves it using a ramification library~(\S\ref{sec:ramifylib}).  The second, $?F |- P_5' --* P_6'$, is solved automatically by \li{unlocalize}.







%This engineering solution shows the intrinsic correlation between frame rule and ramification-P rule, i.e. the entailment relation that $?F$ should satisfy at the localize operation is exactly the ramification criterion.

%\textbf{Tracking Guarded Criteria.} Notice that not all separation logic assertions can be used to fill the uninstantiated frame, there are two criteria to follow. Using the example in Figure \ref{figure:backend}, the frame should satisfy: (1) $P_1 \vdash ?F * P_2$ (2) $?F$ ignores $\mathrm{ModVar}(\text{c2; c3})$.

%{\color{magenta}Our tactic system keeps track with these two guarded criteria to ensure soundness. When people do localize, the tactic system will create the uninstantiated frame tagged with these two criteria (using dependent types in Coq). The first criterion is fully specified while the second one is only semi-specified at the time of localizing because it is unknown which commands will be involved in this ramification block.}

%When people do unlocalize, the tactic system will use the magic wand operation between local and global assertions under universal modal operator ($[\text{c2;c3}](P_8 \wand P_9)$ in this example) to instantiate the frame and check whether these two criteria is satisfied. The first criterion ($P_1 \vdash P_2 * [\text{c2;c3}](P_8 \wand P_9)$ in this example) is usually left for manual proofs and the second criterion is always true because of the modality's property.

%\textbf{Simply Ramification Criterion.} Besides implementing localize and unlocalize operation, our tactic system simplifies the ramification criterion for users. The target of this simplification is eliminating the $[c]$ modality.

\hide{

\lstset{numbers=none}

\begin{figure}[h]
\begin{tabular}{@{\hspace{1pt}}l@{\hspace{3pt}}|@{\hspace{3pt}}l@{\hspace{1pt}}}
\begin{lstlisting}
$\{\mathrm{x} = 1 * P\}$
$\searrow \{\mathrm{emp}\}$
      y = 0;
$\swarrow \{\mathrm{y} = 0 * \mathrm{emp} \}$
$\{\mathrm{x} = 1 \wedge \mathrm{y} = 0 * Q \}$
\end{lstlisting} &
\begin{lstlisting}
$\{\mathrm{x} = 1 * P \}$
$\searrow \{\mathrm{emp} \}$
      y = 0;
$\swarrow \{\exists y_0, \mathrm{y} = y_0 \wedge y_0 = 0 * \mathrm{emp} \}$
$\{\exists y_0, \mathrm{y} = y_0 \wedge \mathrm{x} = 1 \wedge y_0 = 0 * Q[\mathrm{y} \mapsto y_0] \}$
\end{lstlisting}
\end{tabular}
\caption{Eliminating Modality}
\label{figure:modality}
\end{figure}

\lstset{numbers=left}

The left column of Figure \ref{figure:modality} shows a simple application of ramification. However, it is not obvious to how to eliminate the modality in its ramification criterion, i.e.
$$\mathrm{x} = 1 * P \vdash \mathrm{emp} * [\mathrm{y}](\mathrm{y} = 0 * \mathrm{emp} \wand \mathrm{x} = 1 \wedge \mathrm{y} = 0 * Q)$$

The assertions on the right side are equivalent with those on the left side, line by line. But in comparison, the ramification criterion of it (following RAMIF-PQ rule) is easier to simplify. Specifically,
\begin{eqnarray*}
&&~~~~[\mathrm{y}]\forall y_0. (\mathrm{y} = y_0 \wedge y_0 = 0 * \mathrm{emp} \wand \\
&& ~~~~~~~~~   \mathrm{y} = y_0 \wedge \mathrm{x} = 1  \wedge y_0 = 0 * Q[\mathrm{y} \mapsto y_0]) \\
&&= [\mathrm{y}]\forall y_0. (\mathrm{y} = y_0 \to \\
&& ~~~~~~~~~ (y_0 = 0 * \mathrm{emp} \wand \mathrm{x} = 1  \wedge y_0 = 0 * Q[\mathrm{y} \mapsto y_0])) \\
&&= \forall y_0. [\mathrm{y}](\mathrm{y} = y_0 \to \\
&& ~~~~~~~~~ (y_0 = 0 * \mathrm{emp} \wand \mathrm{x} = 1  \wedge y_0 = 0 * Q[\mathrm{y} \mapsto y_0])) \\
&&= \forall y_0. [\mathrm{y}](y_0 = 0 * \mathrm{emp} \wand \mathrm{x} = 1  \wedge y_0 = 0 * Q[\mathrm{y} \mapsto y_0]) \\
&&= \forall y_0. (y_0 = 0 * \mathrm{emp} \wand \mathrm{x} = 1  \wedge y_0 = 0 * Q[\mathrm{y} \mapsto y_0])
\end{eqnarray*}
}

%We build this trick of using existential quantifiers and meta variables to these elimination as a common pattern into our tactic system. As a result, the tactic system always leave a modality free ramification criterion for users to prove. 


%\textbf{Remark.} We extend VST's tactic system such that people can interactively build Hoare proofs with ramification involved. When localize and unlocalize are applied on a paper proof, we use frame rule and uninstantiated frame to build corresponding proof in the underlying logic. Simplified and modality free ramification criteria are left for users to prove as side condition of unlocalize.

