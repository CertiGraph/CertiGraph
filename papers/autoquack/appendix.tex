
\appendix

\section{Junk}
{\color{magenta} Universally-quantified metavariables can appear free in the predicates to make further connections.
Assuming that the abstracted pre- and postconditions $A$, $B$, $C$, and $D$ above all use \li{x}, we proceed
as follows.  First we introduce a new fresh metavariable $x$ whose value will be equal to \li{x} after the localization, and then choose $F \stackrel{\Delta}{=} [\li{x} |-> x] (C -* D)$, that is we substitute the program
variable \li{x} for the metavariable $x$.  Since we have substituted away \li{x}, $F$ ignores it and so we satisfy the side condition on \infrulestyle{Solve Ramify-P}.  We then must strengthen $C$ into $C' \stackrel{\Delta}{=} C /| \li{x} = x$ to make the connection at the appropriate program point.  Now we are left with the entailments
\[
\begin{array}{lcl}
\li{x} = 5 /| A & |- & (\li{x} = 5 /| B) * F \\
F & |- & (\li{x} = 6 /| C') -* (x = 6 /| D)
\end{array}
\]
To further relate the earlier and later values of \li{x} in $F$ we can introduce a second fresh $x'$ and use $B' \stackrel{\Delta}{=} B /| \li{x} = x'$.
}

The \infrulestyle{Ramify} rule is sound but interacts poorly with modified program variables (as in lines~\ref{code:markbeforetripleramify}--\ref{code:markaftertripleramify} of Figure~\ref{fig:markgraph}) {\color{magenta} and
localized existentials (as in lines~\ref{code:beforemarkl}--\ref{code:aftermarkl})}.  Both of these limitations are annoying enough in paper proofs and graduate to major headaches in mechanized ones.  Happily, we show how to overcome both limitations in \S\ref{sec:freevars} and \S\ref{sec:existentials}, respectively, by presenting new variants of \infrulestyle{Ramify}.  Our notation carries over without significant change: just use the new rules to enable the more general ramification entailments they permit.
%When in doubt the most general rule, \infrulestyle{Ramify-PQ} from \S\ref{sec:existentials}, implies all of the others.

\section{Remaining proof of \infrulestyle{Ramify-PQ}}
\label{apx}

See figure \ref{fig:remainrampq}.

\begin{figure*}[t]
\[
\infrule{}
{
  L_1 |- L_1 \\
  \infrule{}
  {
    \infrule{}
    {
      \infrule{}
      {
        \infrule{}
        {
          \infrule{}
          {
            \infrule{}
            {
              \infrule{}
              {
                \infrule{}
                {
                  \infrule{}
                  {
                    \infrule{}
                    {
                      [x |-> x_0] (L_2 -* G_2) |- [x |-> x_0](L_2 -* G_2)
                    } {
                      \forall x.~ (L_2 -* G_2) |- [x |-> x_0](L_2 -* G_2)
                    } {\forall \mathsf{e}}
                  } {
                    \forall x.~ (L_2 -* G_2) |- ([x |-> x_0]L_2) -* ([x |-> x_0]G_2)
                  } {\textrm{substitute}}
                } {
                  \big(\forall x.~ (L_2 -* G_2)\big) * [x |-> x_0]L_2 |- [x |-> x_0]G_2
                } {(3)}
              } {
                \big(\forall x.~ (L_2 -* G_2)\big) * [x |-> x_0]L_2 |- \exists x.~ G_2
              } {\exists \mathsf{i}}
            } {
            \big(\forall x.~ (L_2 -* G_2)\big) * (\exists x.~ L_2) |- \exists x.~ G_2
            } {\exists \mathsf{e}}
          } {
            \forall x.~ (L_2 -* G_2) |- (\exists x.~ L_2) -* (\exists x.~ G_2)
          } {(3)}
        } {
          |- \big(\forall x.~ (L_2 -* G_2)\big) => \big((\exists x.~ L_2) -* (\exists x.~ G_2)\big)
        } {=> \mathsf{i}}
      } {
        |- \pguards{c}\Big(\big(\forall x.~ (L_2 -* G_2)\big) => \big((\exists x.~ L_2) -* (\exists x.~ G_2)\big)\Big)
      } {\mathsf{N}}
    } {
      |- \Big(\pguards{c}\big(\forall x.~ (L_2 -* G_2)\big) \Big) => \Big( \pguards{c}\big((\exists x.~ L_2) -* (\exists x.~ G_2)\big) \Big)
    } {\mathsf{K}}
  } {
    \pguards{c}\big(\forall x.~ (L_2 -* G_2)\big) |- \pguards{c}\big((\exists x.~ L_2) -* (\exists x.~ G_2)\big)
  } {\mathsf{i} =>}
} {
  L_1 * \pguards{c}\big(\forall x.~ (L_2 -* G_2)\big) |- L_1 * \pguards{c}\big((\exists x.~ L_2) -* (\exists x.~ G_2)\big)
} {* \textrm{ split} }
\]
\caption{Remaining proof of \infrulestyle{Ramify-PQ}}
\label{fig:remainrampq}
\end{figure*}
