\begin{figure}[t]
  \begin{lstlisting}
struct Node {$\label{code:nodedefstart}$
  int  _Alignas(16) m;
  struct Node * _Alignas(8) l;
  struct Node * r; };$\label{code:nodedefend}$

void mark(struct Node * x) { // $\label{code:markstart}\{\p{graph}(\tx{x},\gamma)\}$
  struct Node * l, * r; int root_mark; $\label{code:inmark}$
  if (x == 0) return;
// $\{\p{graph}(\tx x,\gamma) /| \exists m,l,r.~ \gamma(\tx{x}) = (m,l,r)\}$
// $\label{code:globalbeforerootmark}\{\p{graph}(\tx x,\gamma) /| \gamma(\tx{x}) = (m,l,r)\}$
// $\label{code:beforerootmark}\searrow \{\tx x|-> m,-,l,r \}$
      root_mark = x -> m;
// $\label{code:afterrootmark}\swarrow \{\tx x|-> m,-,l,r /| m = \tx{root\_mark} \}$
// $\label{code:globalafterrootmark}\{\p{graph}(\tx x,\gamma) /| \gamma(\tx{x}) = (m,l,r) /| m = \tx{root\_mark}\}$
  if (root_mark == 1) return;
// $\{\p{graph}(\tx x,\gamma) /| \gamma(\tx{x}) = (0,l,r) \}$
// $\label{code:markbeforetripleramify}\searrow \{\tx x|-> 0,-,l,r /| \gamma(\tx{x}) = (0,l,r)\}$
      l = x -> l;
      r = x -> r;
      x -> m = 1;
// $\label{code:markaftertripleramify}\swarrow \{\tx x|-> 1,-,\tx{l},\tx{r} /| \gamma(\tx{x}) = (0,\tx{l},\tx{r}) /| \exists \gamma'.~ \m{mark1}(\gamma, \tx{x}, \gamma')\}$
// $\{\exists \gamma'.~ \p{graph}(\tx x,\gamma') /| \gamma(\tx{x}) = (0,\tx{l},\tx{r}) /| \m{mark1}(\gamma, \tx{x}, \gamma')\}$
// $\label{code:beforemarkl}\{\p{graph}(\tx x,\gamma') /| \gamma(\tx{x}) = (0,\tx{l},\tx{r}) /| \m{mark1}(\gamma, \tx{x}, \gamma')\}$
// $\searrow \{\p{graph}(\tx l, \gamma')\}$
      mark(l);
// $\swarrow \{\exists \gamma''.~ \p{graph}(\tx l, \gamma'') /| \m{mark}(\gamma', \tx{l}, \gamma'')\}$
// $\label{code:aftermarkl}\left\{\!\!\!\begin{array}{l@{}}\exists \gamma''.~ \p{graph}(\tx x,\gamma'') /| \gamma(\tx{x}) = (0,\tx{l},\tx{r}) /| \null \\ \m{mark1}(\gamma, \tx{x}, \gamma') /| \m{mark}(\gamma', \tx{l}, \gamma'')\end{array}\right\}$
// $\left\{\!\!\!\begin{array}{l@{}}\p{graph}(\tx x,\gamma'') /| \gamma(\tx{x}) = (0,\tx{l},\tx{r}) /| \null \\ \m{mark1}(\gamma, \tx{x}, \gamma') /| \m{mark}(\gamma', \tx{l}, \gamma'')\end{array}\right\}$
// $\searrow \{\p{graph}(\tx r, \gamma'')\}$
      mark(r);
// $\swarrow \{\exists \gamma'''.~ \p{graph}(\tx r, \gamma''') /| \m{mark}(\gamma'', \tx{r}, \gamma''')\}$
// $\label{code:outmark}\left\{\!\!\!\begin{array}{l@{}}\exists \gamma'''.~ \p{graph}(\tx x,\gamma''') /| \gamma(\tx{x}) = (0,\tx{l},\tx{r}) /| \null \\ \m{mark1}(\gamma, \tx{x}, \gamma') /| \m{mark}(\gamma', \tx{l}, \gamma'') /| \m{mark}(\gamma'', \tx{r}, \gamma''')\end{array}\right\}$
} // $\label{code:markend}\{\exists \gamma'''.~ \p{graph}(\tx x,\gamma''') /| \m{mark}(\gamma, \tx{x}, \gamma''')\}$
\end{lstlisting}
%% \vspace{-8pt}
\caption{Clight code and proof sketch for bigraph mark.}
% {\color{magenta} The steps that induce
%  ramifications are indicated with $\ramify_i$, where the associated ramification entailment is equation number $i$.}} %whose numbers point with their associated ramification entailment reference.}
%\vspace{-19pt}
\label{fig:markgraph}
\end{figure}


In Figure~\ref{fig:markgraph} we put the code and proof sketch of the classic \li{mark} algorithm that visits and colors every reachable node in a heap-represented graph.  The \li{mark} algorithm is good to start with because it is complex enough to require some care to verify while being simple enough that the invariants are straightforward.  In \S\ref{sec:application} we will discuss more complex examples that \emph{e.g.} add/chang/remove edges and/or vertices.

The code in Figure~\ref{fig:markgraph} is written in Clight~\cite{blazy:clight}, an input language to the CompCert certified compiler~\cite{leroy:compcert}, which compiles our code exactly as written.
The paper-format verification sketch for \li{mark} in Figure~\ref{fig:markgraph} is extracted from
a ``Floyd'' proof in VST~\cite{appel:programlogics}, with only minor cleanup to aid the presentation.  
Accordingly, there is an unbroken certified chain from our specification of \li{mark} all the way to the assembly code.  In \S\ref{sec:hipsleek} we use HIP/SLEEK~\cite{chin:hipsleek} to verify a Java version of \li{mark}; the program invariants generated by HIP/SLEEK are slightly different due to HIP/SKEEK's heavier automation.
% but the overall structure is the same.

The specification we certify (lines \ref{code:markstart} and \ref{code:markend}) is
\[
\{\p{graph}(\li{x},\gamma)\}~\li{mark(x)}~\{\exists \gamma'.~ \p{graph}(\li{x},\gamma') /| \m{mark}(\gamma, \li{x}, \gamma')\}
\]
The specification is for full functional correctness, stated using \emph{mathematical} graphs~$\gamma$; until \S\ref{sec:mathgraph} consider $\gamma$ to be a function that maps a vertex $v \in V$ to triples $(m,l,r)$, where $m$ is a ``mark'' bit (0 or 1) and $\{l,r\} \subseteq V \uplus \{0\}$ are the neighbors of $v$.
The \emph{spatial} \p{graph} predicate describes how the mathematical graph $\gamma$ is implemented in the heap.  Until~\S\ref{sec:spacegraph} it is enough to know that \p{graph} satisfies the fold/unfold relationship in
equation \eqref{eqn:bigraphintrofoldunfold}, located just under the code in Figure~\ref{fig:markgraph}.

This fold/unfold relationship deserves attention.  
First, as we explain in~\S\ref{sec:fixpointfail}, it is a terrible mistake to write~\eqref{eqn:bigraphintrofoldunfold} as a definition using $\stackrel{\Delta}{=}$ rather than as a biimplication using $<=>$.  Second, \eqref{eqn:bigraphintrofoldunfold} uses the ``overlapping conjunction'' $\ocon$ of separation logic; informally $P ** Q$ means that $P$ and $Q$ may overlap in the heap (\emph{e.g.}, nodes in the left subgraph can also be in the right subgraph or even be the root $x$).  The presence of the unspecified sharing indicated by the $\ocon$ connective is exactly why graph-manipulating algorithms are so hard to verify (\emph{e.g.}, it is hard to apply the \infrulestyle{Frame} rule).  The semantics of the separation logic connectives used in this paper in Figure~\ref{fig:seplogsem}.
Third, \eqref{eqn:bigraphintrofoldunfold} illustrates how industrial-strength settings complicate verification.  Lines~\mbox{\ref{code:nodedefstart}--\ref{code:nodedefend}} define the data type \li{Node} used by \li{mark}.  The \li{_Allignas($n$)} directives tell CompCert to align fields on $n$-byte boundaries.  As explained in~\S\ref{sec:foldunfold}, this alignment is necessary in C-like memory models to prove fold-unfold \eqref{eqn:bigraphintrofoldunfold}, which is why \eqref{eqn:bigraphintrofoldunfold} includes an alignment restriction $x~\mathsf{mod}~16 = 0$ and an existentially-quantified ``blank'' second field for the root $x \mapsto m,-,l,r$.
%{\color{magenta}(In our Floyd proofs the alignment restriction and blank second field are nicely hidden ``behind the scenes''.)}

Notice that the postcondition of \li{mark} is specified \emph{relationally}, \emph{i.e.} $\{\exists \gamma'.~ \p{graph}(\li{x},\gamma') /| \m{mark}(\gamma, \li{x}, \gamma')\}$ instead of \emph{functionally}, \emph{i.e.} $\{\p{graph}\big(\li{x},\m{mark}(\gamma, \li{x})\big)\}$. In the first case $\m{mark}$ is a relation that specifies that~$\gamma'$ is the result of correctly marking~$\gamma$ from~\li{x}, whereas in the second $\m{mark}$ is a function that \textbf{computes} the result of marking~$\gamma$ from~\li{x}. For both theoretical and practical reasons a relational approach is better.
Theoretically, relations are preferable because they are more general.  For example, relations allow ``inputs'' to have no ``outputs'' (\emph{i.e.} be partial) or alternatively have many outputs (\emph{i.e.} be nondeterministic).  Our graph \li{copy} algorithm is specified nondeterministically to avoid specifying how \li{malloc} allocates fresh blocks of memory.  Relations are also preferable to functions because they are more compositional.
We take advantage of compositionality by using $\m{mark}(\gamma,x,\gamma') /| \ldots$ to specify both our ``spanning tree'' and ``graph copy'' algorithms in~\S\ref{sec:application}, which also mark nodes while carrying out their primary task.

\begin{figure}
\[
\begin{array}{lcl}
\sigma |= P * Q & \defeq & \exists \sigma_1, \sigma_2.~ \sigma_1 \oplus \sigma_2 = \sigma /| \null \\ && ~~ (\sigma_1 |= P) /| (\sigma_2 |= 2)\\
\sigma |= P ** Q & \defeq & \exists \sigma_1, \sigma_2, \sigma_3.~ \sigma_1 \oplus \sigma_2 \oplus \sigma_3 = \sigma /| \null \\ && ~~ (\sigma_1 \oplus \sigma_2 |= P) /| (\sigma_2 \oplus \sigma_3 |= Q) \\
\sigma |= P --* Q & \defeq & \forall \sigma_1, \sigma_2.~ \sigma_1 \oplus \sigma = \sigma_2 /| \null \\ && ~~
(\sigma_1 |= P) => (\sigma_2 |= Q) \\
\sigma |= P --o Q & \defeq & \exists \sigma_1, \sigma_2.~ \sigma_1 \oplus \sigma = \sigma_2 /| \null \\ && ~~
(\sigma_1 |= P) /| (\sigma_2 |= Q)
\end{array}
\]
\vspace{-2ex}
\caption{Separation logic connectives; $\oplus$ is the join operation on states, usually some kind of disjoint union on heaps}
\label{fig:seplogsem}
\vspace{-2ex}
\end{figure}

Practically, it is painful to define computational functions over graphs in a proof assistant like Coq, and portions of this pain are overkill.  For example, Coq requires that all functions terminate, a nontrivial proof obligation over cyclic structures like graphs, but our verification of \li{mark} is only for partial correctness.  Defining relations is much easier because \emph{e.g.} one can use quantifiers and do not have to prove termination.
The $\m{mark}$ and $\m{mark1}$ relations we use are defined straightforwardly at the bottom of Figure~\ref{fig:markgraph}.

Turning to the body of the verification (lines~\ref{code:inmark}--\ref{code:outmark}), readers may already have noticed our new notation: blocks of proof sketch bracketed by the symbols $\searrow$ and $\swarrow$, such as lines~\ref{code:beforerootmark}--\ref{code:afterrootmark}.  We call a bracketed set of lines like this a ``localization block''; localization blocks were inspired by our new \li{localize} and \li{unlocalize} tactics in Floyd.
The intuitive idea is that we zoom in from a larger ``global'' context to a smaller ``local'' one.  After verifying some commands locally to arrive at a local postcondition, we zoom back out to the global context.  Although we do not do so in Figure~\ref{fig:markgraph}, localization blocks can safely nest.

In lines~\ref{code:beforerootmark}--\ref{code:afterrootmark}, imagine unfolding the \p{graph} predicate in line~\ref{code:globalbeforerootmark} using equation \eqref{eqn:bigraphintrofoldunfold} and then zooming in to the root node \li{x} for lines~\ref{code:beforerootmark}--\ref{code:afterrootmark}, before zooming back out in line~\ref{code:globalafterrootmark}.

To define localization blocks formally we need to first understand the \infrulestyle{Frame} and \infrulestyle{Ramify} rules.
 
\subsection{Frames and ramifications are localizations}
\label{sec:localizations}
The key rule of separation logic is \infrulestyle{Frame}~\cite{rey02}:
\[
\infrule{Frame}
{\{ P \} ~ c ~ \{Q \}}
{\{P * F \} ~ c ~ \{ Q * F \}}
{\begin{array}{c}F \textrm{ ignores } \MV(c) \end{array}} \qquad \qquad
\]
The reason \infrulestyle{Frame} is so important is because it enables local verifications.  That is, a verifier can focus on the portions of the heap that are relevant to command $c$ and ``frame away'' the rest.  The side condition $F \textrm{ ignores } \MV(c)$ relates to modified program variables and will be discussed in \S\ref{sec:freevars}.
 
Hobor and Villard observed that \infrulestyle{Frame} is bit rigid because it forces verifiers to split program assertions into syntactically $*$-separated parts~\cite{hobor:ramification}.  This rigidity is particularly unpleasant when verifying programs that manipulate data structures with intrinsic unspecified sharing such as DAGs and graphs.  Hobor and Villard proposed the \infrulestyle{Ramify} rule to circumvent this rigidity:
\[
\infrule{Ramify}
{\{L_1\} ~ c ~ \{L_2\} \\ G_1 |- L_1 * (L_2--* G_2)}
{\{G_1\} ~ c ~ \{G_2\}}
{\begin{array}{c}(L_2 --* G_2) \\ \textrm{ignores} \\ \MV(c) \end{array}} \qquad \qquad \qquad
%{$\begin{array}{l}\m{fv}(Q --* R') \cap \null \\ \m{modif}(c) = \emptyset\end{array}$} \qquad \qquad \qquad
\]
That is, we can verify a ``global'' specification $\{G_1\}~c~\{G_2\}$ by combining a ``local'' specification $\{L_1\}~c~\{L_2\}$ with a \emph{ramification entailment} $G_1 |- L_1 * (L_2--* G_2)$.  This entailment uses the ``magic wand'' operator $--*$ of separation logic\footnote{$--*$ is the adjunct of $*$, \emph{i.e.} $(P * Q |- R) <=> (P |- Q --* R)$.} to express a notion of ``substate update'': inside $G_1$ replace $L_1$ with $L_2$ to reach $G_2$.  Essentially the ramification entailment ensures that the change in state specified locally fits properly into the global context.  In exchange for proving the ramification entailment, a verifier can use \infrulestyle{Ramify} at any time, \emph{i.e.} they need not worry about syntactically matching their assertions with the $*$ in the \infrulestyle{Frame} rule.

We are now ready to give a formal meaning to the ``localization'' pattern employed in Figure~\ref{fig:markgraph}.  When we write:
%\begin{minipage}{.25\textwidth}
\begin{lstlisting}
// $\label{code:prelocal}\{ G_1 \}$
// $\label{code:inlocal}\searrow \{ L_1 \}$
$\ramify(i)$      $c_1$; ... ; $c_n$;
// $\label{code:outlocal}\swarrow \{ L_2 \}$
// $\label{code:postlocal}\{ G_2 \}$
\end{lstlisting}
we mean apply \infrulestyle{Ramify} with $G_1 |- L_1 * (L_2 --* G_2)$.
An advantage of this notation is crystal clarity on the predicates used in the ramification entailment.  For convenience, the optional $\ramify(i)$ specification can reference an equation or lemma number that solves the ramification entailment.  For example, in Figure \ref{fig:markgraph} line \ref{code:markram2} references Lemma \ref{lem:updategraphnode} whereas we omit $\ramify$ around line \ref{code:markram1} since the heap is unchanged and so the entailment is straightforward. If we wish to save a line or two we can compress the line pairs \ref{code:prelocal}--\ref{code:inlocal} and \ref{code:outlocal}--\ref{code:postlocal}
to the single lines $\{ G_1 \} \searrow \{ L_1 \}$ and $\{ G_2 \} \swarrow \{ L_2 \}$ without sacrificing clarity.

Hobor and Villard pointed out that \infrulestyle{Ramify} implies \infrulestyle{Frame} (modulo the modified program variables issue we fix in \S\ref{sec:freevars}), meaning that our notation can clarify uses of \infrulestyle{Frame} as well.  This is particularly useful in multi-line contexts with nontrivial $F$, for which the current popular notation to express \infrulestyle{Frame} involves a liberal use of ``\ldots'', \emph{e.g.}:

\vspace{5pt}

\begin{minipage}{.2\textwidth}
Old notation:
\begin{lstlisting}
// $\{ P_1 * F_1 * F_2 * F_3 \}$
   $c_1$;
// $\{ P_2 * \ldots \}$
   $c_2$;
// $\{ P_3 * \ldots \}$
   $c_3$;
// $\{ P_4 * F_1 * F_2 * F_3 \}$
\end{lstlisting}
\end{minipage}
\begin{minipage}{.2\textwidth}
New notation:
\begin{lstlisting}
// $\{ P_1 * F_1 * F_2 * F_3 \} \searrow \{ P_1 \}$
      $c_1$;
//    $\{ P_2 \}$
      $c_2$;
//    $\{ P_3 \}$
      $c_3$;
// $\{ P_4 * F_1 * F_2 * F_3 \} \swarrow \{ P_4 \}$
\end{lstlisting}
\end{minipage}

\subsection{The program variable bugaboo}
\label{sec:freevars}

The \infrulestyle{Ramify} rule is sound but interacts poorly with modified program variables (as in lines~\ref{code:markbeforetripleramify}--\ref{code:markaftertripleramify} of Figure~\ref{fig:markgraph}) {\color{magenta} and
localized existentials (as in lines~\ref{code:beforemarkl}--\ref{code:aftermarkl})}.  Both of these limitations are annoying enough in paper proofs and graduate to major headaches in mechanized ones.  Happily, we show how to overcome both limitations in \S\ref{sec:freevars} and \S\ref{sec:existentials}, respectively, by presenting new variants of \infrulestyle{Ramify}.  Our notation carries over without significant change: just use the new rules to enable the more general ramification entailments they permit.
%When in doubt the most general rule, \infrulestyle{Ramify-PQ} from \S\ref{sec:existentials}, implies all of the others.

%In \S\ref{sec:freevars} and \S\ref{sec:existentials} we present .



{
\color{magenta}
Our \textbf{seventh} improvement is best illustrated by lines~\ref{code:markbeforetripleramify}--\ref{code:markaftertripleramify}.  As explained more fully in \S\ref{sec:freevars}, Hobor and Villard's \infrulestyle{Ramify} rule interacts very poorly with modified program variables such as \li{l} and \li{r}.  They hacked their way around this issue by proposing a variant of \infrulestyle{Ramify} called \infrulestyle{RamifyAssign}, which could reason about the special case of a single assignment \li{x=$f(\ldots)$}, assuming the verifier can make the local program translation to \li{x'=$f(\ldots)$;x=x'}, where \li{x'} is fresh.  They proposed no way to verify unmodified program code or to ramify/localize across multiple assignments; we do both in lines~\ref{code:markbeforetripleramify}--\ref{code:markaftertripleramify} using a new \infrulestyle{Ramify-P} rule.

{\color{green}
existentials \S\ref{sec:existentials} 4.2. Using existential quantifier in post condition.
}
}


\infrulestyle{Frame}'s side condition ``$F \text{ ignores } \MV(c)$'' can be defined in two ways.
In the more traditional syntactic style, it means that $\FV(F) \cap \MV(c) = \emptyset$.
By ``syntactic style'' we mean that the side condition is written using a function $\FV(F)$ that takes an arbitrary formula and returns the set of free variables within that formula.  To define this $\FV(F)$ function
we need a fixed inductive \textbf{syntax} for formulas.  In contrast, in this paper we follow a ``semantic style'' in which formulas are not given a fixed syntax in advance but can be defined \textbf{semantically} on the fly using an appropriate model~\cite{appel:programlogics}.  In a semantic style, the side condition on the frame rule is defined as:
\[
\begin{array}{ll}
\sigma \stackrel{S}{\cong} \sigma' & \stackrel{\Delta}{=} ~~ \sigma \text{ and } \sigma' \text{ coincide everywhere except } S\\
P \text{ ignores } S & \stackrel{\Delta}{=} ~~ \forall \sigma, \sigma'.~ \sigma \stackrel{S}{\cong} \sigma' => \null \\
& \qquad ~~ (\sigma |= P) <=> (\sigma' |= P)
\end{array}
\]
That is, we consider two program states $\sigma$ and $\sigma'$ equivalent up to program variable set $S$ when they agree everywhere except for on the values of variables in $S$ (typically, a state $\sigma$ is a pair of a heap $h$ and program variables $\rho$).  A predicate $P$ is then stable with respect to $S$ when its truth is independent of all program variables in $S$.  %{\color{magenta} Notice both the syntactic and semantic styles use the $\MV(c)$ function defined via straightforward recursive case analysis on program syntax; programming languages typically do have a fixed syntactic structure.}

%{\color{magenta} Intro blah. lines~\ref{code:beforemarkl}--\ref{code:aftermarkl} of Figure~\ref{fig:markgraph}}

Now consider using ramification to verify this program:
\begin{lstlisting}
// $\{ \tx{x} = 5 /| A \}$
// $\searrow \{\tx{x} = 5 /| B \}$
      ...;
      x = x + 1;
      ...;
// $\swarrow \{\tx{x} = 6 /| C \}$
// $\{ \tx{x} = 6 /| D \}$
\end{lstlisting}
Suppose that the other (elided) lines of the program make localization desirable, even though it is overkill for a single assignment.  The key issue is that the program variable {\li{x}} appears in all four positions in the ramification entailment
\[
(\li{x} = 5 /| A) \vdash (\li{x} = 5 /| B) * \big((\li{x} = 6 /| C) --* (\li{x} = 6 /| D)\big)
\]
One problem is that $(\li{x} = 6 /| C) --* (\li{x} = 6 /| D)$ does \textbf{not} ignore the modified program variable \tx{x}, preventing us from applying the \infrulestyle{Ramify} rule.  Intuitively, the stability side condition on the \infrulestyle{Ramify} rule is a bit too strong since it prevents us from mentioning variables in the postconditions that \textbf{have} been modified by code $c$.

The obvious thing to try is to weaken the side condition in \infrulestyle{Ramify} to $\big(\FV(G_2) \cap \MV(c)\big) \subseteq \FV(L_2)$, the idea being that information about modified program variables mentioned in the local postcondition $L_2$ can be carried to the global postcondition $G_2$.  Unfortunately, this idea is unsound because \li{x} cannot simultaneously be both~5 and~6, \emph{i.e.} the above entailment is vacuous.  A better idea is: % the following :
\[
\infrule{Ramify-P (Program variables)}
{\{ L_1 \} ~ c ~ \{L_2 \} \\
 G_1 \vdash L_1 * \pguards{c}  (L_2 --* G_2)}
{\{ G_1 \} ~ c ~ \{ G_2 \}}{}
\]
The ramification entailment now incorporates a new (universal/boxy) modal operator $\pguards{c}$.  The intuitive meaning of $\pguards{c}$ is that program variables modified by command $c$ can change value inside its scope.    Note that it is vital that $L_2$ appears as the antecedent of a (spatial) implication since the change in program variables is universally quantified.  This means that if we want to say anything specific about modified program variables in the global postcondition $G_2$ then we had better say something about them in the local postcondition $L_2$.

Let us return to our earlier entailment:
\[
\begin{array}{l}
(\li{x} = 5 /| A) \vdash (\li{x} = 5 /| B) * \null \\
~~ \pguards{\li{...; x = x + 1; ...;}} \big((\li{x} = 6 /| C) --* (\li{x} = 6 /| D)\big)
\end{array}
\]
Since \li{x} is modified, its value can change from the first line, in which \li{x} must be 5, to the second, in which \li{x} must be 6.

Here is the definition of $\pguards{c}$, writing $\langle c \rangle$ for $\MV(c)$:
\[
%\begin{array}{lcl}
%\langle c \rangle & \stackrel{\Delta}{=} & \MV(c) \\
\sigma |= \pguards{c} P ~~ \stackrel{\Delta}{=} ~~ \forall \sigma'.~ (\sigma \stackrel{\langle c \rangle}{\cong} \sigma') => (\sigma' |= P)% ~~~~ \text{where $\mathsf{MV}(c)$ is $\MV(c)$}\\
%\end{array}
\]
In other words, $\pguards{c}$ is exactly the universal modal operator~$\Box$ over the relation that considers equivalent all states that differ only on program values modified by $c$.  Since $\stackrel{\langle c \rangle}{\cong}$ is an equivalence relation, $\pguards{c}$ forms an S5 modal logic.

Note that \infrulestyle{Ramify-P} has no free variable side condition, which is unnecessary because $\forall P.~ \pguards{c}P \text{ ignores } \MV(c)$.  However, in practice this side condition reappears because to actually prove a ramification entailment containing $\pguards{c}$ one typically applies the following \infrulestyle{Solve Ramify-P} rule:
\[
\infrule{Solve Ramify-P}
{G_1 |- L_1 * F \\
{\color{magenta} F |- L_2 --* G_2}}
{G_1 \vdash L_1 * \pguards{c}  (L_2 --* G_2)}{F \textrm{ ignores } \MV(c)} \qquad \qquad \qquad \qquad
\]
That is, we can handle the $\pguards{c}$ by breaking apart the single entailment into a pair.  Using two entailments allows modified program variables to change between the preconditions and postconditions.  To connect the pair, we must choose a suitable predicate $F$ that ignores modified variables in $c$. Finding a suitable $F$ and proving the associated entailments can be tricky in the abstract but in practice is guided by using a ``ramification library'' as given in \S\ref{sec:ramifylib}.

With \infrulestyle{Ramify-P} and \infrulestyle{Solve Ramify-P} we can prove the \infrulestyle{Frame} rule with its canonical side condition as follows:
\[
\infrule{}{\raisebox{1.4ex}{$\infrule{}{P * F |- P * F \\ F |- Q --* (Q * F)}
{\raisebox{-4pt}[0pt][0pt]{$P * F |- P * \pguards{c}\big(Q --* (Q * F)\big)$}}
{\hspace{-1.1ex}\raisebox{0.9ex}{$\begin{array}{c}F \text{ ignores} \\ \MV(c)\end{array}$}}$}
\\ \{P\}~c~\{Q\}}
{\{P * F\}~c~\{Q * F\}}
{}
\]
This justifies our point in \S\ref{sec:localizations} that our new localization notation can also be used for frames.
%Conversely, we can also prove \infrulestyle{Ramify-P} from \infrulestyle{Frame} and \infrulestyle{Consequence}:

\subsection{The existential ogre}
\label{sec:existentials}

We call the action in line 2 \emph{localize} and call the action in line 6 \emph{unlocalize}.

~\ref{code:beforemarkl}--\ref{code:aftermarkl}

Existential rule by Floyd~\cite{floydlogic} (1967!)

\[
\infrule{Existential extraction}
{\forall x.~ \big(\{ P \} ~ c ~ \{Q \}\big)}
{\{ \exists x. P \} ~ c ~ \{ \exists x.~ Q \}}{}
\]

\begin{lstlisting}
// $\{ P \}$
  $c$
// $\{ \exists x.~Q \}$
// $\{ Q \}$
\end{lstlisting}


But:
\begin{lstlisting}
// $\{ G_1 \}$
// $\searrow \{ L_1 \}$
      $c$
// $\swarrow \{ \exists x.~L_2 \}$
// $\{ \exists x.~G_2 \}$
// $\{ G_2 \}$
\end{lstlisting}

\Rule{RAM-P-FRAME }
{G \vdash L * \Box^{\llbracket c \rrbracket} (L' --* G') \\
F \text{ is stable w.r.t. } \MV(c) \\}
{G * F \vdash L * \Box^{\llbracket c \rrbracket} (L' --* G' * F) }

\Rule{RAM-P-SPLIT }
{G_1 \vdash L_1 * \Box^{\llbracket c \rrbracket} (L_1' --* G_1') \\
G_2 \vdash L_2 * \Box^{\llbracket c \rrbracket} (L_2' --* G_2') \\}
{G_1 * G_2 \vdash L_1 * L_2 * \Box^{\llbracket c \rrbracket} (L_1' * L_2' --* G_1' * G_2') }

\[\text{UNSOUND-RAM-Q-SPLIT}\]
\Rule{}
{G_1 \vdash L_1 * (\exists x, L_1' (x) --* \exists x, G_1'(x)) \\
G_2 \vdash L_2 * (\exists x, L_2' (x) --* \exists x, G_2'(x)) \\}
{G_1 * G_2 \vdash L_1 * L_2 * (\exists x, L_1'(x) * L_2'(x) --* \exists x, G_1'(x) * G_2'(x)) }


\[
\infrule{Ramify-Q}
{\{ L \} ~ c ~ \{\exists x.~ L' \} \\
 G \vdash L * \big(\forall x.~ (L' --* G')\big)}
{\{ G \} ~ c ~ \{ \exists x.~G' \}}{}
\]

\Rule{Ramification-PQ}
{\{ L \} c \{ \exists x, L' (x) \} \\
 G \vdash L * \Box^{\llbracket c \rrbracket} (\forall x, L' (x) --* G' (x)) \\}
{\{ G \} c \{ \exists x, G' (x)\}}


%Related work

\subsection{Soundness of our rules}
Figure~\ref{fig:rampqproofs}!

\begin{figure*}
Proof of \infrulestyle{Ramify-P} from \infrulestyle{Frame} and \infrulestyle{Consequence}:
\vspace{-3em}
\[
\begin{array}{c}
\infrule{}{
  G_1 |- L_1 * \pguards{c}(L_2 --* G_2) \\
  \infrule{}{\{L_1\}~c~\{L_2\}}
            {\{L_1 * \pguards{c}(L_2 --* G_2)\}~c~\{L_2 * \pguards{c}(L_2 --* G_2)\}}{(1)} \\
  \infrule{}{
            \infrule{}{\stackrel{\langle c \rangle}{\cong} \text{ is reflexive}}{\pguards{c}(L_2 --* G_2) |- L_2 --* G_2}{(2)}}
            {L_2 * \pguards{c}(L_2 --* G_2) |- G_2}{(3)}}
{\{G_1\}~c~\{G_2\}}
{} \\
[5pt]
(1)~ \forall P.~ \pguards{c}P \text{ ignores } \FV(c) \qquad (2)~ \text{Axiom T of modal logic} \qquad (3)~ --* \text{ is the adjunct of } *
\end{array}
\]

Proof of \infrulestyle{Ramify-PQ} from \infrulestyle{Ramify-P}:
\vspace{-4em}
\[
\begin{array}{c}
\infrule{}
{
  \{L_1\}~c~\{\exists x.~ L_2\} \hspace{-0.5em} \\
  \infrule{}
  {
    G_1 |- L_1 * \pguards{c}\big(\forall x.~ (L_2 --* G_2)\big) \hspace{-0.5em} \\
    \infrule{}{
      \infrule{}{
        \infrule{}{
          \vdots
        } {
          \forall x.~ (L_2 --* G_2) |- (\exists x.~ L_2) --* (\exists x.~ G_2)
        } {(1)}
      } {
        \pguards{c}\big(\forall x.~ (L_2 --* G_2)\big) |- \pguards{c}\big((\exists x.~ L_2) --* (\exists x.~ G_2)\big)
      } {(2)}
    } {
      L_1 * \pguards{c}\big(\forall x.~ (L_2 --* G_2)\big) |- L_1 * \pguards{c}\big((\exists x.~ L_2) --* (\exists x.~ G_2)\big)
    } {}
  } {
    G_1 |- L_1 * \pguards{c}\big((\exists x.~ L_2) --* (\exists x.~ G_2)\big)
  } {}
} {
  \{G_1\}~c~\{\exists x.~ G_2\}
} {}
\\
[5pt]
(1)~ \text{tautology using $(P * Q |- R) <=> (P |- Q --* R)$} \qquad (2)~ \text{reduction using modal axioms K and N} %\qquad (3)~ (P |- Q) => (P * F |- Q * F)
\end{array}
\]
\caption{Proofs of \infrulestyle{Ramify-P} and \infrulestyle{Ramify-PQ}}
\label{fig:rampqproofs}
\end{figure*}
