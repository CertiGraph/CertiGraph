\paragraph{Mark example.} In Qinxiang's new format.

% inline code
\makeatletter
\newlength{\@mli}
\newcommand{\mli}[1]{%
  \settowidth{\@mli}{\lstinline/#1/}
  \hspace{-.5ex}\begin{minipage}[t]{\@mli}\lstinline/#1/\end{minipage}}
\makeatother
\newcommand{\li}[1]{\ifmmode\mbox{\mli{#1}}\else\mbox{\lstinline/#1/}\fi}

\newcommand{\tx}[1]{\text{#1}}
\newcommand{\p}[1]{\ensuremath{\mathsf{#1}}} % predicate font
\newcommand{\m}[1]{\ensuremath{\mathit{#1}}} % math font
\let\ramify\lightning
\newcommand{\infrulestyle}[1]{\textsc{#1}}
\newcommand{\infrule}[4]{\inferrule*[lab={\infrulestyle{#1}},right=$\mathrlap{#4}$]{#2}{#3}}
\newcommand{\RuleS}[4]{\infrulestyle{#1}\frac{#2}{#3} \textit{#4}}
\newcommand{\Rule}[3]{\[\RuleS{#1}{\begin{array}{c} #2 \end{array}}{#3}{}\]}
\newcommand{\MV}{\ensuremath{\mathsf{ModVar}}}
\newcommand{\FV}{\ensuremath{\mathsf{FreeVar}}}

\begin{figure}[t]
  \begin{lstlisting}
struct Node {$\label{code:nodedefstart}$
  int  _Alignas(16) m;
  struct Node * _Alignas(8) l;
  struct Node * r; };$\label{code:nodedefend}$

void mark(struct Node * x) { // $\label{code:markstart}\{\p{graph}(\tx{x},\gamma)\}$
  struct Node * l, * r; int root_mark; $\label{code:inmark}$
  if (x == 0) return;
// $\{\p{graph}(\tx x,\gamma) /| \exists m,l,r.~ \gamma(\tx{x}) = (m,l,r)\}$
// $\label{code:globalbeforerootmark}\{\p{graph}(\tx x,\gamma) /| \gamma(\tx{x}) = (m,l,r)\}$
// $\label{code:beforerootmark}\searrow \{\tx x|-> m,-,l,r \}$
$\label{code:markram1}$      root_mark = x -> m;
// $\label{code:afterrootmark}\swarrow \{\tx x|-> m,-,l,r /| m = \tx{root\_mark} \}$
// $\label{code:globalafterrootmark}\{\p{graph}(\tx x,\gamma) /| \gamma(\tx{x}) = (m,l,r) /| m = \tx{root\_mark}\}$
  if (root_mark == 1) return;
// $\{\p{graph}(\tx x,\gamma) /| \gamma(\tx{x}) = (0,l,r) \}$
// $\label{code:markbeforetripleramify}\searrow \{\tx x|-> 0,-,l,r /| \gamma(\tx{x}) = (0,l,r)\}$
      l = x -> l;
$\label{code:markram2}\ramify(\ref{lem:updategraphnode})$      r = x -> r;
      x -> m = 1;
// $\label{code:markaftertripleramify}\swarrow \{\tx x|-> 1,-,\tx{l},\tx{r} /| \gamma(\tx{x}) = (0,\tx{l},\tx{r}) /| \exists \gamma'.~ \m{mark1}(\gamma, \tx{x}, \gamma')\}$
// $\{\exists \gamma'.~ \p{graph}(\tx x,\gamma') /| \gamma(\tx{x}) = (0,\tx{l},\tx{r}) /| \m{mark1}(\gamma, \tx{x}, \gamma')\}$
// $\label{code:beforemarkl}\{\p{graph}(\tx x,\gamma') /| \gamma(\tx{x}) = (0,\tx{l},\tx{r}) /| \m{mark1}(\gamma, \tx{x}, \gamma')\}$
// $\searrow \{\p{graph}(\tx l, \gamma')\}$
$\label{code:markram3}\ramify(\ref{lem:updatesubgraph})$      mark(l);
// $\swarrow \{\exists \gamma''.~ \p{graph}(\tx l, \gamma'') /| \m{mark}(\gamma', \tx{l}, \gamma'')\}$
// $\label{code:aftermarkl}\left\{\!\!\!\begin{array}{l@{}}\exists \gamma''.~ \p{graph}(\tx x,\gamma'') /| \gamma(\tx{x}) = (0,\tx{l},\tx{r}) /| \null \\ \m{mark1}(\gamma, \tx{x}, \gamma') /| \m{mark}(\gamma', \tx{l}, \gamma'')\end{array}\right\}$
// $\left\{\!\!\!\begin{array}{l@{}}\p{graph}(\tx x,\gamma'') /| \gamma(\tx{x}) = (0,\tx{l},\tx{r}) /| \null \\ \m{mark1}(\gamma, \tx{x}, \gamma') /| \m{mark}(\gamma', \tx{l}, \gamma'')\end{array}\right\}$
// $\searrow \{\p{graph}(\tx r, \gamma'')\}$
$\label{code:markram4}\ramify(\ref{lem:updatesubgraph})$      mark(r);
// $\swarrow \{\exists \gamma'''.~ \p{graph}(\tx r, \gamma''') /| \m{mark}(\gamma'', \tx{r}, \gamma''')\}$
// $\label{code:outmark}\left\{\!\!\!\begin{array}{l@{}}\exists \gamma'''.~ \p{graph}(\tx x,\gamma''') /| \gamma(\tx{x}) = (0,\tx{l},\tx{r}) /| \null \\ \m{mark1}(\gamma, \tx{x}, \gamma') /| \m{mark}(\gamma', \tx{l}, \gamma'') /| \m{mark}(\gamma'', \tx{r}, \gamma''')\end{array}\right\}$
} // $\label{code:markend}\{\exists \gamma'''.~ \p{graph}(\tx x,\gamma''') /| \m{mark}(\gamma, \tx{x}, \gamma''')\}$
\end{lstlisting}
%% \vspace{-8pt}
\caption{Clight code and proof sketch for bigraph mark.}
% {\color{magenta} The steps that induce
%  ramifications are indicated with $\ramify_i$, where the associated ramification entailment is equation number $i$.}} %whose numbers point with their associated ramification entailment reference.}
%\vspace{-19pt}
\label{fig:markgraph}
\end{figure}


\subsection{Frames and ramifications are localizations}
The distinguishing rule of separation logic is \infrulestyle{Frame}~\cite{rey02}:
\[
\infrule{Frame}
{\{ P \} ~ c ~ \{Q \}}
{\{P * F \} ~ c ~ \{ Q * F \}}
{\begin{array}{c}F \textrm{ is stable w.r.t.} \\ \MV(c) \end{array}}
\]
The side condition ``$F \text{ is stable w.r.t.} \MV(c)$'' can be defined in two ways.
In the more traditional syntactic style, it means that $\FV(F) \cap \MV(c) = \emptyset$.
By ``syntactic style'' we mean that the side condition is written using a function $\FV(F)$ that takes an arbitrary formula and returns the set of free variables within that formula.  To define this $\FV(F)$ function
we need a fixed inductive \textbf{syntax} for formulas.  In contrast, in this paper we follow a ``semantic style'' in which formulas are not given a fixed syntax in advance but can be defined \textbf{semantically} on the fly using an appropriate model~\cite{appel:programlogics}.  In a semantic style, the side condition on the frame rule is defined as:
\[
\begin{array}{l}
\sigma \stackrel{S}{\cong} \sigma' ~~ \stackrel{\Delta}{=} ~~ \sigma \text{ and } \sigma' \text{ coincide everywhere except } S\\
P \text{ is stable w.r.t. } S ~~ \stackrel{\Delta}{=} ~~ \forall \sigma, \sigma'.~ \sigma \stackrel{S}{\cong} \sigma' => \null \\
\qquad \qquad \qquad \qquad \qquad \qquad ~~ (\sigma |= P) <=> (\sigma' |= P)
\end{array}
\]
That is, we consider two program states $\sigma$ and $\sigma'$ equivalent up to program variable set $S$ when they agree everywhere except for on the values of variables in $S$ (typically, a state $\sigma$ is a pair of a heap $h$ and program variables $\rho$).  A predicate $P$ is then stable with respect to $S$ when it ignores all local variables in $S$.  Notice both the syntactic and semantic styles use the $\MV(c)$ function defined via straightforward recursive case analysis on program syntax; programming languages typically do have a fixed syntactic structure.

Ramification
\[
\infrule{Ramify}
{\{L_1\} ~ c ~ \{L_2\} \\ G_1 |- L_1 * (L_2--* G_2)}
{\{G_1\} ~ c ~ \{G_2\}}
{\begin{array}{c}(L_2 --* G_2) \\ \textrm{is stable} \\ \textrm{w.r.t. } \MV(c) \end{array}} \qquad \qquad \qquad
%{$\begin{array}{l}\m{fv}(Q --* R') \cap \null \\ \m{modif}(c) = \emptyset\end{array}$} \qquad \qquad \qquad
\]

Notation
\begin{lstlisting}
// $\{ G_1 \}$
// $\searrow \{ L_1 \}$
      $c_1$; . . . ; $c_n$;
// $\swarrow \{ L_2 \}$
// $\{ G_2 \}$
\end{lstlisting}




Ramify implies frame

\subsection{The program variable bugaboo}

Consider using ramification to verify the following program:
\begin{lstlisting}
// $\{ \tx{x} = 5 /| G_1 \}$
// $\searrow \{\tx{x} = 5 /| L_1 \}$
      ...
      x = x + 1;
      ...
// $\swarrow \{\tx{x} = 6 /| L_2 \}$
// $\{ \tx{x} = 6 /| G_2 \}$
\end{lstlisting}
Suppose that the other (unshown) lines of the program make localization desirable, even though it is overkill for a single assignment.  The key issue is that the program variable {\li{x}} appears in all four positions in the ramification entailment
\[
(\li{x} = 5 /| G_1) \vdash (\li{x} = 5 /| L_1) * \big((\li{x} = 6 /| L_2) -* (\li{x} = 6 /| G_2)\big)
\]
This entailment has two problems.  {\color{magenta} First, the formula $(\li{x} = 6 /| L_2) -* (\li{x} = 6 /| G_2)$ is \textbf{not} stable with respect to the modified program variable \tx{x}.}  Thus, we cannot

\footnote{An alternative semantics that gets around this problem is to use ``variables as resource''~\cite{bornat:var}, but this introduces other unpleasantness.}


Unfortunately, the 

Old ramify can only handle:
\begin{lstlisting}
// $\{ \tx{x} = 5 \}$
// $\searrow \{ \tx{x} = 5 \}$
      x = x + 1;
// $\swarrow \{ \top \}$
// $\{ \top \}$
\end{lstlisting}

New rule:
\[
\infrule{Ramification-P}
{\{ L \} c \{L' \} \\
 G \vdash L * \Box^{\llbracket c \rrbracket} (L' -* G')}
{\{ G \} c \{ G' \}}{}
\]


Related?: variables as resource

\subsection{The existential ogre}

Existential rule by Floyd~\cite{floydlogic} (1967!)

\[
\infrule{Existential extraction}
{\forall x.~ \big(\{ P \} ~ c ~ \{Q \}\big)}
{\{ \exists x. P \} ~ c ~ \{ {\color{magenta}\exists x.~ Q} \}}{}
%existential_ret_assert =
%fun (A : Type) (R : A -> ret_assert) (ek : exitkind) (vl : option val) =>
%(EX  x : A, R x ek vl)%assert
%     : forall A : Type,
%       (A -> ret_assert) -> exitkind -> option val -> environ -> mpred
%Axiom extract_exists:
%  forall  {Espec: OracleKind}{CS: compspecs} ,
%  forall (A : Type)  (P : A -> environ->mpred) c (Delta: tycontext) (R: A -> ret_assert),
%  (forall x, @semax CS Espec Delta (P x) c (R x)) ->
%   @semax CS Espec Delta (EX x:A, P x) c (existential_ret_assert R).
\]

\begin{lstlisting}
// $\{ P \}$
  $c$
// $\{ \exists x.~Q \}$
// $\{ Q \}$
\end{lstlisting}


But:
\begin{lstlisting}
// $\{ G_1 \}$
// $\searrow \{ L_1 \}$
      $c$
// $\swarrow \{ \exists x.~L_2 \}$
// $\{ \exists x.~G_2 \}$
// $\{ G_2 \}$
\end{lstlisting}

Need:





