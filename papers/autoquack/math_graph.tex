Since the fully functional correctness proof needs reasoning about
pure facts of graphs, we formally built a proof framework for
mathematical graphs and proved many useful theorems in graph
theory.

In mathematics, a (directed) graph is presented as an ordered pair
$(V, E)$ where $V$ is a set of vertices or nodes and $E$ is a set of
edges comprising a source node and a destination node. In our
framework, we defined a similar structure with a small augmentation:
we attached ``validity'' property to each vertex and edge in case of
presenting partial graph structures with dangling edges. We call this
structure \textbf{PreGraph}. Many concepts such as subgraph,
reachability and structural equivalence are defined based
on \textbf{PreGraph} in our framework.

However, there are still many useful and practical properties can not
derived barely from \textbf{PreGraph}. For example, when we compute
some properties of a graph, we always hope the graph is finite
connected. When we are proving the correctness of graph algorithm, we
need a spectial node which represents null pointer. In some cases, we
have to specify a graph in which the outdegree of each nodes is 2. All
these additional properties are abstracted as different property
bundles. We proved quite many theorems for these property
bundles. This abstraction made those theorems more flexible and
general.

* the DFS algorithm on math graphs

1.1. A lot of important theorems about graph are proved.

2. Bi-graph.
2.1. BiMaFin is defined as one kind of general graph.
2.2. \texttt{pSpatialGraph\_Graph\_Bi} defined in "\texttt{spatial\_graph\_bi.v}" is the premise type class for defining BiMaFin.

1. Classification of various graphs: MathGraph, BiGraph and FinGraph. Though now we put them all in a BiMaFin, there are still many general theorems based on only one of them. The abstraction and classification is useful.

1.4. Dag predicate has not been defined as a whole. It is not hard to get fixed.
