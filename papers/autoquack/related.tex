\paragraph{Comparison with Hobor and Villard.}
The most direct ancestor of our work is \cite{hobor:ramification}, which focused on verifying graph algorithms using and introduced the \infrulestyle{Ramify} rule.  We have generalized this rule to better handle modified program variables and existential quantifiers in postconditions; they hacked their way around these issues by proposing a variant of \infrulestyle{Ramify} called \infrulestyle{RamifyAssign}, which could reason about the special case of a single assignment $\li{x=}f(\ldots)$, assuming the verifier can make the local program translation to $\li{x'=}f(\ldots)\li{; x=x'}$, where \li{x'} is fresh.  They proposed no way to verify unmodified program code, to modify program variables inside nested localization blocks, or to do a ramification across multiple assignments as we do in lines~\ref{code:markbeforetripleramify}--\ref{code:markaftertripleramify} of figure~\ref{fig:markgraph}.  Hobor and Villard avoided existentials in localized postconditions because they defined all of their mathematical operations (\emph{e.g.} $\m{mark}$, $\m{mark1}$) as functions rather than as relations.

Hobor and Villard treated mathematical graphs very simply, as triples $(V,E,L)$ of vertices, edges, and a labeling function on vertices.  Vertices had no more than two neighbors.  In contrast, our mathematical graph framework~(\S\ref{sec:mathgraph}) is very modular and general and has been tuned to work smoothly in a mechanized context.

Hobor and Villard fell into the trap of defining spatial graphs recursively~(\S\ref{sec:fixpointfail}); unfortunately other members of the research community have since followed them in.  We exposed this error and provided a sound and quite general definition for \p{graph}~(\S\ref{sec:goodgraph}) that recovers fold/unfold reasoning.  We developed a much more general and more modular set of related lemmas and connect our spatial reasoning to two very different verification tools~(\S\ref{sec:ramifylib}), VST~(\S\ref{sec:vst}) and HIP/SLEEK~(\S\ref{sec:hipsleek}).  Our development is entirely machine-checked~(\S\ref{sec:development}) whereas they used only pen and paper.

\paragraph{Other verification of graph algorithms, with or without $**$.}
Yang's verification of the Schorr-Waite algorithm is a landmark in the early separation logic literature~\cite{hongseok:phd}.  Bornat \emph{et al.} gave an early attempt to reason about graph algorithms in separation logic in a more general way~\cite{bornat:aliasing04}.

Reynolds was the first to document the overlapping conjunction~$**$, although he did not present any strategy to reason about it using Hoare rules~\cite{rey-slnotes}.  Gardner \emph{et al.} were the first to reason about a program using $**$ in Javascript~\cite{GardnerMS12}.  Raad \emph{et al.} used $**$ to reason about a concurrent spanning algorithm using a kind of ``concurrent localization''~\cite{RaadVG15}.  Sergey \emph{et al.} also verified a concurrent spanning tree algorithm and mechanized their proofs in Coq~\cite{ilya-graphs}.

Almost a decade after Yang verified Schoor-Waite on paper, Dafny automated its verification~\cite{Leino10}.

\paragraph{Local variables.}
An alternative way to avoid local variable issues is to use ``variables as resource''~\cite{bornat:var}.
However, most mechanized verification systems do not use variables as resource~\cite{Beckert:2007,DistefanoP08,chin:hipsleek,Leino10,bengtson:charge,appel:programlogics}.

\paragraph{Verification tools.}
Our work heavily interacts with the Floyd~\cite{appel:programlogics} and HIP/SLEEK~\cite{chin:hipsleek} verification tools.  Like Floyd, Charge! uses Coq tactics to work with a shallow embedding of higher order separation logic, but focuses on OO programs written in Java/C\#~\cite{bengtson:charge}.  A  more automated approach to verification of low level programs using Coq is implemented in the Bedrock framework \cite{chlipala:bedrock}.

Many automated verification tools also use separation logic in a forward reasoning style as does HIP/SLEEK, including Smallfoot~\cite{berdine:smallfoot}, jStar~\cite{DistefanoP08}, and Verifast~\cite{jacobs:verifast}.  One of HIP/SLEEK's distinguishing features is good support for user-defined inductive predicates rather than a library of pre-defined predicates for lists, trees etc.

Dafny~\cite{Leino10} and KeY~\cite{Beckert:2007} are two notable verifiers not based on separation logic; KeY uses an interactive verifier while Dafny pursues more automation by using the SMT solver Z3~\cite{Moura2008}.

\paragraph{Mechanized mathematical graph theory.} There is a long history,
going back at least 25 years, of mechanized reasoning about mathematical
graphs~\cite{wong1991}.  The most famous mechanically verified ``graph theorem''
is the Four Color Theorem~\cite{gonthier2005computer}; however
the development actually uses hypermaps instead of graphs.
Noschinski built a graph library in Isabelle/HOL whose formalization
is the most similar to ours~\cite{Noschinski2015}, \emph{e.g.} supporting
graphs with labeled and parallel arcs.

Noschinski and Dubois \emph{et al.} used proof assistants to design verifiable
checkers for solutions to graph problems~\cite{noschinski2015formalizing,dubois2015graphes}.
Yamamoto \emph{et al.} and Bauer and Nipkow use an alternative inductive
encoding of graphs to formalize planar graph theory~\cite{yamamoto1995formalization,bauer20025}.


%{\color{magenta}
%5.2. An alternative way of verifying marking program is reasoning about the whole history of marking operations. The disadvantage of it is that it currently needs more work in a Hoare logic framework. The advantage of it is that its reasoning structure are more similar with the way we understand it in our first algorithm class.
%5.3. I take some effort on garbage collector like graph structural. Though it is only connecting this special structural with 5.1.1 and 1.3, it takes much time and it is not finished yet.}


