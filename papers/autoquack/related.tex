\section{Related and future work}

The most famous graph related theorem which has been mechanically
verified is the Four Color Theorem: Any planar map can be colored with
only four colours. In 2005, Benjamin Werner and Georges Gonthier
formalized a proof of the theorem \cite{gonthier2005computer} inside
Coq. It is very easy and natural to rephrase the problem in graph
theory: by taking regions as nodes and connecting each pair of
adjacent regions as edges, coloring the map is equivalent to coloring
the graph obtained. However, they used a different kind of
combinatorial structure, known as hypermaps, instead of
graphs. Basically, a hypermap is a type ``dart'' with several
functions mapping dart to dart. The combinatorial and geometrical
properties are encoded as certain permutation properties of those
functions. It is quite a very different structure from graph.

Lars Noschinski built a formalized graph library for the Isabelle/HOL
proof assistant and verified a method of checking Kuratowski subgraphs
used in the LEDA library. It supports general infinite directed graphs
with labeled and parallel arcs \cite{Noschinski2015}.

5.2. An alternative way of verifying marking program is reasoning about the whole history of marking operations. The disadvantage of it is that it currently needs more work in a Hoare logic framework. The advantage of it is that its reasoning structure are more similar with the way we understand it in our first algorithm class.
5.3. I take some effort on garbage collector like graph structural. Though it is only connecting this special structural with 5.1.1 and 1.3, it takes much time and it is not finished yet.
