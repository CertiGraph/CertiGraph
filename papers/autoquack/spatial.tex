To prove the functional correctness of real graph-manipulating algorithms implemented in a real language, we need to connect the heap representation of graphs, the memory model of the programming language, and the mathematical properties of graphs from \S\ref{sec:mathgraph}.  The first of these turns out to be surprisingly subtle as we shall see in \S\ref{sec:fixpointfail} and \S\ref{sec:goodgraph}.  The main challenge for the others is to engineer a framework that is generic enough and modular enough to be useful in practice in a variety of settings; we cover it in~\S\ref{sec:ramifylib}.

\subsection{Traditional fixpoints fail to define good \p{graph} predicates}\label{sec:fixpointfail}

\newcommand{\graphkt}{\p{graph}_T}
\newcommand{\grapham}{\p{graph}_A}

\begin{figure*}
\[
\infrule{}
{\infrule{}
  {100 |-> 42,100,0 ~ |- ~ 100 |-> 42,100,0 ** \graphkt(100,\hat{\gamma})}
  {100 |-> 42,100,0 ~ |- ~ \hat{\gamma}(100) = (42,100,0) ~ /| ~ 100 |-> 42,100,0 ** \graphkt(100,\hat{\gamma}) ** \graphkt(0,\hat{\gamma})}
  {(2)}
}
{100 |-> 42,100,0 ~ |- ~ \graphkt(100,\hat{\gamma})}
{(1)}
\]
(1) Unfold $\graphkt$, dismiss first disjunct (contradiction), introduce existentials (which must be 42,100,0) \\
(2) simplify using $P * \p{emp} -|- P$ and remove pure conjunct

\caption{An honest academic tries to prove a ``simple'' entailment}
\label{fig:badcycle}
\end{figure*}

Recursive predicates are ubiquitous in separation logic---so
much so that when a person writes the definition of a predicate as
\mbox{$P$ ``$\defeq$'' $\ldots P \! \ldots$}, no one raises an eyebrow despite the
dangers of circularity in mathematics. Indeed, the vast majority of the time there
is no danger thanks to the magic of the Knaster-Tarski fixpoint
$\mu_{\mathsf{T}}$ \cite{tarski:fixpoint}.  Formally what is going on
is instead of defining $P$ directly, one defines a functional
\mbox{$F_P \defeq \lambda P.~ \ldots P \! \ldots$} and then defines $P$ itself as
\mbox{$P \defeq \mu_{\mathsf{T}} \, F_P$}.  Assuming (as one typically does
without comment) that $F_P$ is \emph{covariant}, i.e. $(P \vdash Q)
\Rightarrow (F \, P \vdash F \, Q)$, one then enjoys the fixpoint
equation $P \Leftrightarrow \ldots P \ldots$, formally justifying
typically written pseudodefinition (``$\defeq$'').

%Definition corec {B A: Type}  (F: (B -> pred A) -> (B -> pred A)) : B -> pred A :=
%fun x w => forall P: B -> pred A, (forall x, F P x |-- P x) -> P x w.

Suppose we define a graph predicate $\graphkt$ this way, \emph{e.g.} along the lines of the fold/unfold definition we used in Figure~\ref{fig:markgraph}, \emph{i.e.}
\[
\begin{array}{@{}l@{}l@{}}
\graphkt(x, \gamma) ~ \stackrel{\Delta}{=} ~ & (x = 0 /| \p{emp}) |/ \exists m,l,r.~ \gamma(x)=(m,l,r) /| \null \\
& x |-> m,l,r ** \graphkt(l, \gamma) ** \graphkt(r, \gamma)
\end{array}
\]
We have removed the alignment-related portions of equation~\eqref{eqn:bigraphintrofoldunfold} to focus on a more serious issue, even though as we will explain in~\S\ref{sec:foldunfold} alignment concerns are also necessary for the fold/unfold relationship to hold in C-like memory models.  The functional needed to define $\graphkt$ is covariant, so we can apply Knaster-Tarski soundly.  However, the resulting predicate can be hard to use.

Consider the following partial memory $m$ for a toy machine:

\begin{minipage}{.24\textwidth}
\qquad \[
\begin{array}{l|l}
\textrm{address} & \textrm{value} \\
\hline
102 & 0 \\
101 & 100 \\
100 & 42 \\
\end{array}
\]
\end{minipage}
\begin{minipage}{.19\textwidth}
\centering
\beginpgfgraphicnamed{selfref}
\begin{tikzpicture}
[->/.style={thick,arrows={-Stealth}},
   propG/.style={shape=circle, draw}]
   \path[use as bounding box] (-1, -1) rectangle (0.5, 0.5);
   \node[propG] (P) at (0, 0) {42};
   \draw[->] (P.west) .. controls (-1.5, 0) and (0, -1.5) .. (P.south);
\end{tikzpicture}
\endpgfgraphicnamed
\end{minipage}
\vspace{0.75ex}

\noindent Clearly $m |= 100 |-> 42,100,0$.  But it seems also clear that this memory represents a one-cell cyclic graph as illustrated in the accompanying diagram, \emph{i.e.} we want $m |= \graphkt(100,\hat{\gamma})$, where $\hat{\gamma}(100) = (42,100,0)$.  This is equivalent to wanting to be able to prove $100 |-> 42,100,0 |- \graphkt(100,\hat{\gamma})$.  Unfortunately, as hinted at in Figure~\ref{fig:badcycle}, this seems rather difficult to do so since applying the natural proof techniques have only strengthened the goal.  In fact we do not know if this entailment is provable or not, but the difficulties encountered in proving what ``should be'' straightforward suggest that Knaster-Tarski should be treated with caution when defining spatial predicates for graphs.

Part of the problem is that the recursive structure interacts very badly with $**$: if the recursion involved $*$ then it \textbf{would} be provable, by induction on the finite memory (each ``recursive call'' would be on a strictly smaller subheap).  This is why Knaster-Tarski works so well with list, tree, and DAG predicates in separation logic.  Note that the other direction, \mbox{$\graphkt(100,\hat{\gamma}) |- 100 |-> 42,100,0$},
\textbf{is} true but is not easy to prove, relying on the constructions in \S\ref{sec:foldunfold} and the fact that $\mu_{\mathsf{T}}$ constructs the least fixpoint.  In contrast, $\graphkt(100,\hat{\gamma}) |- 100 |-> 42,100,0 * \top$ is very easy to prove. % via fold/unfold.


%As explained in~\S\ref{sec:foldunfold}, this alignment is necessary in C-like memory models to prove fold-unfold \eqref{eqn:bigraphintrofoldunfold}, which is why \eqref{eqn:bigraphintrofoldunfold} includes an alignment restriction $x~\mathsf{mod}~16 = 0$ and an existentially-quantified ``blank'' second field for the root $x \mapsto m,-,l,r$.

%As shown in (\ref{eqn:bigraphintrofoldunfold}) of Section
%\ref{sec:orientation}, Hobor and Villard\cite{hobor:ramification}
%defined the separation logic graph predicate
%$\mathsf{graph}(x,\gamma)$ in direct analogy to the standard
%separation logic definition of a tree. Note that the two-neighborhood
%means $\gamma$ is a BiGraph. However, it is peculiarly challenging in
%rigorously formalizing $\p{graph}$.

%* Showing that neither traditional fixpoint method works

Appel and McAllester proposed another fixpoint $\mu_{\mathsf{A}}$
that is sometimes used to define recursive predicates in separation
logic \cite{appel:fixpoint}.  This time the functional $F_P$ needs to be
\emph{contractive}, which to a first order of approximation means that
all recursion needs to be guarded by the ``approximation
modality''~$\rhd$~\cite{appel:vmm}, \emph{i.e.} our graph predicate would
look like
\[
\begin{array}{@{}l@{}l@{}}
\grapham(x, \gamma) ~ \stackrel{\Delta}{=} ~ & (x = 0 /| \p{emp}) |/ \exists m,l,r.~ \gamma(x)=(m,l,r) /| \null \\
& x |-> m,l,r ** \rhd \graphkt(l, \gamma) ** \rhd \graphkt(r, \gamma)
\end{array}
\]

As we will see in \S\ref{sec:hipsleek}, the forward style of reasoning employed by
HIP/SLEEK is greatly aided when predicates are \emph{precise}, \emph{i.e.}
\[
\begin{array}{@{}l@{}l@{}}
\m{precise}(P) ~~ \defeq ~~ & (\sigma_1 |= P) => (\sigma_2 |= P) => \\
& ~~ (\sigma_1 \oplus \sigma_1' = \sigma) => (\sigma_2 \oplus \sigma_2' = \sigma) => \sigma_1 = \sigma_2
\end{array}
\]
Unfortunately, $\rhd P$ is not precise for all $P$, so $\grapham$ is not precise either.  The approximation modality's universal imprecision has never been mentioned previously in the literature.  %We must do better.

\subsection{Defining a good \p{graph} predicate}\label{sec:goodgraph}

We choose an alternative path.  Rather than trying to define \p{graph} directly as a recursive fixpoint, we will give it a flat structure and then \textbf{prove} that it satisfies fold/unfold.  Our path starts with the iterated separating conjunction or ``big star'', defined as follows:
\begin{equation*}
  \underset{\{l_1, l_2,\dots,l_n\}}{\bigstar}P ~~ \defeq ~~ P(l_1) *
  P(l_2) * \dots * P(l_n).
\end{equation*}
Notice that formally $\bigstar$ is defined over a list rather than a set, and is parameterized by a predicate $P$.  It is natural to extend it to a set $S$ using an existentially-quantified duplicate-free list $L$ as follows:
\[
\underset{S}{\bigstar} P ~~ \defeq ~~ \exists L.~ (\p{NoDup}\ L) /| (\forall x.~ x\ \p{in}\ L <=> x \in S) /| \underset{L}{\bigstar}P
\]
We use the same $\bigstar$ notation since the concepts are similar, but the existential can add a little pain since it means that we need to prove that all choices of list $L$ yield equivalent predicates.

We are now ready to give a good \p{graph} predicate:
\begin{equation}\label{eqn:iter_def}
  \p{graph}(x, \gamma) ~~ \defeq ~~ \underset{v \in \mathit{reach}(\gamma, x)}{\bigstar} v\mapsto\gamma(v)
\end{equation}
Here~$\gamma$ is a GeneralGraph from~\S\ref{sec:mathinfra} and ``$x |-> \gamma(x)$'' is a predicate that describes how single nodes fit in memory; in Figure~\ref{fig:markgraph} it was
\[
\exists m,l,r.~\gamma(x) = (m,l,r) /| x |-> m,-,l,r /| x\ \p{mod}\ 16 = 0
\]
In general $\gamma$ need not be a bigraph, but \emph{e.g.} can have many edges.
%,  it can have more edges or data fields.
%; $\gamma$ is in no way limited to bigraphs.

Our definition of \p{graph} is flat in the sense that there is no obvious way to follow the link structure recursively.  Happily, we can recover a general recursive fold/unfold, assuming our \p{graph} and GeneralGraph give us the necessary properties.  To state a general fold/unfold lemma we need the iterated overlapping conjunction:
\[
\bigocon_{l_1,\dots,l_n} \!\! P ~~ \defeq ~~ P(l_1) ** \ldots ** P(l_n)
\]
Now we can state the general fold/unfold as follows:
\begin{equation}
\label{eqn:unfold_graph}
\p{graph}(x,\gamma) ~~ <=> ~~ x |-> \gamma(x) ** \Big(\underset{n \in \p{neighbors}(\gamma,x)}{\raisebox{-0.3ex}{\resizebox{0.75em}{!}{$\scon$}} \hspace{-2.45ex} \bigcup} \p{graph}(\gamma,n) \Big)
\end{equation}
In other words, we get a full equivalence between a graph and its ``unfolded'' structure, regardless of how many neighbors $x$ has.

Both directions of this lemma are a little subtle.  In the $=>$ direction, the key difficulty is that we need to take the existentially-quantified list of vertices on the left $L$ and divide it into (not necessarily disjoint) sublists $L_1, \dots, L_n$ such that $L_i$ is exactly those vertices reachable from neighbor $i$.  To construct these lists we need to use the computable reachability lemma~\ref{lem:computereach} to explicitly test, for each neighbor $i$ and $v\ \p{in}\ L$, whether $i$ can reach $v$; accordingly we require that the GeneralGraph be appropriately finite.

In the $<=$ direction, the difficulty is that if two nodes $x |-> \gamma(x)$ and $x' |-> \gamma(x')$ are \emph{skewed}, \emph{i.e.} ``partially overlaping'' with some---but not all---of $x$'s memory cells shared with $x'$, then the $\bigstar$ on the left hand side cannot separate them.  To avoid skewing we require $x |-> \gamma(x)$ be \emph{alignable}.  A predicate $P$ is alignable when
\[
\forall x,y.~ \Big(P(x) ** P(y) |- \big(P(x) /| x = y\big) |/ \big(P(x) * P(y)\big)\Big)
\]
In other words, either they are completely on top of one another or they do not interfere at all.  In a Java-like memory model such as in HIP/SLEEK this property is automatic because pointers in such a model always point to the root/beginning of an object.  In contrast, in a C-like memory model such as in VST/CompCert, this property is not automatic because pointers can point anywhere.  In such a model, alignment is most easily enforced by storing graph nodes at addresses that are multiples of an appropriate size (16 in Figure~\ref{fig:markgraph}).

Some of our VST proofs do not use fold/unfold, instead preferring to use the lemmas in~\S\ref{sec:ramifylib} directly.  On the other hand, for HIP/SLEEK fold/unfold is vital, and knowing that the recursive relationship holds produces a pleasant feeling.  We also prove a general fold/unfold lemma for DAGs in which we get a $*$ between the root and its $\medocon$-joined neighbors rather than the $**$ present in \eqref{eqn:unfold_graph}.

\subsection{Ramification Libraries}\label{sec:ramifylib}
%\subsection{Infrastructure: Isolate memory model}

In order to ensure proofs reusability, we adopt
a layered style to get things done.

First we build a logic layer between
memory models and our ramification library. 

As shown in Figure~\ref{fig:infra}, the second bottom layer is a separation logic layer which are composed by atomic
notations ($\scon$, $\ocon$, $|->$, $\p{precise}$, etc.) and a small set of axioms. These axioms are proved sound
w.r.t. different memory models at the bottom layer. Using this structure, all of our further development of ramification library can be just depended on this small set of axioms and all properties proved from it will be sound w.r.t. all memory models.

\begin{figure}[htbp]
\centering
\beginpgfgraphicnamed{infrastructure}
\begin{tikzpicture}[
->/.style={thick, arrows={-Stealth}},
ent/.style={shape=rectangle, rounded corners=4pt, draw, on grid}]
\node[ent] (SM) at (0, 0) {\small Step-Indexed Model};
\node[ent] (DM) [right=4.4 of SM] {\small Direct Model};
\node[ent] (CL) [above=1 of SM] {\small Core Logic};
\node[ent] (SL) [above=1 of DM] {\small Supplementary Logic};
\node[ent] (LF) [above left=1 and 1.2 of CL] {\small Logic Facts};
\node[ent] (RF) [above right=1 and 1.2 of CL] {\small Basic Ramification};
\node[ent] (BF) [above=1 of LF] {\small $\bigstar$ Facts};
\node[ent] (BR) [above=1 of RF] {\small $\bigstar$ Ramification};
\node[ent] (GF) [above=1 of BF] {\small Graph Facts};
\node[ent] (GR) [above=1 of BR] {\small Graph Ramification};
\node[ent] (SLF) [above=1 of SL] {\small Supplementary Logic Facts};
\node[ent] (SBF) [above=1 of SLF] {\small Supplementary $\bigstar$ Facts};
\node[ent] (SGF) [above=1 of SBF] {\small Supplementary Graph Facts};
\draw [double, ->] (SM) to (CL);
\draw [double, ->] (SM) to (SL);
\draw [double, ->] (DM) to (CL);
\draw [double, ->] (DM) to (SL);
\draw [->] (CL) to (LF);
\draw [->] (CL) to (RF);
\draw [->] (CL) to (SL);
\draw [->] (SL) to (SLF);
\draw [->] (SLF) to (SBF);
\draw [->] (SBF) to (SGF);
\draw [->] (LF) to (RF);
\draw [->] (LF) to (BF);
\draw [->] (RF) to (SLF);
\draw [->] (RF) to (BR);
\draw [->] (BF) to (BR);
\draw [->] (BF) to (GF);
\draw [->] (GF) to (GR);
\draw [->] (GR) to (SGF);
\draw [->] (BR) to (GR);
\draw [->] (BR) to (SBF);
\node (legend1) [below right=0.2 and -1.2 of SM] {\small Dependence};
\coordinate[left=0.8 of legend1]  (l1);
\draw [->] (l1) to (legend1);
\node (legend2) [right=1 of legend1] {\small Instantialization Choices};
\coordinate[left=0.8 of legend2]  (l2);
\draw [double, ->] (l2) to (legend2);
\end{tikzpicture}
\endpgfgraphicnamed
\vspace{1ex}
\caption{Infrastructure of ramification library}\label{fig:infra}
\end{figure}

Besides the isolation of memory models, all of our ramification entailments and spatial graph properties are also constructed in a modular way.

We first build the some basic facts about ramification upon the core logic. For example, the following rule is useful in building ramification entailments combinationally.
\[
\infrule{Ramify-Q-SPLIT}
{G_1 \vdash L_1 * \forall x.  (L_2 --* G_2) \\
 G'_1 \vdash L'_1 * \forall x.  (L_2' --* G'_2)}
{G_1 * G'_1 \vdash (L_1 * L'_1) * \forall x.  ((L_2 * L'_2)--* (G_2 * G'_2))}{}
\]

MSL also provides a rich and useful separation logic library based on the core logic.

The next layer is separation logic theorems and ramification entailments for iterated separating conjunction. For example:
\[
\infrule{}
{A \cap B = \emptyset}
{\underset{x\in A}{\bigstar} P(x) *   \underset{x\in B}{\bigstar} P(x) \Leftrightarrow \underset{x\in A \cup B}{\bigstar} P(x)}{}
\]
\[
\infrule{}
{A \cap B = \emptyset,  ~~~  A' \cap B = \emptyset}
{\underset{x\in A\cup B}{\bigstar} P(x) \vdash \underset{x\in A}{\bigstar} P(x) * ( \underset{x\in A'}{\bigstar} P(x) \wand \underset{x\in A' \cup B}{\bigstar} P(x))}{}
\]

As in Figure~\ref{fig:infra}, the top layer in this library is the general separation logic theorems and ramification entailments for $\mathtt{graph}$ predicate. For example, the following rule describes the behavior of updating one single vertex in a graph. 
\[
\infrule{}
{\gamma(x) = \gamma'(x) \text{ for any } x \neq x_0 }
{\mathtt{graph}(x, \gamma) \vdash x \mapsto \gamma(x) * (x \mapsto \gamma'(x) \wand \mathtt{graph}(x, \gamma'))}{}
\]

This layered structure mostly enables proof reuse. All of the theorems for $\mathtt{graph}$ are proved from the properties of iterated separating conjunction while this library for \emph{big star} can be used in other structures than graphs.

Also, all of our verifications of different graph algorithms use the proof rules of $\mathtt{graph}$ at the top level in the library. Taking the marking algorithm we introduced in \S\ref{sec:orientation} as an example, we prove the following theorem from the library:
\begin{eqnarray*}
&& \text{If } \gamma(x) = (1, l, r) \text{ then } \\
&& \mathtt{graph}(x, \gamma) \vdash \mathtt{graph}(l, \gamma) * \\
&& ~~ (\forall \gamma', \text{mark}(\gamma, l, \gamma') \wedge \mathtt{graph}(l, \gamma') \wand \\
&& ~~~~~~~~~~~~~~~~ \text{mark}(\gamma, l, \gamma') \wedge \mathtt{graph}(x, \gamma'))
\end{eqnarray*}


%One benefit of the definition in (\ref{eqn:iter_def}) is that the pure
%mathematical graph $\gamma$ in $\mathtt{graph}$ is not necessarily a
%BiGraph. (\ref{eqn:iter_def}) can represent a general graph with
%variant number of neighors as long as extending the definition of
%$\gamma(x)$ to data mapped by the label function and every neighbor of
%node $x$.
%
%Moreover, it turns out that the $\bigstar$ notation is a more useful
%and fundamental concept than $\mathtt{graph}$. There are two parts of
%the $\bigstar$ in (\ref{eqn:iter_def}): one is the predicate $\mapsto$
%and the other is the node set which the $\mapsto$ iterates on. They
%both bind to $\gamma$ in (\ref{eqn:iter_def}) for $\p{graph}(x,
%\gamma)$, which is a special case. In section \ref{sec:applicable}, we
%will see the specification of a spanning tree algorithm which uses
%$\bigstar$ directly instead of $\p{graph}$ because in that
%specification, the predicate $\mapsto$ and the node set bind to
%different mathematical graphs. Furthermore, we generalize the
%ramification rules for $\p{graph}$ in \cite{hobor:ramification}, which
%uses $\bigstar$ so as to be applied in all verification examples.

%% 1.2. \texttt{Iter\_sepcon} and \texttt{pred\_sepcon} are defined. And related ramification rules are proved.
%% 1.3. The most general graph-spatial-predicate \texttt{vertices\_at} are defined (for all possible styles of graphs). Related ramification rules are proved. Graph and graphs are defined as special cases of vertices at.

%% 2. A minor implementation trick. There are many tactics defined in \texttt{msl\_ext/ramify\_tactics.v}, which can manipulate low level heaps efficiently.

%% * Separating the material into the general vs. tool-specific part.  Measurements of etc.
