Dijkstra's eponymous shortest-path algorithm~\cite{DBLP:journals/nm/Dijkstra59} finds
the cost-minimal paths from a distinguished \emph{source} vertex
source to all reachable vertices in a finite directed graph.
The algorithm is classic and ubiquitous, appearing widely in textbooks~\cite{clrs}
and in real routing protocols. Further, the algorithm has been in
use for over $60$ years, suggesting that for all practical purposes
its safety and correctness have been verified by decades of application.

%Recent efforts [cite Mizar, cite ACL2, cite Coq] have implemented the algorithm
%in proof assistants and formally proved claims about its behavior.
%However, because they operate entirely within idealized formal checkers,
%these works inadvertently gloss over certain classes of issues
%that routinely crop up in real-world settings.

Here we verify a~C~implementation of Dijkstra's one-to-all shortest path algorithm.
We implement textbook~C~code~\cite{clrs} in CompCert~C~\cite{leroy:compcert} so that
we can use the Verified Software Toolchain~\cite{appel:programlogics} and
Wang~\emph{et al.}'s recent framework for verifying graph algorithms~\cite{DBLP:journals/pacmpl/WangCMH19} to state and prove the full functional
correctness of the code using the Coq proof assistant~\cite{coq}.
We expose a subtle overflow issue in the~C~code and make a nontrivial refinement to the precondition of the algorithm so that users know when the code will calculate the correct paths.

The remainder of this paper is organized as follows:
\begin{itemize}
    \item[\S\ref{sec:overview}] We briefly present and explain
    our~C~implementation of Dijkstra's algorithm and the key loop invariants
    we use to certify its specification.
    \item[\S\ref{sec:overflow}] We explain how overflow can occur and give a
    refinement to the precondition of the function that restricts edge weights
    and the standard \texttt{INF} definition
    appropriately in a way that prevents said overflow.
    \item[\S\ref{sec:conclusion}] We conclude by putting this result in the context of our
    previous and ongoing work. We discuss related work in certifying Dijkstra's algorithm, and revisit standard algorithms textbook presentations.
\end{itemize} 

\noindent Our code is available at \url{https://github.com/anshumanmohan/TODO}.
