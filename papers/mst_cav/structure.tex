%\subsection{Modularity of library and VST verification}
In previous work by both VST and CertiGraph, the C functions and algorithms implemented and verified are isolated programs with little to no dependency on each other. Even the garbage collector verified by Wang et al\cite{DBLP:journals/pacmpl/WangCMH19} was independent of the other verified algorithms in the library. VST has noted the importance of having a client for a verified program, to judge whether the actual utility of the verified specifications\cite{DBLP:conf/iwmm/AppelN20}. Our work on Kruskal's algorithm is the first step in verifying code that uses \textit{previously} verified C code. To make use of Wang's verified union-find, we reorganised the internal hierarchies in his CertiGraph library, providing a clearer separation between mathematical lemmas, VST specifications and proofs.

%\includegraphics[scale=0.50]{structure_placeholder.jpg} %Please check in this file.

We reorganised CertiGraph into three layers: The mathematical layer which contains ``pure Coq" mathematical models and lemmas; the spatial layer to represent graphs in Verifiable C; and the verification layer, for specifications and verifications of C code, whose ASTs were retrieved from CompCert's \textit{clightgen} utility. We further separate this third layer into specifications and verifications. This allows reuse of a previous specification by another system without being burdened by the verification, as illustrated above. The development and verification of components can then be performed in parallel.

In addition, we have worked to improve on the modularity of the mathematical layer, ensuring that similar graph properties and lemmas can be reused by different proofs with little need for repetition. Our new contributions to CertiGraph include approximately:
\begin{itemize}
\item2000 lines of lemmas of undirected graph properties
\item200 lines to link unionfind properties with undirected graph properties
\item1500 lines for mathematical properties of graphs that fit in symmetric matrices, including a proof of the existence of minimal spanning forests
\item900 lines for that of edge lists
\item1800 lines \textit{each} for the verification of Prim's and Kruskal's algorithms, including the mathematical proofs of the resultant forest's minimality and helper C functions.
\end{itemize}