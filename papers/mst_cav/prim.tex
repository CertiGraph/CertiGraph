Here we discuss our verifications of the classic MST algorithms Prim and Kruskal.  Although our machine-checked proofs are about real~C~code, in this section we take a higher-level approach than we did in \S\ref{sec:dijkstra}, focusing on our key algorithmic findings and overall experience.  Accordingly, we only provide pseudocode for Prim's algorithm rather than a decorated program and do not show any code for Kruskal's.  Our development contains our~C~code and formal proofs~\cite{blah}.

\subsection{Verified Prim's Algorithm in C}
\label{sec:prim}

\begin{figure}
\[
\begin{array}{@{}l@{~~}|@{~}l@{}}
\begin{minipage}{0.5\textwidth}
\begin{lstlisting} %[mathescape=true,showlines=true]
MST-PRIM(G,w,r):
 for each u in G.V
  u.key = INF
  u.parent = NIL
 r.key = 0 $\label{code:primsetroot}$
 Q = G.V
 while Q $\neq$ $\emptyset$
  u = EXTRACT-MIN(Q)
  for each v in G.Adj[u]
   if v $\in$ Q and w(u,v) $<$ v.key
    v.parent = u
    v.key = w(u,v) $\hide{code:primeditpri}$
\end{lstlisting} \end{minipage} &
\begin{minipage}{0.5\textwidth}
%\lstset{firstnumber=3}
\begin{lstlisting}
MST-NOROOT-PRIM(G,w):
 for each u in G.V
  u.key = INF
  u.parent = NIL

 Q = G.V
 while Q $\neq$ $\emptyset$
  u = EXTRACT-MIN(Q)
  for each v in G.Adj[u]
   if v $\in$ Q and w(u,v) $<$ v.key
    v.parent = u
    v.key = w(u,v)
\end{lstlisting}
\end{minipage}
\end{array}
\]
\caption{Left: Prim's algorithm from CLRS~\cite{clrs}. Right: the same omitting line~\ref{code:primsetroot}.}
\label{fig:prims}
\end{figure}

We put the pseudocode for Prim's algorithm in figure~\ref{fig:prims}; the code on the left-hand side is directly from CLRS, whereas the code on the right omits line~\ref{code:primsetroot} and will be discussed in~\S\ref{sec:primforest}.  Note that line~\ref{code:primeditpri} contains an implicit call to the PQ's \texttt{edit\_priority}.  Since the pseudocode only compares keys (\emph{i.e.}, edge weights) rather than doing arithmetic on them \emph{\`a la} Dijkstra, there are no potential overflows and it is reasonable to set \texttt{INF} to \texttt{INT\_MAX} in~C.

Indeed, our initial verifications of~C~code were largely ``turning the crank'' once we had developed the definitions and associated lemma support for pure/abstract undirected graphs, trees, etc. \note{discussed in \S\ref{sec:undirected}}.  Accordingly, from an engineering perspective, our initial contribution was a demonstration that this new graph machinery was sufficient to verify real code.  \note{We also demonstrated that our pure reasoning was isolated from our spatial reasoning by verifying three versions of~C~code with slightly different memory layouts for adjacency matrices with the same pure theorems (our proof base for these versions differs by less than 1\%, exactly for the portion of the~C~code that accesses the graph representation in memory)}.  By reusing much pure and spatial reasoning that had been originally developed for our verification of Dijkstra, we also demonstrated that our extensions to CertiGraph \note{described in \S\ref{sec:undirected}} were generic rather than verification-specific.

\subsection{Base algorithm of Prim's returning a forest}
\label{sec:primforest}

%  We make two immediate observations on CLRS's pseudocode: first, that they do not give an explicit definition for

% Note that  does not explicitly define \texttt{EXTRACT-MIN}

Textbook discussions of Prim's algorithm are often limited to the case where the input graph is a connected graph. This is reasonable, as their purpose is to teach the concept of minimum-spanning \textit{tree} algorithms. However, they seldom expand on the subject of spanning forests for disconnected graphs. For example, CLRS does not mention the case of disconnected graphs. \textit{Discrete Mathematics and Its Applications}~\cite{rozen} leaves it as an exercise to the reader, while \textit{Graph Algorithms in the Language of Linear Algebra}~\cite{kepnergilbert} suggests running Prim's on each component of the disconnected graph to obtain a minimum spanning forest. The last appears to be the most common solution, with lectures and practical implementations suggesting this.

We observed that our code is able to return a minimum spanning forest when the input graph is disconnected, \textit{in a single run}. The reason for this is: We treat a vertex of weight \texttt{INF} and a vertex's membership in the priority queue as \textit{separate} matters. A vertex can be in the priority queue with weight \texttt{INF}, represented by \texttt{key[u] = INF}, and this indicates that \texttt{u} is \textit{not} connected to any previously popped vertex; otherwise its weight would have been previously lowered. However, our priority queue does not care about its items having specific weights, only that the item it pops is always the one with the lowest weight. Thus, when a vertex \texttt{u} is returned, it is possible that \texttt{u} has weight \texttt{INF} - the scenario where all vertices remaining in the queue has weight \texttt{INF}, indicating all of them are disconnected from the current forest.

\begin{figure}[H]
\begin{tikzpicture} [auto, node distance =2 cm and 2cm ,on grid, semithick, state/.style ={circle}]
\node[state] (V0) {$\m{v_0}$};
\node[state] (V1) [right=of V0] {$\m{v_1}$};
\node[state] (V2) [below=of V0] {$\m{v_2}$};
\node[state] (V3) [below right=of V0] {$\m{v_3}$};
\node[color=green] (V4) [right=of V3] {$\m{v_3}$};
\node[color=green] (V5) [right=of V4] {$\m{v_5}$};
\draw[color=red] (V0) edge [ultra thick] node[above=0.15 cm] {$5$} (V1);
\path (V0) edge node[left=0.15 cm] {$6$} (V2);
\path (V0) edge [thin] node[above right=0.05 cm] {$5$} (V3);
\path[color=red] (V1) edge [ultra thick] node[right=0.15 cm] {$5$} (V3);
\path[color=red] (V2) edge [ultra thick] node[below=0.15 cm] {$4$} (V3);
\path (V4) edge node[below=0.15 cm] {$1$} (V5);
\end{tikzpicture}
\caption{Figure of a partial Prim's execution with root $\m{v_0}$. A spanning tree for the left component has been found as indicated in red, while $\m{v_4}$ and $\m{v_5}$ are in the priority queue with weight \texttt{INF}. As our \texttt{EXTRACT-MIN} tolerates \texttt{INF}, our implementation will pop them from the priority queue and proceed as usual, instead of terminating with only the left tree.}
\end{figure}

From the above figure, we demonstrate that our simple priority queue will always pop vertices regardless of its weight. If the popped vertex has weight \texttt{INF}, its parent will also be at its default invalid value, indicating that no edge was added to the graph. We then continue the algorithm as usual, updating the weights of adjacent vertices. As a result, our Prim implementation can return a minimal spanning forest without ``premature termination".

Is our assumption that \texttt{EXTRACT-MIN} can pop \texttt{INF} weight vertices reasonable? We argue that it is, because the abstract algorithm in CLRS makes no statement about \texttt{INF} beyond the initialization. As the algorithm pushes the vertices into queue with weight \texttt{INF}, it is reasonable to say the priority queue can tolerate items with weight \texttt{INF}. Doing so simplifies \texttt{EXTRACT-MIN} to a \texttt{popMin} operation using the priority queue's API, without requiring additional lines of code to further check for specific weights.

Consequently, our code allows a simple implementation of Prim's to return a forest for a disconnected input graph, in a single run of the algorithm without needing to identify the disconnected components beforehand. It is important to note that we do not explicitly define ``components", nor do we tag \texttt{u} as a``new root". It is sufficient to prove that the loop invariant is satisfied whether \texttt{key[u]~< INF} or \texttt{key[u] = INF}.

Note that the priority queue used for this verification is still tied to \texttt{INF}, and thus in the verification our argument is weakened to ``\texttt{INF} is a valid weight in the priority queue" rather than ``membership in the priority queue is independent of \texttt{INF}". Work on a stronger priority queue which completely dissociates from \texttt{INF} was to be delivered by Aquinas Hobor since May 2020.

A further observation is that if we dissociate \texttt{INF} and the priority queue, then the input root required by Prim's is no longer necessary. The root's key is artificially set to 0 in Prim's, which kickstarts the main loop. However, since our emptiness check and \texttt{EXTRACT-MIN} implementation can return vertices with \texttt{INF} weight, the loop will start as long as there are un-popped vertices. Thus, we suggest it is possible to remove the \texttt{rt} parameter. Doing so simplifies the proof, because we do not have to reason about the artificially weighted \texttt{key[rt]}. The noroot-variant is in Figure 4 at the beginning of this section, while the verified implementation of Prim's without root is in \texttt{noroot\_prim.c}. 