\note{We detail the major extensions we made to the 
CertiGraph library.}

\subsection{Undirectedness in a directed world}

The CertiGraph library presented in~\cite{DBLP:journals/pacmpl/WangCMH19} 
supports only directed graphs, baking direction-reliant 
idioms 
% properties?
% constructs?
such as 
\texttt{src} and \texttt{dst} : \texttt{EdgeType} $\rightarrow$ \texttt{VertexType}
deep into its development.
Our challenge is to add support for undirected graphs. 

One approach is to handle undirected edges entirely separately 
from directed edges, \emph{e.g.} using Coq \texttt{Ensemble}s. 
This has the benefit of allowing 
mixed graphs, which contain directed and undirected edges simultaneously. 
Sadly, CertiGraph's tried-and-tested suite of pure formalized graph theory is
% fundamentally 
% hopelessly
incompatible with such undirected edges, and so a new 
``tower'' of development is necessary. 
This is unpleasant both because it forces us to start
from scratch and because it creates 
a
%an insurmountable
%divide
fissure 
within an otherwise unified library. 

The approach we take is more harmonious: 
we observe that every directed graph can be treated as an undirected 
graph by ignoring edge direction.
As we will soon detail, we develop a lightweight layer of 
``undirected flavored'' definitions atop of the existing 
``directed flavored'' definitions, state and prove connections 
between these, and are off to the races. 
We sacrifice the ability to represent mixed graphs---a given
graph needs to either consider edge direction or ignore~it---but 
we gain full access to CertiGraph's graph theory formalizations 
modulo some mathematical bridging.
% It also allows us to use CertiGraph's rich set of lemmas about the addition and removal of edges.
% This is backed by our observation that undirected graph properties are largely mathematical in nature, 

\hide{\begin{lstlisting}
Record PreGraph {EV: EqDec Vertex eq} {EE: EqDec Edge eq} := {
	vvalid : Ensemble Vertex;
	evalid : Ensemble Edge;
	src : Edge -> Vertex;
	dst : Edge -> Vertex
}.
\end{lstlisting}
}% end hide
% Anshuman: keeping this in hide in case you want to show it to explain 
% "how very ingrained" directedness is in CertiGraph.

% This makes sense in the first use cases of CertiGraph, especially when reasoning about C pointers.


\subsubsection{Our basic ``undirected'' definitions.}
%Our definitions are largely based on CLRS~\cite{clrs}.
Two vertices \m{u} and \m{v} are \texttt{adjacent} if there exists
an edge between them in either direction; vertices are self-adjacent.
An valid \texttt{upath} (undirected path) is list of 
\note[say more? cut?]{valid} 
vertices that form a  
pairwise-adjacent chain. Two vertices are \texttt{connected} when a valid \texttt{upath}
features them as head and foot; essentially this is the transitive 
closure of \texttt{adjacenct}.
%A similar approach is proposed in Halsbeck~\cite{DBLP:journals/afp/HaslbeckLB19}.
% Note that the definition of path varies between textbooks and papers. For example, \textit{Discrete Mathematics and its Applications}~\cite{rozen} defines paths as a sequence of edges with an implicit sequence of vertices, whereas CLRS, which we have followed, defines it as a sequence of vertices with an implicit sequence of edges. 

\subsubsection{Bridging with directedness.}
The definitions above sync up nicely with preexisting directed
definitions. 
\note[cut?]{Intuitively, undirectedness is more lax than directedness, 
and so it is unsurprising that these connections are straightforward
weakenings of directed properties.}
For example, a directed \texttt{path~: VType~* list~EdgeType}
is valid when the \texttt{src} of the first edge in the list 
is the vertex supplied, and the subsequent edges are linked up 
in \texttt{src}/\texttt{dst} pairs
as one may expect. A straightforward lemma shows that 
any valid \texttt{path} is a valid \texttt{upath}, after projecting the
\texttt{list EdgeType} to \texttt{list VertexType}.
Similarly, the directed idiom \texttt{reachable~g~u~v} claims the 
existence of a \texttt{path} in~\texttt{g} from~\texttt{u}~to~\texttt{v}, 
and this easily weakens to give \texttt{connected~g~u~v}.

\subsubsection{Building out our undirectedness infrastructure.}
With our basic undirectedness definitions at hand and with the assurance
of mathematical compatability with prior directed work, we can begin in 
earnest. We flesh out a number of definitions that culminate in 
\texttt{minimum\_spanning\_forest}, which is exactly the our postcondition
of both Prim's and Kruskal's algorithms.\footnote{That Prim's algorithm has
a \emph{forest} in its postcondition should raise an eyebrow. We defend this claim
in~\S\ref{sec:primforest}.}

%A \texttt{simple\_upath} is a valid \texttt{upath} that has no duplicate vertices. 
An undirected cycle (\texttt{ucycle}) is a valid \texttt{upath} whose first 
and last vertices are the same.  A \texttt{simple\_ucycle} is a \texttt{ucycle} whose
only duplicate is the first/last vertex. A \texttt{connected\_graph}
has the property that any two valid vertices in the graph are \texttt{connected}.
We proceed with:
\lstset{style=CoqStyle}
\begin{lstlisting}
Definition uforest g := 
 ($\forall$ e, evalid g e -> strong_evalid g e) $/|$
 ($\forall$ p l, $\lnot$simple_ucycle g p l).

Definition spanning g g' := 
 $\forall$ u v, connected g u v <-> connected g' u v.

Definition spanning_uforest f g :=
  is_partial_graph f g $/|$ uforest f $/|$ spanning f g. 
\end{lstlisting}
Trees are easy to define:
\texttt{utree~g~:= uforest~g} $\wedge$ \texttt{connected\_graph~g.}
The \texttt{strong\_evalid} predicate means that the 
\texttt{src} and \texttt{dst} of the edge are also valid, so
\emph{e.g.} a valid edge cannot point to a deleted/absent vertex.
The second predicate is the critical constraint, saying that a
forest has no simple undirected cycles. The other definitions presented
are straightforward from there, and 
\texttt{minimum\_spanning\_forest f g} means that the total edge cost 
of~\texttt{f} is \note[minimal among other... struggling to phrase this.]{minimal}.

\hide{We also highlight a difference in our definition compared to mathematical textbooks. Prim's and Kruskal's algorithms are presented as minimum-spanning \textit{tree} algorithms, and often have the implicit assumption that the graph is fully connected. They may or may not discuss forests - CLRS does not, while \textit{Discrete Mathematics} informally defines forests as ``containing no simple circuits that are not necessarily connected [...] and have the property that each of their connected components is a tree." In short, these sources define forests from ``bottom-up" using trees. We define them ``top-down" instead, recognising forests as acyclic graphs and trees as connected forests. This definition was also used by Lammich et al~\cite{DBLP:journals/afp/LammichN19}.}
% Anshuman: I've hidden this for now, may cut or shorten a lot after discussion.


\subsection{New spatial representations for edge-labeled graphs}

In service of our new verifications, we develop spatial support in separation 
logic for two new styles of graph representation:
adjacency matrices and edge lists. These styles are particularly
amenable to graphs that label their edges, and so are suited to
the algorithms we are working with.


\subsubsection{Adjacency matrices.}

Adjacency matrices are a useful way to represent graphs
in problems where efficient label reads are required. 
We support the three most common 
flavors of adjacency matrix representation:
stack-allocated 2D~array \texttt{int~graph[size][size]},
a stack-allocated 1D~array \texttt{int~graph[size$\times$size]}, 
and a heap-allocated 2D~array \texttt{int~**graph}. 
To the casual observer, these are essentially interchangeable, but 
that is a mistake when thinking spatially. Apart from the obvious 
arithmetic that the second flavor does to access cells, there is a 
more subtle point: the first and second can
assume a contiguous block of memory, but the third needs to allocate
a ``spine'' and then separately allocate each array within the spine.
For a taste, the spatial representation of the first looks like:
\vspace{-0.5em}
\begin{equation*}
\begin{split}
\m{arr\_addr} ((\m{block}, \m{offset}), \m{i}) &\defeq
  (\m{block}, \m{offset} + (\m{i} \times \texttt{sizeof}(\texttt{int}) \times \texttt{size})) \\
\p{arr\_rep}(\gamma, \m{i}, \m{base\_ptr}) &\defeq \mathsf{array}(\mathsf{graph2mat}(\gamma)[\m{i}], \m{arr\_addr}(\m{base\_ptr}, \m{i})) \\
\vspace{1em}
\p{graph\_rep}(\gamma, \m{g\_addr}) &\defeq \underset{\m{v} \in \gamma}{\bigstar} \m{v}  \mapsto\p{arr\_rep}(\gamma, \m{v}, \m{g\_addr})
\end{split}
\end{equation*}
Of particular note are $\mathsf{graph2mat}$, which performs two projections to
drag out the graph's nested edge labels into a 2D matrix, and 
\m{arr\_addr}, which in this instance is able to simply compute
the address of any array \m{i} by adding to the base address of the graph.
When representing the heap-allocated 2D~array, we still benefit from 
$\mathsf{graph2mat}$'s projections but \m{arr\_addr} is powerless: no relationship
between the addresses of individual arrays is guaranteed.
Instead, we require that the user provide a list of base addresses
of each array. 

While handling these spatial wrinkles, we develop utilities that easily 
unfold and refold the adjacency matrix, thus smoothing user 
experience when reading and writing arrays and cells. Of course
these utilities themselves vary by flavor of representation, but 
the net effect is that users of our adjacency matrices really can 
be agnostic to the style of representation they are using 
(see \S\ref{sec:dijkoverview}).

The property of undirectedness requires no special condition 
when it comes to spatial representation in separation logic, 
and so we share spatial representations and utilities 
across directed and undirected graphs.

\hide{We observe one quirk about a symmetric undirected matrix. In our abstract graphs, which are fundamentally directed, the abstract edges $(u,v)$ and $(v,u)$ where $u \neq v$ are clearly distinct from each other. However, in a symmetric matrix both $g[u][v]$ and $g[v][u]$ will be marked, thus it is unclear which edge is in the graph. To disambiguate, we impose the following condition: If $g[u][v]$ and $g[v][u]$ are marked, and $u \leq v$, then we consider $(u,v)$ to be the valid edge; if $v \textless u$, then $(v,u)$ is the valid edge.
}%end hide
% discuss with Aquinas

\subsubsection{Edge lists.}

Edge lists are the representation of choice 
for sparse graphs. A straightforward~C implementation 
defines an \texttt{edge} as a 
\texttt{struct} containing \texttt{src}, \texttt{dst}, and 
\texttt{weight}, and then defines a \texttt{graph} as a 
\texttt{struct} containing 
the graph's size, the number of edges in the graph, 
and an array of \texttt{edge}s. Ours spatial representation 
follows this pattern:
\vspace{-0.5em}
\begin{equation*}
\begin{split}
\p{graph\_rep}(\gamma, \m{g\_addr}, &\m{e\_addr})~\defeq \\
&\mathsf{struct}\big((\texttt{size}, \lvert\gamma.E\rvert,\p{array}(\gamma.\texttt{edges},\m{e\_addr})),\m{g\_addr}\big)
\end{split}
\end{equation*}

% Need to discuss the above.
% data_at sh (t_wedgearray_graph) (Vint (Int.repr (size)), (Vint (Int.repr (numE g)), pointer_val_val orig_eptr)) (pointer_val_val orig_gptr);
%         data_at sh (tarray t_struct_edge MAX_EDGES)
%           (map wedge_to_cdata glist ++ (Vundef_cwedges (MAX_EDGES - numE g))) (pointer_val_val orig_eptr)

\hide{We observe mathematical differences between the graphs that can be represented by an edge list and an adjacency-matrix representation. As mentioned in the Prim's section, in a symmetric adjacency matrix, $g[u][v])$ and $g[v][u]$ are considered the same edge; however, this is not true for an edge list, where every edge is effectively an ordered pair due to the memory positions of the data. As a result, this edge list can contain multiple edges between two vertices whereas a 2D adjacency matrix cannot. Further, we prove that Kruskal's algorithm can handle such inputs and still return a minimal spanning forest.
} % end hide
% Well we can't say the above if it's not true... 

\subsection{Pure reasoning for adjacency matrix-represented graphs}

We discussed above the spatial representation of an adjacency 
matrix, but not every mathematical graph is fits that mould.
We supply a unified, reusable, and extendable \emph{soundness condition} 
\texttt{SoundAdjMat} that a graph must satisfy in order for it 
to be represented as an adjacency matrix.

\texttt{SoundAdjMat} is premised on the \texttt{size} of the graph
and a distinguished number \texttt{inf}.
First, we restrict soundness to finite graphs having 
vertices of type \texttt{Z}, edges of \texttt{Z*Z}, and 
edge labels of \texttt{Z}. 
We ask that the parameters \texttt{size} and \texttt{inf} be
strictly positive and representable on the machine. 
Further, we give semantics to \texttt{src} and \texttt{dst} 
(\m{fst} and \m{snd} respectively)
and \texttt{vvalid} (a valid vertex is indexed between $0$ and \texttt{size-1}).
Most critical, however, is the semantics of \texttt{evalid}:
a valid edge must have a machine-representable label and that label
cannot have value \texttt{inf}; an invalid edge \emph{must} have label \texttt{inf}.

The restriction on edge labels is necessary because we are working 
with labeled adjacency matrices. An unlabeled 
adjacency matrix can get by with labels of $0$/$1$ to 
indicate the absence/presence of an edge, but in a labeled 
setting, any label could be bona-fide depending on the use case. 
The solution is to 
allow the user the choice of a distinguished number \texttt{inf},
which thereafter indicates the \emph{absence} of an edge.

\note{Add a little about undirected, after discussing with Aquinas.}

In our verifications, 
the soundness condition of DijkstraGraph is \texttt{SoundAdjMat}
along with a few further constraints. An obvious constraint is that Dijkstra 
requires positive edge weights; further restrictions on 
\texttt{size} and \texttt{inf} are discussed in~\S\ref{sec:dijkoverflow}.
PrimGraph uses the \texttt{SoundUAdjMat} wholesale and places no further restrictions
on it.


