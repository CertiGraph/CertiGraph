We begin with the briefest of introductions to CertiGraph's core structure and then detail our extensions we make to various levels of
CertiGraph in service of our Dijkstra, Prim, and Kruskal verifications.
Ignoring modularity and eliding elements not used in our work, 
a mathematical graph in CertiGraph is a tuple:
($\mathcal{V}$, $\mathcal{E}$, \texttt{vvalid}, \texttt{evalid},
\texttt{src}, \texttt{dst}, \texttt{vlabel}, \texttt{elabel}, 
\texttt{sound}). $\mathcal{V}$/$\mathcal{E}$ are the carrier
types of vertices/edges, \texttt{vvalid}/\texttt{evalid}
place restrictions specifying whether a vertex/edge is 
valid\footnote{Validity denotes presence in the graph: \emph{e.g.}, if we are using $\mathbb{Z}$ as the carrier type $\mathcal{V}$, and have
only 7 vertices, then $\texttt{vvalid}(x)$ is probably the proposition $0 \le x < 7$).},
and $\texttt{src}/\texttt{dst} : \mathcal{E} \to \mathcal{V}$ map edges to their source/destination.
Labels are allowed on vertices and edges, and
a \texttt{sound}ness condition allows custom application-specific
restrictions~\cite{DBLP:journals/pacmpl/WangCMH19}.
Mathematical graphs connect to graphs in computer memory via spatial predicates in separation logic.
%We explain here the extensions we made .

\subsection{Pure reasoning for adjacency matrix-represented graphs}
\label{sec:adjmatpure}

Two of our algorithms operate over graphs represented as adjacency matrices.
Not every legal graph can be represented as an adjacency matrix,
so we develop a unified, reusable, and extendable \texttt{sound}ness condition
\texttt{SoundAdjMat} that a graph must satisfy in order for it
to be represented as an adjacency matrix.

\texttt{SoundAdjMat} is parameterized by the graph's \texttt{size} and a distinguished number \texttt{inf}.
We restrict most fields in the tuple:
($\mathcal{V}=\mathbb{Z}$,\ $\mathcal{E}=\mathbb{Z} \times \mathbb{Z}$,\ $\texttt{vvalid}=\lambda v.~\mbox{$0 \le v < \texttt{size}$}$,\ $\texttt{evalid}=\ldots$,\ $\texttt{src}=\m{fst}$,\ $\texttt{dst}=\m{snd}$,\ $\texttt{vlabel}$,\ $\texttt{elabel}$,\ $\texttt{sound}=\ldots$).  We also restrict the carrier type of vertex labels to \texttt{unit} and edge labels to $\mathbb{Z}$.
We require the parameters \texttt{size} and \texttt{inf} be
strictly positive and representable on the machine.
%Further, we give semantics to \texttt{vvalid}: a valid vertex is indexed
%between $0$ and \texttt{size-1}.
Most critical, however, is the semantics of \texttt{evalid}:
a valid edge must have a machine-representable label and that label
cannot have value \texttt{inf}; an invalid edge \emph{must} have label \texttt{inf}.
Last, the graph must be finite.

The restriction on edge labels is necessary because we are working
with labeled adjacency matrices on a real system: we need to set aside
a distinguished number \texttt{inf} such that edgeweight \texttt{inf}
indicates the \emph{absence} of an edge. We cannot
prescribe some \texttt{inf} because client needs can vary widely. For
instance, our verifications of Dijkstra's and Prim's algorithms
require subtly different \texttt{inf}s.

\texttt{SoundAdjMat} guarantees spatial representability
as an adjacency matrix, but it can be extended with further
algorithm-specific restrictions before
being plugged in for \texttt{sound}. Dijkstra's algorithm
requires positive edge weights, and---as we will discuss in~\S\ref{sec:dijkoverflow}---nontrivial restrictions on \texttt{size} and \texttt{inf}.

\subsection{New spatial representations for edge-labeled graphs}
\label{sec:newspatial}

We give predicates for adjacency matrices and edge lists
for edge-labeled graphs.

\subsubsection{Adjacency matrices.}

Adjacency matrices enable efficient label access for edge-labeled graphs.
We support three common
adjacency matrix representations:
a stack-allocated 2D~array \texttt{int~graph[size][size]},
a stack-allocated 1D~array \texttt{int~graph[size$\times$size]},
and a heap-allocated 2D~array \texttt{int~**graph}.
To the casual observer, these are essentially interchangeable, but
that is a mistake when thinking spatially. Apart from the
arithmetic that the second flavor uses to access cells, there is a
more subtle point: the first and second enjoy a contiguous block of
memory, but the third does not: it is an allocated ``spine'' with pointers to separately-allocated rows.
For a taste, the spatial representation of the first is:
\vspace{-0.5em}
\begin{equation*}
\begin{split}
\m{arr\_addr} (\m{ptr}, \m{i},\texttt{size}) \defeq~&
  \m{ptr} + (\m{i} \times \hide{\texttt{sizeof}(\texttt{int}) \times} \texttt{size}) \\
\mathsf{array}(\m{ptr},\m{list}) \defeq~& \underset{\m{i} \in [0, \lvert\m{list}\rvert)}{\bigstar} (\m{ptr} + \m{i}) \mapsto \m{list}[\m{i}]) \\
\p{arr\_rep}(\gamma, \m{i}, \m{ptr}) \defeq~& \texttt{let }\m{row} \texttt{ := }\mathsf{graph2mat}(\gamma)[\m{i}] \texttt{ in } \\
&\mathsf{array}(\m{arr\_addr}(\m{ptr},\m{i},\lvert\m{row}\rvert), \m{row}) \\
\vspace{1em}
\p{graph\_rep}(\gamma, \m{g\_addr}, \_) \defeq~& \underset{\m{v} \in \gamma}{\bigstar} \p{arr\_rep}(\gamma, \m{v}, \m{g\_addr})
\end{split}
\end{equation*}
We use the separation logic $\ast$ in its iterated form
to say that the arrays are separate in memory.  We elide details relating to object sizes, pointer alignment, and so forth, although our formal proofs handle such matters.
Of particular note are $\mathsf{graph2mat}$, which performs two projections to
drag out the graph's nested edge labels into a 2D matrix, and
\m{arr\_addr}, which in this instance simply computes
the address of any legal row \m{i} from the base address of the graph.
Notice that this $\p{graph\_rep}$ predicate ignores its third argument.
To represent a heap-allocated 2D~array we can still use $\mathsf{graph2mat}$
but can no longer use address arithmetic; the third
parameter is then a list of pointers to the row sub-arrays.

While ironing out these spatial wrinkles, we develop utilities that easily
unfold and refold our adjacency matrices, thus smoothing user
experience when reading and writing arrays and cells. Of course
these utilities themselves vary by flavor of representation, but
the net effect is that users of our adjacency matrices really can
be agnostic to the style of representation they are using
(see \S\ref{sec:dijkoverview}).


\hide{We observe one quirk about a symmetric undirected matrix. In our abstract graphs, which are fundamentally directed, the abstract edges $(u,v)$ and $(v,u)$ where $u \neq v$ are clearly distinct from each other. However, in a symmetric matrix both $g[u][v]$ and $g[v][u]$ will be marked, thus it is unclear which edge is in the graph. To disambiguate, we impose the following condition: If $g[u][v]$ and $g[v][u]$ are marked, and $u \leq v$, then we consider $(u,v)$ to be the valid edge; if $v \textless u$, then $(v,u)$ is the valid edge.
}%end hide
% discuss with Aquinas

\subsubsection{Edge lists.}

Edge lists are the representation of choice
for sparse graphs. Our~C implementation
defines an \texttt{edge} as a
\texttt{struct} containing \texttt{src}, \texttt{dst}, and
\texttt{weight}, and defines a \texttt{graph} as a
\texttt{struct} containing
the graph's size, edge count,
and an array of \texttt{edge}s. Our spatial representation
follows this pattern:
\vspace{-0.5em}
\begin{equation*}
\begin{split}
\p{graph\_rep}(\gamma, \m{g\_addr}, &\m{e\_addr})~\defeq~ \\
&\big(\m{g\_addr} \mapsto (\lvert\gamma.V\rvert, \lvert\gamma.E\rvert, \m{e\_addr})\big)
\ast
\p{array}(\m{e\_addr},\gamma.E)
\end{split}
\end{equation*}

\hide{We abuse notation slightly here: we simplify \m{g\_addr}'s target
from a \texttt{struct} to a tuple, and also overload the \p{array} predicate to represent
an array of \texttt{structs}.}

% Need to discuss the above.
% data_at sh (t_wedgearray_graph) (Vint (Int.repr (size)), (Vint (Int.repr (numE g)), pointer_val_val orig_eptr)) (pointer_val_val orig_gptr);
%         data_at sh (tarray t_struct_edge MAX_EDGES)
%           (map wedge_to_cdata glist ++ (Vundef_cwedges (MAX_EDGES - numE g))) (pointer_val_val orig_eptr)

\hide{We observe mathematical differences between the graphs that can be represented by an edge list and an adjacency-matrix representation. As mentioned in the Prim's section, in a symmetric adjacency matrix, $g[u][v])$ and $g[v][u]$ are considered the same edge; however, this is not true for an edge list, where every edge is effectively an ordered pair due to the memory positions of the data. As a result, this edge list can contain multiple edges between two vertices whereas a 2D adjacency matrix cannot. Further, we prove that Kruskal's algorithm can handle such inputs and still return a minimal spanning forest.
} % end hide
% Well we can't say the above if it's not true...


\subsection{Undirectedness in a directed world}
\label{sec:newundirected}

The CertiGraph library presented in~\cite{DBLP:journals/pacmpl/WangCMH19}
supports only directed graphs, and, as we have seen, bakes
direction-reliant idioms
such as \texttt{src} and \texttt{dst} deep into its development.
Our challenge is to add support for undirected graphs atop of this.

\hide{One approach is to handle undirected edges entirely separately
from directed edges, \emph{e.g.} using Coq \texttt{Ensemble}s.
This has the benefit of allowing
mixed graphs, which contain directed and undirected edges simultaneously.
Sadly, CertiGraph's tried-and-tested suite of pure formalized graph theory is
% fundamentally
% hopelessly
incompatible with such undirected edges, and so a new
``tower'' of development is necessary.
This is unpleasant both because it forces us to start
from scratch and because it creates
a
%an insurmountable
%divide
fissure
within an otherwise unified library.}%end hide

Our approach is to observe that every directed graph can be
treated as an undirected graph by ignoring edge direction.
We develop a lightweight layer of
``undirected flavored'' definitions atop of the existing
``directed flavored'' definitions, state and prove connections
between these, and then build the undirected infrastructure we need.
The result is that we retain full access to CertiGraph's graph theory formalizations
modulo some mathematical bridging.
% It also allows us to use CertiGraph's rich set of lemmas about the addition and removal of edges.
% This is backed by our observation that undirected graph properties are largely mathematical in nature,

\hide{\begin{lstlisting}
Record PreGraph {EV: EqDec Vertex eq} {EE: EqDec Edge eq} := {
	vvalid : Ensemble Vertex;
	evalid : Ensemble Edge;
	src : Edge -> Vertex;
	dst : Edge -> Vertex
}.
\end{lstlisting}
}% end hide
% Anshuman: keeping this in hide in case you want to show it to explain
% "how very ingrained" directedness is in CertiGraph.

% This makes sense in the first use cases of CertiGraph, especially when reasoning about C pointers.


Our basic ``undirected flavored'' definitions are
standard~\cite{clrs}.
%Our definitions are largely based on CLRS~\cite{clrs}.
Vertices \m{u} and \m{v} are \texttt{adjacent} if there is
an edge between them in either direction; vertices are self-adjacent.
An valid \texttt{upath} (undirected path) is list of
valid vertices that form a
pairwise-adjacent chain. Two vertices are \texttt{connected} when a valid \texttt{upath}
features them as head and foot (essentially the transitive
closure of \texttt{adjacenct}).
%A similar approach is proposed in Halsbeck~\cite{DBLP:journals/afp/HaslbeckLB19}.
% Note that the definition of path varies between textbooks and papers. For example, \textit{Discrete Mathematics and its Applications}~\cite{rozen} defines paths as a sequence of edges with an implicit sequence of vertices, whereas CLRS, which we have followed, defines it as a sequence of vertices with an implicit sequence of edges.

% \subsubsection{Our basic ``undirected'' definitions.}
The definitions above sync up with preexisting ``directed
flavored'' definitions.
Intuitively, undirectedness is more lax than directedness,
and so it is unsurprising that these connections are straightforward
weakenings of directed properties.
We next give standard definitions~\cite{clrs} that culminate in
\texttt{minimum\_spanning\_forest}, which is exactly our postcondition
of both Prim's and Kruskal's algorithms.\footnote{That Prim's postcondition has
a \emph{forest} may raise an eyebrow. See~\S\ref{sec:primforest}.}


\hide{
For example, a directed \texttt{path~: $\mathcal{V}$~* list~$\mathcal{E}$}
is valid when the \texttt{src} of the first edge in the list
is the vertex supplied, and the subsequent edges are linked up
in \texttt{src}/\texttt{dst} pairs
as one may expect. A straightforward lemma shows that
any valid \texttt{path} is a valid \texttt{upath}, after projecting the
\texttt{list $\mathcal{E}$} to \texttt{list $\mathcal{V}$}.
The directed idiom \texttt{reachable~g~u~v} claims the
existence of a \texttt{path} in~\texttt{g} from~\texttt{u}~to~\texttt{v},
and this easily weakens to give \texttt{connected~g~u~v}.}

% \subsubsection{Building out our undirectedness infrastructure.}
\hide{
With our basic undirectedness definitions at hand and with the assurance
of mathematical compatability with prior directed work, we can begin in
earnest.} 


%A \texttt{simple\_upath} is a valid \texttt{upath} that has no duplicate vertices.
An undirected cycle (\texttt{ucycle}) is a valid non-empty \texttt{upath} whose first
and last vertices are equal.  \hide{A \texttt{simple\_ucycle} is a \texttt{ucycle} whose
only duplicate is the first/last vertex.} \note[not used]{A \texttt{connected\_graph}
means that any two valid vertices are \texttt{connected}.}
\texttt{is\_partial\_graph f g} means everything in \texttt{f} is in \texttt{g}.
We proceed:
\lstset{style=CoqStyle}
\begin{lstlisting}
Definition uforest g :=
 ($\forall$ e, evalid g e $\rightarrow$ strong_evalid g e) $/|$
 ($\forall$ p l, $\lnot$ucycle g p l).
Definition spanning g g' :=
 $\forall$ u v, connected g u v $\leftrightarrow$ connected g' u v.
Definition spanning_uforest f g :=
  is_partial_graph f g $/|$ uforest f $/|$ spanning f g.
\end{lstlisting}
\hide{Trees are easy to define:
\texttt{utree~g~:= uforest~g} $\wedge$ \texttt{connected\_graph~g.}}
The \texttt{strong\_evalid} predicate means that the
\texttt{src} and \texttt{dst} of the edge are also valid, so
\emph{e.g.} a valid edge cannot point to a deleted/absent vertex.
The second conjunct of \texttt{uforest} is critical: a
forest has no undirected cycles. The other definitions 
are straightforward from there, and
\texttt{minimum\_spanning\_forest f g} means that no other spanning forest has
lower total edge cost than \texttt{f}.

\hide{We also highlight a difference in our definition compared to mathematical textbooks. Prim's and Kruskal's algorithms are presented as minimum-spanning \textit{tree} algorithms, and often have the implicit assumption that the graph is fully connected. They may or may not discuss forests - CLRS does not, while \textit{Discrete Mathematics} informally defines forests as ``containing no simple circuits that are not necessarily connected [...] and have the property that each of their connected components is a tree." In short, these sources define forests from ``bottom-up" using trees. We define them ``top-down" instead, recognising forests as acyclic graphs and trees as connected forests. This definition was also used by Lammich et al~\cite{DBLP:journals/afp/LammichN19}.}
% Anshuman: I've hidden this for now, may cut or shorten a lot after discussion.

Our undirected work is also compatible with our new developments
in~\S\ref{sec:adjmatpure} and~\S\ref{sec:newspatial}.
An adjacency matrix-representable undirected graph
has all the pure properties we covered in \texttt{SoundAdjMat},
and also has symmetry across the left diagonal.
We extend \texttt{SoundAdjMat} into
\texttt{Sound{\underline{U}}AdjMat} by adding the condition that
all valid edges have src $\le$ dst. This effectively ``turns off'' the
matrix on one half of the diagonal and avoids double-counting. Prim's algorithm uses
\texttt{SoundUAdjMat} and places no further restrictions.
Further, spatial representations and fold/unfold utilities are shared
across directed and undirected adjacency matrices.


\lstset{style=myTinyStyle}


