\note{We detail the major extensions we made to the 
CertiGraph library. The following two sections then take advantge of 
these extensions to complete three verifications.}

\subsection{Undirectedness in a directed world}

The CertiGraph library presented in~\cite{DBLP:journals/pacmpl/WangCMH19} 
supports only directed graphs, baking direction-reliant 
idioms 
% properties?
% constructs?
such as 
\texttt{src} and \texttt{dst} : \texttt{EdgeType} $\rightarrow$ \texttt{VertexType}
deep into its development.
Our challenge is to add support for undirected graphs. 

One approach is to handle undirected edges entirely separately 
from directed edges, \emph{e.g.} using Coq \texttt{Ensemble}s. 
This has the benefit of allowing 
mixed graphs, which contain directed and undirected edges simultaneously. 
Sadly, CertiGraph's tried-and-tested suite of pure formalized graph theory is
% fundamentally 
% hopelessly
incompatible with such undirected edges, and so a new 
``tower'' of development is necessary. 
This is unpleasant both because it forces us to start
from scratch and because it creates 
a
%an insurmountable
%divide
fissure 
within an otherwise unified library. 

The approach we take is more harmonious: 
we observe that every directed graph can be treated as an undirected 
graph by ignoring edge direction.
As we will soon detail, we develop a lightweight layer of 
``undirected flavored'' definitions atop of the existing 
``directed flavored'' definitions, state and prove connections 
between these, and are off to the races. 
We sacrifice the ability to represent mixed graphs---a given
graph needs to either consider edge direction or ignore~it---but 
we gain full access to CertiGraph's graph theory formalizations 
modulo some mathematical bridging.
% It also allows us to use CertiGraph's rich set of lemmas about the addition and removal of edges.
% This is backed by our observation that undirected graph properties are largely mathematical in nature, 

% The property of undirectedness requires no special conditions when it comes to spatial representation in separation logic, and so we seamlessly share spatial representations and related lemmas across directed and undirected graphs.
% The above line is mine (Anshuman's) but I think it's a distraction to put it here. Let's discuss later.

\hide{\begin{lstlisting}
Record PreGraph {EV: EqDec Vertex eq} {EE: EqDec Edge eq} := {
	vvalid : Ensemble Vertex;
	evalid : Ensemble Edge;
	src : Edge -> Vertex;
	dst : Edge -> Vertex
}.
\end{lstlisting}
}% end hide
% Anshuman: keeping this in hide in case you want to show it to explain 
% "how very ingrained" directedness is in CertiGraph.

% This makes sense in the first use cases of CertiGraph, especially when reasoning about C pointers.


\subsubsection{Our basic ``undirected'' definitions.}
%Our definitions are largely based on CLRS~\cite{clrs}.
Two vertices \m{u} and \m{v} are \texttt{adjacent} if there exists
an edge between them in either direction; vertices are self-adjacent.
An valid \texttt{upath} (undirected path) is list of 
\note[say more? cut?]{valid} 
vertices that form a  
pairwise-adjacent chain. Two vertices are \texttt{connected} when a valid \texttt{upath}
features them as head and foot; essentially this is the transitive 
closure of \texttt{adjacenct}.
%A similar approach is proposed in Halsbeck~\cite{DBLP:journals/afp/HaslbeckLB19}.
% Note that the definition of path varies between textbooks and papers. For example, \textit{Discrete Mathematics and its Applications}~\cite{rozen} defines paths as a sequence of edges with an implicit sequence of vertices, whereas CLRS, which we have followed, defines it as a sequence of vertices with an implicit sequence of edges. 

\subsubsection{Bridging with directedness.}
The definitions above sync up nicely with preexisting directed
definitions. 
\note[cut?]{Intuitively, undirectedness is more lax than directedness, 
and so it is unsurprising that these connections are straightforward
weakenings of directed properties.}
For example, a directed \texttt{path~: VType~* list~EdgeType}
is valid when the \texttt{src} of the first edge in the list 
is the vertex supplied, and the subsequent edges are linked up 
in \texttt{src}/\texttt{dst} pairs
as one may expect. A straightforward lemma shows that 
any valid \texttt{path} is a valid \texttt{upath}, after projecting the
\texttt{list EdgeType} to \texttt{list VertexType}.
Similarly, the directed idiom \texttt{reachable~g~u~v} claims the 
existence of a \texttt{path} in~\texttt{g} from~\texttt{u}~to~\texttt{v}, 
and this easily weakens to give \texttt{connected~g~u~v}.

\subsubsection{Building out our undirectedness infrastructure.}
With our basic undirectedness definitions at hand and with the assurance
of mathematical compatability with prior directed work, we can begin in 
earnest. We flesh out a number of definitions that culminate in 
\texttt{minimum\_spanning\_forest}, which is exactly the our postcondition
of both Prim's and Kruskal's algorithms.\footnote{That Prim's algorithm has
a \emph{forest} in its postcondition should raise an eyebrow. We defend this claim
in~\S\ref{sec:primforest}.}

%A \texttt{simple\_upath} is a valid \texttt{upath} that has no duplicate vertices. 
An undirected cycle (\texttt{ucycle}) is a valid \texttt{upath} whose first 
and last vertices are the same.  A \texttt{simple\_ucycle} is a \texttt{ucycle} whose
only duplicate is the first/last vertex. A \texttt{connected\_graph}
has the property that any two valid vertices in the graph are \texttt{connected}.
These give us:
\lstset{style=CoqStyle}
\begin{lstlisting}
Definition uforest g := 
 ($\forall$ e, evalid g e -> strong_evalid g e) $/|$
 ($\forall$ p l, $\lnot$simple_ucycle g p l).

Definition spanning g g' := 
 $\forall$ u v, connected g u v <-> connected g' u v.

Definition spanning_uforest f g :=
  is_partial_graph f g $/|$ uforest f $/|$ spanning f g. 
\end{lstlisting}
The \texttt{strong\_evalid} predicate means that the 
\texttt{src} and \texttt{dst} of the edge are also valid, so
\emph{e.g.} a valid edge cannot point to a deleted/absent vertex.
The second predicate is the critical constraint, saying that a
forest has no simple undirected cycles. The other definitions presented
are straightforward from there, and 
\texttt{minimum\_spanning\_forest f g} means that the total edge cost 
of~\texttt{f} is \note[minimal among other... struggling to phrase this.]{minimal}.
For completeness, trees are just a special kind of forest:
\texttt{utree~g~:= uforest~g} $\wedge$ \texttt{connected\_graph~g.}

\hide{We also highlight a difference in our definition compared to mathematical textbooks. Prim's and Kruskal's algorithms are presented as minimum-spanning \textit{tree} algorithms, and often have the implicit assumption that the graph is fully connected. They may or may not discuss forests - CLRS does not, while \textit{Discrete Mathematics} informally defines forests as ``containing no simple circuits that are not necessarily connected [...] and have the property that each of their connected components is a tree." In short, these sources define forests from ``bottom-up" using trees. We define them ``top-down" instead, recognising forests as acyclic graphs and trees as connected forests. This definition was also used by Lammich et al~\cite{DBLP:journals/afp/LammichN19}.}
% Anshuman: I've hidden this for now, may cut or shorten a lot after discussion.


\subsection{Undirected adjacency matrix representation}

We chose a symmetrical matrix representation for the undirected graph to be used in Prim's algorithm. This bears similarities to the adjacency matrix used in Dijkstra's, allowing us to reuse many properties about adjacency matrices.

We observe one quirk about a symmetric undirected matrix. In our abstract graphs, which are fundamentally directed, the abstract edges $(u,v)$ and $(v,u)$ where $u \neq v$ are clearly distinct from each other. However, in a symmetric matrix both $g[u][v]$ and $g[v][u]$ will be marked, thus it is unclear which edge is in the graph. To disambiguate, we impose the following condition: If $g[u][v]$ and $g[v][u]$ are marked, and $u \leq v$, then we consider $(u,v)$ to be the valid edge; if $v \textless u$, then $(v,u)$ is the valid edge.

\subsection{Edge list representation, multi-graph trimming, and subtle differences of graphs in different representations}
Kruskal's algorithm differs from Prim's in that it directly operates on the edges of the input graph, rather than Prim's vertex-based approaches. Consequently, in C implementations it is useful to pass the graph in as a list of graph edges, rather than an adjacency matrix or list. This also simplifies the sorting necessary in Kruskal's.

We define a simple edge list representation in C below, and observe mathematical differences between the graphs that can be represented by an edge list and an adjacency-matrix representation. As mentioned in the Prim's section, in a symmetric adjacency matrix, $g[u][v])$ and $g[v][u]$ are considered the same edge; however, this is not true for an edge list, where every edge is effectively an ordered pair due to the memory positions of the data. As a result, this edge list can contain multiple edges between two vertices whereas a 2D adjacency matrix cannot. Further, we prove that Kruskal's algorithm can handle such inputs and still return a minimal spanning forest.
\newline
\begin{lstlisting}
struct edge {
	int weight;
	int src;
	int dst;
};

struct graph {
	int V; //number of vertices {0...V-1}
	int E; //number of edges in the edgelist
	struct edge *edge_list;
};
//Consider a graph with edges {weight=1,src=0,dst=1}
//	and {weight=2,src=1,dst=0}
\end{lstlisting}
\begin{figure}
\begin{tikzpicture} [auto, node distance =2 cm and 2cm ,on grid, semithick, state/.style ={ circle}]
	\node[state] (V0) {$\m{v_0}$};
	\node[state] (V1) [right=of V0] {$\m{v_1}$};
	\path (V0) edge [bend left] node[above=0.15 cm] {$\{1,\m{v_0},\m{v_1}\}$} (V1);
	\path (V1) edge [bend left] node[below=0.15 cm] {$\{2,\m{v_1},\m{v_0}\}$} (V0);
\end{tikzpicture}
\caption{Example of one kind of multi-graph representable by the above C edge-list implementation, which Kruskal's algorithm is able to prune.}
\end{figure}
