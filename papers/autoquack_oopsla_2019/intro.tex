Over the last fifteen years, separation logic has facilitated great strides 
in verifying programs that manipulate tree-shaped data structures. \cite{berdine:smallfoot,chin:hipsleek,jacobs:verifast,chlipala:bedrock,bengtson:charge,appel:programlogics}. 
Unfortunately, programs that manipulate graph-shaped data structures 
(i.e. structures with \emph{intrinsic sharing}) have proved harder to verify.  
Indeed, such programs were formidable enough that a number of the early 
landmark results in separation logic devoted substantial effort to verifying 
single examples such as Schorr-Waite~\cite{hongseok:phd} or 
union-find~\cite{NeelThesis} with pen and paper.  
More recent landmarks have moved to a machine-checked context, but have still
been devoted to either single examples or to classes of closely-related examples, 
such as garbage collectors~\cite{A,CakeMLGC,C}.
Due to the complexity of these examples, they tend to require a large number 
of custom predicates and subtle reasoning, which generally does not carry across 
to the verification of other graph-manipulating programs.

We present a general toolkit for verifying graph-manipulating programs in a 
machine-checked context. Our techniques are \emph{general} in that they handle a diverse 
range of graph-manipulating programs, {\color{blue}including union-find, visiting/marking 
each node in a DAG, and pruning a cyclic graph to a spanning tree.} {\color{magenta}This 
will be repeated shortly...}
Our techniques are \emph{modular}, allowing us to reuse definitions and proofs 
(\emph{e.g.} facts about reachability) and giving us {\color{magenta} significant 
modularity} between abstract mathematical graphs and their concrete representation 
in the heap.
Our techniques are \emph{powerful} enough to reason about real C code as compiled by 
CompCert~\cite{blah}, and also \emph{lightweight} enough to integrate into the Verified 
Software Toolchain (VST)~\cite{appel:programlogics} without serious reengineering.
Finally, our techniques \emph{scale} well beyond short toy programs: we certify the 
correctness of a generational garbage collector for the CertiCoq project~\cite{blah} 
($\approx400$ devilish lines of C).

We proceed in three steps. First, we develop a ``mathematical graph library'' that is general enough to reason about a wide variety of algorithms.  We modularize this library carefully so that common ideas---\emph{e.g.} subgraphs, reachability, and isomorphism---can be reused in a number of algorithms.  
Second, we connect these abstract graphs to
concrete graphs in the heap using separation logic in a way that facilitates the reuse of key definitions and theorems across algorithms.  Finally, we develop a notion of \emph{localization blocks} to
enable the key Hoare triples to be proved in a modular fashion using our \textsc{Localize}
rule:

% old two-column version:
\hide{
\begin{equation}
\label{eq:localize}
\begin{array}{@{}l@{}}
\infrule{}%Localize}
{G_1 |- L_1 * R \\
\{ L_1 \} ~ c ~ \{ \exists x.~ L_2 \} \\
R |- \forall x.~ (L_2 --* G_2) }
{\{ G_1 \} ~ c ~ \{ \exists x.~ G_2 \}} {(\dagger)} \\
[3pt]
(\dagger)~ \mathit{freevars}(R) \cap \MV(c) = \emptyset
\end{array}
\end{equation} \marginpar{Can we typeset this a little better?}
} % end hide

% new one-column version:
\begin{equation}
\label{eq:localize}
\begin{array}{@{}l@{}}
\infrule{Localize}
{G_1 |- L_1 * R \\
\{ L_1 \} ~ c ~ \{ \exists x.~ L_2 \} \\
R |- \forall x.~ (L_2 --* G_2) }
{\{ G_1 \} ~ c ~ \{ \exists x.~ G_2 \}} {\mathit{freevars}(R) \cap \MV(c) = \emptyset} \\
[3pt]
\end{array}
\end{equation} \marginpar{Can we typeset this a little better?}

\textsc{Localize} connects the behavior of a command~$c$ in a ``local'' context~$L_1$ 
and~$L_2$ to its effect on a ``global'' context~$G_1$ and~$G_2$ using a carefully-chosen 
predicate~$R$. This is a more general version of the well-known \textsc{Frame} rule, 
which does the same task in the simpler case when $G_i = L_i * F$ for some frame~$F$ that 
is untouched by~$c$.  {\color{magenta}Indeed, \textsc{Frame} can be derived 
from \textsc{Localize} by setting $R \defeq F$, which makes both entailments 
tautologies\footnote{For readers relatively unfamiliar with the \emph{separating 
implication}~$--*$, \emph{a.k.a.} ``magic wand'',~$*$ and~$--*$ are related in the same 
way that~$/|$ and~$->$ are in propositional logic, \emph{i.e.} 
adjointness:~$(P * Q |- R) \Leftrightarrow (P |- Q --* R)$.} and trivially satisfies the 
side condition.} {\color{blue}I think the magenta section can be a footnote. It explains 
how Localize is a general version of Frame. I don't know if it deserves prime 
real estate.} 
Localization blocks are an upgrade to Hobor and Villard's concept of 
\emph{ramification}~\cite{blah} with two key improvements: support for existential 
quantifiers in postconditions, and smoother treatment of modified program variables.  
We will see how all three ingredients work together in an example 
in~\S\ref{sec:orientation}.

%as a kind of proof pattern or framework to verify graph-manipulating programs on pen and paper~\cite{hobor:ramification}.  The major focus of this paper is to develop methods to verify realistic graph programs in a mechanized context.  We do so by upgrading the theory of ramification and by developing a general and modular library for graph-related reasoning in separation logic.  We incorporate our approach into two sizeable separation logic-based verification tools: the Floyd system of the Verified Software Toolchain (VST)~\cite{appel:programlogics} and the HIP/SLEEK program verifier~\cite{chin:hipsleek}.  VST and HIP/SLEEK inhabit quite different points in the design space for verification tools, with VST primarily focusing on heavily human-guided verifications with an emphasis on end-to-end machine-checked proofs, and HIP/SLEEK focusing on more automation.  Despite these differences, the vast majority of our Coq code base is shared between them,
%giving us hope that our work will be applicable to other tools.

%% \marginpar{\color{magenta} computable mathgraphs, null, pregraphs Problem with ``later'' not being precise.}
Our contributions are organized as follows:
\begin{itemize}
\item[\S\ref{sec:orientation}] We use the classic ``union-find'' disjoint set algorithm to show {\color{magenta}how our three key ingredients---mathematical graphs, spatial graphs, and localization blocks---come together to verify 
graph-manipulating algorithms.} 
{\color{blue}I worry this is repetetive. Ref: the last few words of the previous paragraph.}
Although we use it primarily as a running example, this is the first machine-checked verification of this algorithm in C to the best of our knowledge. {\color{blue} Does anyone have any data on this?}  {\color{magenta} We introduce \emph{localization blocks} as a new notation for ramification. We briefly discuss three additional VST-certified examples to give a sense of the breadth of algorithms we can verify: marking a DAG, pruning a graph into a spanning tree, and making a structure-preserving copy of a 
graph.}
{\color{blue}The above reads a lot like something in the previous page. Can one 
of them be shortened?}
\item[\S\ref{sec:localizations}] {\color{magenta} Existentials for fun and profit: we upgrade Hobor and Villard's \infrulestyle{Ramify} rule to handle both modified program variables and existential quantifiers in postconditions more gracefully.}
\item[\S\ref{sec:mathgraph}] We develop a general mechanization of mathematical graphs powerful enough to support realistic verification. {\color{magenta}Probably we will merge this with the next section.} %{\color{magenta} What else can we say here?}
\item[\S\ref{sec:spacegraph}] We suggest that the Knaster-Tarski fixpoint~\cite{tarski:fixpoint} cannot define a usable separation logic graph predicate.  We propose a better definition for general spatial graphs that still enjoys a ``recursive'' fold/unfold.  We prove general theorems about spatial graphs in a way that can be utilized in multiple flavors of separation logic. %{\color{magenta} Line about modularity?}
%\item[\S\ref{sec:hipsleek}] We explain how we modified HIP/SLEEK to introduce ramifications when programs modify data structures with intrinsic sharing and to automatically discharge the associated obligations using Coq-verified external lemmas.
%\item[\S\ref{sec:vst}] We explain how we integrated ramification into VST by developing .
\item[\S\ref{sec:certigc}] We discuss the certification of the CertiCoq garbage collector (GC). While certifying this GC we identify two places where the semantics of C is too weak to define an OCaml-style GC. We also found and fixed a rather subtle overflow error in the original C code for the GC, justifying the effort of developing the machine-checked proof.
\item[\S\ref{sec:development}] We integrate our techniques into VST with two new Floyd tactics, \li{localize} and \li{unlocalize}, and document some statistics related to our overall development. {\color{blue} I think it's a bit funny to put these two in the same section.}
\item[\S\ref{sec:related}] We discuss related work.
\item[\S\ref{sec:conclusion}] We discuss directions for future work and conclude.
\end{itemize}
All of our results are machine checked in Coq.

