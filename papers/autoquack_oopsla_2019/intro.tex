Over the last fifteen years, separation logic has facilitated great strides
in verifying programs that manipulate tree-shaped data structures.
\citep{berdine:smallfoot,chin:hipsleek,jacobs:verifast,
chlipala:bedrock,bengtson:charge,appel:programlogics}.
Unfortunately, programs that manipulate graph-shaped data structures
(i.e. structures with \emph{intrinsic sharing}) have proved harder to verify.
Indeed, such programs were formidable enough that a number of the early
landmark results in separation logic devoted substantial effort to verifying
single examples such as Schorr-Waite~\cite{hongseok:phd} or
union-find~\cite{neelthesis} with pen and paper.
More recent landmarks have moved to a machine-checked context, but have still
been devoted to either single examples or to classes of closely-related examples
such as garbage collectors~\cite{gcexample3,cakemlgc}.
These kinds of examples tend to require a large number
of custom predicates and subtle reasoning, which generally 
does not \note[carry over? carry forward?]{carry}
to the verification of other graph-manipulating programs.

In contrast, we present a general toolkit for verifying graph-manipulating programs in a
machine-checked context. Our techniques are \emph{general} in that they handle a diverse
range of graph-manipulating programs, and \emph{modular} in that they allow code
reuse (\emph{e.g.} facts about reachability) and encourage separation of concerns
(\emph{e.g.} between abstract mathematical graphs and their concrete representation
in the heap).
Our techniques are \emph{powerful} enough to reason about real C code as compiled by
CompCert~\cite{leroy:compcert}, and also
\emph{lightweight} enough to integrate into the Verified
Software Toolchain (VST)~\cite{appel:programlogics} without requiring major reengineering.
Both CompCert and VST are distributed as optional packages in the CoqIDE installer and through 
opam, so they have a sizable userbase that can take advantage of our techniques. %``out of the box''.
Finally, our techniques \emph{scale} well beyond short toy programs: we certify the
correctness of a generational garbage collector for the CertiCoq
project~\cite{certicoqpaper} ($\approx400$ rather devilish lines of C).

We proceed in three steps. First, we develop a ``mathematical graph library'' that is general enough to reason about a wide variety of algorithms and expressive enough to describe the behavior of these algorithms in real machines.  We modularize this library carefully so that common ideas---\emph{e.g.}~subgraphs, reachability, and isomorphism---can be efficiently reused
in different algorithms.  Second, we use separation logic to express how these abstract graphs
are actualized in the heap as concrete graphs in a way that facilitates the reuse of key definitions and theorems across algorithms.  Finally, we develop a notion of \emph{localization blocks} that
allows us to carry out our Hoare proofs in a modular fashion, even in the presence of the
implicit sharing intrinsic to graphs, by using our \textsc{Localize} rule:

% old two-column version:
\hide{
\begin{equation}
\label{eq:localize}
\begin{array}{@{}l@{}}
\infrule{Localize}
{G_1 |- L_1 * R \\
\{ L_1 \} ~ c ~ \{ \exists x.~ L_2 \} \\
R |- \forall x.~ (L_2 --* G_2) }
{\{ G_1 \} ~ c ~ \{ \exists x.~ G_2 \}} {(\dagger)} \\
[3pt]
(\dagger)~ \mathit{freevars}(R) \cap \MV(c) = \emptyset
\end{array}
\end{equation} \marginpar{Can we typeset this a little better?}
} % end hide

% new one-column version:
\vspace{-1em}
\begin{equation}
\label{eq:localize}
\inferrule[Localize]	
{G_1 |- L_1 * R \\
\{ L_1 \} ~ c ~ \{ \exists x.~ L_2 \} \\
R |- \forall x.~ (L_2 --* G_2) }
{\{ G_1 \} ~ c ~ \{ \exists x.~ G_2 \}} \; \FV(R) \cap \MV(c) = \emptyset \qquad
\vspace{-0.3em}
\end{equation}
%\marginpar{\tiny \color{blue} Maybe this should be moved to \S2? Just formula for intro?}
\textsc{Localize} connects the ``local'' effect of a command~$c$, \emph{i.e.}
transforming~$L_1$ to~$L_2$, with its ``global'' effect, \emph{i.e.} from~$G_1$ to~$G_2$.
The key is carefully choosing a \emph{ramification frame}~$R$ that satisfies a pair of
delicately-stated entailments\footnote{Readers less familiar with the separating implication $P --* Q$, also known as \emph{magic wand}, can refer to its semantics in Figure~\ref{fig:seplogsem} (page~\pageref{fig:seplogsem}), which also models the other separation logic operators
we use in this paper.  The key proof rule for magic wand is its adjointness
with the separating conjunction:~$(P * Q |- R) \Leftrightarrow (P |- Q --* R)$.} and
the side condition on modified local program variables.
\textsc{Localize} is a more general version of the well-known \textsc{Frame} rule,
which does the same task in the simpler case when $G_i = L_i * F$
for some frame~$F$ that is untouched by~$c$.  Said differently, \textsc{Frame} works well
for tree-manipulating programs, while \textsc{Localize} handles more subtle
graph-manipulating programs.
\textsc{Localize} %Localization blocks upgrade
upgrades the \textsc{Ramify} rule~\cite{hobor:ramification} in two key ways:
support for existential
quantifiers in postconditions and smoother treatment of modified program variables.

%as a kind of proof pattern or framework to verify graph-manipulating programs on pen and paper~\cite{hobor:ramification}.  The major focus of this paper is to develop methods to verify realistic graph programs in a mechanized context.  We do so by upgrading the theory of ramification and by developing a general and modular library for graph-related reasoning in separation logic.  We incorporate our approach into two sizeable separation logic-based verification tools: the Floyd system of the Verified Software Toolchain (VST)~\cite{appel:programlogics} and the HIP/SLEEK program verifier~\cite{chin:hipsleek}.  VST and HIP/SLEEK inhabit quite different points in the design space for verification tools, with VST primarily focusing on heavily human-guided verifications with an emphasis on end-to-end machine-checked proofs, and HIP/SLEEK focusing on more automation.  Despite these differences, the vast majority of our Coq code base is shared between them,
%giving us hope that our work will be applicable to other tools.

%% \marginpar{\color{magenta} computable mathgraphs, null, pregraphs Problem with ``later'' not being precise.}
Our contributions are organized as follows:
\begin{itemize}
\item[\S\ref{sec:orientation}] We use the classic ``union-find'' disjoint set algorithm~\cite{clrs} to show how our three key ingredients---mathematical graphs, spatial graphs, and localization blocks---come together to verify
graph-manipulating algorithms.
%\marginpar{\tiny \color{blue} TODO: check.}
%Although we use it primarily as a running example, 
To the best of our knowledge this is the first machine-checked verification of this algorithm that starts with real C code.
We introduce \emph{localization blocks} as a notation for \textsc{Localize} in decorated programs. 
\item[\S\ref{sec:localizations}] We show that \textsc{Localize} and \textsc{Frame} are equivalent,
\note[unclear what you mean]{and a delicate technique to properly handle modified local variables.}  We show a mark-graph
program that explores a graph in a fold/unfold style and discuss the utility of linked 
existentials in postconditions. We also briefly discuss some additional examples
to give a sense of the breadth of algorithms we can verify: marking a DAG, an array-based 
version of union-find, and pruning a graph into a spanning tree.
%{\color{magenta}Existentials for fun and profit: we upgrade Hobor and Villard's \infrulestyle{Ramify} rule to handle both modified program variables and existential quantifiers in postconditions more gracefully.
%    We briefly discuss four additional VST-certified examples to give a sense of the breadth of algorithms we can verify: marking a DAG, marking a graph, an array-based version of union-find, and pruning a graph into a spanning tree.} %, and making a structure-preserving copy of a graph.
\item[\S\ref{sec:mathgraph}] We develop a general framework of mathematical graphs powerful enough to support realistic verification in a mechanized context.  We give a sampling of key definitions and show how our framework is modularized to facilitate code reuse.
\item[\S\ref{sec:spacegraph}] We suggest that the Knaster-Tarski fixpoint~\cite{tarski:fixpoint} cannot define a usable separation logic graph predicate.  We propose a better definition for general spatial graphs that still enjoys a ``recursive'' fold/unfold.  We prove general theorems about spatial graphs in a way that can be utilized in multiple flavors of separation logic. %{\color{magenta} Line about modularity?}
%\item[\S\ref{sec:hipsleek}] We explain how we modified HIP/SLEEK to introduce ramifications when programs modify data structures with intrinsic sharing and to automatically discharge the associated obligations using Coq-verified external lemmas.
%\item[\S\ref{sec:vst}] We explain how we integrated ramification into VST by developing .
\item[\S\ref{sec:certigc}] We discuss the certification of the CertiCoq garbage collector (GC). While certifying this GC we identify two places where the semantics of C is too weak to define an OCaml-style GC. We also find and fix a rather subtle overflow error in the original C code for the GC, justifying the effort of developing the machine-checked proof.
%\marginpar{\tiny \color{blue} Bit funny to put these two in the same section.}
\item[\S\ref{sec:development}] We discuss how our techniques are integrated into the
``Floyd'' module of VST, a separation-logic based engine to help users verify
CompCert~C programs, via two new Floyd tactics \li{localize} and \li{unlocalize}.
We also document statistics related to our overall development.
\item[\S\ref{sec:related}] We discuss related work.
\item[\S\ref{sec:conclusion}] We discuss directions for future work and conclude.
\end{itemize}
All of our results are machine checked in Coq and available at~\cite{github}.
