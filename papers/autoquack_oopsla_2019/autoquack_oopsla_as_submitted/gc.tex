% need to split this up into different figures so they look neater.
% will do when we have a better sense of:
% 1. exactly what code we want
% 2. do we want comments?
% 3. should I make a flowchart of who calls whom and use that instead?

\begin{figure}[t]
\vspace{-1ex}
  \begin{lstlisting}
#include <stdlib.h>
#include <stdio.h>
#include <assert.h>
#include "config.h"
#include "gc.h"

/* The following 5 functions should (in practice) compile correctly in CompCert,
   but the CompCert correctness specification does not _require_ that
   they compile correctly:  their semantics is "undefined behavior" in
   CompCert C (and in C11), but in practice they will work in any compiler. */

int test_int_or_ptr (value x) /* returns 1 if int, 0 if aligned ptr */ {
    return (int)(((intnat)x)&1);
}

intnat int_or_ptr_to_int (value x) /* precondition: is int */ {
    return (intnat)x;
}

void * int_or_ptr_to_ptr (value x) /* precond: is aligned ptr */ {
    return (void *)x;
}

value int_to_int_or_ptr(intnat x) /* precondition: is odd */ {
    return (value)x;
}

value ptr_to_int_or_ptr(void *x) /* precondition: is aligned */ {
    return (value)x;
}

int Is_block(value x) {
    return test_int_or_ptr(x) == 0;
}

/* A "space" describes one generation of the generational collector. */
struct space {
  value *start, *next, *limit;
};
/* Either start==NULL (meaning that this generation has not yet been created),
   or start <= next <= limit.  The words in start..next  are allocated
   and initialized, and the words from next..limit are available to allocate. */

#define MAX_SPACES 12  /* how many generations */

#ifndef RATIO
#define RATIO 2   /* size of generation i+1 / size of generation i */
/*  Using RATIO=2 is faster than larger ratios, empirically */
#endif

#ifndef NURSERY_SIZE
#define NURSERY_SIZE (1<<16)
#endif
/* The size of generation 0 (the "nursery") should approximately match the
   size of the level-2 cache of the machine, according to:
      Cache Performance of Fast-Allocating Programs,
      by Marcelo J. R. Goncalves and Andrew W. Appel.
      7th Int'l Conf. on Functional Programming and Computer Architecture,
      pp. 293-305, ACM Press, June 1995.
   We estimate this as 256 kilobytes
     (which is the size of the Intel Core i7 per-core L2 cache).
    http://www.tomshardware.com/reviews/Intel-i7-nehalem-cpu,2041-10.html
    https://en.wikipedia.org/wiki/Nehalem_(microarchitecture)
   Empirical measurements show that 64k works well
    (or anything in the range from 32k to 128k).
*/

#ifndef DEPTH
#define DEPTH 0  /* how much depth-first search to do */
#endif

#ifndef MAX_SPACE_SIZE
#define MAX_SPACE_SIZE (1 << 29)
#endif
/* The restriction of max space size is required by pointer
   subtraction.  If the space is larger than this restriction, the
   behavior of pointer subtraction is undefined.
*/

struct heap {
  /* A heap is an array of generations; generation 0 must be already-created */
  struct space spaces[MAX_SPACES];
};

#ifdef DEBUG

int in_heap(struct heap *h, value v) {
  int i;
  for (i=0; i<MAX_SPACES; i++)
    if (h->spaces[i].start != NULL)
      if (h->spaces[i].start <= (value*)v &&
    (value *)v <= h->spaces[i].limit)
  return 1;
  return 0;
}

void printtree(FILE *f, struct heap *h, value v) {
  if(Is_block(v))
    if (in_heap(h,v)) {
      header_t hd = Field(v,-1);
      int sz = Wosize_hd(hd);
      int i;
      fprintf(f,"%d(", Tag_hd(hd));
      for(i=0; i<sz-1; i++) {
  printtree(f,h,Field(v,i));
  fprintf(f,",");
      }
      if (i<sz)
  printtree(f,h,Field(v,i));
      fprintf(f,")");
    }
    else {
      fprintf(f,"%8x",v);
    }
  else fprintf(f,"%d",v>>1);
}

void printroots (FILE *f, struct heap *h,
      fun_info fi,   /* which args contain live roots? */
      struct thread_info *ti) /* where's the args array? */
 {
   value *args; int n; uintnat i, *roots;
   roots = fi+2;
   n = fi[1];
   args = ti->args;

  for(i = 0; i < n; i++) {
    fprintf(f,"%d[%8x]:",roots[i],args[roots[i]]);
    printtree(f, h, args[roots[i]]);
    fprintf(f,"\n");
  }
  fprintf(f,"\n");
}

#endif

void abort_with(char *s) {
  fprintf(stderr, s);
  exit(1);
}

int Is_from(value* from_start, value * from_limit,  value * v) {
    return (from_start <= v && v < from_limit);
}

/* #define Is_from(from_start, from_limit, v)     \ */
/*    (from_start <= (value*)(v) && (value*)(v) < from_limit) */
/* Assuming v is a pointer (Is_block(v)), tests whether v points
   somewhere into the "from-space" defined by from_start and from_limit */

void forward (value *from_start,  /* beginning of from-space */
        value *from_limit,  /* end of from-space */
        value **next,       /* next available spot in to-space */
        value *p,           /* location of word to forward */
        int depth)          /* how much depth-first search to do */
/* What it does:  If *p is a pointer, AND it points into from-space,
   then make *p point at the corresponding object in to-space.
   If such an object did not already exist, create it at address *next
    (and increment *next by the size of the object).
   If *p is not a pointer into from-space, then leave it alone.

   The depth parameter may be set to 0 for ordinary breadth-first
   collection.  Setting depth to a small integer (perhaps 10)
   may improve the cache locality of the copied graph.
*/
 {
    value * v;
    value va = *p;
    if(Is_block(va)) {
        v = (value*)int_or_ptr_to_ptr(va);
    if(Is_from(from_start, from_limit, v)) {
      header_t hd = Hd_val(v);
      if(hd == 0) { /* already forwarded */
  *p = Field(v,0);
      } else {
  int i;
  int sz;
  value *new;
        sz = Wosize_hd(hd);
  new = *next+1;
        *next = new+sz;
        Hd_val(new) = hd;
  for(i = 0; i < sz; i++) {
    Field(new, i) = Field(v, i);
  }
  Hd_val(v) = 0;
  Field(v, 0) = ptr_to_int_or_ptr((void *)new);
  *p = ptr_to_int_or_ptr((void *)new);
  if (depth>0)
    for (i=0; i<sz; i++)
      forward(from_start, from_limit, next, &Field(new,i), depth-1);
      }
    }
  }
}

void forward_roots (value *from_start,  /* beginning of from-space */
        value *from_limit,  /* end of from-space */
        value **next,       /* next available spot in to-space */
        fun_info fi,        /* which args contain live roots? */
        struct thread_info *ti) /* where's the args array? */
/* Forward each live root in the args array */
 {
   value *args; int n; uintnat i;
   const uintnat *roots = fi+2;
   n = fi[1];
   args = ti->args;

   for(i = 0; i < n; i++) {
     assert (roots[i] < MAX_ARGS);
     forward(from_start, from_limit, next, args+roots[i], DEPTH);
   }
}

#define No_scan_tag 251
#define No_scan(t) ((t) >= No_scan_tag)

void do_scan(value *from_start,  /* beginning of from-space */
       value *from_limit,  /* end of from-space */
       value *scan,        /* start of unforwarded part of to-space */
       value **next)       /* next available spot in to-space */
/* Forward each word in the to-space between scan and *next.
  In the process, next may increase, so keep doing it until scan catches up.
  Leave alone:  header words, and "no_scan" (nonpointer) data.
*/
{
  value *s;
  s = scan;
  while(s < *next) {
    header_t hd = *((header_t *) s);
    mlsize_t sz = Wosize_hd(hd);
    int tag = Tag_hd(hd);
    if (!No_scan(tag)) {
      intnat j;
      for(j = 1; j <= sz; j++) {
  forward (from_start, from_limit, next, &Field(s, j), DEPTH);
      }
    }
    s += 1+sz;
  }
}
  
void do_generation (struct space *from,  /* descriptor of from-space */
        struct space *to,    /* descriptor of to-space */
        fun_info fi,         /* which args contain live roots? */
        struct thread_info *ti)  /* where's the args array? */
/* Copy the live objects out of the "from" space, into the "to" space,
   using fi and ti to determine the roots of liveness. */
{
  value *p = to->next;
  assert(from->next-from->start <= to->limit-to->next);
  forward_roots(from->start, from->limit, &to->next, fi, ti);
  do_scan(from->start, from->limit, p, &to->next);
  if(0)  fprintf(stderr,"%5.3f%% occupancy\n",
    (to->next-p)/(double)(from->next-from->start));
  from->next=from->start;
}

#if 0
/* This "gensize" function is only useful if the desired ratio is >2,
   but empirical measurements show that ratio=2 is better than ratio>2. */
uintnat gensize(uintnat words)
/* words is size of one generation; calculate size of the next generation */
{
  uintnat maxint = 0u-1u;
  uintnat n,d;
  /* The next few lines calculate a value "n" that's at least words*2,
     preferably words*RATIO, and without overflowing the size of an
     unsigned integer. */
  /* minor bug:  this assumes sizeof(uintnat)==sizeof(void*)==sizeof(value) */
  if (words > maxint/(2*sizeof(value)))
    abort_with("Next generation would be too big for address space\n");
  d = maxint/RATIO;
  if (words<d) d=words;
  n = d*RATIO;
  assert (n >= 2*words);
  return n;
}
#endif

void create_space(struct space *s,  /* which generation to create */
      uintnat n) /* size of the generation */
  /* malloc an array of words for generation "s", and
     set s->start and s->next to the beginning, and s->limit to the end.
  */

 {
  value *p;
  if (n >= MAX_SPACE_SIZE)
      abort_with("The generation is too large to be created\n");
  p = (value *)malloc(n * sizeof(value));
  if (p==NULL)
    abort_with ("Could not create the next generation\n");
  /*  fprintf(stderr, "Created a generation of %d words\n", n); */
  s->start=p;
  s->next=p;
  s->limit = p+n;
}

struct heap *create_heap()
/* To create a heap, first malloc the array of space-descriptors,
   then create only generation 0.  */
{
  int i;
  struct heap *h = (struct heap *)malloc(sizeof (struct heap));
  if (h==NULL)
    abort_with("Could not create the heap\n");
  create_space(h->spaces+0, NURSERY_SIZE);
  for(i=1; i<MAX_SPACES; i++) {
    h->spaces[i].start = NULL;
    h->spaces[i].next = NULL;
    h->spaces[i].limit = NULL;
  }
  return h;
}

struct thread_info *make_tinfo(void) {
  struct heap *h;
  struct thread_info *tinfo;
  h = create_heap();
  tinfo = (struct thread_info *)malloc(sizeof(struct thread_info));
  if (!tinfo)
    abort_with("Could not allocate thread_info struct\n");

  tinfo->heap=h;
  tinfo->alloc=h->spaces[0].start;
  tinfo->limit=h->spaces[0].limit;

  return tinfo;
}

void resume(fun_info fi, struct thread_info *ti)
/* When the garbage collector is all done, it does not "return"
   to the mutator; instead, it uses this function (which does not return)
   to resume the mutator by invoking the continuation, fi->fun.
   But first, "resume" informs the mutator
   of the new values for the alloc and limit pointers.
*/
 {
  struct heap *h = ti->heap;
  value *lo, *hi;
  uintnat num_allocs = fi[0];
  assert (h);
  lo = h->spaces[0].start;
  hi = h->spaces[0].limit;
  if (hi-lo < num_allocs)
    abort_with ("Nursery is too small for function's num_allocs\n");
  ti->alloc = lo;
  ti->limit = hi;
}

void garbage_collect(fun_info fi, struct thread_info *ti)
/* See the header file for the interface-spec of this function. */
{
  struct heap *h = ti->heap;
  int i;
  /* assert (h->spaces[0].limit == ti->limit);   */
  h->spaces[0].next = ti->alloc; /* this line is probably unnecessary */
  for (i=0; i<MAX_SPACES-1; i++) {
    /* Starting with the youngest generation, collect each generation
       into the next-older generation.  Usually, when doing that,
       there will be enough space left in the next-older generation
       so that we can break the loop by resuming the mutator. */

    /* If the next generation does not yet exist, create it */
    if (h->spaces[i+1].start==NULL) {
      int w = h->spaces[i].limit-h->spaces[i].start;
      create_space(h->spaces+(i+1), RATIO*w);
    }
    /* Copy all the objects in generation i, into generation i+1 */
    if(0)
      fprintf(stderr, "Generation %d:  ", i);
      do_generation(h->spaces+i, h->spaces+(i+1), fi, ti);
      /* If there's enough space in gen i+1 to guarantee that the
         NEXT collection into i+1 will succeed, we can stop here */
      if (h->spaces[i].limit - h->spaces[i].start
    <= h->spaces[i+1].limit - h->spaces[i+1].next) {
          resume(fi,ti);
          return;
      }
  }
    /* If we get to i==MAX_SPACES, that's bad news */
  abort_with("Ran out of generations\n");

  /* Can't reach this point */
  assert(0);
}

/* REMARK.  The generation-management policy in the garbage_collect function
   has a potential flaw.  Whenever a record is copied, it is promoted to
   a higher generation.  This is generally a good idea.  But there is
   a bounded number of generations.  A useful improvement would be:
   when it's time to collect the oldest generation (and we can tell
   it's the oldest, at least because create_space() fails), do some
   reorganization instead of failing.
 */

void reset_heap (struct heap *h) {
  fprintf(stderr, "Debug: in reset_heap\n");
  int i;
  for (i=0; i<MAX_SPACES; i++)
    h->spaces[i].next = h->spaces[i].start;
}

void free_heap (struct heap *h) {
  fprintf(stderr, "Debug: in free_heap\n");
  int i;
  for (i=0; i<MAX_SPACES; i++) {
    value *p = h->spaces[i].start;
    if (p!=NULL)
      free(p);
  }
  free (h);
}

\end{lstlisting}

\vspace{-0.4em}
\caption{Clight code for garbage collection}
\label{fig:forward}
\vspace{-1em}
\end{figure}