In the future we plan to improve the pure reasoning of graphs and
similar data structures, in particular to add automation.  We have
also begun to investigate integrating our techniques into the 
HIP/SLEEK toolchain~\cite{chin:hipsleek}, which as compared to VST 
provides more automation at the cost of lower expressivity.  We
are also interested in investigating better ways to handle the
kinds of undefined behavior upon which real~C systems code sometimes
relies.

\hide{\color{magenta}We are in the process of verifying a garbage
collector for the ``CertiCoq'' project, which is building
a certified compiler from Gallina to Clight. We would like to investigate
using our externally verified lemmas in HIP/SLEEK to verify code such as fast
exponentiation and more graph algorithms. We also would like to make
the interface between Coq and H/S simpler and cleaner.
One final direction we would like to investigate is using our new
connection to Coq to have H/S output certificates as it
verifies programs so that the system becomes more trustworthy.}

Our main contributions were as follows.  We developed a mathematical
graph library that was powerful enough to reason about graph-manipulating
algorithms written in real~C code.  We connected these mathematical graphs
to spatial graphs in the heap via separation logic.  We developed 
localization blocks to smoothly reason about a local action's effect on
a global context in a mechanized context, including a robust treatment
of modified program variables and existential quantifiers in postconditions.
We demonstrated our techniques on several nontrivial examples, including union-find
and spanning tree.  Our flagship example is the verification of the garbage collector 
for the CertiCoq project, during which we found two places in which the~C semantics
is too weak to define an OCaml-style GC.  We integrated our techniques into the
VST toolset. % and reported on our proof development.

