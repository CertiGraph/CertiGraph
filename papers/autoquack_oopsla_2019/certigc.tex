Our certification of a realistic garbage collector for the 
CertiCoq Project serves to show that our techniques 
scale well beyond short toy programs. 

\hide{The GC is our most complicated example,
and we will discuss some of its key proofs, but the larger
point here is that we completed this certification using 
exactly the framework and principles we have discussed
thus far. 

We enjoyed significant code 
reuse from our prior certifications, and when we stated 
new lemmas for the GC, we filed them away at the appropriate
``layers'' so that they may be reused in the future.
}%end hide

\subsection{Background}
\label{sec:gcbackground}

\marginpar{\tiny \color{blue} Highlight features discussed later.}
The CertiCoq Project compiles Gallina code into 
C-light, which it then compiles to assembly using CompCert.
CertiCoq provides garbage collection at
the C-light level in order to support Gallina's presumption of
infinite heap memory. The generational copying garbage 
collector {\color{magenta} supports variable-sized words, ... .}
Since CertiCoq aims to be end-to-end certified, the GC 
also needs certification.

\marginpar{\tiny \color{blue}The mutator can "drop"
a block, showing that it won't need it anymore}
The GC's mutator controls
an array of {\color{magenta}arguments} that it 
{\color{magenta}cares about}. 
These arguments may point at memory blocks 
in the heap, and, recursively, the fields of those
blocks may point at other blocks in the heap, to indefinite depth.
If the mutator {\color{magenta}can reach a block}, either directly
or indirectly, via this {\color{magenta}dependency graph}, 
it reserves the right to access it for reading or writing. 

When invoked by the mutator, the GC 
is allowed to modify the heap as
it sees fit, with the condition that it not damage the 
mutator's dependency graph. The mutator is agnostic 
to the specific layout of the heap, so long as 
it can access the same fields by following the same 
links.
% Other features of the original web, such as where 
% items were situated in the heap, may be changed. 
In \S\ref{sec:movetomathgraph}, we will explain how we abstracted 
this dependency graph into a mathematical graph 
(as seen in \S\ref{sec:mathinfra}), where the problem becomes
analogous to showing graph isomorphism.

%``We thus say that the arguments array is owned by the 
% mutator, but the heap is owned by the GC.''

\subsection{Structure of the Program {\color{magenta}(move to appendix?)}}
\label{sec:gcstructure}

% My goal here is to start with a broad overview
% and then work my way down to forward.
% At the end of this subsection I want it to be pretty
% clear that the whole game is just a series of calls
% to forward.
% This will set us up nicely for the decorated proof of
% forward in the next subsection.

% maybe a diagram showing how everything calls forward? 
CertiCoq uses a generational copying garbage collector 
that is inspired by the OCaml GC. 
\hide{It leans on the empirical observation that
new blocks often need to be collected soon after their
allocation, while blocks that survive this initial
culling tend to live for much longer.
% I'm wondering if we can elide this. 
% Treat the above as adequate revision and assume
% they know the rest of the story.
}%end hide
The heap is divided into a series of disjoint
spaces called \emph{generations}. The size of the first generation
is carefully calculated, and then subsequent generations
{\color{magenta}double in size}.
The mutator only ever allocates new memory in the first, 
smallest generation of heap, which is called the nursery. 
If it finds that the nursery is full, 
the mutator calls the GC to free up space.
The GC collects the nursery 
(now called the \emph{from} generation)
into the second generation (the \emph{to} generation): 
it examines the elements 
in \emph{from}, sees if they are accessible by
the mutator, and, if they are, 
copies them over to \emph{to}. This copying is achieved over a few steps, 
and we will explain these shortly, but the larger picture is that 
everything of importance in \emph{from} gets copied to \emph{to}, 
and so \emph{from} can safely be reset. 

An important subtlety here is that \emph{to} had enough 
room to accept \emph{from}'s items. 
In the (empirically improbable) worst case, 
\emph{all} of \emph{from}'s fields were copied over to \emph{to}.
Because \emph{to} has twice the capacity of \emph{from},
\emph{to} could not have been more than half full when 
the collection started.
This guarantee must be renewed before the next collection. 
So, in case the collection of the nursery caused
the second generation to become more than half full, 
the second generation is collected into the third. This makes 
both the first and second generations empty, thus ensuring 
the guarantee trivially. It should be 
clear to see that this may also trigger further collections in 
a cascade effect. The GC's task is only complete once this 
cascade (if any) is over. It returns control to the mutator,
which goes ahead with 
the allocation that it was trying to perform in the nursery.

\marginpar{\tiny \color{blue}should I mention the distinguished 
pointer that points at the copy?}
Having shown that the overall collection 
works via (a series of) two-generational collections, 
we now zoom in and explain a two-generation collection.
The GC starts at the mutator-owned arguments array, whose fields
are either data, or pointers that point at memory blocks in the 
heap. It ignores the data entirely, and, among the pointers, 
cares only for the pointers that point into the \emph{from} generation. 
For each pointer that points into \emph{from}, it copies its
target block to \emph{to}, simply adding it in its entirety 
after \emph{to}'s last-used memory field, which is called \emph{next}. 

This operation only takes care of the blocks in the heap that 
the arguments array was pointing at directly, so the GC still has to copy 
over indirectly-accessible blocks. Of course, the only way to 
access an indirect 
block is via one of the direct blocks that it has
just finished copying into a contiguous array. 
It starts at the old \emph{next} in \emph{to}
and works its way ``upwards'' through the freshly copied blocks, 
again looking exclusively for pointers that point into \emph{from}
and copying over their target blocks into \emph{to}. 
In the \emph{to} generation, these newly copied blocks 
simply get stacked atop our first batch of copied blocks. 

The mutator's dependency graph has indefinite depth, so the second 
batch of copied blocks may still have pointers into \emph{from}. 
However, thanks to this systematic
copying strategy, it is very easy to take care of all indirect
blocks. The GC simply keeps scanning upwards in 
\emph{to}, copying over blocks from \emph{from} as necessary, 
until the scanning pointer catches up to the last-used field in
\emph{to}. This completes a collection 
from \emph{from} to \emph{to}, copying all blocks that lived
in \emph{from} and were of interest to the mutator. 
\emph{from} is now reset.

A good question at this juncture is why this rather selective scan 
of the args array and the heap is good enough to collect \emph{from}. 
The GC definitely collected every direct block by scanning the args array,
but what of the indirect blocks? Couldn't there be valid indirect links
that start either below \emph{next} in the \emph{to} generation, 
or from other generations altogether? 

\marginpar{\tiny \color{blue}Gallina is purely functional, C-light is not.}
Both of these turn out to be impossible because the 
{\color{magenta}mutator is a purely functional language.}
The heap is chronologically
faithful, in that higher-indexed generations host
objects that were allocated earlier. Because
of the immutability of objects in a purely functional language, 
it is impossible for objects to point ``backwards'' to 
a younger generation, as the older object would not have
had known about the younger at the time of its allocation, and could not 
have been modified after its allocation. Fields living in generations 
younger than \emph{from} can point into \emph{from} during 
normal mutator activity, but this is
impossible at the time of collection: 
\emph{from} is only ever collected either if it is
the nursery or if a cascade effect has caused all generations younger
than it to be collected and reset. 
In fact, the only time the GC ever sees backwards pointers
is when it creates (and quickly fixes) them during during its 
activities.

\subsection{From C to Mathematical Graphs}
\label{sec:movetomathgraph}
TODO

\subsection{Forward}
\label{sec:gcforward}
The GC's main workhorse is the function \emph{forward}.
Whenever we have mentioned any kind of copying in the preceeding 
discussion, we have always meant a call to \emph{forward}. 
Given a pointer, it checks that the pointer's target is a block
in the \emph{from} generation, and, if so, 
makes a copy of that block in the \emph{to} generation. 
{\color{magenta}The function 
is robust and versatile, and gets called in a variety of different 
situations.}

Figure \ref{fig:forward} shows a decorated proof of the forward function.
The function's behavior is subtly different depending on 
whether the the pointer $p$ lives in the arguments array or in the 
heap itself. The difference lies in lines \ref{dummyref} and \ref{dummyref}, where 
we wish to update $p$ to point to the new forwarded block. 
When $p$ lives in the arguments array, this update is easy to 
reason about since $p$ is $\bigstar$-separated from our heap. 
When $p$ is itself in the heap, this update itself constitutues a
write to the heap. In the decorated proof, we show the second 
case. To this end, we assume in line \ref{dummyref} that p is (todo). 

% for second branch
%$\left\{\!\!\!\begin{array}{l@{}} boo \end{array}\right\}$
% 
\begin{figure}[t]
\vspace{-1ex}
  \begin{lstlisting}
// $\left\{\!\!\!\begin{array}{l@{}} \forall \gamma, \m{f\_info}, \m{t\_info}, \m{outlier}, \m{from}, \m{to}, \m{forp}.~ \null \\ \p{gc\_graph}(\gamma) * \p{fun\_info}(\m{f\_info}, \m{fi}) * \p{thread\_info}(\m{t\_info}, \m{ti}) /| \null \\ \m{compat}(\gamma, \m{f\_info}, \m{t\_info}, \m{outlier}, \m{from}, \m{to}, \m{forp}) /| \null \\ \tx{from\_start} = \m{gen\_start}(\gamma, \m{from}) /| \null \\ \tx{from\_limit} = \tx{from\_start} + \m{gen\_size}(\gamma, \m{from}) /| \null \\ \tx{next} = \m{nextaddr}(\m{t\_info}, \m{to}) /| \null \\ \exists \m{v},\m{n}.~\m{forp}=(\m{v},\m{n}) /| \tx{p} = \m{vaddr}(\gamma, \m{v}) + \m{n}\end{array}\right\}$
void forward (value *from_start,
              value *from_limit,
              value **next,
              value *p,
              int depth) {
  value * v;
// $\searrow \left\{\!\!\!\begin{array}{l@{}} \tx{...} \tx{p} = \m{vaddr}(\gamma, \m{v}) + \m{n} /| \null \\ \exists \m{hd},\m{flds}.~ \m{hd} = \m{mkhd}(\gamma, \m{v}) /| \m{flds} = \m{mkfv}(\gamma, \m{v}) /| \m{v} |-> \m{hd},\m{flds} \end{array}\right\}$
// $\left\{\!\!\!\begin{array}{l@{}} \m{above} + \tx{p} = \tx{\&}(\m{flds}[\m{n}]) /| \m{iopi}(\tx{*p}) \end{array}\right\}$
  value va = *p; 
?? $\swarrow \{\m{line\_1} + \exists...~ /| \tx{va} = \m{flds}[\m{n}] /| \m{iopi}(\tx{va})\}$
// magic - we know that $\m{flds}[\m{n}]$ is an edge
// $\{\exists \m{e}, \m{v'}, \m{b}, \m{i}.~ \tx{va} = \m{flds}[\m{e}] = \m{vaddr}(\gamma, \m{dst}(\gamma, \m{e})) = \m{vaddr}(\gamma, \m{v'}) = \m{Vptr}(\m{b},\m{i})\}$
  if(Is_block(va)) {
    v = (value*)int_or_ptr_to_ptr(va);
// $\{\tx{v} = \m{Vptr}(\m{b},\m{i}) = \m{vaddr}(\gamma, \m{v'})\}$
// $\{(\m{Is\_from()} /| \tx{...}) \lor (\neg \m{Is\_from()} /| \tx{...})\}$
    if(Is_from(from_start, from_limit, v)) {
// $\searrow \left\{\!\!\!\begin{array}{l@{}} \exists \m{hd'},\m{flds'}.~ \m{hd'} = \m{mkhd}(\gamma, \m{v'}) /| \m{flds'} = \m{mkfv}(\gamma, \m{v'}) /| \m{v'} |-> \m{hd'},\m{flds'} \end{array}\right\}$
      header_t hd = Hd_val(v);
// $\swarrow \left\{\!\!\!\begin{array}{l@{}} \tx{hd} = \m{Z2val}(\m{hd'})  /| ((\tx{hd == 1}  /| tx{...}) \lor \null \\ (\tx{hd==0} /| \m{flds'}[0] = \m{vaddr}(\m{copvert}(\gamma, \m{v'}))))\end{array}\right\}$
      if(hd == 0) {
// $\searrow \left\{\!\!\!\begin{array}{l@{}} \exists \m{hd'},\m{flds'}.~ \m{hd'} = \m{mkhd}(\gamma, \m{v'}) /| \m{flds'} = \m{mkfv}(\gamma, \m{v'}) /| \null \\ \m{v'} |-> \m{hd'},\m{flds'} /| \m{flds'}[0] = \m{vaddr}(\m{copvert}(\gamma, \m{v'})) \end{array}\right\}$
        t = Field(v,0);
// $\swarrow \left\{\!\!\!\begin{array}{l@{}} \tx{t} = \m{vaddr}(\m{copvert}(\gamma, \m{v'})) \end{array}\right\}$
// $\searrow \left\{\!\!\!\begin{array}{l@{}} \exists \m{hd},\m{flds}.~ \m{hd} = \m{mkhd}(\gamma, \m{v}) /| \m{flds} = \m{mkfv}(\gamma, \m{v}) /| \null \\ \m{v} |-> \m{hd},\m{flds} /| \tx{p} = \tx{\&}(\m{flds}[\m{n}]) \end{array}\right\}$
        *p = t;
?? $\left\{\!\!\!\begin{array}{l@{}} \m{v} |-> \m{hd},(\m{upd\_nth}(\m{flds}, \m{vaddr}(\m{copvert}(\gamma, \m{v'}))))\end{array}\right\}$
// $\swarrow$ ??
// $\left\{\!\!\!\begin{array}{l@{}} \exists \gamma'.~ \p{uf\_graph}(\gamma') /| \gamma' = \m{lgd}(\gamma,\m{e},\m{copvert}(\gamma, \m{v'})) /| \null \\ \tx{postcondition}\end{array}\right\}$
      } else {
// remind them of old facts...
// $\{ \tx{hd} = \m{Z2val}(\m{hd'})\}$
        int i; int sz; value *new;
        sz = Wosize_hd(hd); new = *next+1; *next = new+sz;
// $\{\tx{sz} = \m{getvertsz}(\m{hd'}) = \m{lenflds}(\gamma, \m{v'}) /| \tx{new} = \m{sp\_start}(\m{to}) + \m{used}(\m{to}) + 1 /| \tx{??}\}$        
        Hd_val(new) = hd; // ??
        for(i = 0; i < sz; i++) {
          Field(new, i) = Field(v, i);
        }
        Hd_val(v) = 0;
        Field(v, 0) = ptr_to_int_or_ptr((void *)new);
// $\searrow$
        *p = ptr_to_int_or_ptr((void *)new);
// $\swarrow$
        if (depth>0)
        for (i=0; i<sz; i++)
          forward(from_start, from_limit, 
            next, &Field(new,i), depth-1);
      }
    }
  }
}
\end{lstlisting}

\vspace{-0.4em}
\caption{Clight code and proof sketch for forward}
\label{fig:forward}
\vspace{-1em}
\end{figure}

\subsection{Issues}
\label{sec:gccsemantics}
\paragraph{Double-Bounded Pointer Comparisons.}
\paragraph{Pointer Subtraction.}
\paragraph{Block Alignment}

%\begin{figure}[t]
\vspace{-1ex}
  \begin{lstlisting}
if (h->spaces[i+1].start==NULL) {
  int w = h->spaces[i].limit-h->spaces[i].start;
  create_space(h->spaces+(i+1), RATIO*w);
}
\end{lstlisting}

\caption{Boundary Issues}
\label{fig:boundary}
\vspace{-1em}
\end{figure}