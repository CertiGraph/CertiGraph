\appendix

\section{Spanning Trees}
\label{apx:spanning}
\begin{figure}[htbp]
  \begin{lstlisting}
struct Node {
    int m;
    struct Node * l;
    struct Node * r; };

// We use $R$ to represent $\p{reachable}(\gamma,\tx x)$

void spanning(struct Node * x) { 
// $\{\p{graph}(\tx{x},\gamma)/|\gamma(\tx{x}).1=0\}$ 
    struct Node * l, * r; int root_mark;
// $\{\p{graph}(\tx x,\gamma) /| \exists l,r.~ \gamma(\tx{x}) = (0,l,r)\}$
// $\{\p{graph}(\tx x,\gamma) /| \gamma(\tx{x}) = (0,l,r)\}$
// $\{\p{vertices\_at}(\p{reachable}(\gamma,\tx x), \gamma) /| \gamma(\tx{x}) = (0,l,r)\}$
// $\{\p{vertices\_at}(R, \gamma) /| \gamma(\tx{x}) = (0,l,r)\}$
// $\searrow \{\tx x|-> 0,l,r /| \gamma(\tx{x}) = (0,l,r)\}$
    l = x -> l;
    r = x -> r;
    x -> m = 1;
// $\swarrow \{\tx x|-> 1,\tx{l},\tx{r} /| \gamma(\tx{x}) = (0,\tx{l},\tx{r}) /| \exists \gamma_1.~ \m{mark1}(\gamma, \tx{x}, \gamma_1)\}$
// $\{\exists \gamma_1.~\p{vertices\_at}(R, \gamma_1) /| \gamma(\tx{x}) = (0,\tx{l},\tx{r}) /| \m{mark1}(\gamma, \tx{x}, \gamma_1)\}$
// $\{\p{vertices\_at}(R, \gamma_1) /| \gamma(\tx{x}) = (0,\tx{l},\tx{r}) /| \m{mark1}(\gamma, \tx{x}, \gamma_1)\}$
    if (l) {
        root_mark = l -> m;
        if (root_mark == 0) {
            spanning(l);
        } else { x -> l = 0; } }
// $\left\{\!\!\!\begin{array}{l@{}}\exists\gamma_2. ~\p{vertices\_at}(R,\gamma_2)/| \gamma(\tx{x}) = (0,\tx{l},\tx{r})  /| \null \\ \m{mark1}(\gamma, \tx{x}, \gamma_1) /| \m{e\_span}(\gamma_1,\tx{x}.\text{L},\gamma_2)\end{array}\right\}$
// $\left\{\!\!\!\begin{array}{l@{}}\p{vertices\_at}(R,\gamma_2)/| \gamma(\tx{x}) = (0,\tx{l},\tx{r})  /| \null \\ \m{mark1}(\gamma, \tx{x}, \gamma_1) /|  \m{e\_span}(\gamma_1,\tx{x}.\text{L},\gamma_2)\end{array}\right\}$
    if (r) {
        root_mark = r -> m;
        if (root_mark == 0) {
           spanning(r);
        } else { x -> r = 0; } }
// $\left\{\!\!\!\begin{array}{l@{}}\exists\gamma_3.~\p{vertices\_at}(R,\gamma_3)/| \gamma(\tx{x}) = (0,\tx{l},\tx{r})  /| \null \\ \m{mark1}(\gamma, \tx{x}, \gamma_1) /| \m{e\_span}(\gamma_1,\tx{x}.\text{L},\gamma_2) /| \m{e\_span}(\gamma_2,\tx{x}.\text{R},\gamma_3)\end{array}\right\}$
} // $\{\exists \gamma_3.~\p{vertex\_at}(\p{reachable}(\gamma, \tx{x}), \gamma_3)/|\m{span}(\gamma,\tx{x},\gamma_3)\}$
  \end{lstlisting}
  \small
\begin{gather*}
  \p{vertices\_at}(\p{reachable}(\gamma_1, \tx x), \gamma_2)\defeq \underset{v\in\p{reachable}(\gamma_1, \tx x)}{\bigstar}v \mapsto \gamma_2(v)\\
  \begin{split}
  \m{span}(\gamma_1, \tx x, \gamma_2)\defeq &\m{mark}(\gamma_1,\tx x, \gamma_2) /| \gamma_1\!\uparrow(\lambda v. x\mathrel{{\stackrel{\gamma_1~}{\leadsto^{\star}_{0}}}} v) \text{ is a tree} /| \null\\
  & \gamma_1\!\uparrow\!(\lambda v.\neg x\mathrel{{\stackrel{\gamma_1~}{\leadsto^{\star}_{0}}}} v) = \gamma_2\!\uparrow\!(\lambda v.\neg x\mathrel{{\stackrel{\gamma_1~}{\leadsto^{\star}_{0}}}} v) /| \null\\
  & (\forall v.~x\mathrel{{\stackrel{\gamma_1~}{\leadsto^{\star}_{0}}}} v => \gamma_2\models x \leadsto v) /| \null\\
  & (\forall a,b.~x\mathrel{{\stackrel{\gamma_1~}{\leadsto^{\star}_{0}}}} a => \neg x\mathrel{{\stackrel{\gamma_1~}{\leadsto^{\star}_{0}}}} b => \neg \gamma_2\models a \leadsto b)\\
  \end{split}\\
  \m{e\_span}(\gamma_1, e, \gamma_2)\defeq
  \begin{cases}
    \gamma_1 - e = \gamma_2  & t(\gamma_1,e)=1\\
    \m{span}(\gamma_1, t(\gamma_1,e), \gamma_2) & t(\gamma_1,e)=0\\
  \end{cases}
\end{gather*}
\caption{Clight code and proof sketch for bigraph spanning tree.}
\label{fig:spanning}

\end{figure}


In Figure~\ref{fig:spanning} we show a simplified proof script for
the spanning tree algorithm.  Unlike graph marking, the spanning tree algorithm changes the
structure of the graph, leading to a more complicated specification,
in both the pure part and the spatial part. Observe that the $\m{span}$ relation is
rather long; the $\m{e\_span}$ handles the case of either calling spanning tree or deleting an edge.
Just like other parts of the paper, this algorithm has been machine-verified.

\section{Explanation of the Garbage Collector Program}
\label{apx:gcstructure}

% My goal here is to start with a broad overview
% and then work my way down to forward.
% At the end of this subsection I want it to be pretty
% clear that the whole game is just a series of calls
% to forward.
% This will set us up nicely for the decorated proof of
% forward in the next subsection.

% maybe a diagram showing how everything calls forward? 
CertiCoq uses a generational copying garbage collector 
that is inspired by the OCaml GC. 
\hide{It leans on the empirical observation that
new blocks often need to be collected soon after their
allocation, while blocks that survive this initial
culling tend to live for much longer.
% I'm wondering if we can elide this. 
% Treat the above as adequate revision and assume
% they know the rest of the story.
}%end hide
The heap is divided into a series of disjoint
spaces called \emph{generations}. The size of the first generation
is carefully calculated, and then subsequent generations
double in size.
The mutator only ever allocates new memory in the first, 
smallest generation of heap, which is called the nursery. 
If it finds that the nursery is full, 
the mutator calls the GC to free up space.
The GC collects the nursery 
(now called the \emph{from} generation)
into the second generation (the \emph{to} generation): 
it examines the elements 
in \emph{from}, sees if they are accessible by
the mutator, and, if they are, 
copies them over to \emph{to}. This copying is achieved over a few steps, 
and we will explain these shortly, but the larger picture is that 
everything of importance in \emph{from} gets copied to \emph{to}, 
and so \emph{from} can safely be reset. 

An important subtlety here is that \emph{to} had enough 
room to accept \emph{from}'s items. 
In the (empirically improbable) worst case, 
\emph{all} of \emph{from}'s fields were copied over to \emph{to}.
Because \emph{to} has twice the capacity of \emph{from},
\emph{to} could not have been more than half full when 
the collection started.
This guarantee must be renewed before the next collection. 
So, in case the collection of the nursery caused
the second generation to become more than half full, 
the second generation is collected into the third. This makes 
both the first and second generations empty, thus ensuring 
the guarantee trivially. It should be 
clear to see that this may also trigger further collections in 
a cascade effect. The GC's task is only complete once this 
cascade (if any) is over. It returns control to the mutator,
which goes ahead with 
the allocation that it was trying to perform in the nursery.

Having shown that the overall collection 
works via (a series of) two-generational collections, 
we now zoom in and explain a two-generation collection.
The GC starts at the mutator-owned arguments array, whose fields
are either data, or pointers that point at memory blocks in the 
heap. It ignores the data entirely, and, among the pointers, 
cares only for the pointers that point into the \emph{from} generation. 
For each pointer that points into \emph{from}, it copies its
target block to \emph{to}, simply adding it in its entirety 
after \emph{to}'s last-used memory field, which is called \emph{next}. 

This operation only takes care of the blocks in the heap that 
the arguments array was pointing at directly, so the GC still has to copy 
over indirectly-accessible blocks. Of course, the only way to 
access an indirect 
block is via one of the direct blocks that it has
just finished copying into a contiguous array. 
It starts at the old \emph{next} in \emph{to}
and works its way ``upwards'' through the freshly copied blocks, 
again looking exclusively for pointers that point into \emph{from}
and copying over their target blocks into \emph{to}. 
In the \emph{to} generation, these newly copied blocks 
simply get stacked atop our first batch of copied blocks. 

The mutator's dependency graph has indefinite depth, so the second 
batch of copied blocks may still have pointers into \emph{from}. 
However, thanks to this systematic
copying strategy, it is very easy to take care of all indirect
blocks. The GC simply keeps scanning upwards in 
\emph{to}, copying over blocks from \emph{from} as necessary, 
until the scanning pointer catches up to the last-used field in
\emph{to}. This completes a collection 
from \emph{from} to \emph{to}, copying all blocks that lived
in \emph{from} and were of interest to the mutator. 
\emph{from} is now reset.

A good question at this juncture is why this rather selective scan 
of the args array and the heap is good enough to collect \emph{from}. 
The GC definitely collected every direct block by scanning the args array,
but what of the indirect blocks? Couldn't there be valid indirect links
that start either below \emph{next} in the \emph{to} generation, 
or from other generations altogether? 

Both of these turn out to be impossible because the 
mutator behaves in a purely functional manner.
The heap is chronologically
faithful, in that higher-indexed generations host
objects that were allocated earlier. Because
of the immutability of objects in a purely functional language, 
it is impossible for objects to point ``backwards'' to 
a younger generation, as the older object would not have
had known about the younger at the time of its allocation, and could not 
have been modified after its allocation. Fields living in generations 
younger than \emph{from} can point into \emph{from} during 
normal mutator activity, but this is
impossible at the time of collection: 
\emph{from} is only ever collected either if it is
the nursery or if a cascade effect has caused all generations younger
than it to be collected and reset. 
In fact, the only time the GC ever sees backwards pointers
is when it creates (and quickly fixes) them during during its 
activities.
\newpage
\section{Code for Forward Relation}
\label{apx:forwardrelation}
  \begin{lstlisting}[basicstyle=\normalfont\tiny\tt],
  Inductive
forward_relation (from to : nat) : nat -> forward_t -> LGraph -> LGraph -> Prop :=
    fr_z : forall (depth : nat) (z : Z) (g : LGraph),
           forward_relation from to depth (inl (inl (inl z))) g g
  | fr_p : forall (depth : nat) (p : GC_Pointer) (g : LGraph),
           forward_relation from to depth (inl (inl (inr p))) g g
  | fr_v_not_in : forall (depth : nat) (v : VType) (g : LGraph),
                  vgeneration v <> from ->
                  forward_relation from to depth (inl (inr v)) g g
  | fr_v_in_forwarded : forall (depth : nat) (v : VType)
                          (g : LabeledGraph VType EType raw_vertex_block unit
                                 graph_info),
                        vgeneration v = from ->
                        raw_mark (vlabel g v) = true ->
                        forward_relation from to depth (inl (inr v)) g g
  | fr_v_in_not_forwarded_O : forall (v : VType)
                                (g : LabeledGraph VType EType raw_vertex_block unit
                                       graph_info),
                              vgeneration v = from ->
                              raw_mark (vlabel g v) = false ->
                              forward_relation from to 0 
                                (inl (inr v)) g (lgraph_copy_v g v to)
  | fr_v_in_not_forwarded_Sn : forall (depth : nat) (v : VType)
                                 (g : LabeledGraph VType EType raw_vertex_block unit
                                        graph_info) (g' : LGraph),
                               vgeneration v = from ->
                               raw_mark (vlabel g v) = false ->
                               let new_g := lgraph_copy_v g v to in
                               forward_loop from to depth
                                 (vertex_pos_pairs new_g (new_copied_v g to)) new_g
                                 g' ->
                               forward_relation from to (S depth) (inl (inr v)) g g'
  | fr_e_not_to : forall (depth : nat) (e : EType) (g : LGraph),
                  vgeneration (dst (pg_lg g) e) <> from ->
                  forward_relation from to depth (inr e) g g
  | fr_e_to_forwarded : forall (depth : nat) (e : EType) (g : LGraph),
                        vgeneration (dst (pg_lg g) e) = from ->
                        raw_mark (vlabel g (dst (pg_lg g) e)) = true ->
                        let new_g :=
                          labeledgraph_gen_dst g e
                            (copied_vertex (vlabel g (dst (pg_lg g) e))) in
                        forward_relation from to depth (inr e) g new_g
  | fr_e_to_not_forwarded_O : forall (e : EType) (g : LGraph),
                              vgeneration (dst (pg_lg g) e) = from ->
                              raw_mark (vlabel g (dst (pg_lg g) e)) = false ->
                              let new_g :=
                                labeledgraph_gen_dst
                                  (lgraph_copy_v g (dst (pg_lg g) e) to) e
                                  (new_copied_v g to) in
                              forward_relation from to 0 (inr e) g new_g
  | fr_e_to_not_forwarded_Sn : forall (depth : nat) (e : EType) (g g' : LGraph),
                               vgeneration (dst (pg_lg g) e) = from ->
                               raw_mark (vlabel g (dst (pg_lg g) e)) = false ->
                               let new_g :=
                                 labeledgraph_gen_dst
                                   (lgraph_copy_v g (dst (pg_lg g) e) to) e
                                   (new_copied_v g to) in
                               forward_loop from to depth
                                 (vertex_pos_pairs new_g (new_copied_v g to)) new_g
                                 g' ->
                               forward_relation from to (S depth) (inr e) g g'
  with forward_loop (from to : nat)
         : nat -> list forward_p_type -> LGraph -> LGraph -> Prop :=
    fl_nil : forall (depth : nat) (g : LGraph), forward_loop from to depth [] g g
  | fl_cons : forall (depth : nat) (g1 g2 g3 : LGraph) (f : forward_p_type)
                (fl : list forward_p_type),
              forward_relation from to depth (forward_p2forward_t f [] g1) g1 g2 ->
              forward_loop from to depth fl g2 g3 ->
              forward_loop from to depth (f :: fl) g1 g3
\end{lstlisting}


% \section{Code of Garbage Collector}
% Below we present the code of the garbage collector in its entirety.
% % need to split this up into different figures so they look neater.
% will do when we have a better sense of:
% 1. exactly what code we want
% 2. do we want comments?
% 3. should I make a flowchart of who calls whom and use that instead?

\begin{figure}[t]
\vspace{-1ex}
  \begin{lstlisting}
#include <stdlib.h>
#include <stdio.h>
#include <assert.h>
#include "config.h"
#include "gc.h"

/* The following 5 functions should (in practice) compile correctly in CompCert,
   but the CompCert correctness specification does not _require_ that
   they compile correctly:  their semantics is "undefined behavior" in
   CompCert C (and in C11), but in practice they will work in any compiler. */

int test_int_or_ptr (value x) /* returns 1 if int, 0 if aligned ptr */ {
    return (int)(((intnat)x)&1);
}

intnat int_or_ptr_to_int (value x) /* precondition: is int */ {
    return (intnat)x;
}

void * int_or_ptr_to_ptr (value x) /* precond: is aligned ptr */ {
    return (void *)x;
}

value int_to_int_or_ptr(intnat x) /* precondition: is odd */ {
    return (value)x;
}

value ptr_to_int_or_ptr(void *x) /* precondition: is aligned */ {
    return (value)x;
}

int Is_block(value x) {
    return test_int_or_ptr(x) == 0;
}

/* A "space" describes one generation of the generational collector. */
struct space {
  value *start, *next, *limit;
};
/* Either start==NULL (meaning that this generation has not yet been created),
   or start <= next <= limit.  The words in start..next  are allocated
   and initialized, and the words from next..limit are available to allocate. */

#define MAX_SPACES 12  /* how many generations */

#ifndef RATIO
#define RATIO 2   /* size of generation i+1 / size of generation i */
/*  Using RATIO=2 is faster than larger ratios, empirically */
#endif

#ifndef NURSERY_SIZE
#define NURSERY_SIZE (1<<16)
#endif
/* The size of generation 0 (the "nursery") should approximately match the
   size of the level-2 cache of the machine, according to:
      Cache Performance of Fast-Allocating Programs,
      by Marcelo J. R. Goncalves and Andrew W. Appel.
      7th Int'l Conf. on Functional Programming and Computer Architecture,
      pp. 293-305, ACM Press, June 1995.
   We estimate this as 256 kilobytes
     (which is the size of the Intel Core i7 per-core L2 cache).
    http://www.tomshardware.com/reviews/Intel-i7-nehalem-cpu,2041-10.html
    https://en.wikipedia.org/wiki/Nehalem_(microarchitecture)
   Empirical measurements show that 64k works well
    (or anything in the range from 32k to 128k).
*/

#ifndef DEPTH
#define DEPTH 0  /* how much depth-first search to do */
#endif

#ifndef MAX_SPACE_SIZE
#define MAX_SPACE_SIZE (1 << 29)
#endif
/* The restriction of max space size is required by pointer
   subtraction.  If the space is larger than this restriction, the
   behavior of pointer subtraction is undefined.
*/

struct heap {
  /* A heap is an array of generations; generation 0 must be already-created */
  struct space spaces[MAX_SPACES];
};

#ifdef DEBUG

int in_heap(struct heap *h, value v) {
  int i;
  for (i=0; i<MAX_SPACES; i++)
    if (h->spaces[i].start != NULL)
      if (h->spaces[i].start <= (value*)v &&
    (value *)v <= h->spaces[i].limit)
  return 1;
  return 0;
}

void printtree(FILE *f, struct heap *h, value v) {
  if(Is_block(v))
    if (in_heap(h,v)) {
      header_t hd = Field(v,-1);
      int sz = Wosize_hd(hd);
      int i;
      fprintf(f,"%d(", Tag_hd(hd));
      for(i=0; i<sz-1; i++) {
  printtree(f,h,Field(v,i));
  fprintf(f,",");
      }
      if (i<sz)
  printtree(f,h,Field(v,i));
      fprintf(f,")");
    }
    else {
      fprintf(f,"%8x",v);
    }
  else fprintf(f,"%d",v>>1);
}

void printroots (FILE *f, struct heap *h,
      fun_info fi,   /* which args contain live roots? */
      struct thread_info *ti) /* where's the args array? */
 {
   value *args; int n; uintnat i, *roots;
   roots = fi+2;
   n = fi[1];
   args = ti->args;

  for(i = 0; i < n; i++) {
    fprintf(f,"%d[%8x]:",roots[i],args[roots[i]]);
    printtree(f, h, args[roots[i]]);
    fprintf(f,"\n");
  }
  fprintf(f,"\n");
}

#endif

void abort_with(char *s) {
  fprintf(stderr, s);
  exit(1);
}

int Is_from(value* from_start, value * from_limit,  value * v) {
    return (from_start <= v && v < from_limit);
}

/* #define Is_from(from_start, from_limit, v)     \ */
/*    (from_start <= (value*)(v) && (value*)(v) < from_limit) */
/* Assuming v is a pointer (Is_block(v)), tests whether v points
   somewhere into the "from-space" defined by from_start and from_limit */

void forward (value *from_start,  /* beginning of from-space */
        value *from_limit,  /* end of from-space */
        value **next,       /* next available spot in to-space */
        value *p,           /* location of word to forward */
        int depth)          /* how much depth-first search to do */
/* What it does:  If *p is a pointer, AND it points into from-space,
   then make *p point at the corresponding object in to-space.
   If such an object did not already exist, create it at address *next
    (and increment *next by the size of the object).
   If *p is not a pointer into from-space, then leave it alone.

   The depth parameter may be set to 0 for ordinary breadth-first
   collection.  Setting depth to a small integer (perhaps 10)
   may improve the cache locality of the copied graph.
*/
 {
    value * v;
    value va = *p;
    if(Is_block(va)) {
        v = (value*)int_or_ptr_to_ptr(va);
    if(Is_from(from_start, from_limit, v)) {
      header_t hd = Hd_val(v);
      if(hd == 0) { /* already forwarded */
  *p = Field(v,0);
      } else {
  int i;
  int sz;
  value *new;
        sz = Wosize_hd(hd);
  new = *next+1;
        *next = new+sz;
        Hd_val(new) = hd;
  for(i = 0; i < sz; i++) {
    Field(new, i) = Field(v, i);
  }
  Hd_val(v) = 0;
  Field(v, 0) = ptr_to_int_or_ptr((void *)new);
  *p = ptr_to_int_or_ptr((void *)new);
  if (depth>0)
    for (i=0; i<sz; i++)
      forward(from_start, from_limit, next, &Field(new,i), depth-1);
      }
    }
  }
}

void forward_roots (value *from_start,  /* beginning of from-space */
        value *from_limit,  /* end of from-space */
        value **next,       /* next available spot in to-space */
        fun_info fi,        /* which args contain live roots? */
        struct thread_info *ti) /* where's the args array? */
/* Forward each live root in the args array */
 {
   value *args; int n; uintnat i;
   const uintnat *roots = fi+2;
   n = fi[1];
   args = ti->args;

   for(i = 0; i < n; i++) {
     assert (roots[i] < MAX_ARGS);
     forward(from_start, from_limit, next, args+roots[i], DEPTH);
   }
}

#define No_scan_tag 251
#define No_scan(t) ((t) >= No_scan_tag)

void do_scan(value *from_start,  /* beginning of from-space */
       value *from_limit,  /* end of from-space */
       value *scan,        /* start of unforwarded part of to-space */
       value **next)       /* next available spot in to-space */
/* Forward each word in the to-space between scan and *next.
  In the process, next may increase, so keep doing it until scan catches up.
  Leave alone:  header words, and "no_scan" (nonpointer) data.
*/
{
  value *s;
  s = scan;
  while(s < *next) {
    header_t hd = *((header_t *) s);
    mlsize_t sz = Wosize_hd(hd);
    int tag = Tag_hd(hd);
    if (!No_scan(tag)) {
      intnat j;
      for(j = 1; j <= sz; j++) {
  forward (from_start, from_limit, next, &Field(s, j), DEPTH);
      }
    }
    s += 1+sz;
  }
}
  
void do_generation (struct space *from,  /* descriptor of from-space */
        struct space *to,    /* descriptor of to-space */
        fun_info fi,         /* which args contain live roots? */
        struct thread_info *ti)  /* where's the args array? */
/* Copy the live objects out of the "from" space, into the "to" space,
   using fi and ti to determine the roots of liveness. */
{
  value *p = to->next;
  assert(from->next-from->start <= to->limit-to->next);
  forward_roots(from->start, from->limit, &to->next, fi, ti);
  do_scan(from->start, from->limit, p, &to->next);
  if(0)  fprintf(stderr,"%5.3f%% occupancy\n",
    (to->next-p)/(double)(from->next-from->start));
  from->next=from->start;
}

#if 0
/* This "gensize" function is only useful if the desired ratio is >2,
   but empirical measurements show that ratio=2 is better than ratio>2. */
uintnat gensize(uintnat words)
/* words is size of one generation; calculate size of the next generation */
{
  uintnat maxint = 0u-1u;
  uintnat n,d;
  /* The next few lines calculate a value "n" that's at least words*2,
     preferably words*RATIO, and without overflowing the size of an
     unsigned integer. */
  /* minor bug:  this assumes sizeof(uintnat)==sizeof(void*)==sizeof(value) */
  if (words > maxint/(2*sizeof(value)))
    abort_with("Next generation would be too big for address space\n");
  d = maxint/RATIO;
  if (words<d) d=words;
  n = d*RATIO;
  assert (n >= 2*words);
  return n;
}
#endif

void create_space(struct space *s,  /* which generation to create */
      uintnat n) /* size of the generation */
  /* malloc an array of words for generation "s", and
     set s->start and s->next to the beginning, and s->limit to the end.
  */

 {
  value *p;
  if (n >= MAX_SPACE_SIZE)
      abort_with("The generation is too large to be created\n");
  p = (value *)malloc(n * sizeof(value));
  if (p==NULL)
    abort_with ("Could not create the next generation\n");
  /*  fprintf(stderr, "Created a generation of %d words\n", n); */
  s->start=p;
  s->next=p;
  s->limit = p+n;
}

struct heap *create_heap()
/* To create a heap, first malloc the array of space-descriptors,
   then create only generation 0.  */
{
  int i;
  struct heap *h = (struct heap *)malloc(sizeof (struct heap));
  if (h==NULL)
    abort_with("Could not create the heap\n");
  create_space(h->spaces+0, NURSERY_SIZE);
  for(i=1; i<MAX_SPACES; i++) {
    h->spaces[i].start = NULL;
    h->spaces[i].next = NULL;
    h->spaces[i].limit = NULL;
  }
  return h;
}

struct thread_info *make_tinfo(void) {
  struct heap *h;
  struct thread_info *tinfo;
  h = create_heap();
  tinfo = (struct thread_info *)malloc(sizeof(struct thread_info));
  if (!tinfo)
    abort_with("Could not allocate thread_info struct\n");

  tinfo->heap=h;
  tinfo->alloc=h->spaces[0].start;
  tinfo->limit=h->spaces[0].limit;

  return tinfo;
}

void resume(fun_info fi, struct thread_info *ti)
/* When the garbage collector is all done, it does not "return"
   to the mutator; instead, it uses this function (which does not return)
   to resume the mutator by invoking the continuation, fi->fun.
   But first, "resume" informs the mutator
   of the new values for the alloc and limit pointers.
*/
 {
  struct heap *h = ti->heap;
  value *lo, *hi;
  uintnat num_allocs = fi[0];
  assert (h);
  lo = h->spaces[0].start;
  hi = h->spaces[0].limit;
  if (hi-lo < num_allocs)
    abort_with ("Nursery is too small for function's num_allocs\n");
  ti->alloc = lo;
  ti->limit = hi;
}

void garbage_collect(fun_info fi, struct thread_info *ti)
/* See the header file for the interface-spec of this function. */
{
  struct heap *h = ti->heap;
  int i;
  /* assert (h->spaces[0].limit == ti->limit);   */
  h->spaces[0].next = ti->alloc; /* this line is probably unnecessary */
  for (i=0; i<MAX_SPACES-1; i++) {
    /* Starting with the youngest generation, collect each generation
       into the next-older generation.  Usually, when doing that,
       there will be enough space left in the next-older generation
       so that we can break the loop by resuming the mutator. */

    /* If the next generation does not yet exist, create it */
    if (h->spaces[i+1].start==NULL) {
      int w = h->spaces[i].limit-h->spaces[i].start;
      create_space(h->spaces+(i+1), RATIO*w);
    }
    /* Copy all the objects in generation i, into generation i+1 */
    if(0)
      fprintf(stderr, "Generation %d:  ", i);
      do_generation(h->spaces+i, h->spaces+(i+1), fi, ti);
      /* If there's enough space in gen i+1 to guarantee that the
         NEXT collection into i+1 will succeed, we can stop here */
      if (h->spaces[i].limit - h->spaces[i].start
    <= h->spaces[i+1].limit - h->spaces[i+1].next) {
          resume(fi,ti);
          return;
      }
  }
    /* If we get to i==MAX_SPACES, that's bad news */
  abort_with("Ran out of generations\n");

  /* Can't reach this point */
  assert(0);
}

/* REMARK.  The generation-management policy in the garbage_collect function
   has a potential flaw.  Whenever a record is copied, it is promoted to
   a higher generation.  This is generally a good idea.  But there is
   a bounded number of generations.  A useful improvement would be:
   when it's time to collect the oldest generation (and we can tell
   it's the oldest, at least because create_space() fails), do some
   reorganization instead of failing.
 */

void reset_heap (struct heap *h) {
  fprintf(stderr, "Debug: in reset_heap\n");
  int i;
  for (i=0; i<MAX_SPACES; i++)
    h->spaces[i].next = h->spaces[i].start;
}

void free_heap (struct heap *h) {
  fprintf(stderr, "Debug: in free_heap\n");
  int i;
  for (i=0; i<MAX_SPACES; i++) {
    value *p = h->spaces[i].start;
    if (p!=NULL)
      free(p);
  }
  free (h);
}

\end{lstlisting}

\vspace{-0.4em}
\caption{Clight code for garbage collection}
\label{fig:forward}
\vspace{-1em}
\end{figure}
