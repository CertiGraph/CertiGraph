In this section we first (\S\ref{sec:rulessound}) ensure our feet are solidly planted by proving why \textsc{Localize} is sound, and indeed equivalent to \textsc{Frame}.  Second (\S\ref{sec:freevars}),
we address the bug from \S\ref{sec:localblocks} and show that \textsc{Localize} is indeed
strong enough to robustly handle modified local program variables.  Third (\S\ref{sec:linkedex}),
we showcase two additional features of our framework, linked existentials and a fold/unfold style
for spatial graphs, by covering the verification of a program that marks a cyclic graph.
Finally (\S\ref{sec:application}), we briefly discuss some additional examples we have handled.
Our flagship example, the garbage collector for the CertiCoq project, will be covered in~\S\ref{sec:certigc}.


%aim to detail additional aspects of our framework, particularly
%how the \textsc{Localize} rule can handle modified program variables and existentials
%in postconditions, but also how we can explore a spatial graph ``recursively'' via
%fold-unfold.  We also .

%Here we explain two important improvements the \infrulestyle{Localize} rule
%makes on the \infrulestyle{Ramify} rule.
%It can link existentially-quantified predicates from the local postcondition
%$L_2$ to the global postcondition $G_2$, and also handle program variables
%that are modified during a localization block. We need to understand
%{\color{magenta}and solve} the former in order to provide a robust,
%lightweight, and fully-automatable solution to the latter.


\subsection{Soundness of $\infrulestyle{Localize}$}
\label{sec:rulessound}

\begin{figure*}
{\footnotesize
\[
\hspace{-1em}
\begin{array}{l}
\infrule{}{
  G_1 |- L_1 * R \\
  \infrule{}{\{L_1\}~c~\{\exists x.~ L_2\}}{\{L_1 * R\}~c~\{(\exists x.~ L_2) * R\}}{\infrulestyle{Frame}} \\
  \qquad \quad \infrule{}{\infrule{}{\vdots}{(\exists x.~ L_2) * (\forall x.~ (L_2 --* G_2)) |- \exists x.~ G_2}{\infrulestyle{Tauto}}}{(\exists x.~ L_2) * R |- \exists x.~ G_2}{\infrulestyle{Cut}~(\dagger)}
  }
  {\{G_1\}~c~\{\exists x.~ G_2\}}{\textsc{Cons}} \\ \\
[-5pt]
(\dagger)~R |- \forall x.~ (L_2 --* G_2) ~\text{is a premise of \infrulestyle{Localize}}
\end{array}
\]


\[
\hspace{-1em}
\infrule{}{
\infrule{}{
%  P * F |- P * F \\
  \infrule{}{\{P\}~c~\{Q\} \\ Q |- \exists x_f.~ Q}{\{P\}~c~\{\exists x_f.~ Q\}}{\infrulestyle{Cons}} \\
  \qquad F |- \forall x_f.~(Q --* (Q * F))
  %\quad \infrule{}{\infrule{}{\vdots}{(\exists x.~ L_2) * (\forall x.~ (L_2 --* G_2)) |- \exists x.~ G_2}{\infrulestyle{Tauto}}}{(\exists x.~ L_2) * R |- \exists x.~ G_2}{\infrulestyle{Cut}}
  }
  {\{P * F\}~c~\{\exists x_f.~(Q * F)\}}{\textsc{Localize}} \\
  \qquad \qquad \exists x_f.~(Q * F) |- Q * F
  }
  {\{P * F\}~c~\{Q * F\}}{\textsc{Cons}}
\]
}
\vspace{-1em}
\caption{Proving \infrulestyle{Localize} from \infrulestyle{Frame}, and conversely \infrulestyle{Frame} from \infrulestyle{Localize}}
\label{fig:rampqproofs}
\end{figure*}

In Figure~\ref{fig:rampqproofs} we put the proof sketches that show that
\infrulestyle{Localize} and \infrulestyle{Frame} are equivalent.  They require a
little care with quantifiers, but are in essence straightforward.
In the latter proof set $R \defeq F$, choose $x_f$ fresh, and range the quantifiers
over the unit type.  Notice that in both directions the restriction on modified
program variables is satisfied: in the first proof, \textsc{Localize}'s side
condition that $\FV(R) \cap \MV(c) = \emptyset$ is exactly what \textsc{Frame} needs;
in the second, \textsc{Frame}'s side condition that $\FV(F) \cap \MV(c) = \emptyset$
is exactly what \textsc{Localize} needs (since $R \defeq F$).
The equivalence between \textsc{Frame} and \textsc{Localize} means that %is important because
our techniques will be sound in any separation logic.

\paragraph{Note on notation.} Although we do not do so in Figure~\ref{fig:find}, localization blocks can safely nest.  When the ramification entailment is not noteworthy we can omit the $\ramify(i)$ reference in pen-and-paper proofs.  When we wish to save vertical space we can write $\{ G_1 \} \searrow \{ L_1 \}$ and $\{ G_2 \} \swarrow \{ L_2 \}$.
We also note that since \infrulestyle{Localize} can derive \infrulestyle{Frame}, our notation for localization blocks clarifies pen-and-paper uses of \infrulestyle{Frame}, especially in multi-line contexts with nontrivial frame $F$.
\hide{, for which the current popular notation to express \infrulestyle{Frame} involves a liberal use of
ellipses, \emph{e.g.}:

\vspace{5pt}

\begin{minipage}{.25\textwidth}
Old notation:
\begin{lstlisting}
// $\{ P_1 * F_1 * F_2 * F_3 \}$
   $c_1$;
// $\{ P_2 * \ldots \}$
   $c_2$;
// $\{ P_3 * \ldots \}$
   $c_3$;
// $\{ P_4 * F_1 * F_2 * F_3 \}$
\end{lstlisting}
\end{minipage} \quad \vline \; ~~~
\begin{minipage}{.3\textwidth}
New notation:
\begin{lstlisting}[numbers=none]
// $\{ P_1 * F_1 * F_2 * F_3 \} \searrow \{ P_1 \}$
      $c_1$;
//    $\{ P_2 \}$
      $c_2$;
//    $\{ P_3 \}$
      $c_3$;
// $\{ P_4 * F_1 * F_2 * F_3 \} \swarrow \{ P_4 \}$
\end{lstlisting}
\end{minipage}
\vspace{-0.75ex}
}

\hide{
\subsection{Forward ramification}
\marginpar{\tiny \color{blue} What to do about this now that H/S has been cut?}

{\color{magenta}The forward style of reasoning employed by HIP/SLEEK uses the existential wand $--o$ to express ramifications instead of the more typical universal wand $--*$.  The standard $--*$ form of ramification is weaker, but the strongest postcondition style using $--o$ can also get the job done without too much extra work since:}
\vspace*{5.25ex}
%\[\infrulestyle{WandToEwand}~~~~~~~~~~~~~~~~~~~~~~~~~~~~~~~~~~~~~~~~~~~~~~~~~~~~~~~~~~~~~~~~~~~~~~~\]
\vspace*{-6ex}
\[
\inferrule[WandToEwand]
{G_1 |- L_1 * (L_2 --* G_2)}
{(L_1 --o G_1) * L_2 |- G_2} \m{precise}(L_1)
\]
\[
\begin{array}{@{}l@{}l@{}}
\m{precise}(P) ~ \defeq ~ & (\sigma_1 |= P) => (\sigma_2 |= P) => \\
& ~~ (\sigma_1 \oplus \sigma_1' \! = \! \sigma) => (\sigma_2 \oplus \sigma_2' \! = \! \sigma) => \sigma_1 \! = \! \sigma_2
\end{array}
\]
In \S\ref{sec:ramifylib} we will discuss the ``supplemental'' spatial libraries, which ensures
the preciseness of our key predicates.
}

\subsection{Smoothly handling modified program variables}
\label{sec:freevars}

Consider using \textsc{Localize} to verify the program
\vspace{-1ex}
\begin{lstlisting}
  $// \{ \tx{x} = 5 /| A \} \searrow \{\tx{x} = 5 /| B \}$
    `\hl{...; x = x + 1; ...;}`
  $// \{\tx{x} = 6 /| C \}$
  $// \swarrow \{ \tx{x} = 6 /| D \}$ 
\end{lstlisting}
\vspace{-1ex}
Suppose that other (elided) lines of the program make localization desirable, even though it is overkill for a single assignment.  The key issue is that if one sets $R \defeq L_2 --* G_2$, as
we tried to do in \S\ref{sec:localblocks}, then the program variable {\li{x}} appears in all four positions in the ramification entailment
\vspace{-1ex}
\[
\overbrace{(\li{x} \! = \! 5 /| A)}^{G_1} |- \overbrace{(\li{x} \! = \! 5 /| B)}^{L_1} * \big(\overbrace{(\li{x} \! = \! 6 /| C)}^{L_2} --* \overbrace{(\li{x} \! = \! 6 /| D)}^{G_2}\big)
\vspace{-1ex}
\]
For the sake of simplicity, assume that in the above snippet only \li{x} is modified
and that \li{x} does not appear free in $A$, $B$, $C$ or $D$.  Let us further assume
that, modulo the local variable issue we are trying to solve, the entailment holds.
In other words, let us assume that $A |- B * (C --* D)$.

Turning to the local variable issue itself, in~\S\ref{sec:localblocks} we observed
that $L_2 --* G_2$ does \textbf{not} ignore the modified program variable \tx{x},
preventing us from meeting \infrulestyle{Localize}'s side condition\footnote{There is
another problem: in the standard model for local variable treatment in separation logic,
the separating implication is vacuously true since \tx{x} cannot simultaneously be
both $5$ and $6$.  But since two fatal problems are overkill, let us move on.}.
Intuitively, the side condition on \infrulestyle{Localize} is a bit too strong
since it prevents us from mentioning variables in the postconditions that have been modified
by code $c$.  As in other cases when life gets tough, what we need is an elegant little dance, and as with most dances, one should lead by example.

First, define $\hat{L_2}(x_f) \defeq (x_f \! = \! 6 /| C)$ and
$\hat{G_2}(x_f) \defeq (x_f \! = \! 6 /| D)$, \emph{i.e.} replace the troublesome program variable \tx{x} in $L_2$ and $G_2$ with a harmless fresh metavariable $x_f$.  Next, notice that with a carefully chosen existential quantifier, we can express the original $L_2$ with the new $\hat{L_2}$
while keeping the troublesome program variable \tx{x} isolated and shift the above decorated program
into the form
\begin{lstlisting}
  $// \{ \tx{x} \! = \! 5 /| A \} \searrow \{\tx{x} \! = \! 5 /| B \}$
    `\hl{...; x = x + 1; ...;}`
  $// \{\tx{x} \! = \! 6 /| C\}$
  $\label{code:locexamplepresw}// \swarrow \{\exists x_f.~ (x_f = \tx{x}) /| (x_f \! = \! 6 /| C) \} $
  $\label{code:locexamplepostsw}// \{ \exists x_f.~ (x_f = \tx{x}) /| (x_f \! = \! 6 /| D) \}  \}$
  $// \{ \tx{x} \! = \! 6 /| D \}  \}$
\end{lstlisting}
Notice that lines~\ref{code:locexamplepresw} and~\ref{code:locexamplepostsw}
are exactly in the form $\exists x_f. \hat{\hat{L_2}}(x_f)$ and $\exists x_f. \hat{\hat{G_2}}(x_f)$, \emph{i.e.} exactly in the format permitted by \textsc{Localize}, where $\hat{\hat{L_2}}(x_f) \defeq (x_f = \tx{x}) /| \hat{L_2}(x_f)$, \emph{i.e.} $(x_f = \tx{x}) /| (x_f \! = \! 6 /| C)$ and $\hat{\hat{G_2}}(x_f)$ is similar. % \defeq (x_f = \tx{x}) /| (x_f \! = \! 6 /| C)$
Now apply \textsc{Localize} with $R \defeq \forall x_f.~\hat{L_2}(x_f) --* \hat{G_2}(x_f)$, \emph{i.e.} $\forall x_f.~(x_f \! = \! 6 /| C) --* (x_f \! = \! 6 /| D)$.  By construction, $R$
is free from all program variables modified by $c$, so \textsc{Localize}'s side condition is
satisfied.  All that remains is to prove \textsc{Localize}'s two entailments.  Let us consider
them in reverse order.  The second one is $R |- \forall x_f.~ \big(\hat{\hat{L_2}}(x_f) --* \hat{\hat{G_2}}(x_f)\big)$, \emph{i.e.}
\[
\forall x_f.~\big(\hat{L_2}(x_f) --* \hat{G_2}(x_f)\big) |- \forall x_f.~\Big(\big((\tx{x} = x_f) /| \hat{L_2}(x_f)\big) --* \big((\tx{x} = x_f) /| \hat{G_2}(x_f)\big)\Big)
\]
This turns out to be just a long-winded tautology, and can be handled automatically by a tool.

The first of \textsc{Localize}'s entailments is $G_1 |- L_1 --* R$, \emph{i.e.}
\[
(\li{x} \! = \! 5 /| A) |- (\li{x} \! = \! 5 /| B) * \big(\forall x_f.~ (x_f \! = \! 6 /| C) --* (x_f \! = \! 6 /| D) \big)
\]
This can be broken into the ``variable-related'' part $\li{x} \! = \! 5 |- (\li{x} \! = \! 5) * \big(\forall x_f.~ (x_f \! = \! 6 --* x_f \! = \! 6)\big)$, which is also a tautology, and the ``spatial'' part $A |- B * (C --* D)$, which is true by assumption above.

With careful engineering, the entire modified-variable dance detailed above 
can be done fully automatically, in a way that is completely hidden to end-users.  
The only remaining proof
goal is the spatial part, which captures the key action of the localization block.  To solve
these in practice, we end up applying generic lemmas from \S\ref{sec:spacegraph}.

\paragraph{Discussion.} The delicacy and detail in the dance above may seem to be making
mountains out of molehills, since a careful treatment of modified program variables is
hardly a sexy topic.  Indeed, in pen-and-paper systems they are molehills, with any
number of workarounds including: making local program transformations to introduce
fresh variables and arguing for program equivalence, using
variables-as-resource~\cite{bornat:var}, or even just sweeping the issue under the rug.

In a mechanized context, working with existing toolsets, these kinds of solutions are
not viable.  Either we must reinvent a \textbf{very, very} large wheel---combined,
VST and CompCert contain about 840k LOC---or we must dance within their constraints.
VST does not use variables as resource, nor does it have modules to reason about program equivalence.  Moreover, it is hardly unique in these respects: most other mechanized verification systems do not support these solutions either~\cite{beckert:2007,distefanop08,bengtson:charge,chin:hipsleek}.  By respecting the design
decisions taken by most existing tools, our solutions can be incorporated more easily; in~\S\ref{sec:development} we will see that our additions to VST are less than 1\% of its codebase.

\subsection{Linked existentials}
\label{sec:linkedex}

\begin{figure}[t]
  \begin{lstlisting}
struct Node {$\label{code:nodedefstart}$
  int  _Alignas(16) m;
  struct Node * _Alignas(8) l;
  struct Node * r; };$\label{code:nodedefend}$

void mark(struct Node * x) { // $\label{code:markstart}\{\p{graph}(\tx{x},\gamma)\}$
  struct Node * l, * r; int root_mark; $\label{code:inmark}$
  if (x == 0) return;
// $\{\p{graph}(\tx x,\gamma) /| \exists m,l,r.~ \gamma(\tx{x}) = (m,l,r)\}$
// $\label{code:globalbeforerootmark}\{\p{graph}(\tx x,\gamma) /| \gamma(\tx{x}) = (m,l,r)\}$
// $\label{code:beforerootmark}\searrow \{\tx x|-> m,-,l,r \}$
      root_mark = x -> m;
// $\label{code:afterrootmark}\swarrow \{\tx x|-> m,-,l,r /| m = \tx{root\_mark} \}$
// $\label{code:globalafterrootmark}\{\p{graph}(\tx x,\gamma) /| \gamma(\tx{x}) = (m,l,r) /| m = \tx{root\_mark}\}$
  if (root_mark == 1) return;
// $\{\p{graph}(\tx x,\gamma) /| \gamma(\tx{x}) = (0,l,r) \}$
// $\label{code:markbeforetripleramify}\searrow \{\tx x|-> 0,-,l,r /| \gamma(\tx{x}) = (0,l,r)\}$
      l = x -> l;
      r = x -> r;
      x -> m = 1;
// $\label{code:markaftertripleramify}\swarrow \{\tx x|-> 1,-,\tx{l},\tx{r} /| \gamma(\tx{x}) = (0,\tx{l},\tx{r}) /| \exists \gamma'.~ \m{mark1}(\gamma, \tx{x}, \gamma')\}$
// $\{\exists \gamma'.~ \p{graph}(\tx x,\gamma') /| \gamma(\tx{x}) = (0,\tx{l},\tx{r}) /| \m{mark1}(\gamma, \tx{x}, \gamma')\}$
// $\label{code:beforemarkl}\{\p{graph}(\tx x,\gamma') /| \gamma(\tx{x}) = (0,\tx{l},\tx{r}) /| \m{mark1}(\gamma, \tx{x}, \gamma')\}$
// $\searrow \{\p{graph}(\tx l, \gamma')\}$
      mark(l);
// $\swarrow \{\exists \gamma''.~ \p{graph}(\tx l, \gamma'') /| \m{mark}(\gamma', \tx{l}, \gamma'')\}$
// $\label{code:aftermarkl}\left\{\!\!\!\begin{array}{l@{}}\exists \gamma''.~ \p{graph}(\tx x,\gamma'') /| \gamma(\tx{x}) = (0,\tx{l},\tx{r}) /| \null \\ \m{mark1}(\gamma, \tx{x}, \gamma') /| \m{mark}(\gamma', \tx{l}, \gamma'')\end{array}\right\}$
// $\left\{\!\!\!\begin{array}{l@{}}\p{graph}(\tx x,\gamma'') /| \gamma(\tx{x}) = (0,\tx{l},\tx{r}) /| \null \\ \m{mark1}(\gamma, \tx{x}, \gamma') /| \m{mark}(\gamma', \tx{l}, \gamma'')\end{array}\right\}$
// $\searrow \{\p{graph}(\tx r, \gamma'')\}$
      mark(r);
// $\swarrow \{\exists \gamma'''.~ \p{graph}(\tx r, \gamma''') /| \m{mark}(\gamma'', \tx{r}, \gamma''')\}$
// $\label{code:outmark}\left\{\!\!\!\begin{array}{l@{}}\exists \gamma'''.~ \p{graph}(\tx x,\gamma''') /| \gamma(\tx{x}) = (0,\tx{l},\tx{r}) /| \null \\ \m{mark1}(\gamma, \tx{x}, \gamma') /| \m{mark}(\gamma', \tx{l}, \gamma'') /| \m{mark}(\gamma'', \tx{r}, \gamma''')\end{array}\right\}$
} // $\label{code:markend}\{\exists \gamma'''.~ \p{graph}(\tx x,\gamma''') /| \m{mark}(\gamma, \tx{x}, \gamma''')\}$
\end{lstlisting}
%% \vspace{-8pt}
\caption{Clight code and proof sketch for bigraph mark.}
% {\color{magenta} The steps that induce
%  ramifications are indicated with $\ramify_i$, where the associated ramification entailment is equation number $i$.}} %whose numbers point with their associated ramification entailment reference.}
%\vspace{-19pt}
\label{fig:markgraph}
\end{figure}


We have already seen that allowing existentials in postconditions lets us handle modified program
variables properly.  However, these ``linked existentials''---recall that our previous technique
hinged on the fact that the existential witness to the variable $x_f$ in the local postcondition
$L_2$ was carried over to the corresponding existential witness in the global postcondition---have
other uses as well.  To illustrate them, and to demonstrate other aspects of our system, in particular our ability to explore a graph recursively via fold/unfold, we consider another example.

In Figure~\ref{fig:markgraph} we put the code and proof sketch of the classic \li{mark} algorithm that visits and colors every reachable node in a heap-represented graph.  The \li{mark} example contrasts from \li{find} in several respects.  First, it modifies the labels of nodes instead of the edges.  Second, each node has two outgoing edges rather than one, so the graph can have a more complex shape.  Third, the graph can be nontrivially cyclic.  Lastly, the specification we certify (lines \ref{code:markstart} and \ref{code:markend}) is \emph{local} rather than \emph{global}:
\vspace*{-1ex}
\[
\{\p{m\_graph}(\li{x},\gamma)\}~\li{mark(x)}~\{\exists \gamma'.~ \p{m\_graph}(\li{x},\gamma') /| \m{mark}(\gamma, \li{x}, \gamma')\}
\vspace*{-1ex}
\]
The specification is again stated with mathematical $\gamma$, although in this case $\gamma(x)$ maps to triples $(m,l,r)$, where $m$ is a ``mark'' bit (0 or 1) and $\{l,r\} \subseteq V(\gamma) \uplus \{\mathtt{null}\}$ are the neighbors of $v$.  By ``local'', we mean that the predicate $\p{m\_graph}(\li{x},\gamma)$ says that the heap represents \emph{only the nodes in $\gamma$ that are reachable from \li{x}}.  Rather than passing the entire graph around as \li{find} does, \li{mark} will use the fold/unfold relationship given in equation~\eqref{eqn:bigraphintrofoldunfold}, located just under the code in Figure~\ref{fig:markgraph}, to ``unfold'' the graph as if it were an inductive predicate.

This fold/unfold relationship deserves attention.
First, \eqref{eqn:bigraphintrofoldunfold} uses the ``overlapping conjunction''~$\ocon$ of separation logic; informally $P ** Q$ means that $P$ and $Q$ may overlap
in the heap (\emph{e.g.}, nodes in the left subgraph can also be in the right subgraph
or even be the root $x$).  The presence of the unspecified sharing indicated by the
$\ocon$ connective\footnote{Recall that the
standard semantics of the separation logic connectives used in this paper are in
Figure~\ref{fig:seplogsem} on page~\pageref{fig:seplogsem}.} is part of why graph-manipulating algorithms are so hard to verify
(\emph{e.g.}, it is hard to apply the \infrulestyle{Frame} rule).
Second, \eqref{eqn:bigraphintrofoldunfold} illustrates how industrial-strength settings complicate verification.  Lines~\mbox{\ref{code:nodedefstart}--\ref{code:nodedefend}} define the data type \li{Node} used by \li{mark}.  The \li{\_Alignas(}$n${)} directives tell CompCert to align fields on $n$-byte boundaries.
As explained in~\S\ref{sec:goodgraph}, this alignment is necessary in C-like memory models to prove fold-unfold \eqref{eqn:bigraphintrofoldunfold}, which is why \eqref{eqn:bigraphintrofoldunfold} includes an alignment restriction $x~\mathsf{mod}~16 = 0$ and an existentially-quantified ``blank'' second field for the root $x \mapsto m,-,l,r$.
In our Floyd proofs, the alignment restriction and blank second field are nicely hidden ``behind the scenes''.

Just as with \li{find}, the postcondition of \li{mark} is specified \emph{relationally}, \emph{i.e.} $\{\exists \gamma'.~ \p{m\_graph}(\li{x},\gamma') /| \m{mark}(\gamma, \li{x}, \gamma')\}$ instead of \emph{functionally}, \emph{i.e.} $\{\p{m\_graph}\big(\li{x},\m{mark}(\gamma, \li{x})\big)\}$. In the first case $\m{mark}$ is a relation that specifies that~$\gamma'$ is the result of correctly marking~$\gamma$ from~\li{x}, whereas in the second $\m{mark}$ is a function that \textbf{computes} a new graph, which is the result of marking~$\gamma$ from~\li{x}. A relational approach is better for both theoretical and practical reasons.
%\marginpar{\tiny \color{blue}Needs polishing}
%{\color{magenta}
Theoretically, relations are preferable because they are more general.  For example, relations allow ``inputs'' to have no ``outputs'' (\emph{i.e.} be partial) or alternatively have many outputs (\emph{i.e.} be nondeterministic).  Nondeterminism can be quite useful when specifying programs; for example, the CertiCoq garbage collector (\S\ref{sec:certigc}) is specified nondeterministically to avoid, among other things, specifying how \li{malloc} allocates fresh blocks of memory.  Relations are also preferable to functions because they are more compositional.
%We take advantage of compositionality by using $\m{mark}(\gamma,x,\gamma') /| \ldots$ to specify both our ``spanning tree'' and {\color{blue}``graph copy''} algorithms in~\S\ref{sec:application}, which also need to mark nodes in order to carry out their primary tasks.

Practically, it is painful to define computational functions over graphs in a proof assistant like Coq.  For example, Coq requires that all functions terminate, a nontrivial proof obligation over cyclic structures like graphs, but our verification of \li{mark} is only for partial correctness.  Defining relations is much easier because \emph{e.g.} one can use quantifiers and does not have to prove termination.
The $\m{mark}$ and $\m{mark1}$ relations we use are defined straightforwardly at the bottom of Figure~\ref{fig:markgraph}.

The highlights of the proof are as follows.
In lines~\ref{code:beforerootmark}--\ref{code:afterrootmark}, imagine unfolding the \p{m\_graph} predicate in line~\ref{code:globalbeforerootmark} using equation \ref{eqn:bigraphintrofoldunfold} and then zooming in to the root node \li{x} for lines~\ref{code:beforerootmark}--\ref{code:afterrootmark}, before zooming back out in line~\ref{code:globalafterrootmark}.
\hide{\color{magenta}Here we upgrade the \infrulestyle{Ramify} rule in two important respects.  The first is the treatment of modified program variables.  Although prosaic, a robust treatment of modified variables is essential to verifying programs of any length.  The second is better handling of existential variables in the postconditions of localization blocks, which occur frequently when using relations in specifications.}
Lines~\ref{code:beforemarkl}--\ref{code:aftermarkl} of Figure~\ref{fig:markgraph} contain an example where we use the power of linked existentials in the \textsc{Localize} rule to ``extract'' the existentially-quantified~$\gamma''$ from inside the localization block to outside it.
The rest of the proof is relatively routine.

%Moreover, rather surprisingly, the robust treatment of modified program variables requires the extraction of existential quantifiers from localization blocks.  Accordingly, we develop the \infrulestyle{Ramify-PQ} rule that can robustly handle both modified \underline{p}rogram variables as well as existential \underline{q}uantifiers.
%\marginpar{\tiny \color{blue}Update to the latest stats, remove H/S.}
%Hobor and Villard noted that their \infrulestyle{Ramify} rule handled modified program variables poorly~\cite{hobor:ramification} and proposed two solutions: using ``variables as resource''~\cite{bornat:var} and making local program transformations.  While these solutions are reasonable in a pen-and-paper context, both fall well short in mechanized ones.
%VST does not use variables as resource, nor does it have modules to reason about program equivalence; moreover, most other mechanized verification systems do not support these solutions either~\cite{beckert:2007,distefanop08,bengtson:charge}.  We want a solution that is widely applicable to tools as they actually exist today; in~\S\ref{sec:development} {\color{magenta}we will see that our modifications to VST and H/S are approximately 1.4\% of their respective (sizable) codebases.}

\hide{
\subsection{Modified program variables}
\label{sec:freevars}

\infrulestyle{Frame}'s side condition ``$F \text{ ignores } \MV(c)$'' can be defined in two ways.
In the more traditional syntactic style, it means that $\FV(F) \cap \MV(c) = \emptyset$.
By ``syntactic style'' we mean that the side condition is written using a function $\FV(F)$ that takes an arbitrary formula and returns the set of free variables within that formula.  To define this $\FV(F)$ function
we need a fixed inductive \textbf{syntax} for formulas.  In contrast, in this paper we follow a ``semantic style'' in which formulas are not given a fixed syntax in advance but can be defined \textbf{semantically} on the fly using an appropriate model~\cite{appel:programlogics}.  In a semantic style, the side condition on the frame rule is defined as:
\[
\begin{array}{ll}
\sigma \stackrel{S}{\cong} \sigma' & \defeq ~~ \sigma \text{ and } \sigma' \text{ coincide everywhere except } S\\
P \text{ ignores } S & \defeq ~~ \forall \sigma, \sigma'.~ \sigma \stackrel{S}{\cong} \sigma' => \big( (\sigma |= P) <=> (\sigma' |= P) \big)
\end{array}
\]
That is, we consider two program states $\sigma$ and $\sigma'$ equivalent up to program variable set~$S$ when they agree everywhere except on the values of $S$ (typically, a state $\sigma$ is a pair of a heap $h$ and program variables $\rho$).  A predicate $P$ ignores~$S$ when its truth is independent of all program variables in $S$.  %{\color{magenta} Notice both the syntactic and semantic styles use the $\MV(c)$ function defined via straightforward recursive case analysis on program syntax; programming languages typically do have a fixed syntactic structure.}


Now consider using ramification to verify this program:
\vspace{-1ex}
\begin{lstlisting}
// $\{ \tx{x} = 5 /| A \} \searrow \{\tx{x} = 5 /| B \}$
      ...; x = x + 1; ...;
// $\{ \tx{x} = 6 /| D \} \swarrow \{\tx{x} = 6 /| C \}$
\end{lstlisting}
\vspace{-1ex}
Suppose that other (elided) lines of the program make localization desirable, even though it is overkill for a single assignment.  The key issue is that the program variable {\li{x}} appears in all four positions in the ramification entailment
\vspace{-1ex}
\[
\overbrace{(\li{x} \! = \! 5 /| A)}^{G_1} \vdash \overbrace{(\li{x} \! = \! 5 /| B)}^{L_1} * \big(\overbrace{(\li{x} \! = \! 6 /| C)}^{L_2} --* \overbrace{(\li{x} \! = \! 6 /| D)}^{G_2}\big)
\vspace{-1ex}
\]
One problem is that $L_2 --* G_2$ does \textbf{not} ignore the modified program variable \tx{x}, preventing us from applying \infrulestyle{Ramify}.  Intuitively, the side condition on the \infrulestyle{Ramify} rule is a bit too strong since it prevents us from mentioning variables in the postconditions that have been modified by code $c$.

We could weaken the side condition in \infrulestyle{Ramify} to $\big(\FV(G_2) \cap \MV(c)\big) \subseteq \FV(L_2)$, with the hope that information about modified program variables mentioned in the local postcondition $L_2$ can be carried to the global postcondition $G_2$.  Unfortunately, this idea is unsound because \li{x} cannot simultaneously be both~5 and~6, \emph{i.e.} the above entailment is vacuous.  A better idea is: % the following :
\[
\inferrule[Ramify-P (Program variables)]
{\{ L_1 \} ~ c ~ \{L_2 \} \\
 G_1 \vdash L_1 * \pguards{c}  (L_2 --* G_2)}
{\{ G_1 \} ~ c ~ \{ G_2 \}}
\]
The ramification entailment now incorporates a new (universal/boxy) modal operator $\pguards{c}$.  The intuitive meaning of $\pguards{c}$ is that program variables modified by command $c$ can change value inside its scope.    Note that it is vital that $L_2$ appears as the antecedent of a (spatial) implication since the change in program variables is universally quantified.  This means that if we want to say anything specific about modified program variables in the global postcondition $G_2$ then we had better say something about them in the local postcondition $L_2$.

Let us return to our earlier entailment:
\[
\begin{array}{l}
(\li{x} = 5 /| A) \vdash (\li{x} = 5 /| B) *
\pguards{\li{...; x = x + 1; ...;}} \big((\li{x} = 6 /| C) --* (\li{x} = 6 /| D)\big)
\end{array}
\]
Since \li{x} is modified, its value can change from the first line, in which \li{x} must be 5, to the second, in which \li{x} must be 6.

Here is the definition of $\pguards{c}$, writing $\langle c \rangle$ for $\MV(c)$:
\vspace{-1ex}
\[
%\begin{array}{lcl}
%\langle c \rangle & \stackrel{\Delta}{=} & \MV(c) \\
\sigma |= \pguards{c} P ~~ \defeq ~~ \forall \sigma'.~ (\sigma \stackrel{\langle c \rangle}{\cong} \sigma') => (\sigma' |= P)% ~~~~ \text{where $\mathsf{MV}(c)$ is $\MV(c)$}\\
%\end{array}
\vspace{-1ex}
\]
In other words, $\pguards{c}$ is exactly the universal modal operator~$\Box$ over the relation that considers equivalent all states that differ only on program values modified by $c$.  Since $\stackrel{\langle c \rangle}{\cong}$ is an equivalence relation, $\pguards{c}$ forms an S5 modal logic.

Note that \infrulestyle{Ramify-P} has no free variable side condition, which is unnecessary because \\ $\forall P.~ \pguards{c}P \text{ ignores } \MV(c)$. However, in practice this side condition reappears because to actually prove a ramification entailment containing $\pguards{c}$ one typically applies the following \infrulestyle{Solve Ramify-P} rule:
\[
\inferrule[Solve Ramify-P]
{G_1 |- L_1 * F \\ F |- L_2 --* G_2}
%{\color{magenta} F |- L_2 --* G_2}
%F * L_2 |- G_2
{G_1 \vdash L_1 * \pguards{c}  (L_2 --* G_2)}
F \textrm{ ignores } \MV(c)
\]
We can handle the $\pguards{c}$ by breaking apart the single entailment into a pair.  Using two entailments allows modified program variables to change between the preconditions and postconditions\footnote{Entailment procedures for separation logic may prefer to use $F * L_2 |- G_2$ as the second premise of \infrulestyle{Solve Ramify-P} because it is free from $--*$.}.  To connect the pair, we must choose a suitable predicate~$F$ that ignores modified variables in~$c$.

With \infrulestyle{Ramify-P} and \infrulestyle{Solve Ramify-P} we can prove the \infrulestyle{Frame} rule with its canonical side condition as follows:
\vspace{-2ex}
\[
\infrule{}{\raisebox{1.4ex}{$\infrule{}{P * F |- P * F \\ F |- Q --* (Q * F)}
{\raisebox{-4pt}[0pt][0pt]{$P * F |- P * \pguards{c}\big(Q --* (Q * F)\big)$}}
{\hspace{-1.1ex}\raisebox{0.9ex}{$\begin{array}{c}F \textrm{ ignores} \MV(c)\end{array}$}}$}
\\ \{P\}~c~\{Q\}}
{\{P * F\}~c~\{Q * F\}}
{}
\vspace{-2ex}
\]
This justifies our point in \S\ref{sec:localizations} that our new localization notation can also be used for frames.

The choice of $F$ in a concrete setting is delicate. In our example, we replace\footnote{In a semantic setting, substitution is defined with a modal operator and not textual replacement, but the effect is the same.}~\li{x} with~$6$ in $L_2 --* G_2$:
\[
F ~~ \defeq ~~ (6 = 6 /| [\li{x} |-> 6]C) --* (6 = 6 /| [\li{x} |-> 6]D)
\]
The first premise of \infrulestyle{Solve Ramify-P} is
\[
\begin{array} {l}
\li{x} = 5 /| A ~ |- ~ (\li{x} = 5 /| B) * \big((6 = 6 /| [\li{x} |-> 6]C) --* (6 = 6 /| [\li{x} |-> 6]D)\big)
\end{array}
\]
This entailment is the key proof that our localization was sound.  To solve it we first substitute away the remaining program variables (\emph{e.g.} replace \li{x} with 5) to obtain a program-variable-free and modality-free entailment and then apply our ramification library (\S\ref{sec:ramifylib}); as previously explained we sometimes use $\ramify(n)$ to explicitly reference a library lemma.

The second premise, shown below, is a tautology using $(P * Q |- R) <=> (P |- Q --* R)$:
\vspace*{-0.75ex}
\begin{equation}
\label{eqn:sndpremisetauto}
\begin{array}{l}
(6 = 6 /| [\li{x} |-> 6]C) --* (6 = 6 /| [\li{x} |-> 6]D) ~ |- (\li{x} = 6 /| C) --* (\li{x} = 6 /| D)
\end{array}
\vspace*{-0.75ex}
\end{equation}

%: introduce the $L_2$ premise of the right $--*$ to the left side of the entailment and, since the clause $\li{x} = 6$ is now on the left, substitute it everywhere.
\iffalse
This strategy is sufficient to handle all of the localization blocks in Figure~\ref{fig:markgraph}.  For example, in lines~\ref{code:markbeforetripleramify}--\ref{code:markaftertripleramify}, choose $F \defeq \null$
\vspace*{-0.75ex}
\[
\begin{array}{@{}l@{}}
\big(\li{x} |-> 1,-,l,r /| \gamma(\li{x}) = (0,l,r) /| \exists \gamma'.~ \m{mark1}(\gamma, \li{x}, \gamma')\big) \\ \null --* \big(\exists \gamma'.~ \p{graph}(\li{x},\gamma') /| \gamma(\li{x}) = (0,l,r) /| \m{mark1}(\gamma, \li{x}, \gamma') \big)
\end{array}
%// $\{\exists \gamma'.~ \p{graph}(\tx x,\gamma') /| \gamma(\tx{x}) = (0,\tx{l},\tx{r}) /| \m{mark1}(\gamma, \tx{x}, \gamma')\}$
\vspace*{-0.75ex}
\]
Note the use of the metavariables $l$ and $r$ rather than \li{l} and \li{r} in $F$, added to the metacontext in lines~\ref{code:globalbeforerootmarkwithex}--\ref{code:globalbeforerootmark} using Floyd's \infrulestyle{Existential extraction} rule~\cite{floydlogic}:
\vspace*{-0.75ex}
\[
\infrule{Existential extraction}
{\forall x.~ \big(\{ P \} ~ c ~ \{Q \}\big)}
{\{ \exists x. P \} ~ c ~ \{ \exists x.~ Q \}}{}
\vspace*{-0.75ex}
\]
Pen and paper Hoare proofs are often a little casual with existentials, \emph{e.g.} omitting line~\ref{code:globalbeforerootmarkwithex}; we wrote it because we wanted to be clear that the metavariables $l$ and $r$ were properly ``in scope'' over the localization blocks.
\fi

%Conversely, we can also prove \infrulestyle{Ramify-P} from \infrulestyle{Frame} and \infrulestyle{Consequence}:

\subsection{Existential quantifiers in postconditions}
\label{sec:existentials}

What happens when we \textbf{cannot} calculate a substitution using globally-scoped metavariables?  Consider the following: %example:
\begin{lstlisting}
// $\{ A \} \searrow \{ B \}$
      ...; x = malloc(sizeof(int));
      if (x == 0) then y = 0 else y = 1; ...;
// $\label{toycode:localpost}\swarrow \{ \big((\tx{x} |-> {-} /| \tx{y} = 1) |/ (\tx{x} = 0 /| \tx{y} = 0)\big) * C \}$
// $\label{toycode:globalpost}\{ (\tx{y} = 1 /| D_1) |/  (\tx{y} = 0 /| D_2) \}$
\end{lstlisting}
\vspace*{-1.5ex}
Within a localization block we call the nondeterministically specified function \li{malloc} and use the program variable~\li{y} as a flag to keep track of whether the allocation succeeded.  Call the postconditions in lines~\ref{toycode:localpost} and~\ref{toycode:globalpost} just above $L_2$ and $G_2$. %respectively.

Now the choice of $F$ is not very straightforward because we do not know the values to substitute for \li{x} or \li{y}: $[\li{x} |-> ?][\li{y} |-> ?] (L_2 --* G_2)$. \label{eqn:unclearsubst}

\hide{
\vspace*{-1.5ex}
\begin{equation}
\label{eqn:unclearsubst}
[\li{x} |-> ?][\li{y} |-> ?] (L_2 --* G_2)
%\begin{array}{@{}l@{}}
%\Big(\big((? |-> {-} /| ? = 1) |/ (? = 0 /| ? = 0)\big) * [\li{x} |-> ?][\li{y} |-> ?]C\Big) \\
%--* \! (? = 1 \! /| \! [\li{x} |-> ?][\li{y} |-> ?]D_1) \! |/ \! (? = 0 \! /| \! [\li{x} |-> ?][\li{y} |-> ?]D_2)
%\end{array}
\vspace*{-1.5ex}
\end{equation}
} % hiding because I just made it inline.

We proceed as follows.  First, rewrite the postconditions in lines~\ref{toycode:localpost} and~\ref{toycode:globalpost} just above to introduce fresh existentially-quantified  variables $x$ and $y$ and bind them to \li{x} and \li{y}:
\begin{lstlisting}[firstnumber=4]
//   $\;\{ L_2 \}$
// $\label{code:L2p}\swarrow \{\exists x,y. ~ x = \tx{x} /| y = \tx{y} /| [\tx{x} |-> x][\tx{y} |-> y] L_2 \}$
// $\label{code:G2p}\{\exists x,y. ~ x = \tx{x} /| y = \tx{y} /| [\tx{x} |-> x][\tx{y} |-> y] G_2\}$
// $\{ G_2 \}$
\end{lstlisting}
Call these equivalent postconditions $L_2'$ and $G_2'$ (lines~\ref{code:L2p} and
\ref{code:G2p}).


%\begin{lstlisting}[firstnumber=4]
%// $\swarrow \left\{\begin{array}{@{}l@{}l@{}} \exists x,y.~ & x \! = \! \tx{x} /| y \! = \! \tx{y} /| \big((x |-> \! {-} /| y \! = \! 1) {|/} (x \! = \! 0 /| y \! = \! 0)\big) \\ & * \, [\tx{x} |-> x] [\tx{y} |-> y] C \end{array} \right\}$
%// $\left\{\begin{array}{@{}l@{}l@{}} \exists x,y. ~ x = \tx{x} /| y = \tx{y} /| \big(&(y = 1 /| [\tx{x} |-> x] [\tx{y} |-> y]D_1) |/  \null \\ & (y = 0 /| [\tx{x} |-> x] [\tx{y} |-> y]D_2)\big) \end{array}\right\}$
%\end{lstlisting}

%\[
%\begin{array}{@{}l@{}}
%\forall x, y. \Big(\!\big((x \! |-> \! {-} \! /| \! y \! = \! 1) \! |/ \! (x \! = \!0 /| y \! = \! 0)\big) \! * \! [\li{x} \! |-> \! x][\li{y} \! |-> \! y]C\Big) \\
%--* \! (y \! = \! 1 \! /| \! [\li{x} |-> x][\li{y} |-> y]D_1) \! |/ \! (y \! = \! 0 \! /| \! [\li{x} \! |-> \! x][\li{y} \! |-> \! y]D_2)
%\end{array}
%\]
Next apply \infrulestyle{Ramify-P} and \infrulestyle{Solve Ramify-P} with
$F \defeq \forall x, y.~ [\tx{x} |-> x][\tx{y} |-> y](L_2 --* G_2)$.
In other words, replace the ``?'' from the unclear substitution in
\S\ref{eqn:unclearsubst} with universally-quantified metavariables $x$ and $y$ scoped over the entire $--*$.

Now consider the first premise of \infrulestyle{Solve Ramify-P}:
\[
\begin{array}{@{}l|l@{}}
G_1 \! |- \! L_1 \! * \! F & A |- B * \forall x, y.~ [\tx{x} |-> x][\tx{y} |-> y](L_2 --* G_2)
\end{array}
\]
This is essentially the same ramification entailment we had before, and so the general strategy is to apply the ramification library~\S\ref{sec:ramifylib}.  The second premise is more interesting:
\[
\begin{array}{@{}l|l@{}}
F |- & \big(\forall x, y.~ [\tx{x} |-> x][\tx{y} |-> y](L_2 --* G_2) \big) |- \null \\
(L_2' --* & ~~ (\exists x,y.~ x = \tx{x} /| y = \tx{y} /| [\tx{x} |-> x][\tx{y} |-> y] L_2) --* \null \\
G_2') & \quad (\exists x,y. ~ x = \tx{x} /| y = \tx{y} /| [\tx{x} |-> x][\tx{y} |-> y] G_2)
\end{array}
\]
Like equation~\eqref{eqn:sndpremisetauto}, this turns out to also be a tautology, albeit a more complicated one.
Since $L_2$ and $G_2$ are equivalent to $L_2'$ and $G_2'$, we can therefore verify the specification all the way from $A$ to $G_2$ despite the presence of the existentially-quantified modifications to the program variables \li{x} and \li{y}.

We package all of this reasoning into the following rule:
\[
\inferrule[Ramify-PQ (Program variables and Quantifiers)]
{\{ L \} ~ c ~ \{ \exists x.~ L_2 \} \\
 G_1 \vdash L_1 * \pguards{c} \big(\forall x.~ (L_2 --* G_2)\big) }
{\{ G_1 \} ~ c ~ \{ \exists x.~ G_2 \}}
\]
Essentially \infrulestyle{Ramify-PQ} allows us to shift existential variables from the local context to the global one in a smooth way, especially in conjunction with the following rule:
\[
\inferrule[Solve Ramify-PQ]
{G_1 |- L_1 * F \\ F |- \forall x.~ (L_2 --* G_2)}
%{\color{magenta} F |- L_2 --* G_2}
%F * L_2 |- G_2
{G_1 \vdash L_1 * \pguards{c}  \big(\forall x.~ (L_2 --* G_2)\big)}
F \textrm{ ignores} \MV(c)
\]
%Since we use a relational style to verify graph algorithms (\emph{e.g.} in Figure~\ref{fig:markgraph}), existentials appear frequently and a smooth treatment is very helpful in practice.  To make this point a little more clearly
We were more explicit about existentials in \emph{e.g.} lines~\ref{code:beforemarkl}--\ref{code:aftermarkl} than is typical so that we could prove that we handle them correctly.  Fortified by the \infrulestyle{Ramify-PQ} rule, we could reasonably (albeit less formally) have \emph{e.g.} written line~\ref{code:postmark1} as below, allowing us to omit line~\ref{code:aftermarkl} entirely.
 \begin{lstlisting}[firstnumber=25]
// $\swarrow \{\p{graph}(\tx l, \gamma'') /| \m{mark}(\gamma', \tx{l}, \gamma'')\}$
\end{lstlisting}

%\ref{code:postmark1}]

Although our technique to handle modified program variables is rather intricate, it is handled mechanically by our \li{localize} and \li{unlocalize} tactics, which use \infrulestyle{Ramify-PQ}.
% since it is the most general rule.
}

\subsection{Our six verified examples}
\label{sec:application}

In addition to \li{find} shown above, we do a proof of \li{union}
in the same style, thus completing the verification of 
union-find for \li{malloc}-allocated nodes.
We also verify a \emph{second} version of union-find that uses
arrays rather than nodes. The two programs look different spatially, but 
we employ exactly the same abstract mathematical definitions,
\emph{e.g.} for $\m{findS}$. 
This shows that we have
separated the concerns of abstract algorithmic reasoning from the specific details
of heap representation.

In addition to the graph \li{mark}ing algorithm discussed above, 
we verify a version of \li{mark} for directed acyclic graphs (DAGs).  
In general, using DAGs is both easier and harder than cyclic graphs.  
On the one hand, we get genuine separation
between the root and its children; on the other hand, we need to maintain acyclicity if
we modify the link structure.

As an example of more aggressive modifications to the link structure 
of a graph, we verify
\li{spanning}, which prunes a graph into its spanning tree. We elide a discussion 
about it in the interest of space, but the verification code and an annotated
proof sketch \note[todo]{are available as online resources}.
Our flagship example, the CertiCoq garbage collector, will be discussed in~\S\ref{sec:certigc}.

\hide{
\subsection{Forward ramification}

\note[Check refs]
{Sometimes it is convenient to reason in a ``forward'' (strongest postcondition)
style rather than in a ``backward'' (weakest precondition) style---many 
automated tools use a forward style~\cite{chin:hipsleek,X,XX,XXX}.}  

As we saw in \S\ref{sec:hipsleek}, the forward style of reasoning employed by HIP/SLEEK uses the existential wand $--o$ to express ramifications instead of the more typical universal wand $--*$.  The standard $--*$ form of ramification is weaker, but the strongest postcondition style using $--o$ can also get the job done without too much extra work since:
\vspace*{0.25ex}
\[\infrulestyle{WandToEwand}~~~~~~~~~~~~~~~~~~~~~~~~~~~~~~~~~~~~~~~~~~~~~~~~~~~~~~~~~~~~~~~~~~~~~~~\]
\vspace*{-8ex}
\[
~~~~~~~~\infrule{}
{G_1 |- L_1 * (L_2 --* G_2)}
{(L_1 --o G_1) * L_2 |- G_2}
{\m{precise}(L_1)}
\vspace*{-1.5ex}
\]
\[
\begin{array}{@{}l@{}l@{}}
\m{precise}(P) ~ \defeq ~ & (\sigma_1 |= P) => (\sigma_2 |= P) => \\
& ~~ (\sigma_1 \oplus \sigma_1' \! = \! \sigma) => (\sigma_2 \oplus \sigma_2' \! = \! \sigma) => \sigma_1 \! = \! \sigma_2
\end{array}
\]
In \S\ref{sec:ramifylib} we will discuss the ``supplemental'' spatial libraries, which ensures
the preciseness of our key predicates.

}

%In addition to \li{mark} for graphs, we have also verified the same program for DAGs.
%More interestingly

%two algorithms that modify the link structure:
%for graphs, and \li{copy}
%for graphs, which makes a structure-preserving copy.  For space reasons we put
%\li{spanning} and \li{copy}
