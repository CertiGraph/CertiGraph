\hide{
In order to verify the functional correctness of graph
algorithms, we need to first reason about mathematical graphs.
Our graph library is powerful and expressive, allowing
us to verify realistic algorithms that work in an end-to-end
system. One of its strengths is its modularity,
which allows us to intuitively reuse and compose our proofs when
mechanising our verifications. In this section, we present our
mathematical graph framework with an emphasis on this modularity.
{\color{magenta} We continue to use Union-Find from }
} % end hide

When verifying the functional correctness of graph algorithms it is natural
to want to use graph theory to describe
program behavior.
%\note[a bit off? Consider: ...graph theory has deduced that ``it is...'']{As discussed in \S\ref{sec:related},
%25 years of research into mechanized graph theory can
%be succinctly summarized as ``it is a little tricky''.}
Over the past 25 years, researchers in mechanized proofs have explored many choices for the 
critical definitions of graph theory~(\S\ref{sec:related}).  Unfortunately, most such choices
are not powerful enough, or expressive enough, to mechanically verify realistic algorithms 
written in~C.
%, since as is so often the
%case, the proofs are straightforward only once the proper definitions have
%been found.
Accordingly, an essential component of our
work is a library of formalized graph theory that can support such verifications.
As will be shown in \S\ref{sec:development}, our mathematical
graph constructions comprise a considerable fraction of our
codebase, so it is vital that our framework be highly modular to
enable reuse of definitions and proofs from one
example to the next. We present our mathematical graph
framework with an emphasis on this modularity.

\hide{\color{magenta} We continue to use Union-Find from
\S\ref{sec:orientation} as our motivating example.} % end hide

% saving old version just in case...
\hide{As will be shown in \S\ref{sec:development}, our mathematical
graph constructions comprise a considerable fraction of our
codebase. Indeed, as discussed in \S\ref{sec:related},
25 years of research into mechanized graph theory can
be summarized as ``it is a little tricky''.
First, as demonstrated in \S\ref{sec:orientation},
our development is expressive and powerful enough to verify realistic
algorithms---that is, it actually works in an end-to-end system.
Second, we have taken considerable care to develop a modular and
general-purpose framework for such mathematical graphs to allow
such verifications to be mechanized without undue pain.
Accordingly, in this section we will present our framework
at a high level to communicate the overall architecture rather
than focusing on the nitty-gritty details.} % end hide

\begin{figure}[t!]
\centering
\begin{adjustbox}{scale=0.85}
  \begin{subfigure}[t]{0.5\textwidth}
  \centering
  \beginpgfgraphicnamed{variousgraph}
\begin{tikzpicture}
[->/.style={thick,arrows={-Stealth}},
-->/.style={thick,arrows={-Stealth}, decorate, decoration={snake, amplitude=.4mm,segment length=2mm,post length=2mm}},
   realG/.style={shape=rectangle, rounded corners=4pt, draw, fill=gray!40},
   propG/.style={shape=rectangle, rounded corners=4pt, draw}]
\node[realG] (PG) at (0, 0) {\small PreGraph};
\node[realG] (LG) [right=0.8 of PG] {\small LabeledGraph};
\node[realG] (GG) [right=2 of LG] {\small GeneralGraph};
\draw [double, ->] (PG) -- (LG) node [pos=0.5, above] {\small Label} ;
\draw [double, ->] (LG) -- (GG) node (SC) [pos=0.5, above, align=center]
{\small Soundness \\ \small Condition};
\node[propG] (Prop) [below=0.6 of SC] {\small Property};
\node[propG] (PropL) [below=0.4 of Prop] {\small Property Lemmas};
\node[propG] (PGL) [below=2 of PG, align=center] {\small PreGraph \\\small Lemmas};
\node[propG] (LGL) [below=2 of LG, align=center] {\small LabeledGraph \\\small Lemmas};
\node[propG] (GGL) [below=2 of GG, align=center] {\small GeneralGraph \\\small Lemmas};
\draw [double, ->] (PGL) to (LGL);
%% \draw [double, ->] (LGL) to (GGL);
\draw [->] (PG) to (PGL);
\draw [->] (Prop) to (PropL);
\draw [-->] (Prop) to (SC);
\coordinate [left=0.2 of LG.south] (LGs1);
\coordinate [left=0.2 of LGL.north] (LGLn1);
\draw [->] (LGs1) to (LGLn1);
\coordinate [right=0.2 of LG.south] (LGs2);
\coordinate [right=0.2 of LGL.north] (LGLn2);
\draw [->] (LGs2) |- (Prop);
\draw [double, ->] (LGLn2) |- (PropL);
\coordinate [right=0.2 of GG.south] (GGs);
\coordinate [left=0.2 of GGL.north] (GGLn1);
\coordinate [right=0.2 of GGL.north] (GGLn2);
\draw [double, ->] (PropL) -| (GGLn1);
\draw [->] (GGs) to (GGLn2);
\node [draw, thick, rectangle, dashed, fit=(Prop) (PropL)] {};
\node (legend1) [below right=0.2 and -0.3 of PGL] {\small Depends};
\coordinate[left=0.8 of legend1]  (l1);
\draw [->] (l1) to (legend1);
\node (legend2) [right=1 of legend1] {\small Inherits};
\coordinate[left=0.8 of legend2]  (l2);
\draw [double, ->] (l2) to (legend2);
\node (legend3) [right=1 of legend2] {\small Instantializes};
\coordinate[left=0.8 of legend3]  (l3);
\draw [-->] (l3) to (legend3);
\end{tikzpicture}
\endpgfgraphicnamed
\vspace{1ex}
\captionof{figure}{Structure of the Mathematical Graph Library}\label{fig:graphs}
\end{subfigure}
  ~
  \begin{subfigure}[t]{0.5\textwidth}
\centering
\beginpgfgraphicnamed{pregraphexp}
\begin{tikzpicture}
[vad/.style={circle, fill=black, inner sep=0pt, minimum size=4pt},
 inv/.style={circle, draw=black, thick, inner sep=0pt, minimum size=4pt},
 ->/.style={thick, arrows={-Stealth}}]
\node[vad] (n1) at (0, 0) {};
\node[vad] (n2) at (1, 1) {};
\node[inv] (n3) at (1, -1) {};
\node[vad] (n4) at (-2,2) {};
\node[vad] (n5) at (-2,-2) {};
\node[vad] (n6) at (-3,0) {};
\node[vad] (n7) at (3,1.5) {};
\node[inv] (n8) at (3,-1.5) {};
\node[vad] (n9) at (3.5, 1) {};
\node[inv] (n10) at (3.5, 0.5) {};
\node at (3.5, 1) [right=1.5pt] {\small Valid node};
\node at (3.5, 0.5) [right=1.5pt] {\small Invalid node};
\node at (3.5, 0) [right=1.5pt] {\small Valid edge};
\node at (3.5, -0.5) [right=1.5pt] {\small Invalid edge};
\draw[->] (n1) to (n2);
\draw[->,dashed] (n1) to (n3);
\draw[->,dashed] (n3) to (n5);
\draw[->] (n2) to (n3);
\draw[->] (n2) to (n7);
\draw[->] (n2) to (n8);
\draw[->,dashed] (n3) to (n8);
\draw[->] (n4) to (n1);
\draw[->] (n1) to (n5);
\draw[->] (n2) to (n4);
\draw[->] (n1) to [bend left=20] (n6);
\draw[->] (n6) to (n5);
\draw[->] (n8) to [bend left=20] (n7);
\draw[->] (n4) to [bend right=35] (n1);
\draw[->] (3.0, 0) -- (3.6, 0);
\draw[->,dashed] (3.0, -0.5) -- (3.6, -0.5);
\node at (1, 1.3) {$v_0$};
\node at (1, -1.3) {$v_1$};
\node at (1.2, 0) {$e_0$};
\end{tikzpicture}
\endpgfgraphicnamed
\vspace{1ex}
\captionof{figure}{A PreGraph with valid/invalid vertices and edges.}\label{fig:pregraph}
\end{subfigure}
\end{adjustbox}
\caption{The Mathematical Graph Library, and a Pregraph}
\end{figure}


\subsection{Definitions of graphs}\label{sec:mathinfra}

Our first challenge is that graph theory is usually based on
\emph{set theory} but our formalization in Coq is
based on \emph{type theory}. 
Instead of formalizing set theory in Coq and then building
graph theory atop it, we formalize graph theory
directly in Coq, as this lets us take advantage of Coq's built-in
support for type-related constructions.
To resolve the dichotomy between
the generality and the speciality of the library, we develop
our graphs gradually over three structures:
PreGraph, LabeledGraph and GeneralGraph.
Figure~\ref{fig:graphs} shows the
architecture of the library.

\hide{The most basic kind of graph is PreGraph, out of which we build
LabeledGraph, and which in turn are used
to build GeneralGraphs. Each kind has some lemmas and also inherits the lemmas of the
previous kind.  The dashed box represents a ``plugin'' system for attaching arbitrary
properties to LabeledGraphs (\ref{subsec:graphplugins}). %and will be discussed later.
%We will consider each in turn.
} % end hide

\paragraph{PreGraph.}
A PreGraph is a hextuple $(\mathcal{V}, \mathcal{E}, \mathtt{V}, \mathtt{E}, \mathtt{src}, \mathtt{dst})$.  Arguments $\mathcal{V}$ and $\mathcal{E}$ are the underlying
carrier types of vertices and edges.  $\mathtt{V}$ and $\mathtt{E}$ are predicates over
$\mathcal{V}$ and $\mathcal{E}$ that specify the notion
of \emph{validity} in the graph.  Finally, $\mathtt{src}$ and $\mathtt{dst}$ map each edge to
its source and destination.
% With an
%eye to flexibility, we make no further requirements of a legal PreGraph.

The benefits of introducing validity are twofold. The first is a
neat resolution of the incompatibility between type theory and set theory.
In set theory, one
element can belong to multiple sets, and
adding or removing vertices or edges is as easy as altering
the set directly to represent the result of the operation.
In type theory, however, a term can only belong
to one type, which makes it difficult\hide{if not impossible}
to analogously change the
type to represent the result. As is common practice, the
predicates $\mathtt{V}$ and $\mathtt{E}$ specify whether a vertex/edge is \emph{valid}
(in the graph) or \emph{invalid} (out). Adding or removing vertices/edges
is as simple as weakening or strengthening these two predicates.

The second benefit is the ability to represent incomplete graphs.
Consider starting from a graph $\gamma$ and then removing a subgraph; the remaining
``doughnut'' structure is \textbf{not} necessarily a graph, since there may be dangling
edges pointing into the ``hole''.  Figure \ref{fig:pregraph} shows just such a situation,
where a connected graph (everything that is reachable from $\m{v}_0$) has had the connected
subgraph reachable from $\m{v}_1$ removed; \emph{e.g.}, the edge $e$ is dangling.
The last conjunct in the $\m{findS}$ relation from Figure~\ref{fig:find} is an example
of where a real verification needs to reason about just such a doughnut, in particular
to specify that the unreachable portion of a graph has not changed.
%PreGraph supplies the necessary flexibility to express such notions.

\hide{an incomplete graph,
where vertices $v_0$ and $v_1$
are the source and destination of a valid edge $e_0$. Both
vertices are in $\mathcal{V}$, but $v_0$ is valid
and $v_1$ is not, meaning $v_1$ is not in the
graph. Drawing $v_1$ in Figure \ref{fig:pregraph} only indicates
that $d(e_0) = v_1$.
Incomplete graphs of this sort are needed, \emph{e.g.}, by programs
that traverse selective portions of graphs and manipulate them under
certain criteria.
After the traversal, the graph $g$ can be partitioned into two parts: the subgraph
$g_1$, which is composed of vertices and edges processed by the algorithm,
and the residue $g_2$, which is the difference $g - g_1$. In
the specification of the program, it may be necessary to state that
$g_2$ is unchanged. However, $g_2$ is not guaranteed to be a ``well-formed''
graph in the conventional sense:
if $g$ is an undirected graph,
$g_2$ is just a collection of \note[disconnected?]{connected} components,
and if $g$ is a directed graph, then $g_2$ may contain dangling edges or
vertices. PreGraph is accommodating enough to permit reasoning
about~$g_2$ in either case.}

\begin{figure}
	
  \begin{minipage}{.5\textwidth}
  \begin{equation*}
  {\footnotesize
  \begin{array}{r@{}c@{}l}
    	\mathrm{path} \; &\defeq \;& (\m{v}_0, [e_0, e_1, \dots, e_k]) \\  
      \mathtt{end}\big(\gamma, (\m{v}, [])\big)\;&\defeq\;& \m{v}\\
      \mathtt{end}\big(\gamma, (\m{v},[e_1,\dots,e_n])\big)\;&\defeq\;&
      \mathtt{dst}_{\gamma}(e_{n}) \\
      \mathtt{s\_evalid}(\gamma, e) \; &\defeq \;& \mathtt{E}_{\gamma}(e) /| \null \\
      &&\mathtt{V}_{\gamma}(\mathtt{src}_{\gamma}(e)) /| \null \\
      &&\mathtt{V}_{\gamma}(\mathtt{dst}_{\gamma}(e)) \\
      \mathtt{valid\_path}\big(\gamma, (\m{v},[])\big) \;&\defeq \;& \mathtt{V}_{\gamma}(\m{v})\\
      \mathtt{valid\_path}\big(\gamma, (\m{v},[e_1,e_2,\dots,e_n])\big) \; &\defeq \;& \m{v}=\mathtt{src}_{\gamma}(e_1) \wedge \null \\
      &&\mathtt{s\_evalid}(\gamma,e_1) \wedge \null \\
      &&\mathtt{dst}_{\gamma}(e_1)=\mathtt{src}_{\gamma}(e_2) \wedge \null \\
      &&\mathtt{s\_evalid}(\gamma, e_2) \wedge \dots /| \null \\
      &&\mathtt{dst}_{\gamma}(e_{n-1})=\mathtt{src}_{\gamma}(e_n)   
  \end{array}}
  \end{equation*}
  \end{minipage}%
  \begin{minipage}{.5\textwidth}
  \begin{equation*} 
  {\footnotesize
  \begin{array}{r@{}c@{}l}
      \gamma_1 \cong \gamma_2 \; &\defeq\;&
        \forall e.~\mathtt{E}_{\gamma_1}(e) <=> \mathtt{E}_{\gamma_2}(e) /| \null \\
        &&\forall \m{v}.~\mathtt{V}_{\gamma_1}(\m{v}) <=> \mathtt{V}_{\gamma_1}(\m{v}) /| \null \\
        &&\forall e.~\mathtt{E}_{\gamma_1}(e) ==> \mathtt{src}_{\gamma_1}(e) = \mathtt{src}_{\gamma_2}(e) /| \null \\ 
        &&\hspace{5.9em} \mathtt{dst}_{\gamma_1}(e) = \mathtt{dst}_{\gamma_2}(e) \\
      \gamma |= s \overset{\m{p}}{\leadsto} t \;&\defeq\;& 
        \mathtt{valid\_path}(\gamma, p)\wedge \null \\
        &&\mathtt{fst}(p)=s \wedge \mathtt{end}(\gamma,p)=t \\
      \gamma \smallsetminus S \; &\defeq \;&
          \Big(\mathcal{V}_{\gamma},\mathcal{E}_{\gamma}, \mathtt{src}_{\gamma}, \mathtt{dst}_{\gamma}, \\
          &&\qquad \lambda x.~ \gamma_{\mathtt{V}}(x) /| \lnot S_{\mathtt{V}}(x), \\
          &&\qquad \lambda x.~ \gamma_{\mathtt{E}}(x) /| \lnot S_{\mathtt{E}}(x)\;\Big)
  \end{array}}  
  \end{equation*}
  \end{minipage}
\caption{Some PreGraph definitions}
\label{fig:pregraphdefs}
\vspace{-1em}
\end{figure} 

% stash:
% \begin{figure}
%   \begin{gather*}
%       \begin{aligned}
%       \mathrm{path} \; \defeq \; &(\m{v}_0, [e_0, e_1, \dots, e_k]) \\
%       \mathtt{s\_evalid}(\gamma, e) \; \defeq \; & \mathtt{E}_{\gamma}(e) /|
%       \mathtt{V}_{\gamma}(\mathtt{src}_{\gamma}(e)) /|
%       \mathtt{V}_{\gamma}(\mathtt{dst}_{\gamma}(e)) \\
%     \mathtt{valid\_path}\big(\gamma, (\m{v},[])\big) \;\defeq \; & \mathtt{V}_{\gamma}(\m{v})\\
%     \mathtt{valid\_path}\big(\gamma, (\m{v},[e_1,e_2,\dots,e_n])\big) \; \defeq \; &\m{v}=\mathtt{src}_{\gamma}(e_1) \wedge \mathtt{s\_evalid}(\gamma,e_1) \wedge \null \\
%     &\mathtt{dst}_{\gamma}(e_1)=\mathtt{src}_{\gamma}(e_2) \wedge \null\\
%     &\mathtt{s\_evalid}(\gamma, e_2) \wedge \dots /| \mathtt{dst}_{\gamma}(e_{n-1})=\mathtt{src}_{\gamma}(e_n) \\
%         \mathtt{end}\big(\gamma, (\m{v}, [])\big)\;\defeq\;& \m{v}\\
%         \mathtt{end}\big(\gamma, (\m{v},[e_1,e_2,\dots,e_n])\big)\;\defeq\;&
%         \mathtt{dst}_{\gamma}(e_{n})\\
%         \gamma \vDash s \overset{\m{p}}{\leadsto} t \;\defeq\; &
%         \mathtt{valid\_path}(\gamma, p)\wedge
%         \mathtt{fst}(p)=s \wedge \mathtt{end}(\gamma,p)=t \\
%         \gamma_1 \cong \gamma_2 \; \defeq \; &
%           \forall e.~\mathtt{E}_{\gamma_1}(e) <=> \mathtt{E}_{\gamma_2}(e) /| \forall \m{v}.~\mathtt{V}_{\gamma_1}(\m{v}) <=> \mathtt{V}_{\gamma_1}(\m{v}) /| \null \\
%           &\forall e.~\mathtt{E}_{\gamma_1}(e) ==> \mathtt{src}_{\gamma_1}(e) = \mathtt{src}_{\gamma_2}(e) /| \mathtt{dst}_{\gamma_1}(e) = \mathtt{dst}_{\gamma_2}(e) \\
%         \gamma \smallsetminus S \; \defeq \;
%           & \Big(\mathcal{V}_{\gamma},\mathcal{E}_{\gamma}, \mathtt{src}_{\gamma}, \mathtt{dst}_{\gamma}, \lambda x.~ \gamma_{\mathtt{V}}(x) /| \lnot S_{\mathtt{V}}(x),\lambda x.~ \gamma_{\mathtt{E}}(x) /| \lnot S_{\mathtt{E}}(x)\Big)\\
%       \end{aligned}
%     \end{gather*}
% \caption{Some PreGraph definitions}
% \label{fig:pregraphdefs}
% \vspace{-1em}
% \end{figure} 

We define many fundamental graph concepts on PreGraphs,
including structures like \emph{path}$^{*}$, predicates
such as \emph{is\_cyclic} and \emph{reachable}$^{*}$,
operations such as \emph{add\_vertex}
and \emph{remove\_edge}, and relations between PreGraphs such
as \emph{structurally\_identical}$^{*}$ and \emph{subgraph}.
Definitions of the concepts marked with asterisks are
shown in Figure \ref{fig:pregraphdefs} to give a flavor
of the subtleties involved in getting definitions that
really work.
These general concepts, together with around 500 derived lemmas,
provide a
solid foundation for more specific theorems needed in concrete
verifications.

\hide{
In {\color{magenta}Figure \ref{fig:pregraph}}, valid vertices satisfy
$V$ and invalid vertices are just of type $VT$ but do not satisfy
$V$. Importantly, both kinds of vertices are legally part of the
PreGraph. Finally, $s$ and $d$ are functions that map an edge to its
source and destination respectively; {\color{orange}this means that
PreGraphs model directed graphs.}  With an eye to flexibility, we make
no further requirements of a legal PreGraph, not even a specific
notion of how the four sets are related.  Indeed, the PreGraph in
Figure \ref{fig:pregraph} contains invalid vertices and edges in an
arbitrary configuration.

Many graph concepts such as \emph{path}, \emph{reachability}, and \emph{subgraph} are
defined on PreGraphs. In \S\ref{fig:find} we saw \emph{reachable}, written
{\color{orange}
$\m{a} \mathrel{{\stackrel{\gamma~}{\leadsto^{1}}}} \m{b}$. It means that a and b are in $V(\gamma)$ and that there exists an edge (in $E(\gamma)$)
that goes from a to b.}
The reflexive, transitive closure on \emph{reachable} is written
$\m{a} \accentset{\gamma}{\leadsto}^{\star} \m{b}$, and
$\neg (\m{a} \accentset{\gamma}{\leadsto}^{\star} \m{b})$
is written
$\m{a} \accentset{\gamma}{\not\leadsto}^{\star} \m{b}$
$\m{a} \mathrel{{\stackrel{\gamma~}{\not\leadsto^{\star}}}} \m{b}$.

PreGraph's ability to reason about missing vertices and edges is convenient when
verifying real programs. Suppose some $\gamma$ satisfied some stronger notion of
``well-formed'', in the sense that valid vertices have only valid edges and
vice versa. Could we then subtract some vertices and edges from it and reason about the
resulting structure? This is precisely what we needed to do in \ref{fig:find}, where
we argued for a condition of congruence on
$\gamma \smallsetminus (v \in \gamma \mid \m{x}
\mathrel{{\stackrel{\gamma~}{\leadsto^{\star}}}} \m{v})$.
A strong notion of well-formedness may have stopped us short at this point,
declaring the structure ill-formed because of the dangling edges
pointing to recently-removed vertices.
A PreGraph is more accommodating, since
it produces a fresh PreGraph after this selective subtraction
and then allows us to go ahead and reason about congruence as we need to.
}%end hide

\hide{
For example, consider the difference of two graphs, $\gamma_1
- \gamma_2$.  Even if both of these graphs are ``well-formed'' to begin with, in the
sense that valid vertices have only valid edges and vice versa, their difference
may not be since there may be dangling edges pointing to the
now-removed vertices of $\gamma_2$.} % end hide

\hide
{In \S\ref{sec:spacegraph} we will tie a mathematical graph $\gamma$ to
a spatial graph predicate
$\p{graph}(x, \gamma)$.   As we will see, a $\p{graph}$ ``owns'' only the
spatial portion of $\gamma$ that is reachable
from $x$ even though $\gamma$ may have other valid vertices.
} % end hide

%%%%%%%%%%%%%%%%%%%%%%%%%%%%%%%%%%%%%%%%%%%%%%%%%%
%%%%%% Edit 1: Qinxiang's proposal ends
%%%%%%%%%%%%%%%%%%%%%%%%%%%%%%%%%%%%%%%%%%%%%%%%%%


\vspace{-0.75ex}

\paragraph{LabeledGraph.}
A LabeledGraph is septuple $(\mathrm{PreGraph},\ma{L_{V}},\ma{L_{E}},\ma{L_{G}}, \mathtt{vl},\mathtt{el},\mathtt{gl})$ that augments a PreGraph with \emph{labels} on
vertices, edges, and/or the graph as a whole. $\ma{L_{V}}$, $\ma{L_{E}}$, and $\ma{L_{G}}$
are the associated carrier types; $\mathtt{vl}$, $\mathtt{el}$, and $\mathtt{gl}$
are the labeling functions themselves.
Many classic graph problems, from union-find (node ranks)
to Dijkstra (edge weights), require such labels.
The need for a label on the graph as a whole
is a little more subtle; in \S\ref{sec:certigc} we use one in the garbage collector
to keep track of \emph{e.g.} the number of generations and their boundaries.
Since LabeledGraph is built on PreGraph, it inherits all PreGraph lemmas via
Coq's \emph{type coercion} mechanism, while enabling additional lemmas involving labels.

%In addition, we can prove additional lemmas that use labels,
%\emph{e.g.} vertices in the union-find graph have an integer label denoting \emph{rank},
%and we prove that running \li{find} does not alter any vertex's rank.

%% Anshuman's proposal
\hide{\paragraph{LabeledGraph.}
A LabeledGraph is a PreGraph with the addition of \emph{labels} on
vertices, edges, and/or the graph as a whole. The need for such labels
is fairly clear; the bare structure of a graph can only
contain so much information, and many classic graph problems
such as graph coloring, shortest path, and network flow rely on
additional information in the form of labels. In our architecture, a
LabeledGraph inherits any lemmas proved about its associated PreGraph.
In addition, we can define additional lemmas that use labels,
\emph{e.g.} the union-find graph has an integer label denoting \emph{rank}.
We could prove a lemma that running \li{find} does not alter
any vertex's rank.
\hide{add string labels to edges and reason about a trie.}}
% Anshuman's proposal ends

%%%%%%%%%%%%%%%%%%%%%%%%%%%%%%%%%%%%%%%%%%%%%%%%%%
%%%%%% Edit 2

\vspace{-0.75ex}
\paragraph{GeneralGraph.}
PreGraphs and LabeledGraphs let us state
and prove many useful lemmas that follow essentially by the nature
of our graph constructions. However, when proving the correctness of graph
algorithms, we often need more specificity in our mathematical graphs
so that we may model the real program's behaviors closely.
For example, the \p{uf\_graph} used in \li{find}
restricted each vertex to having exactly one out-edge.
On the other hand, these restrictions vary greatly by algorithm, so we do not
want to bake them into our core definitions.
We achieve this flexibility using GeneralGraphs, which augment
LabeledGraphs by adding arbitrarily complex ``soundness conditions'', indicated in
Figure~\ref{fig:graphs} with a dashed border.
%These soundness condition can be arbitrarily
%complex. %, and can thus specify arbitrary restrictions on the graph.
Further, the type coercion we described earlier continues to apply,
meaning that a GeneralGraph can seamlessly behave like a
LabeledGraph or a PreGraph, thereby inheriting their lemmas.
This combination of specificity and generality is
what makes GeneralGraphs versatile. Moreover, we can
compose complicated soundness conditions from reusable pieces,
further enabling code sharing between algorithms.

%we do
%not need to restate

%for
%each program that we verify since we can %. In the next section, we explain how to
%them out of .



\subsection{Composing soundness plugins}
\label{subsec:graphplugins}
\begin{figure}
\begin{equation*}
{\footnotesize
\begin{array}{r@{}c@{}c@{\!}rl}
        \mathrm{MathGraph}(\gamma)\,&\defeq&\,\Big\{&\mathtt{null}:&\; V\\
          &&&\mathtt{valid\_graph}:&\; \forall e\,.\,\mathtt{evalid}(\gamma, e) \Rightarrow
          \mathtt{vvalid}\big(\gamma, \mathtt{src}(\gamma, e)\big)\wedge \null \\
          &&&&\qquad \big(\m{e} = \mathtt{null} \vee \mathtt{vvalid}(\gamma, \m{e})\big)\\
          &&&\mathtt{valid\_not\_null}:&\; \forall \m{v}\,.\,\mathtt{vvalid}(\gamma, \m{v})
          \Rightarrow \m{v} \neq \mathtt{null} \qquad \Big\} \\
        \mathrm{LstGraph}(\gamma)\,&\defeq&\, \Big\{&\mathtt{out}:&\; V \rightarrow E\\
          &&&\mathtt{only\_one\_edge}:&\; \forall \m{v},\,e\,.\,
          \mathtt{vvalid}(\gamma, \m{v}) \Rightarrow \\
          &&&&\qquad \Big(\mathtt{src}(\gamma, e) =\m{v} \wedge
                \mathtt{evalid}(\gamma, e)\Big) \Leftrightarrow e = \mathtt{out}(\m{v})\\
          &&&\mathtt{acyclic\_path}:&\; \forall \m{v},\,p\,.\,
          \gamma \vDash \m{v} \overset{p}{\leadsto} \m{v} \Rightarrow p = (\m{v},[]) \qquad \Big\} \\
       	\mathrm{FiniteGraph}(\gamma)\,&\defeq&\, \Big\{&\mathtt{finite\_v}:&\;\exists\, S_\m{v},\, M_\m{v}.~\;\lvert S_\m{v}\rvert
          \leq M_\m{v} \wedge
          \forall \m{v}\,.\, \mathtt{vvalid}(\gamma,\m{v})\Rightarrow \m{v}\,\in\,S_\m{v}\\
          &&&\mathtt{finite\_e}:&\;\exists\, S_e,\, M_e.~\;\lvert S_e\rvert
          \leq M_e \wedge
          \forall e\,.\, \mathtt{evalid}(\gamma,e)\Rightarrow e\,\in\,S_e \qquad \Big\}
\end{array}}
\end{equation*}
\vspace{-1em}
\caption{Some GeneralGraph definitions}
\label{fig:gengraphdefs}
\vspace{-1.5em}

\end{figure} 

\hide{Our entire library of formal
graph theory is developed around the
three graph structures above. The theorems about the first two
are universal, while some theorems about GeneralGraph
are developed on demand because soundness conditions vary by
algorithm.}
Soundness conditions are often specific %need to be custom-designed
to each algorithm, but they feature some recurring themes.
We take advantage of this pattern by developing %some simple atomic
\emph{soundness plugins}, \emph{i.e.} definitions of soundness
conditions along with related lemmas.  By combining these plugins
we can describe the soundness condition we need for a particular
algorithm.  When proving lemmas about the resulting combination,
we can use known facts about the separate plugins, in addition to
lemmas that emerge due to the various combinations.  This complexity
is managed smoothly by Coq's typeclass system, increasing the
compositionality of the system.
Consider the following oft-used graph properties:

\begin{itemize}
\vspace{-1ex}
\item \p{BiGraph}: there are exactly two outgoing edges per vertex
\item $\p{LstGraph}^{*}$: the graph is structured like a list, meaning that every 
vertex has one outgoing edge, and there no loops except trivial self-loops 
of length $0$, signifying the end of the ``list"
\item $\p{MathGraph}^{*}$: every valid edge has a valid source vertex,
and its destination vertex is either valid or is a 
special invalid vertex called \p{null}
%, which
%is allowed to be the destination of valid edges are allowed to be destinations
%of valid edges, thus allowing null values to represent unused nodes
\item $\p{FiniteGraph}^{*}$: the sets of valid vertices and edges are both finite
%\vspace{-1ex}
\hide{More subtly, consider that many real data structures use special null values to
represent unused nodes.  The  property introduces this concept---
\emph{i.e.} some special invalid nodes are allowed to appear as
destinations for valid edges.} % end hide
%\vspace{-1ex}

%each

\end{itemize}
Definitions of the concepts marked with asterisks are
shown in Figure~\ref{fig:gengraphdefs} for illustration.

\hide{As a first step, we can prove many general, reusable lemmas
about these properties. However, these properties are still
too general to model a real program. The next step is to compose
these plugins to arrive at a more specific set of restrictions
that more closely models our particular graph.}
We can compose
\p{LstGraph}, \p{MathGraph}, and \p{FiniteGraph}
together into a new plugin called \p{LiMaFin}, which, incidentally, is
the soundness condition of mathematical \p{uf\_graph}
we used to verify \li{find} in
Figure~\ref{fig:find}.  In our verification of \li{mark} in
Figure~\ref{fig:markgraph}, we use a similar soundness condition
\p{BiMaFin}, which uses \p{BiGraph} instead of \p{LstGraph}.
The commonalities and differences between \p{LiMaFin}
and \p{BiMaFin} are readily apparent from their construction.
%Composing plugins in this way is much more elegant than creating a
%custom plugin for each new algorithm as it promotes the natural reuse of graph
%properties across algorithms.

\iffalse
\marginpar{\tiny \color{blue} Maybe move this somewhere.}
{\color{magenta}Coq also handles our notion of inherited
lemmas seamlessly: in our verfication of Find, we
work directly with a \p{LiMaFin} GeneralGraph, but, as
we saw, we still use properties such as reachability
and operations such as selective subtraction, which are defined on the
embedded PreGraph, not the GeneralGraph.
Coq handles the appropriate coercions with
remarkable elegance.}
\fi

%% We can apply our framework to define related structures such as DAGs and trees.
%% For example, a DAG has the additional property that for any $x$ and $y$,
%% if $x$ is reachable from $y$, then $x = y$ or $y$ is not reachable
%% from $x$.
%  Similarly, we define tree by saying that for any reachable node $n$ there is
%a unique path from the root to $n$.
\iffalse
\subsection{Reasoning about relations between graphs} %Using our framework to reasonApplication of the framework}

\marginpar{\color{blue} \tiny Needs revised examples based on new Orientation.}

{\color{magenta} In Figure~\ref{fig:markgraph} we defined the relation
$\m{mark}(\gamma, \tx x, \gamma')$
for the graph marking algorithm.  Similarly, we define $\m{span}$ for the
spanning tree program
and $\m{copy}$ for the graph copy program.
These relations all capture how the graph has changed from before to after the program
execution.  By specifying $\m{copy}$ relationally
rather than functionally we avoid explicitly modeling how the memory
allocator works, a major advantage.

As previously mentioned, we reuse $\m{mark}$ and its
related lemmas to prove facts about spanning tree and graph copy
because the latter two programs mark nodes as they work.
Accordingly, we can reuse facts such as the following:
%%%%%%%%%%%%%%%%%%%%%%%%%%%%%%%%%%%%%%%%%%%%%%%%%%
%%%%%% Edit 4
%%%%%%%%%%%%%%%%%%%%%%%%%%%%%%%%%%%%%%%%%%%%%%%%%%
%%%%%%%%%%%%%%%%%%%%%%%%%%%%%%%%%%%%%%%%%%%%%%%%%%
%%%%%% Edit 4: Qinxiang's proposal starts
%%%%%%%%%%%%%%%%%%%%%%%%%%%%%%%%%%%%%%%%%%%%%%%%%%
\[
\begin{array}{@{}l@{}}
\text{if } \gamma(x)=(0, v_1, \dots,v_n), \m{mark1}(\gamma, x, \gamma_1),
\text{ and } \forall i, \m{mark}(\gamma_i,v_i,\gamma_{i+1}) \text{ then } \m{mark}(\gamma,x,\gamma_{n+1}).
\end{array}
\]
}
\fi
%We prove this theorem sound for any LabeledGraph, not just
%\p{BiGraph}s (\emph{i.e.}, we do not assume only two neighbors).
%%%%%%%%%%%%%%%%%%%%%%%%%%%%%%%%%%%%%%%%%%%%%%%%%%
%%%%%% Edit 4: Qinxiang's proposal ends
%%%%%%%%%%%%%%%%%%%%%%%%%%%%%%%%%%%%%%%%%%%%%%%%%%
