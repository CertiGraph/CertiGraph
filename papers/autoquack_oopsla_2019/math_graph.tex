In order to verify the functional correctness of graph
algorithms, we need to first reason about mathematical graphs.
Our graph library is powerful and expressive, allowing 
us to verify realistic algorithms that work in an end-to-end
system. One of its strengths is its modularity, 
which allows us to intuitively reuse and compose our proofs when
mechanising our verifications. In this section, we present our 
mathematical graph framework with an emphasis on this modularity. 
{\color{magenta} We continue to use Union-Find from 
\S\ref{sec:orientation} as our motivating example.}

% saving old version just in case...
\hide{As will be shown in \S\ref{sec:development}, our mathematical
graph constructions comprise a considerable fraction of our
codebase. Indeed, as discussed in \S\ref{sec:related},
25 years of research into mechanized graph theory can
be summarized as ``it is a little tricky''. 
First, as demonstrated in \S\ref{sec:orientation},
our development is expressive and powerful enough to verify realistic
algorithms---that is, it actually works in an end-to-end system.
Second, we have taken considerable care to develop a modular and
general-purpose framework for such mathematical graphs to allow
such verifications to be mechanized without undue pain.
Accordingly, in this section we will present our framework
at a high level to communicate the overall architecture rather
than focusing on the nitty-gritty details.} % end hide

\subsection{Structure of the mathematical graph framework}\label{sec:mathinfra}

\begin{figure}[t!]
\centering
\begin{adjustbox}{scale=0.85}
  \begin{subfigure}[t]{0.5\textwidth}
  \centering
  \beginpgfgraphicnamed{variousgraph}
\begin{tikzpicture}
[->/.style={thick,arrows={-Stealth}},
-->/.style={thick,arrows={-Stealth}, decorate, decoration={snake, amplitude=.4mm,segment length=2mm,post length=2mm}},
   realG/.style={shape=rectangle, rounded corners=4pt, draw, fill=gray!40},
   propG/.style={shape=rectangle, rounded corners=4pt, draw}]
\node[realG] (PG) at (0, 0) {\small PreGraph};
\node[realG] (LG) [right=0.8 of PG] {\small LabeledGraph};
\node[realG] (GG) [right=2 of LG] {\small GeneralGraph};
\draw [double, ->] (PG) -- (LG) node [pos=0.5, above] {\small Label} ;
\draw [double, ->] (LG) -- (GG) node (SC) [pos=0.5, above, align=center]
{\small Soundness \\ \small Condition};
\node[propG] (Prop) [below=0.6 of SC] {\small Property};
\node[propG] (PropL) [below=0.4 of Prop] {\small Property Lemmas};
\node[propG] (PGL) [below=2 of PG, align=center] {\small PreGraph \\\small Lemmas};
\node[propG] (LGL) [below=2 of LG, align=center] {\small LabeledGraph \\\small Lemmas};
\node[propG] (GGL) [below=2 of GG, align=center] {\small GeneralGraph \\\small Lemmas};
\draw [double, ->] (PGL) to (LGL);
%% \draw [double, ->] (LGL) to (GGL);
\draw [->] (PG) to (PGL);
\draw [->] (Prop) to (PropL);
\draw [-->] (Prop) to (SC);
\coordinate [left=0.2 of LG.south] (LGs1);
\coordinate [left=0.2 of LGL.north] (LGLn1);
\draw [->] (LGs1) to (LGLn1);
\coordinate [right=0.2 of LG.south] (LGs2);
\coordinate [right=0.2 of LGL.north] (LGLn2);
\draw [->] (LGs2) |- (Prop);
\draw [double, ->] (LGLn2) |- (PropL);
\coordinate [right=0.2 of GG.south] (GGs);
\coordinate [left=0.2 of GGL.north] (GGLn1);
\coordinate [right=0.2 of GGL.north] (GGLn2);
\draw [double, ->] (PropL) -| (GGLn1);
\draw [->] (GGs) to (GGLn2);
\node [draw, thick, rectangle, dashed, fit=(Prop) (PropL)] {};
\node (legend1) [below right=0.2 and -0.3 of PGL] {\small Depends};
\coordinate[left=0.8 of legend1]  (l1);
\draw [->] (l1) to (legend1);
\node (legend2) [right=1 of legend1] {\small Inherits};
\coordinate[left=0.8 of legend2]  (l2);
\draw [double, ->] (l2) to (legend2);
\node (legend3) [right=1 of legend2] {\small Instantializes};
\coordinate[left=0.8 of legend3]  (l3);
\draw [-->] (l3) to (legend3);
\end{tikzpicture}
\endpgfgraphicnamed
\vspace{1ex}
\captionof{figure}{Structure of the Mathematical Graph Library}\label{fig:graphs}
\end{subfigure}
  ~
  \begin{subfigure}[t]{0.5\textwidth}
\centering
\beginpgfgraphicnamed{pregraphexp}
\begin{tikzpicture}
[vad/.style={circle, fill=black, inner sep=0pt, minimum size=4pt},
 inv/.style={circle, draw=black, thick, inner sep=0pt, minimum size=4pt},
 ->/.style={thick, arrows={-Stealth}}]
\node[vad] (n1) at (0, 0) {};
\node[vad] (n2) at (1, 1) {};
\node[inv] (n3) at (1, -1) {};
\node[vad] (n4) at (-2,2) {};
\node[vad] (n5) at (-2,-2) {};
\node[vad] (n6) at (-3,0) {};
\node[vad] (n7) at (3,1.5) {};
\node[inv] (n8) at (3,-1.5) {};
\node[vad] (n9) at (3.5, 1) {};
\node[inv] (n10) at (3.5, 0.5) {};
\node at (3.5, 1) [right=1.5pt] {\small Valid node};
\node at (3.5, 0.5) [right=1.5pt] {\small Invalid node};
\node at (3.5, 0) [right=1.5pt] {\small Valid edge};
\node at (3.5, -0.5) [right=1.5pt] {\small Invalid edge};
\draw[->] (n1) to (n2);
\draw[->,dashed] (n1) to (n3);
\draw[->,dashed] (n3) to (n5);
\draw[->] (n2) to (n3);
\draw[->] (n2) to (n7);
\draw[->] (n2) to (n8);
\draw[->,dashed] (n3) to (n8);
\draw[->] (n4) to (n1);
\draw[->] (n1) to (n5);
\draw[->] (n2) to (n4);
\draw[->] (n1) to [bend left=20] (n6);
\draw[->] (n6) to (n5);
\draw[->] (n8) to [bend left=20] (n7);
\draw[->] (n4) to [bend right=35] (n1);
\draw[->] (3.0, 0) -- (3.6, 0);
\draw[->,dashed] (3.0, -0.5) -- (3.6, -0.5);
\node at (1, 1.3) {$v_0$};
\node at (1, -1.3) {$v_1$};
\node at (1.2, 0) {$e_0$};
\end{tikzpicture}
\endpgfgraphicnamed
\vspace{1ex}
\captionof{figure}{A PreGraph with valid/invalid vertices and edges.}\label{fig:pregraph}
\end{subfigure}
\end{adjustbox}
\caption{The Mathematical Graph Library, and a Pregraph}
\end{figure}

%\begin{figure}[t]
\centering
%\beginpgfgraphicnamed{variousgraph}
\begin{tikzpicture}
[->/.style={thick,arrows={-Stealth}},
-->/.style={thick,arrows={-Stealth}, decorate, decoration={snake, amplitude=.4mm,segment length=2mm,post length=2mm}},
   realG/.style={shape=rectangle, rounded corners=4pt, draw, fill=gray!40},
   propG/.style={shape=rectangle, rounded corners=4pt, draw}]
\node[realG] (PG) at (0, 0) {\small PreGraph};
\node[realG] (LG) [right=0.8 of PG] {\small LabeledGraph};
\node[realG] (GG) [right=2 of LG] {\small GeneralGraph};
\draw [double, ->] (PG) -- (LG) node [pos=0.5, above] {\small Label} ;
\draw [double, ->] (LG) -- (GG) node (SC) [pos=0.5, above, align=center]
{\small Soundness \\ \small Condition};
\node[propG] (Prop) [below=0.6 of SC] {\small Property};
\node[propG] (PropL) [below=0.4 of Prop] {\small Property Lemmas};
\node[propG] (PGL) [below=2 of PG, align=center] {\small PreGraph \\\small Lemmas};
\node[propG] (LGL) [below=2 of LG, align=center] {\small LabeledGraph \\\small Lemmas};
\node[propG] (GGL) [below=2 of GG, align=center] {\small GeneralGraph \\\small Lemmas};
\draw [double, ->] (PGL) to (LGL);
%% \draw [double, ->] (LGL) to (GGL);
\draw [->] (PG) to (PGL);
\draw [->] (Prop) to (PropL);
\draw [-->] (Prop) to (SC);
\coordinate [left=0.2 of LG.south] (LGs1);
\coordinate [left=0.2 of LGL.north] (LGLn1);
\draw [->] (LGs1) to (LGLn1);
\coordinate [right=0.2 of LG.south] (LGs2);
\coordinate [right=0.2 of LGL.north] (LGLn2);
\draw [->] (LGs2) |- (Prop);
\draw [double, ->] (LGLn2) |- (PropL);
\coordinate [right=0.2 of GG.south] (GGs);
\coordinate [left=0.2 of GGL.north] (GGLn1);
\coordinate [right=0.2 of GGL.north] (GGLn2);
\draw [double, ->] (PropL) -| (GGLn1);
\draw [->] (GGs) to (GGLn2);
\node [draw, thick, rectangle, dashed, fit=(Prop) (PropL)] {};
\node (legend1) [below right=0.2 and -0.3 of PGL] {\small Depends};
\coordinate[left=0.8 of legend1]  (l1);
\draw [->] (l1) to (legend1);
\node (legend2) [right=1 of legend1] {\small Inherits};
\coordinate[left=0.8 of legend2]  (l2);
\draw [double, ->] (l2) to (legend2);
\node (legend3) [right=1 of legend2] {\small Instantializes};
\coordinate[left=0.8 of legend3]  (l3);
\draw [-->] (l3) to (legend3);
\end{tikzpicture}
%\endpgfgraphicnamed
\vspace{1ex}
\caption{Structure of the Mathematical Graph Library}\label{fig:graphs}
\end{figure}

%\begin{figure}[t]
\centering
\beginpgfgraphicnamed{pregraphexp}
\begin{tikzpicture}
[vad/.style={circle, fill=black, inner sep=0pt, minimum size=4pt},
 inv/.style={circle, draw=black, thick, inner sep=0pt, minimum size=4pt},
 ->/.style={thick, arrows={-Stealth}}]
\node[vad] (n1) at (0, 0) {};
\node[vad] (n2) at (1, 1) {};
\node[inv] (n3) at (1, -1) {};
\node[vad] (n4) at (-2,2) {};
\node[vad] (n5) at (-2,-2) {};
\node[vad] (n6) at (-3,0) {};
\node[vad] (n7) at (3,1.5) {};
\node[inv] (n8) at (3,-1.5) {};
\node[vad] (n9) at (3.5, 1) {};
\node[inv] (n10) at (3.5, 0.5) {};
\node at (3.5, 1) [right=1.5pt] {\small Valid node};
\node at (3.5, 0.5) [right=1.5pt] {\small Invalid node};
\node at (3.5, 0) [right=1.5pt] {\small Valid edge};
\node at (3.5, -0.5) [right=1.5pt] {\small Invalid edge};
\draw[->] (n1) to (n2);
\draw[->,dashed] (n1) to (n3);
\draw[->,dashed] (n3) to (n5);
\draw[->] (n2) to (n3);
\draw[->] (n2) to (n7);
\draw[->] (n2) to (n8);
\draw[->,dashed] (n3) to (n8);
\draw[->] (n4) to (n1);
\draw[->] (n1) to (n5);
\draw[->] (n2) to (n4);
\draw[->] (n1) to [bend left=20] (n6);
\draw[->] (n6) to (n5);
\draw[->] (n8) to [bend left=20] (n7);
\draw[->] (n4) to [bend right=35] (n1);
\draw[->] (3.0, 0) -- (3.6, 0);
\draw[->,dashed] (3.0, -0.5) -- (3.6, -0.5);
\end{tikzpicture}
\endpgfgraphicnamed
\vspace{1ex}
\caption{A PreGraph with valid and invalid vertices and edges.}\label{fig:pregraph}
\end{figure}


{\color{magenta}Figure~\ref{fig:graphs}} shows the architecture of our mathematical graph library. 
\hide{The most basic kind of graph is PreGraph, out of which we build 
LabeledGraph, and which in turn are used
to build GeneralGraphs.  Each kind has some lemmas and also inherits the lemmas of the 
previous kind.  The dashed box represents a ``plugin'' system for attaching arbitrary 
properties to LabeledGraphs (\ref{subsec:graphplugins}). %and will be discussed later. 
%We will consider each in turn.
} % end hide

% move to its own file?

\vspace{-0.75ex}
\iftrue
\paragraph{PreGraph.} A PreGraph is a hextuple $(VT, ET, V, E, s, d)$, where $VT$ 
and $ET$ are the underlying carrier sets of vertices and edges, and $V$ and $E$, 
subsets $VT$ and $ET$ respectively, introduce the notion of \emph{validity} in the 
graph. In {\color{magenta}Figure \ref{fig:pregraph}}, valid vertices are in $V$ and 
invalid vertices are in $VE \smallsetminus V$. Importantly, 
both kinds of vertices are legally part 
of the PreGraph. Finally, $s$ and $d$ are functions that map 
an edge to its source and destination respectively; 
{\color{orange}this means that PreGraphs model directed graphs.}
With an eye to flexibility, we make no further 
requirements of a legal PreGraph, not even a specific notion 
of how the four sets are related.
Indeed, the PreGraph in Figure \ref{fig:pregraph} contains invalid 
vertices and edges in an arbitrary configuration.

Many graph concepts such as \emph{path}, \emph{reachability}, and \emph{subgraph} are 
defined on PreGraphs. In \S\ref{fig:find} we saw \emph{reachable}, written
{\color{orange}
$\m{a} \mathrel{{\stackrel{\gamma~}{\leadsto^{1}}}} \m{b}$. It means that a and b are in $V(\gamma)$ and that there exists an edge (in $E(\gamma)$)
that goes from a to b.}
The reflexive, transitive closure on \emph{reachable} is written 
$\m{a} \mathrel{{\stackrel{\gamma~}{\leadsto^{\star}}}} \m{b}$, and 
$\neg (\m{a} \mathrel{{\stackrel{\gamma~}{\leadsto^{\star}}}} \m{b})$ 
is written $\m{a} \mathrel{{\stackrel{\gamma~}{\not\leadsto^{\star}}}} \m{b}$.

PreGraph's ability to reason about missing vertices and edges is convenient when 
verifying real programs. Suppose some $\gamma$ satisfied some stronger notion of
``well-formed'', in the sense that valid vertices have only valid edges and 
vice versa. Could we then subtract some vertices and edges from it and reason about the 
resulting structure? This is precisely what we needed to do in \ref{fig:find}, where
we argued for a condition of congruence on 
$\gamma \smallsetminus (v \in \gamma \mid \m{x} 
\mathrel{{\stackrel{\gamma~}{\leadsto^{\star}}}} \m{v})$. 
A strong notion of well-formedness may have stopped us short at this point, 
declaring the structure ill-formed because of the dangling edges 
pointing to recently-removed vertices. 
A PreGraph is more accommodating, since
it produces a fresh PreGraph after this selective subtraction 
and then allows us to go ahead and reason about congruence as we need to.

\hide{
For example, consider the difference of two graphs, $\gamma_1
- \gamma_2$.  Even if both of these graphs are ``well-formed'' to begin with, in the 
sense that valid vertices have only valid edges and vice versa, their difference 
may not be since there may be dangling edges pointing to the 
now-removed vertices of $\gamma_2$.} % end hide

\hide
{In \S\ref{sec:spacegraph} we will tie a mathematical graph $\gamma$ to 
a spatial graph predicate
$\p{graph}(x, \gamma)$.   As we will see, a $\p{graph}$ ``owns'' only the
spatial portion of $\gamma$ that is reachable
from $x$ even though $\gamma$ may have other valid vertices.
} % end hide
\fi
%%%%%%%%%%%%%%%%%%%%%%%%%%%%%%%%%%%%%%%%%%%%%%%%%%
%%%%%% Edit 1: Qinxiang's proposal ends
%%%%%%%%%%%%%%%%%%%%%%%%%%%%%%%%%%%%%%%%%%%%%%%%%%
%%%%%% Edit 1: Original version starts
%%%%%%%%%%%%%%%%%%%%%%%%%%%%%%%%%%%%%%%%%%%%%%%%%%
\iffalse
\paragraph{Pregraphs.} A PreGraph is a hextuple $(V, E, \phi_V, \phi_E, s, d)$,
where $V$ and $E$ are the underlying carrier set of vertices and edges.  
Not every $v \in V$ or $e \in E$ is actually ``in'' the graph, so we provide 
the predicates $\phi_V$ and $\phi_E$ to classify vertices and edges as 
\emph{valid} (in) or not (out).  Finally, $s$ and $d : E -> V$ are functions that 
map an edges to their source and destination respectively; this model means that 
PreGraphs are directed rather than undirected.  By design, there are no requirements 
for \emph{e.g.} how the validities of edges and vertices relate.  As shown in 
Figure \ref{fig:pregraph}, a PreGraph can contain invalid nodes and edges in an 
arbitrary configuration.

Many graph concepts such as \emph{path}, \emph{reachability}, and \emph{subgraph} are 
defined on PreGraphs.  Wefor write $\gamma\models n_1 \xrightarrow{P} n_2$ to mean 
that there is a valid path from $n_1$ to $n_2$ such that each vertex in the path 
satisfies the predicate $P$.  The set of all reachable vertices from $v$, written $\p{
reachable}(\gamma,v)$, is then just $\{v' ~|~ \gamma\models v \xrightarrow{\top} v'\}$.
In \S\ref{sec:spacegraph} we will tie mathematical graphs $\gamma$ to a spatial graph 
predicate
$\p{graph}(x, \gamma)$.   As we will see, $\p{graph}$ ``owns'' only the
spatial portion of $\gamma$ that is reachable
from $x$ even though $\gamma$ may have other valid vertices.

The advantage of designing a graph type that can reason about missing vertices and 
edges is because concepts necessary to verify real programs require such flexibility.  
For example, consider the difference of two graphs, $\gamma_1 - \gamma_2$.  Even if 
both of these graphs are ``well-formed'' to begin with, in the sense that valid nodes 
have only valid edges and vice versa, their difference may not since there may be 
dangling edges pointing to the now-removed vertices of $\gamma_2$.
\fi
%%%%%%%%%%%%%%%%%%%%%%%%%%%%%%%%%%%%%%%%%%%%%%%%%%
%%%%%% Edit 1: Original version ends
%%%%%%%%%%%%%%%%%%%%%%%%%%%%%%%%%%%%%%%%%%%%%%%%%%


\vspace{-0.75ex}
%% Anshuman's proposal 
\iftrue
\paragraph{LabeledGraph.}
A LabeledGraph is a PreGraph with the addition of \emph{labels} on 
vertices, edges, and/or the graph as a whole. The need for such labels
is fairly clear; the bare structure of a graph can only 
contain so much information, and many classic graph problems 
such as graph coloring, shortest path, and network flow rely on 
additional information in the form of labels. In our architecture, a
LabeledGraph inherits any lemmas proved about its associated PreGraph. 
In addition, we can define additional lemmas that use labels, 
\emph{e.g.} the union-find graph has an integer label denoting \emph{rank}.
We could prove a lemma that running \texttt{find} does not alter
any vertex's rank. 
\hide{add string labels to edges and reason about a trie.}
\fi
% Anshuman's proposal ends

%%%%%%%%%%%%%%%%%%%%%%%%%%%%%%%%%%%%%%%%%%%%%%%%%%
%%%%%% Edit 2
%%%%%%%%%%%%%%%%%%%%%%%%%%%%%%%%%%%%%%%%%%%%%%%%%%
%%%%%% Edit 2: Qinxiang's proposal starts
%%%%%%%%%%%%%%%%%%%%%%%%%%%%%%%%%%%%%%%%%%%%%%%%%%
\iffalse
\paragraph{LabeledGraph.}
A LabeledGraph is a PreGraph with labels, e.g. the ``mark bit'' used in Figure~\ref{fig:markgraph} are labels.
\fi
%%%%%%%%%%%%%%%%%%%%%%%%%%%%%%%%%%%%%%%%%%%%%%%%%%
%%%%%% Edit 2: Qinxiang's proposal ends
%%%%%%%%%%%%%%%%%%%%%%%%%%%%%%%%%%%%%%%%%%%%%%%%%%
%%%%%% Edit 2: Original version starts
%%%%%%%%%%%%%%%%%%%%%%%%%%%%%%%%%%%%%%%%%%%%%%%%%%
\iffalse
\paragraph{LabeledGraph.}
Although many basic lemmas can be proved about PreGraphs, they are inadequate for real program verification.
When reasoning about the concrete graphs manipulated by various algorithms,
we usually need to add a notion of \emph{labels} on vertices and/or edges, such as
the ``mark bit'' used in Figure~\ref{fig:markgraph}, letting us define notions like ``the vertices reachable via an unmarked path''
on LabeledGraphs.
\fi
%%%%%%%%%%%%%%%%%%%%%%%%%%%%%%%%%%%%%%%%%%%%%%%%%%
%%%%%% Edit 2: Original version ends
%%%%%%%%%%%%%%%%%%%%%%%%%%%%%%%%%%%%%%%%%%%%%%%%%%

\vspace{-0.75ex}
\paragraph{GeneralGraph.}
PreGraphs and LabeledGraphs are quite universal, and allow us 
to state and prove useful lemmas that are true by virtue of the 
way the graphs were set up. However, when proving the correctness
of graph algorithms, we often need more specificity in our mathematical graphs
so that we may model the real program's restrictions more closely. 
For example, the graph used in Find earlier had the restriction 
that each vertex have exactly one out-edge. 
We achieve this using GeneralGraphs. 
A GeneralGraph augments a LabeledGraph by adding a 
``soundness condition'' plugin, indicated in 
Figure~\ref{fig:graphs} by a dashed border. 
This soundness condition can be arbitrarily complex, 
and can thus specify arbitrary restrictions on the graph. 
This is what makes GeneralGraphs truly versatile: 
it allows us to state algorithm-specific properties 
of any complexity on the graph, and then derive lemmas based on those properties.
Thankfully, we do not need to state these complicated soundness 
conditions afresh for each program that we verify. In the next section, 
we explain how to compose these out of smaller, reusable pieces.

\subsection{Composing Graph plugins}
\label{subsec:graphplugins}

We use Coq's typeclass system to manage our soundness plugins smoothly, 
benefiting from the \emph{compositionality} of the system. 
If we have two soundness properties, each with its associated proved lemmas, 
we can combine them, 
prove lemmas about their combination using known facts about 
the separate pieces,
and then treat the combination as a new plugin. 

Consider the following oft-used graph properties:
\begin{itemize}
\vspace{-1ex}
\item \p{FiniteGraph}: The validity sets $V$ and $E$ are finite.
%\vspace{-1ex}
\item \p{MathGraph}: invalid nodes are allowed to be destinations
of valid edges, thus allowing null values to represent unused nodes.
\hide{More subtly, consider that many real data structures use special null values to 
represent unused nodes.  The  property introduces this concept---
\emph{i.e.} some special invalid nodes are allowed to appear as 
destinations for valid edges.} % end hide
%\vspace{-1ex}
\item \p{BiGraph}: there are exactly two outgoing edges per node. 
\item \p{LstGraph}: the graph is list-like; loops are forbidden and 
nodes have one out-edge each.
\end{itemize}

\hide{
\begin{figure}[t]
\centering
\beginpgfgraphicnamed{graphproperty}
\begin{tikzpicture}
[->/.style={thick,arrows={-Stealth}},
   group/.style={shape=rectangle, draw, thick, dashed},
   propG/.style={shape=rectangle, rounded corners=4pt, draw}]
\node[propG] (PL12) at (0, 0) {\footnotesize Lemmas of Property 1 and 2};
\coordinate [left=1 of PL12.north] (PL12n1);
\coordinate [right=1 of PL12.north] (PL12n2);
\node[propG] (PL1) [above=0.5 of PL12n1, align=center] {\footnotesize Property 1 \\\footnotesize Lemmas};
\node[propG] (PL2) [above=0.5 of PL12n2, align=center] {\footnotesize Property 2 \\\footnotesize Lemmas};
\node[propG] (P1) [above=0.5 of PL1] {\footnotesize Property 1};
\node[propG] (P2) [above=0.5 of PL2] {\footnotesize Property 2};
\draw [->] (P1) to (PL1);
\draw [->] (P2) to (PL2);
\draw [double, ->] (PL1) to (PL12n1);
\draw [double, ->] (PL2) to (PL12n2);
\draw [->] (P1.west) to [bend right=45] (PL12.west);
\draw [->] (P2.east) to [bend left=45] (PL12.east);
\node (R1) [group, fit=(P1) (PL1)] {};
\node (R2) [group, fit=(P2) (PL2)] {};
\node (R3) [group, fit=(current bounding box)] {};
\node [propG] (P1P2) [above right=-1.1 and 1.2 of R3, align=left] {\footnotesize Property 1$/|$ \\\footnotesize Property 2};
\node [propG] (PL1PL2) [below=0.5 of P1P2, align=center] {\footnotesize Prop. 1 Lemmas \\\footnotesize Prop. 2 Lemmas \\\footnotesize Prop. 1$/|$2 Lemmas};
\draw [->] (P1P2) to (PL1PL2);
\node (R4) [group, fit=(P1P2) (PL1PL2)] {};
\node (EQ) [right=0 of R3] {\bf \Large $\mapsto$};
\end{tikzpicture}
\endpgfgraphicnamed
\vspace{1ex}
\caption{Combining plugins{\color{blue} [cut this?]}}\label{fig:properties}
\end{figure}
} % end hide

As a first step, we can prove many general, reusable lemmas
about these properties. However, these properties are still 
too general to model a real program. The next step is compose 
these plugins to arrive at a more specific set of restrictions 
that more closely models our particular graph. 
For instance, we can compose 
\p{LstGraph}, \p{MathGraph}, and \p{FiniteGraph} 
together into a new plugin called \p{LiMaFin}, which, incidentally, is the 
soundness condition used to verify Find in Figure~\ref{fig:find}.
In our verification of Mark~\ref{fig:markgraph}, we use a similar soundness condition
\p{BiMaFin}, which uses \p{BiGraph} instead of \p{LstGraph}. 
The commonalities and differences between \p{LiMaFin} 
and \p{BiMaFin} are readily apparent from their construction.
Composing plugins in this way is much more elegant than creating a
custom plugin for each new algorithm as it promotes the natural reuse of graph 
properties across algorithms.

\marginpar{\tiny \color{blue} Maybe move this somewhere.} 
{\color{magenta}Coq also handles our notion of inherited 
lemmas seamlessly: in our verfication of Find, we 
work directly with a \p{LiMaFin} GeneralGraph, but, as 
we saw, we still use properties such as reachability 
and operations such as selective subtraction, which are defined on the 
embedded PreGraph, not the GeneralGraph. 
Coq handles the appropriate coercions with 
remarkable elegance.}

%% We can apply our framework to define related structures such as DAGs and trees.
%% For example, a DAG has the additional property that for any $x$ and $y$,
%% if $x$ is reachable from $y$, then $x = y$ or $y$ is not reachable
%% from $x$.
%  Similarly, we define tree by saying that for any reachable node $n$ there is
%a unique path from the root to $n$.

\subsection{Reasoning about relations between graphs} %Using our framework to reasonApplication of the framework}

\marginpar{\color{blue} \tiny Needs revised examples based on new Orientation.}

{\color{magenta} In Figure~\ref{fig:markgraph} we defined the relation 
$\m{mark}(\gamma, \tx x, \gamma')$
for the graph marking algorithm.  Similarly, we define $\m{span}$ for the 
spanning tree program
and $\m{copy}$ for the graph copy program.
These relations all capture how the graph has changed from before to after the program
execution.  By specifying $\m{copy}$ relationally
rather than functionally we avoid explicitly modeling how the memory 
allocator works, a major advantage.

As previously mentioned, we reuse $\m{mark}$ and its
related lemmas to prove facts about spanning tree and graph copy
because the latter two programs mark nodes as they work.
Accordingly, we can reuse facts such as the following:
%%%%%%%%%%%%%%%%%%%%%%%%%%%%%%%%%%%%%%%%%%%%%%%%%%
%%%%%% Edit 4
%%%%%%%%%%%%%%%%%%%%%%%%%%%%%%%%%%%%%%%%%%%%%%%%%%
%%%%%%%%%%%%%%%%%%%%%%%%%%%%%%%%%%%%%%%%%%%%%%%%%%
%%%%%% Edit 4: Qinxiang's proposal starts
%%%%%%%%%%%%%%%%%%%%%%%%%%%%%%%%%%%%%%%%%%%%%%%%%%
\[
\begin{array}{@{}l@{}}
\text{if } \gamma(x)=(0, v_1, \dots,v_n), \m{mark1}(\gamma, x, \gamma_1),
\text{ and } \forall i, \m{mark}(\gamma_i,v_i,\gamma_{i+1}) \text{ then } \m{mark}(\gamma,x,\gamma_{n+1}).
\end{array}
\]
}
%We prove this theorem sound for any LabeledGraph, not just
%\p{BiGraph}s (\emph{i.e.}, we do not assume only two neighbors).
%%%%%%%%%%%%%%%%%%%%%%%%%%%%%%%%%%%%%%%%%%%%%%%%%%
%%%%%% Edit 4: Qinxiang's proposal ends
%%%%%%%%%%%%%%%%%%%%%%%%%%%%%%%%%%%%%%%%%%%%%%%%%%
