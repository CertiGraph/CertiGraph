\begin{figure}[t]
\vspace{-1ex}
  \begin{lstlisting}
struct Node { $\label{code:nodedefstart}$
  unsigned int rank;
  struct Node * parent; $\label{code:nodedefstart}$
  }

struct Node* find(struct Node* x) { // $\label{code:findstart}\{\p{graph}(\gamma)\}$
  struct Node *p, *p0;
// $\label{code:befparentfind}\searrow\{\tx{x} \in vertices(\gamma) /| \gamma(\tx{x}) = (rnk, par) /| \tx{x}|-> rnk,par\}$
  p = x -> parent; $\label{code:findram1}$
// $\label{code:aftparentfind}\swarrow\{\tx x \in vertices(\gamma) /| \gamma(\tx{x}) = (rnk, par) /| \tx{x}|-> rnk,par /| \tx{p} = par\}$ 
  if (p != x) {
// $\label{code:findbeforereassign} \{\tx x \in vertices(\gamma) /| \gamma(\tx{x}) = (rnk, par) /| \tx{x}|-> rnk,par /| \tx{p} \neq \tx{x}\}$
    p0 = find(p); 
// $\{\exists \gamma',rt.~ \m{findS}(\gamma, \tx{p}, \gamma') /| \m{uf\_root}(\gamma',\tx{p},rt) /| \tx{p0} = rt\}$
// $\{\tx{x} \in vertices(\gamma') /| \gamma'(\tx{x}) = (rnk, par) /| \tx{x}|-> rnk,par\}$
// $\{... /| \m{!reachable}(\gamma', rt, \tx{x}) /| ... )\}$   
    p = p0; // what's the point of this step?
// $\{... /| \tx{p} = rt\}$   
// $\label{code:findbeforexparent}\searrow$
    x -> parent = p;
// $\label{code:findafterxparent}\swarrow\{\exists \gamma''.~ \gamma'' = \m{ggrp}(\gamma',\tx{x},rt) /|...\}$
// $\{\m{findS}(\gamma, \tx{p}, \gamma'') /| \m{uf\_root}(\gamma'',\tx{p},rt) /| \tx{x} |-> rnk,rt\}$
  } // $\label{code:findafterreassign}\{\exists \gamma',rt.~ \m{findS}(\gamma, \tx{x}, \gamma') /| \m{uf\_root}(\gamma',\tx{x},rt)\}$   
  return p; // $\{\tx{p} = rt\}$
}
\end{lstlisting}

% need to have some notion of findS down below. probably want to define it at a higher level than what I've reproduced below?

%  Definition uf_root (pg: PreGraph Vertex Edge) (x root: Vertex) : Prop := reachable pg x root /\ (forall y, reachable pg root y -> root = y).

%  Definition uf_equiv (g1 g2: PreGraph Vertex Edge) : Prop :=
%    (forall x, vvalid g1 x <-> vvalid g2 x) /\ (forall x r1 r2, uf_root g1 x r1 -> uf_root g2 x r2 -> r1 = r2).

%  Definition rank_unchanged (g1 g2: Graph) : Prop := forall v, vvalid g1 v -> vvalid g2 v -> vlabel g1 v = vlabel g2 v.

%  Definition findS (g1: Graph) (x: Vertex) (g2: Graph) :=
%    (predicate_partialgraph g1 (fun n => ~ reachable g1 x n)) ~=~ (predicate_partialgraph g2 (fun n => ~ reachable g1 x n)) /\ uf_equiv g1 g2 /\ rank_unchanged g1 g2. (**) why isn't uf_equiv enough?



\vspace{-0.4em}
\caption{Clight code and proof sketch for find}
\label{fig:find}
\vspace{-1em}
\end{figure}

As an initial demonstration of our techniques, we show the decorated code of the
\li{find} function from the classic disjoint-set data structure in
Figure~\ref{fig:find}. The \li{find} function returns the root
(ultimate parent) of a \li{Node x}. A node is a root when its parent
pointer points to itself (line~\ref{code:ufif});
other than such self-loops at roots, the structure is acyclic.
For good amortised performance, \li{find} also performs path
compression (line~\ref{code:ufpathcompress}).
At first, \li{find} appears rather trivial since it only has about 5 lines
of code and a \li{Node} has only a single outgoing pointer.
In actual fact, the rather subtle nature of path compression and the
implicit sharing inherent in parent-pointers make the disjoint-set data
structure very difficult to reason about. Indeed, the first pen-and-paper
verification in separation logic required 20 pages~\cite{neelthesis}.

%\marginpar{\tiny \color{blue} Would you consider $V_\gamma$ and $E_\gamma$?}
We use the following conventions in our invariants.  Pure predicates are written in
\textit{italic}.  We write~$\gamma$ to
mean a ``mathematical'' (or~``pure'') graph: roughly, a set of
labeled vertices $V(\gamma)$ and edges $E(\gamma)$.
When $\m{v} \in V(\gamma)$, we write $\gamma(\m{v}) = (r,p)$ to state that vertex $\m{v}$ has
label $r$ and parent vertex $p$ ($r$ stores the ``rank'' of a node; it is ignored
in \li{find}).  We detail mathematical graphs in~\S\ref{sec:mathgraph}.

Spatial predicates are written in \textsf{sans-serif}.
Each node $\m{v} \in V(\gamma)$ is
represented in the heap by $\m{v} \mapsto \gamma(\m{v})$, %, \emph{i.e.}  = (r,p)$},
where we use the usual pen-and-paper
trick of writing \emph{e.g.} $\m{v} \mapsto r,p$ to mean
\mbox{$\big(\m{v} |-> r\big) * \big((\m{v} + \m{sizeof}(\li{unsigned int})) |-> p\big)$} in the character-addressed C memory model.
%\marginpar{\tiny \color{blue}Bigstar coming in unexplained.}
The whole graph (disjoint-set forest) is represented by
$\p{uf\_graph}(\gamma)$, essentially the
iterated separating conjunction of the representations
of each vertex $\m{v} \in V(\gamma)$.
We detail spatial graphs in~\S\ref{sec:spacegraph}.

The invariants at
each program point are natural despite only minor tidying from our machine-checked
proof.  We also enjoy good separation between the spatial predicates and pure
predicates.  All of this is despite verifying real C code, which entails quite a number
of grungy details. As one example, we will examine some grunginess that occurs in the
verification of line~\ref{code:ufif} shortly.

The precondition on line~\ref{code:findstart} says that we have a disjoint-set forest representing the abstract graph~$\gamma$, and that \li{x} is a valid vertex in~$\gamma$.  The postcondition is on line~\ref{code:findend}: the heap contains a new union-find graph~$\gamma''$, and \li{find} returns the node \m{rt}.
We specify that $\m{rt}$
is the root (ultimate parent) of~\li{x} with the pure relation $\m{uf\_root}$.
The relation $\m{findS}$, which conservatively approximates the action of path
compression, relates the final graph~$\gamma''$ to the original graph~$\gamma$.
The formal definitions for the concepts used in $\m{uf\_root}$ and $\m{findS}$
will be given in~\S\ref{sec:mathgraph}, but briefly:
$\reachable{x}{\gamma}{\star}{}{y}$ expresses that~$y$ is reachable
from~$x$ in~$\gamma$, $\gamma \smallsetminus S$ expresses the result of removing the
vertices in set~$S$ from graph~$\gamma$, and~$\gamma_1 \cong \gamma_2$ expresses that the two
graphs are structurally equivalent.

Most of the verification is straightforward.  To aid human readability, we use the color {\color{red}red} to indicate the changes in the invariants line-by-line.
Each individual line of code
(\ref{code:findram1}, \ref{code:ufif}, \ref{code:ufreccall}, \ref{code:ufpathcompress},
and \ref{code:ufreturn}) is bracketed with invariants leading to relatively easy proofs
of the command (ignoring the symbols~$\searrow \text{, } \ramify(i) \text{, and }
\swarrow$ until~\S\ref{sec:localblocks}).
In addition to improving human comprehensibility, this
also aids mechanical comprehensibility---that is, straightforward
invariants help the underlying verification engine (VST, in our case) handle many grungy
details for us either automatically or with a little human guidance via suitable lemmas.

The pointer comparison in line~\ref{code:ufif} is an example of where
such lemmas are necessary.
Formally, pointer (in-)equality comparison in C is only defined under somewhat delicate
circumstances\footnote{\label{footnote:pointereq}To summarize: it is a mess. Specifically,
whenever (1) \li{x} and \li{p} are both null; or when (2) one of them is null and the
other has offset between~0 and the size of the memory block into which it is pointing;
or when (3) if \li{x} and \li{p} are from the same memory block, then both of their
offsets are between~0 and the size of that block; or when (4) \li{x} and \li{p} are
not in the same memory block and both have offsets between 0 and the size of their
respective memory blocks \emph{minus one}.}.  VST could prove the definedness of the
pointer comparison automatically if we knew
\mbox{$(\li{x} |-> \_) * (\li{p} |-> \_) * \top$}, but
unfortunately this does not follow from line~\ref{code:aftparentfind} since,
when \li{x} is a root, self-loop gives us $\li{x}=\li{p}$ and
%then, because of the way $*$ works,
$\li{x} |-> \_ * \li{x} |-> \_ |- \bot$.  Accordingly, we must prove a
simple lemma that states that when $\p{uf\_graph}(\gamma) /| \li{x}
\in V(\gamma) /| \li{p} \in V(\gamma)$,
the pointer comparison is defined in C.

\subsection{Localization Blocks}
\label{sec:localblocks}

It is time to explain the non-obvious jumps in reasoning bracketed by the
symbols $\searrow \text{, } \ramify(i)$, and $\swarrow$
(lines~\ref{code:ufbefram1}--\ref{code:ufaftram1} and~\ref{code:ufbefram2}--\ref{code:ufaftram2}). We call such bracketed sets of lines ``localization blocks''.
As explained above, the verification of the command itself is entirely
straightforward given its immediate neighbors
(lines~\ref{code:befparentfind}--\ref{code:aftparentfind}
and~\ref{code:findbeforexparent}--\ref{code:findafterxparent}).
What is not so straightforward is how \emph{e.g.} line~\ref{code:ufbefram1}
leads to line~\ref{code:befparentfind} or how line~\ref{code:aftparentfind}
leads to line~\ref{code:ufaftram1}.
Intuitively, a localization block allows us to zoom in from a larger
``global'' context to a smaller ``local'' one, and, after verifying some commands
locally and arriving at a local postcondition, to zoom back out to the global
context. The $\searrow$ and $\swarrow$ symbols formally indicate an application
of the \textsc{Localize} rule (Equation \ref{eq:localize} from page~\pageref{eq:localize}).

%\marginpar{\tiny \color{blue} repetition}
Recall that \textsc{Localize} connects some ``global'' pre- and postconditions $G_1$ and $G_2$
with some ``local'' pre- and postconditions $L_1$ and $L_2$ using a ramification
frame $R$.
The lines adjacent to
the~$\searrow$ and~$\swarrow$ symbols specify~$G_1$ (\emph{e.g.} line~\ref{code:ufbefram1}),
$L_1$~(line~\ref{code:befparentfind}),
$L_2$~(line~\ref{code:aftparentfind}), and~$G_2$~(line~\ref{code:ufaftram1}).
Note that the \textsc{Localize} rule expects quantifiers in the postconditions $L_2$ and $G_2$,
but lines \ref{code:aftparentfind}--\ref{code:ufaftram1} do not have any.  We can overcome
this mismatch since $\forall P.~ (P -|- \exists x : \m{unit}.P)$ for any $x$ not free in $P$. %we will elide until {\color{red}\S\ref{whenever}}.


What is not specified explicitly is the ramification frame $R$, which must satisfy
the entailments $G_1 |- L_1 * R$ and (again eliding the quantifier) $R |- L_2 --* G_2$.
Here things are a little delicate.  To give intuition, it is \textbf{almost} enough to choose
$R \defeq L_2 --* G_2$, which makes the second entailment trivial, and reduces the problem
to three remaining checks.

The first check is the \emph{ramification entailment}: $G_1 |- L_1 * (L_2 --* G_2)$.
Informally, this asks whether replacing $L_1$ with $L_2$ inside $G_1$ yields $G_2$.
This ramification entailment is a nontrivial proof obligation both because
we must prove $L_1$ is located inside $G_1$, and because we must prove that
``replacing'' $L_1$ with $L_2$ yields $G_2$.
When we want to refer to the ramification entailment of a localization block in
subsequent text we use the symbol $\ramify(i)$ to connect it to relevant equation
numbers.  Accordingly, $\ramify(\ref{findram1})$ refers to the ramification entailment
associated with lines~\ref{code:ufbefram1}--\ref{code:ufaftram1}:
\begin{equation}
\label{findram1}
\p{graph}(\gamma) /| x \in V(\gamma) \quad |- \quad x |-> \gamma(x) * \big(x |-> \gamma(x) --* \p{graph}(\gamma)\big)
%\p{graph}(\gamma) /| x \in V(\gamma) |- x |-> \gamma(x) * (\overbrace{x |-> \gamma(x)}^{L_2'} --* \overbrace{\p{graph}(\gamma)}^{G_2'})
\end{equation}
Here we have isolated the key \textbf{spatial} parts of the invariants on lines~\ref{code:ufbefram1}--\ref{code:ufaftram1}.  Notice that this lemma is stated for any whole-graph predicate $\p{graph}(\gamma)$, and not merely for the special class of ``union-find graphs'' $\p{uf\_graph}(\gamma)$ (that \emph{e.g.} have only one outgoing edge per node).  That is useful because we use the same lemma to prove similar goals in all of our examples.
Indeed, this ``unchanged vertex'' ramification entailment is used whenever we need to read from a vertex in a graph.  In~\S\ref{sec:spacegraph} we describe other generic and reusable lemmas that prove other ramification entailments.

The second check is the Hoare proof of the local
change from $L_1$ to $L_2$.  Since lines~\ref{code:befparentfind} and~\ref{code:aftparentfind}
are straightforward---indeed, the point of \textsc{Localize} is to make them
so---verifying line~\ref{code:findram1} is easy.

The third check is the side condition on the modified variables.  Here we have an irritating
problem: the free variables of $R \defeq L_2 --* G_2$ are \textbf{not} disjoint from
the local variables modified by $c$.  Inspection of lines~\ref{code:findram1}--\ref{code:ufaftram1} shows that the program variable \li{p} \textbf{is} modified by~$c$,
and \textbf{is} free in both~$L_2$ and~$G_2$.
This issue is fundamental: the whole point of verifying
a read is to know something about the value that has been read.
Accordingly, our proof fails when we choose $R \defeq (L_2 --* G_2)$.
We will address this problem
head-on in~\S\ref{sec:freevars}, but for now let us content ourselves with knowing
that other than this problem, \textsc{Localize} lets us verify
lines \ref{code:ufbefram1}--\ref{code:ufaftram1}.

%Accordingly,
%although \textsc{Ramify} is sound, it is not strong enough to verify
%lines~\ref{code:ufbefram1}--\ref{code:ufaftram1}.

%: the program variable \li{p} \textbf{is} modified on line~\ref{findram1}, and---since
%the whole point of a read is that we know something about its contents afterwards---\li{p}
%is mentioned in both the local~$L_2$ and global~$G_2$ postconditions.

%As it turns out, $R$ can be selected automatically in a way that makes the second entailment
%a tautology.   which means that the remaining entailment can be

%As it turns out, the second of these two entailments will end up being a tautology.

%We can easily (and as we will see in \S\ref{sec:development}, automatically) adjust for this by setting the .

\hide{
one by using five predicates.  To guide intuition in \S\ref{sec:localblocks}, we will
provisionally use a simpler variant of \textsc{Localize} called \textsc{Ramify}:

\begin{equation}
\label{eq:ramify}
\inferrule[Ramify]
{\{ L_1 \} ~ c ~ \{ L_2 \} \\
G_1 |- L_1 * (L_2 --* G_2)}
{\{ G_1 \} ~ c ~ \{ G_2 \}} \mathit{freevars}(L_2 --* G_2) \cap \MV(c) = \emptyset
\end{equation}

The \emph{ramification entailment} $G_1 |- L_1 * (L_2 --* G_2)$ captures two key ideas:
first, that $L_1$ is spatially contained within $G_1$; and second, that
\emph{replacing} $L_1$ with $L_2$ inside $G_1$ yields $G_2$. \textsc{Ramify} essentially
combines the two entailments in \textsc{Localize} into one by forcing the choice of
$R \defeq L_2 --* G_2$; it also does not support the full power of existential
quantifiers (quantifiers can appear \emph{within} $L_2$ and $G_2$, but
\emph{the witnesses of such quantifiers cannot be related to each other}).
Although \textsc{Ramify} is sound\footnote{Use \textsc{Localize} with
$P -|- \exists x : \m{unit}.P$ for any $x$ not free in $P$.},
we shall soon see that these drawbacks are serious.

Using either \textsc{Localize} or \textsc{Ramify},

Unfortunately, we still have a problem: the side condition above requires that
}

The second localization block (lines~\ref{code:ufbefram2}--\ref{code:ufaftram2}) is both easier and harder than the first.  It is easier because line~\ref{code:ufpathcompress} does not modify any local program variables, so the side condition is trivially satisfied.  Moreover, although line~\ref{code:ufaftram2} does contain an existential, line~\ref{code:findafterxparent} does not, and so there is no need to ``link'' the two associated witnesses.  We will discuss this issue
in more detail in \S\ref{sec:linkedex}, but for now it is enough to
choose $R \defeq L_2 --* G_2$ \note{exactly written} in lines~\ref{code:findafterxparent}
(for $L_2$) and~\ref{code:ufaftram2} (for $G_2$).

On the other hand, the second localization block is harder than the first because there is more going on spatially. $\ramify(\ref{findram2})$ expresses an update to a single node of our graph:
\begin{equation}
\label{findram2}
\inferrule
{x \in V(\gamma') \; \qquad \gamma'' = [x -> (r,rt)]\gamma'}
{\p{graph}(\gamma') \; |- \; x |-> \gamma'(x) * \big(x |-> \gamma''(x) --* \p{graph}(\gamma'')\big)}
\end{equation}
Here we abuse notation a little bit.  The conclusion of the ``rule'' (actually, lemma) is exactly right and appropriately generic, so spatial ramification lemmas of the kind given in \S\ref{sec:spacegraph} can handle the dirty spatial work for us.  However, the second premise uses a notation for ``mathematical graph node update'' that is customized for union-find graphs, since most graphs have more than a rank and single outgoing edge.  More seriously, updating a mathematical graph cannot be done willy-nilly; it is only defined when the properties that restrict the mathematical structure of $\gamma$ are preserved. For example, in the case of union-find graphs,
the graph must be acyclic (other than at roots).
% , as we will discuss in \S\ref{sec:mathgraph},
%\marginpar{\tiny \color{blue}Too early}
In Coq, these properties are carried around via dependent types, as will be explained in \S\ref{sec:mathinfra}.

In our proof of \li{find}, the bulk of the example-specific effort (as opposed to generic lemmas we reuse in other examples) is showing that this mathematical update can be done properly, \emph{i.e.} from
%\marginpar{\tiny \color{blue} Need to come to some decision re: the typesetting.}
\[
\li{x} \in V(\gamma) /| \gamma(\li{x}) = (r,\m{pa}) \quad \text{ and } \quad
\li{x} \neq \m{pa} /|
\m{findS}(\gamma,\m{pa},\gamma') /| \m{uf\_root}(\gamma',\m{pa},\m{rt})
\]
we can prove
\[
\exists \gamma''.~ \gamma'' = [x -> (r,rt)]\gamma' /| \m{uf\_root}(\gamma'',\li{x},\m{rt}) /| \m{findS}(\gamma,\li{x},\gamma'')
\]


This lemma captures the essence of both finding the root and doing path compression: after compressing your parent and finding its root, you can path compress yourself by rerouting your own parent pointer to your (soon-to-be former) parent's root.  The existential in the goal is nontrivial
exactly because the update $[x -> (r,rt)]\gamma'$ is not always kosher.
This lemma requires some effort to prove, but is completely isolated from the grungy details of C.

With the second localization block complete, the remainder of the verification is straightforward.



\hide{
(Assume until~\S\ref{sec:existentials} that $L_2$ and $G_2$ do not contain ``fancy'' existentials; in this case the explicit existentials in \textsc{Localize} can be ,
which effectively makes them trivial.)  What then remains is to choose a predicate $R$ and prove the two entailments in \textsc{Localize}.  If one ignores the side condition about modified local variables

Turning to the body of the verification (lines~\ref{code:inmark}--\ref{code:outmark}), readers may already have noticed our new notation: blocks of proof sketch bracketed

, such as lines~\ref{code:beforerootmark}--\ref{code:afterrootmark}.  We call a bracketed set of lines like this a ``localization block''; localization blocks were inspired by our new \li{localize} $\searrow$ and \li{unlocalize} $\swarrow$ tactics in Floyd (\S\ref{sec:vst}).


In lines~\ref{code:beforerootmark}--\ref{code:afterrootmark}, imagine unfolding the \p{graph} predicate in line~\ref{code:globalbeforerootmark} using equation \eqref{eqn:bigraphintrofoldunfold} and then zooming in to the root node \li{x} for lines~\ref{code:beforerootmark}--\ref{code:afterrootmark}, before zooming back out in line~\ref{code:globalafterrootmark}.

To define localization blocks formally we need to first understand the \infrulestyle{Frame} and \infrulestyle{Ramify} rules.
}

\hide{



In lines~\ref{code:beforerootmark}--\ref{code:afterrootmark}, imagine unfolding the \p{graph} predicate in line~\ref{code:globalbeforerootmark} using equation \eqref{eqn:bigraphintrofoldunfold} and then zooming in to the root node \li{x} for lines~\ref{code:beforerootmark}--\ref{code:afterrootmark}, before zooming back out in line~\ref{code:globalafterrootmark}.

To define localization blocks formally we need to first understand the \infrulestyle{Frame} and \infrulestyle{Ramify} rules.

\subsection{Frames and ramifications are localizations}
%\label{sec:localizations}

The key rule of separation logic is \infrulestyle{Frame}~\cite{rey02}:
\[
\infrule{Frame}
{\{ P \} ~ c ~ \{Q \}}
{\{P * F \} ~ c ~ \{ Q * F \}}
{\begin{array}{c}F \textrm{ ignores } \MV(c) \end{array}} \qquad \qquad
\vspace{-0.75ex}
\]
The reason \infrulestyle{Frame} is so important is because it enables local verifications.  That is, a verifier can focus on the portions of the heap that are relevant to command $c$ and ``frame away'' the rest.  The side condition ``$F \textrm{ ignores } \MV(c)$'' relates to modified program variables and will be discussed in \S\ref{sec:freevars}.

Hobor and Villard observed that \infrulestyle{Frame} is bit rigid because it forces verifiers to split program assertions into syntactically $*$-separated parts~\cite{hobor:ramification}.  This rigidity is particularly troublesome when verifying programs that manipulate data structures with intrinsic unspecified sharing such as DAGs and graphs.  Hobor and Villard proposed the \infrulestyle{Ramify} rule to circumvent this rigidity:
\vspace{-1.5ex}
\[
\infrule{Ramify}
{\{L_1\} ~ c ~ \{L_2\} \\ G_1 |- L_1 * (L_2--* G_2)}
{\{G_1\} ~ c ~ \{G_2\}}
{\begin{array}{c}(L_2 --* G_2) \\ \textrm{ignores} \\ \MV(c) \end{array}} \qquad \qquad \qquad
%{$\begin{array}{l}\m{fv}(Q --* R') \cap \null \\ \m{modif}(c) = \emptyset\end{array}$} \qquad \qquad \qquad
\vspace{-1.5ex}
\]

}


\hide{


%\vspace*{-0.75ex}
\subsection{Marking a graph in VST}
\label{sec:vstgraphmark}
%\vspace*{-0.75ex}


In Figure~\ref{fig:markgraph} we put the code and proof sketch of the classic \li{mark} algorithm that visits and colors every reachable node in a heap-represented graph.  The \li{mark} algorithm is good to start with because it is complex enough to require some care to verify while being simple enough that the invariants are straightforward.  In \S\ref{sec:application} we will discuss more complex examples that \emph{e.g.} add/change/remove edges and/or vertices.

The code in Figure~\ref{fig:markgraph} is written in Clight~\cite{blazy:clight}, an input language to the CompCert certified compiler~\cite{leroy:compcert}, which compiles our code exactly as written.
The paper-format verification sketch for \li{mark} in Figure~\ref{fig:markgraph} is extracted from
a Floyd proof in VST, with only minor cleanup to aid the presentation.
Accordingly, there is an unbroken certified chain from our specification of \li{mark} all the way to the assembly code.  In \S\ref{sec:hipsleek} we use HIP/SLEEK~\cite{chin:hipsleek} to verify a Java version of \li{mark}; the program invariants generated by HIP/SLEEK are slightly different due to HIP/SLEEK's heavier automation.
% but the overall structure is the same.

The specification we certify (lines \ref{code:markstart} and \ref{code:markend}) is
\vspace*{-1ex}
\[
\{\p{graph}(\li{x},\gamma)\}~\li{mark(x)}~\{\exists \gamma'.~ \p{graph}(\li{x},\gamma') /| \m{mark}(\gamma, \li{x}, \gamma')\}
\vspace*{-1ex}
\]
The specification is for full functional correctness, stated using \emph{mathematical} graphs~$\gamma$; until \S\ref{sec:mathgraph} consider $\gamma$ to be a function that maps a vertex $v \in V$ to triples $(m,l,r)$, where $m$ is a ``mark'' bit (0 or 1) and $\{l,r\} \subseteq V \uplus \{0\}$ are the neighbors of $v$.
The \emph{spatial} \p{graph} predicate describes how the mathematical graph $\gamma$ is implemented in the heap.  Until~\S\ref{sec:spacegraph} it is enough to know that \p{graph} satisfies the fold/unfold relationship in
equation~\eqref{eqn:bigraphintrofoldunfold}, located just under the code in Figure~\ref{fig:markgraph}.

This fold/unfold relationship deserves attention.
First, as we explain in~\S\ref{sec:fixpointfail}, it is probably a mistake to write~\eqref{eqn:bigraphintrofoldunfold} as a definition using $\stackrel{\Delta}{=}$ rather than as a biimplication using $<=>$.  Second, \eqref{eqn:bigraphintrofoldunfold} uses the ``overlapping conjunction'' $\ocon$ of separation logic; informally $P ** Q$ means that $P$ and $Q$ may overlap in the heap (\emph{e.g.}, nodes in the left subgraph can also be in the right subgraph or even be the root $x$).  The presence of the unspecified sharing indicated by the $\ocon$ connective is exactly why graph-manipulating algorithms are so hard to verify (\emph{e.g.}, it is hard to apply the \infrulestyle{Frame} rule).  The standard semantics of the separation logic connectives used in this paper are in Figure~\ref{fig:seplogsem}.
Third, \eqref{eqn:bigraphintrofoldunfold} illustrates how industrial-strength settings complicate verification.  Lines~\mbox{\ref{code:nodedefstart}--\ref{code:nodedefend}} define the data type \li{Node} used by \li{mark}.  The \li{_Alignas($n$)} directives tell CompCert to align fields on $n$-byte boundaries.  As explained in~\S\ref{sec:goodgraph}, this alignment is necessary in C-like memory models to prove fold-unfold \eqref{eqn:bigraphintrofoldunfold}, which is why \eqref{eqn:bigraphintrofoldunfold} includes an alignment restriction $x~\mathsf{mod}~16 = 0$ and an existentially-quantified ``blank'' second field for the root $x \mapsto m,-,l,r$.
%{\color{magenta}(In our Floyd proofs the alignment restriction and blank second field are nicely hidden ``behind the scenes''.)}

Notice that the postcondition of \li{mark} is specified \emph{relationally}, \emph{i.e.} $\{\exists \gamma'.~ \p{graph}(\li{x},\gamma') /| \m{mark}(\gamma, \li{x}, \gamma')\}$ instead of \emph{functionally}, \emph{i.e.} $\{\p{graph}\big(\li{x},\m{mark}(\gamma, \li{x})\big)\}$. In the first case $\m{mark}$ is a relation that specifies that~$\gamma'$ is the result of correctly marking~$\gamma$ from~\li{x}, whereas in the second $\m{mark}$ is a function that \textbf{computes} the result of marking~$\gamma$ from~\li{x}. For both theoretical and practical reasons a relational approach is better.
Theoretically, relations are preferable because they are more general.  For example, relations allow ``inputs'' to have no ``outputs'' (\emph{i.e.} be partial) or alternatively have many outputs (\emph{i.e.} be nondeterministic).  Our graph \li{copy} algorithm is specified nondeterministically to avoid specifying how \li{malloc} allocates fresh blocks of memory.  Relations are also preferable to functions because they are more compositional.
We take advantage of compositionality by using $\m{mark}(\gamma,x,\gamma') /| \ldots$ to specify both our ``spanning tree'' and ``graph copy'' algorithms in~\S\ref{sec:application}, which also mark nodes while carrying out their primary tasks.

\begin{figure}
\[
\begin{array}{lcl}
\sigma |= P * Q & \defeq & \exists \sigma_1, \sigma_2.~ \sigma_1 \oplus \sigma_2 = \sigma /| \null \\ && ~~ (\sigma_1 |= P) /| (\sigma_2 |= 2)\\
[-2pt]
\sigma |= P ** Q & \defeq & \exists \sigma_1, \sigma_2, \sigma_3.~ \sigma_1 \oplus \sigma_2 \oplus \sigma_3 = \sigma /| \null \\ && ~~ (\sigma_1 \oplus \sigma_2 |= P) /| (\sigma_2 \oplus \sigma_3 |= Q) \\
[-2pt]
\sigma |= P --* Q & \defeq & \forall \sigma_1, \sigma_2.~ \sigma_1 \oplus \sigma = \sigma_2 /| \null \\ && ~~
(\sigma_1 |= P) => (\sigma_2 |= Q) \\
[-2pt]
\sigma |= P --o Q & \defeq & \exists \sigma_1, \sigma_2.~ \sigma_1 \oplus \sigma = \sigma_2 /| \null \\ && ~~
(\sigma_1 |= P) /| (\sigma_2 |= Q)
\end{array}
\]
\vspace*{-1.5em}
\caption{Separation logic connectives; $\oplus$ is the join operation on states, usually some kind of disjoint union on heaps}
\label{fig:seplogsem}
\vspace*{-1em}
\end{figure}

Practically, it is painful to define computational functions over graphs in a proof assistant like Coq, and portions of this pain are overkill.  For example, Coq requires that all functions terminate, a nontrivial proof obligation over cyclic structures like graphs, but our verification of \li{mark} is only for partial correctness.  Defining relations is much easier because \emph{e.g.} one can use quantifiers and does not have to prove termination.
The $\m{mark}$ and $\m{mark1}$ relations we use are defined straightforwardly at the bottom of Figure~\ref{fig:markgraph}.

Turning to the body of the verification (lines~\ref{code:inmark}--\ref{code:outmark}), readers may already have noticed our new notation: blocks of proof sketch bracketed by the symbols $\searrow$ and $\swarrow$, such as lines~\ref{code:beforerootmark}--\ref{code:afterrootmark}.  We call a bracketed set of lines like this a ``localization block''; localization blocks were inspired by our new \li{localize} $\searrow$ and \li{unlocalize} $\swarrow$ tactics in Floyd (\S\ref{sec:vst}).
The intuitive idea is that we zoom in from a larger ``global'' context to a smaller ``local'' one.  After verifying some commands locally to arrive at a local postcondition, we zoom back out to the global context.  Although we do not do so in Figure~\ref{fig:markgraph}, localization blocks can safely nest.
} % end hide
