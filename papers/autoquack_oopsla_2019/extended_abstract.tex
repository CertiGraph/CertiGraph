\documentclass[sigplan, anonymous=false, review=false]{acmart}

\setcopyright{none}

\usepackage{listings}
\usepackage{scalerel}
\usepackage[export]{adjustbox} % for subfigures
\usepackage{courier}           % for the courier font (optional)
\makeatletter
\newlength{\@mli}
\newcommand\hide[1]{}
\newcommand{\mli}[1]{%
  \settowidth{\@mli}{\lstinline/#1/}
  \hspace{-.5ex}\begin{minipage}[t]{\@mli}\lstinline/#1/\end{minipage}}
\makeatother
\newcommand{\li}[1]{\ifmmode\mbox{\mli{#1}}\else\mbox{\lstinline/#1/}\fi}

\newcommand{\tx}[1]{\text{#1}}
\newcommand{\p}[1]{\ensuremath{\mathsf{#1}}} % predicate font
\newcommand{\m}[1]{\ensuremath{\mathit{#1}}} % math font
\newcommand{\hl}[1]{\colorbox{lightgray}{#1}}

\lstset{%
  language=C,
  morecomment=[n][{\color{red!80!black}}]{/*}{*/},
  morecomment=[l][{\color{red!80!black}}]{//},
  sensitive=true,
  mathescape=true,
  showlines=true,
  basicstyle=\normalfont\smaller\tt,
  keywordstyle=\color{blue},
  numbers=left,
  numberstyle=\tiny,
  numbersep=5pt,
  boxpos=t,
}

\newcommand{\infrulestyle}[1]{\textsc{#1}}
\newcommand{\infrule}[4]{\inferrule*[lab=\infrulestyle{#1},right=$\mathrlap{#4}$]{#2}{#3}}
\usepackage{semantic}
\newcommand{\scon}{\mathbin{\star}}
\renewcommand{\scon}{\mathbin{\ast}} % Comment out this line to return to 5-pt star
\renewcommand{\bigstar}{\raisebox{-0.24em}{{\scaleobj{2.5}{\scon}}}}
\newcommand{\ocon}{%
  \mathbin{\mbox{$\mathrlap{\cup}\hspace*{.15em}
      \raisebox{.01em}[0ex][0ex]{$\scon$}$\hspace*{.07em}}}}
\newcommand{\wand}{%
 \mathrel{\mbox{$\hspace*{-0.03em}\mathord{-}\hspace*{-0.66em}
     \mathord{-}\hspace*{-0.36em}\mathord{\scon}$\hspace*{-0.005em}}}}
\newcommand{\septraction}{%
  \mathrel{\mbox{$\hspace*{-0.03em}\mathord{-}\hspace*{-0.66em}
  \mathord{-}\hspace*{-0.155em}\mathord{\ocircle\hspace*{-.66em}\scon}$\hspace*{0.05em}}}}
\newcommand{\magicwand}{\wand}
\newcommand{\defeq}{\mathbin{\stackrel{\Delta}{=}}}

\newcommand{\bigocon}{
  \raisebox{-0.3ex}{\resizebox{0.75em}{!}{$\scon$}} \hspace{-4.3ex} \bigcup}
  \newcommand{\MV}{\ensuremath{\mathsf{ModVar}}}
\newcommand{\medocon}{
  \raisebox{-0.3ex}{\resizebox{0.63em}{!}{$\scon$}} \hspace{-2.05ex} \bigcup}
\colorlet{red}{red!80!black}
\usepackage{mathpartir}        % for inferrule
\usepackage{mathtools}         % for mathrlap

%\input mathlig
  \mathlig{--*}{\mathrel{\magicwand}}
  \mathlig{--o}{\mathrel{\septraction}}
  \mathlig{|->}{\mathrel{\mapsto}} % tight points-to
  \mathlig{<=>}{\mathrel{\Leftrightarrow}} % equivalence of expressions
  \mathlig{==>}{\mathrel{\Rightarrow}} % meta implication
  \mathlig{-|-}{\mathrel{\mathrlap{\dashv} \hspace{5pt} \vdash}} % entails

  \mathlig{/|}{\mathbin{\wedge}} % additive conjunction
  \mathlig{|/}{\mathbin{\vee}} % additive disjunction
  \mathlig{|-}{\mathrel{\vdash}} % entails
  \mathlig{|=}{\vDash} % models

  \mathlig{**}{\mathbin{\ocon}}
  \mathlig{*}{\mathbin{\scon}}

\usepackage{tikz}
\usetikzlibrary{arrows.meta, positioning, decorations.pathmorphing, fit}

\acmConference[APLAS 2019]{Asian Symposium on Programming Languages
  and Systems}{December 2017}{Bali, Indonesia}
\acmYear{2019}
\copyrightyear{2019}
\begin{document}
\title[Certifying Graph-Manipulating Programs]
{Certifying Graph-Manipulating C Programs via~Localizations within Data Structures}
\subtitle{Extended Abstract}

\author[S. Wang]{Shengyi Wang}
\affiliation{\institution{School of Computing, National University of Singapore}}

\author[Q. Cao]{Qinxiang Cao}
\affiliation{\institution{John Hopcroft Center, Shanghai Jiao Tong University}}
% \email{caoqinxiang@gmail.com}

\author[A. Mohan]{Anshuman Mohan}
\affiliation{\institution{Yale-NUS College and School of Computing, National University of Singapore}}
% \email{anshumanmohan@live.com}

\author[A. Hobor]{Aquinas Hobor}
\affiliation{\institution{Yale-NUS College and School of Computing, National University of Singapore}}

\maketitle

We develop a comprehensive set of techniques to mechanically verify 
executable C programs that manipulate heap-represented graphs. 
The critical problem in this area is that
separation logic does not work its usual miracles: graphs 
exhibit \emph{deep intrinsic sharing}, causing the \infrulestyle{Frame}
rule to fail. 

We construct a modular and general setup for 
reasoning about abstract mathematical graphs~(\S\ref{sec:mathgraph}), and provide
a key extension to separation logic and an improved way of representing 
abstract graphs concretely in the heap~(\S\ref{sec:spacegraph}). 
Our techniques are general, powerful, and scalable: 
we integrate them into the Verified Software Toolchain and produce Coq-checked 
proofs of correctness for seven classic graph-manipulating 
programs written in CompCert C. In the interest of space we discuss
only our flagship example, a garbage collector 
for the CertiCoq project~(\S\ref{sec:certigc}).
We provide brief comparisons with the state-of-the-art throughout.

\section{A Framework for Abstract Graph Theory}\label{sec:mathgraph}

  \begin{figure}[t]
  \begin{adjustbox}{scale=0.80}
  \centering
\begin{tikzpicture}
[->/.style={thick,arrows={-Stealth}},
-->/.style= {thick,arrows={-Stealth}, decorate, decoration={snake, amplitude=.4mm,segment length=2mm,post length=2mm}},
   realG/.style={shape=rectangle, rounded corners=4pt, draw, fill=red!30},
   propG/.style={shape=rectangle, rounded corners=4pt, draw}]
\node[realG] (PG) at (0, 0) {\small PreGraph};
\node[realG] (LG) [right=0.8 of PG] {\small LabeledGraph};
\node[realG] (GG) [right=2 of LG] {\small GeneralGraph};
\draw [double, ->] (PG) -- (LG) node [pos=0.5, above] {\small Label} ;
\draw [double, ->] (LG) -- (GG) node (SC) [pos=0.5, above, align=center]
{\small Soundness \\ \small Condition};
\node[propG] (Prop) [below=0.6 of SC] {\small Property};
\node[propG] (PropL) [below=0.4 of Prop] {\small Property Lemmas};
\node[propG] (PGL) [below=2 of PG, align=center] {\small PreGraph \\\small Lemmas};
\node[propG] (LGL) [below=2 of LG, align=center] {\small LabeledGraph \\\small Lemmas};
\node[propG] (GGL) [below=2 of GG, align=center] {\small GeneralGraph \\\small Lemmas};
\draw [double, ->] (PGL) to (LGL);
%% \draw [double, ->] (LGL) to (GGL);
\draw [->] (PG) to (PGL);
\draw [->] (Prop) to (PropL);
\draw [-->] (Prop) to (SC);
\coordinate [left=0.2 of LG.south] (LGs1);
\coordinate [left=0.2 of LGL.north] (LGLn1);
\draw [->] (LGs1) to (LGLn1);
\coordinate [right=0.2 of LG.south] (LGs2);
\coordinate [right=0.2 of LGL.north] (LGLn2);
\draw [->] (LGs2) |- (Prop);
\draw [double, ->] (LGLn2) |- (PropL);
\coordinate [right=0.2 of GG.south] (GGs);
\coordinate [left=0.2 of GGL.north] (GGLn1);
\coordinate [right=0.2 of GGL.north] (GGLn2);
\draw [double, ->] (PropL) -| (GGLn1);
\draw [->] (GGs) to (GGLn2);
\node [draw=red, thick, rectangle, dashed, fit=(Prop) (PropL)] {};
\node (legend1) [below right=0.2 and -0.3 of PGL] {\small Depends};
\coordinate[left=0.8 of legend1]  (l1);
\draw [->] (l1) to (legend1);
\node (legend2) [right=1 of legend1] {\small Inherits};
\coordinate[left=0.8 of legend2]  (l2);
\draw [double, ->] (l2) to (legend2);
\node (legend3) [right=1 of legend2] {\small Instantializes};
\coordinate[left=0.8 of legend3]  (l3);
\draw [-->] (l3) to (legend3);
\end{tikzpicture}
\end{adjustbox}
\caption{Structure of the Mathematical Graph Library}
\label{fig:graphlib}
\end{figure}

When verifying the functional correctness of graph algorithms, it is natural
to use graph theory to describe program behavior.
We develop a new graph theory library that allows reasoning about a 
wide variety of mathematical graphs. Although real-world 
programs manipulate customized graphs with very specific properties, 
those properties can often be described as a combination of 
simpler properties. Our library takes advantage of this
to allow the systematic construction of mathematical graphs from atomic 
parts that we provide out of the box. 
Figure~\ref{fig:graphlib} shows the architecture of our
mathematical graph library.

\hide{The most basic kind of graph is PreGraph,
out of which we build LabeledGraphs, and which in turn are used to
build GeneralGraphs. Each kind has some lemmas and also inherits the
lemmas of the previous kind. The dashed box represents a ``plugin''
system to attach custom properties.}

A PreGraph is a hextuple $(\mathcal{V}, \mathcal{E}, \mathtt{V}, \mathtt{E}, \mathtt{src}, \mathtt{dst})$.  Arguments $\mathcal{V}$ and $\mathcal{E}$ are the underlying
carrier types of vertices and edges.  $\mathtt{V}$ and $\mathtt{E}$ are predicates over
$\mathcal{V}$ and $\mathcal{E}$ that decide
of \emph{validity} in the graph.  
Finally, $\mathtt{src}$ and $\mathtt{dst}: \mathcal{E} -> \mathcal{V}$ map each edge to
its source and destination. 
We pre-define many fundamental graph concepts on PreGraphs,
including structures like \emph{path}, predicates
such as \emph{is\_cyclic} and \emph{reachable},
operations such as \emph{add\_vertex}
and \emph{remove\_edge}, and relations between PreGraphs such
as \emph{structurally\_identical} and \emph{subgraph}.
These general concepts, together with around 500 derived lemmas,
provide a solid foundation for more specific theorems needed in concrete
verifications.

A LabeledGraph augments a PreGraph with custom \emph{labels} 
of arbitrary type on vertices, edges, and on the graph as a whole. 
Many classic graph problems manipulate labels/weights, and so the need for 
LabeledGraph is unsurprising. A LabeledGraph smoothly inherits any properties 
and lemmas about its internal PreGraph.

PreGraphs and LabeledGraphs let us state
and prove lemmas that follow from their constructions. However, 
modeling a real program's behavior 
calls for more specificity.
For example, an algorithm may require that each vertex have exactly one out-edge
and cycles be forbidden.
Of course, these restrictions vary greatly by algorithm, so we do not
want to bake them into our core definitions.
We achieve this flexibility using GeneralGraphs, which augment
LabeledGraphs by adding arbitrarily complex ``soundness conditions'', indicated in
Figure~\ref{fig:graphlib} with a dashed border.
Further, a GeneralGraph can seamlessly behave like its internal
LabeledGraph or a PreGraph, thereby inheriting their lemmas.
This combination of specificity and generality
makes GeneralGraphs versatile. Moreover, we can
create complicated soundness conditions by composing smaller reusable pieces,
further enabling code sharing between algorithms.

\paragraph{Comparison with the state-of-the-art.} 
Prior works in the verification of graph-manipulating 
programs have also built libraries such as ours, but they tend
to suffer a few shortcomings: 
(1) they avoid linking
with a real executable language, thus avoiding many of the
subtleties that we have worked out; and 
(2) they are typically geared toward the verification of 
specific examples or classes of examples, meaning 
their reusability and modularity is limited.


\section{Graphs on the Heap}
\label{sec:spacegraph}

The new \textsc{Localize} rule
connects the ``local'' effect of a command~$c$, \emph{i.e.}
transforming~$L_1$ to~$L_2$, with its ``global'' effect, \emph{i.e.} from~$G_1$ to~$G_2$.
\textsc{Localize} is a more general version of the well-known \textsc{Frame} rule,
which does the same task in the simpler case when $G_i = L_i * F$
for some frame~$F$ that is untouched by~$c$.
\textsc{Localize} and \textsc{Frame} are coderivable, so we 
know our feet are firmly planted.

\begin{equation*}
\begin{array}{@{}l@{}}
\infrule{Localize}
{G_1 |- L_1 * R \\
\{ L_1 \} ~ c ~ \{ \exists x.~ L_2 \} \\
R |- \forall x.~ (L_2 --* G_2) }
{\{ G_1 \} ~ c ~ \{ \exists x.~ G_2 \}} {(\dagger)} \\
[3pt]
(\dagger)~ \mathit{freevars}(R) \cap \MV(c) = \emptyset
\end{array}
\end{equation*}
% \caption{The \infrulestyle{Localize} rule}
\label{eq:localize}

The standard approach to describing the layout of 
graphs on the heap is to rely on the Knaster-Tarski fixpoint  
and define \p{graph} as a recursive fixpoint. 
We show that this approach is flawed, and that a flat structure is better.

We start with the iterated separating conjunction or ``big star'', 
which is defined over lists and extended to sets. This
allows us to define a better \p{graph} predicate, as follows:

\vspace{-1em}
\[
\!\!\!\!\!\!\!\underset{\{l_1,\dots,l_n\}}{\bigstar P}\!\!\!\!\!\!\! ~~ \defeq ~~ P(l_1) *
  P(l_2) * \dots * P(l_n) 
  \]
\[
\underset{S}{\bigstar} P ~~ \defeq ~~ \exists L.~ (\p{NoDup}\ L) /| (\forall x.~ x\ \p{in}\ L <=> x \in S) /| \underset{L}{\bigstar}P
\]
\[\p{graph}(x, \gamma) \; \defeq \; \!\!\!\!\!\!\!\!\!\!\underset{\m{v} \in \mathit{reachable}(\gamma, x)}{\bigstar}\!\!\!\!\!\!\m{v}\mapsto\gamma(\m{v})\]

%\begin{wrapfigure}{r}{.5\textwidth}
\begin{figure}
\centering
%\beginpgfgraphicnamed{infrastructure}
\begin{adjustbox}{scale=0.80}
\begin{tikzpicture}[
->/.style={thick, arrows={-Stealth}},
ent/.style={shape=rectangle, rounded corners=4pt, draw, on grid},
entcol/.style={shape=rectangle, rounded corners=4pt, draw, on grid, fill=red!30},x=33pt]
\node[ent] (SM) at (0, 0) {\small Step-Indexed Model};
\node[ent] (DM) [right=4.9 of SM] {\small Direct Model};
\node[entcol] (CL) [above=1.5 of SM] {\small Core Logic Interface};
\node[ent] (SL) [above=1.5 of DM] {\small Supplementary Logic Interface};
\node[entcol] (LF) [above left=1 and 1.2 of CL] {\small Logic Facts};
\node[entcol] (RF) [above right=1 and 1.2 of CL] {\small Basic Ramification};
\node[entcol] (BF) [above=1 of LF] {{\scriptsize $\bigstar$} {\small Facts}};
\node[entcol] (BR) [above=1 of RF] {{\scriptsize $\bigstar$} {\small Ramification}};
\node[entcol] (GF) [above=1 of BF] {\small Graph Facts};
\node[entcol] (GR) [above=1 of BR] {\small Graph Ramification};
\node[ent] (SLF) [above=1 of SL] {\small Supplementary Logic Facts};
\node[ent] (SBF) [above=1 of SLF] {\small Supplementary {\scriptsize $\bigstar$} Facts};
\node[ent] (SGF) [above=1 of SBF] {\small Supplementary Graph Facts};
\draw [double, ->] (SM) to (CL);
\draw [double, ->] (SM) to [bend right=12] (SL);
\draw [double, ->] (DM) to [bend left=12] (CL);
\draw [double, ->] (DM) to (SL);
\draw [->] (CL) to (LF);
\draw [->] (CL) to (RF);
\draw [->] (CL) to (SL);
\draw [->] (SL) to (SLF);
\draw [->] (SLF) to (SBF);
\draw [->] (SBF) to (SGF);
\draw [->] (LF) to (RF);
\draw [->] (LF) to (BF);
\draw [->] (RF) to (SLF);
\draw [->] (RF) to (BR);
\draw [->] (BF) to (BR);
\draw [->] (BF) to (GF);
\draw [->] (GF) to (GR);
\draw [->] (GR) to (SGF);
\draw [->] (BR) to (GR);
\draw [->] (BR) to (SBF);
\node (legend1) [below right=0.2 and -1.2 of SM] {\small Dependence};
\coordinate[left=0.8 of legend1]  (l1);
\draw [->] (l1) to (legend1);
\node (legend2) [right=1 of legend1] {\small Instantialization Options};
\coordinate[left=0.8 of legend2]  (l2);
\draw [double, ->] (l2) to (legend2);
\end{tikzpicture}
\end{adjustbox}
%\endpgfgraphicnamed
%% \vspace{1ex}
\caption{Infrastructure of ramification library}\label{fig:infra}
\end{figure}
%\end{wrapfigure}

Armed with these two tools, we go on to develop a second library, which 
allows intuitive reasoning about graphs as they are manipulated on the heap.
Figure~\ref{fig:infra} shows the architecture of our spatial development.
Once again, we pay attention to modularity and reuse. 
We support both the step-indexed and direct heap models.
Out of the box, we
provide an extensive library of lemmas about separation logic, 
$\bigstar$, and \p{graph}. 
Further, we provide a number of lemmas that, in many common cases, 
handle the ``ramification entailments'' of \infrulestyle{Localize} for us, thus 
making its use much simpler. 

\paragraph{Comparison with the state-of-the-art.} 
\textsc{Localize} is an improvement in two key ways: 
it allows support for existential quantifiers in postconditions 
and smoother treatment of modified program variables.
Both of these prove helpful when reasoning about graphs on the heap.
We show that recursively defined graph predicates, which are 
in common use, are a poor idea. We provide a flat definition, 
along with a proof---with the amount of precision
required to convince Coq---that we still enjoy fold/unfold
with our flat definition.

\section{Certification of a Garbage Collector}
\label{sec:certigc}

The CertiCoq compiler's 400-line generationalgarbage collector, which is 
written in the spirit of the OCaml GC,
is realistic and sophisticated, though not
industrial-strength.
Because CertiCoq uses OCaml's representation of blocks and
values, the GC must support features such as
$12$ generations, variable-length memory objects, object fields that may be boxed
or unboxed and must be disambiguated at runtime, and pointers to places
outside the GC-managed heap. 

It is an important example for our purposes because it shows that our
techniques actually \emph{scale} well beyond short toy programs. The manipulations 
that the GC performs are delicate and numerous enough that a brute force
approach involving custom predicates is a tall task indeed. 
Happily, our two libraries and our \infrulestyle{Localize} rule save the day.

Interestingly, we find and fix two bugs in the GC source code, and also 
discover two places where the semantics of Clight is 
unable to specify an OCaml-style GC such as ours. 
Although the Clight code is undefined, we prove that 
CompCert's compiler transformations preserve the necessary invariants
for both operations. 

\paragraph{Comparison with the state-of-the-art.} 
Verifying GCs often involves customized predicates and onerous steps 
such as annotating x86 code by hand, writing the GC in Assembly, \emph{etc.}
Our framework allows us to work at a comfortably high level. 
Further, our GC ranks high in terms of complexity and real-world use:
it has many of the complications of the industrial-strength OCaml GC, 
and it fits into CertiCoq, a real verified compiler.

\end{document}
