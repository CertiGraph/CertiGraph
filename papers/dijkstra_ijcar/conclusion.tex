\paragraph{Our previous and future work.} We have long been interested in
the verification of graph-manipulating programs written in~C~\cite{hobor:ramification}.
More recently, we have fortified our techniques to handle realistic (CompCert~\cite{leroy:compcert})~C
to a machine-checked level of rigour~\cite{DBLP:journals/pacmpl/WangCMH19}.  The present result extends
our previous work in that it uses a adjacency matrices to represent the graph in
memory and involves non-trivial edge labels; both are new for us.

We are investigating techniques to increase the automation of such verifications.  Although
we benefit from some automation at the Hoare-logic level provided by the Verified Software
Toolchain~\cite{appel:programlogics}, building these proofs is still highly labor intensive.  We see potential
for automation in four areas: (A) the Hoare level; (B) the spatial level; (C) the mathematical level; and (D) the interface between the spatial and the mathematical levels.  Our ongoing work
on these challenges include improved tactics for VST for common cases we encounter in graph
algorithms (A); an expanded library of existing graph constructions and associated lemmas (B);
better lemmas about common mathematical graph patterns, investigations into reachability techniques
based on regular expressions over matrices and related semirings~\cite{backhouse,DBLP:journals/jacm/Tarjan81a,dolan2013fun,krishna2017go} (C); improved modularity in our constructions and
automation of common cases, \emph{e.g.} we often compare C pointers to heap-represented graph
nodes for equality, and due to the nature of our representations this equality check will be
well-defined in~C when the associated nodes are present in the mathematical graph.  The key
advantage of having end-to-end machine-checked examples such as the one we presented above is 
that they guide the automation efforts by providing precise goals that are known to be strong 
enough to complete the verification of real code.

\paragraph{Other verifications of Dijkstra.}
Due to its stature in the algorithm pantheon, Dijkstra has been verified several times before, although never to the best of our knowledge for~C: Moore and Zhang verified it in ACL2~\cite{Moore2005}, Mange and Kuhn verified it in Jahob~\cite{mange2007verifying}, and Klasen verified it in KeY~\cite{klasen2010verifying}.  Liu \emph{et al.} took an alternative SMT-based approach to verify a Java implementation.

\paragraph{Conclusion.}
Blah.
