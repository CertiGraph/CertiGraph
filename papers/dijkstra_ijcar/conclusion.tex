\paragraph{Our previous work.} We have long been interested in
the verification of graph-manipulating programs written in~C~\cite{hobor:ramification}.
More recently, we have fortified our techniques to handle realistic (CompCert~\cite{leroy:compcert})~C~to a machine-checked level of rigour~\cite{DBLP:journals/pacmpl/WangCMH19}.  Compared to our previously published 
examples, the present result includes a previously-untried adjacency matrices spatial graph representation as well as utilizing non-trivial edge labels between graph nodes. %, used to represent cost. %; both are new for us.

\paragraph{Ongoing and future work.}
We are investigating techniques to increase the automation of such verifications.  Although
we benefit from some automation at the Hoare-logic level provided by the Verified Software
Toolchain~\cite{appel:programlogics}, building these proofs is still highly labor intensive.  We see potential
for automation in four areas: (A) the Hoare level; (B) the spatial level; (C) the mathematical level; and (D) the interface between the spatial and the mathematical levels.  Our ongoing work
on these challenges include (A) improved tactics for VST for common cases we encounter in graph
algorithms; (B) an expanded library of existing graph constructions such as the adjacency-matrix representation used in this result, as well as associated lemmas;
(C) better lemmas about common mathematical graph patterns, investigations into reachability techniques
based on regular expressions over matrices and related semirings~\cite{backhouse,DBLP:journals/jacm/Tarjan81a,dolan2013fun,krishna2017go}; and (D) improved modularity in our constructions and
automation of common cases, \emph{e.g.} we often compare~C~pointers to heap-represented graph
nodes for equality, and due to the nature of our representations this equality check will be
well-defined in~C~when the associated nodes are present in the mathematical graph.  The key
advantage of having end-to-end machine-checked examples such as the one we presented above is 
that they guide the automation efforts by providing precise goals that are known to be strong 
enough to verify real code.

\paragraph{Other verifications of shortest-path.}
Due to its stature in the algorithm pantheon, Dijkstra has been verified several times before, although never to the best of our knowledge for~C:~Chen verified it in Mizar~\cite{chen2003dijkstra}, Gordan \emph{et al.} formalized the reachability property in HOL~\cite{gordon2003executing}, Moore and Zhang verified it in ACL2~\cite{Moore2005}, Mange and Kuhn verified it in Jahob~\cite{mange2007verifying}, and Klasen verified it in KeY~\cite{klasen2010verifying}.  Liu \emph{et al.} took an alternative SMT-based approach to verify a Java implementation of Dijkstra~\cite{6200101}.  In general the above work operates within
idealized formal environments and thus gloss over certain real-world issues, and so as best we 
can tell do not \emph{e.g.} handle the overflow issue we identified.

%However, because they operate entirely within ,
%these works inadvertently gloss over certain classes of issues
%that routinely crop up in real-world settings.


\paragraph{Shortest-path in algorithms textbooks.}  
We were not able to find a standard textbook that gives a robust, precise,
and full description of the overflow issue we describe in~\S\ref{sec:overflow}.
The landmark CLRS~\cite{clrs} does not address overflow, although---arguably---this is unnecessary in pseudocode.  Sedgewick's book on graph algorithms in~C~\cite{robert2002algorithms} likewise does not address overflow, and moreover gives~C~code
that contains exactly the bug we identify.  Skiena's \emph{Algorithm Design Manual} likewise 
glosses over the issue, both
in pseudocode and~C~\cite{DBLP:books/daglib/0022194}.
To its credit, Heineman \emph{et al.}'s \emph{Algorithms in a Nutshell}~\cite{heineman2008algorithms} vaguely mentions overflow as a possibility and 
proposes casting certain computations to \texttt{long} to avoid them in~C.  Howeveer, Heineman 
\emph{et al.} do not give any
nontrivial bound on the input edge weights, and so users are left in the fog as to the precise
conditions required for the algorithm to produce the correct answer.

%In a related vein, Paulin and Filli\^atre verified Floyd's algorithm in Coq~\cite{paulin}

%@misc{paulin,
%title={The {C}oq proof assistant},
%author={{C}oq development team},
%url={https://coq.inria.fr/}, journal={The Coq Proof Assistant}}
%C. and J.C. Filli\^atre
%http://pauillac.inria.fr/cdrom/www/coq/contribs/floyd.html
%.11. R.Sumners.Corre
%tne



\paragraph{Conclusion.}
We have described a machine-checked proof of correctness for Dijkstra’s 
shortest-path algorithm written in real~C from classic textbook code.
We showed that this code suffers from overflow issues for some 
inputs and described a precise nontrivial precondition on edge weights that 
avoids it.  We described how this result fits into our ongoing work.
