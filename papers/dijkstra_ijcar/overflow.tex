
Point out the key line in the code where overflow could occur

Use this diagram:

\begin{adjustbox}{scale=0.30}
{\centering
\begin{tikzpicture}[
  vert/.style={circle, line width=1pt, draw, fill=red}]
  \node[vert] (A) at (0,0) {\color{white}A};
  \node[vert] (B) [right = 8 of A] {\color{white}B};
  \node[vert] (C) [right = 8 of B] {\color{white}C};
  \draw [->,line width=3pt,arrows={-Stealth}] (A) -- (B);
  \draw [->,line width=3pt,arrows={-Stealth}] (B) -- (C);
  \draw [->,line width=3pt,arrows={-Stealth}] (C.south) .. controls ++(0, -2) .. (B);
  \node at (4,0.6) {3};
  \node at (12,0.6) {3};
  \node at (15,-1.9) {3};
\end{tikzpicture}

}
\end{adjustbox}

Briefly explain that intuition supports the straight approach

Show the nontrivial refinement in the precondition

