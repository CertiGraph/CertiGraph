A little thought makes it clear that Dijkstra's algorithm cannot work when the cumulative path
cost is greater than or equal to \texttt{INT\_MAX}.  A reasonable restriction would seem to be
allowing edge costs to be no more than $\left\lfloor\frac{\texttt{INT\_MAX}}{\texttt{SIZE}-1}\right\rfloor$, since the longest path can have no more than $\texttt{SIZE}-1$ links and so the maximum cumulative path cost will be no more than \texttt{INT\_MAX}.  However, this has two flaws.  First, since are writing real code in~C, rather than pseudocode in an idealized setting, we must reserve some concrete \texttt{int} value for ``infinity'' \texttt{INF}, which has the special semantics that when a vertex $x$'s distance is \texttt{INF} then $x$ must be unreached; if a reached destination vertex could have a legitimate path cost of \texttt{INF} then we would have an unpleasant ambiguity.  Second, even though path costs start at \texttt{INF} and only decrease, the code can overflow in lines~\ref{code:overflow} and~\ref{code:update}.

\begin{figure}[t]
\centering
\begin{tikzpicture}[x=0.3cm, y=0.3cm,
  vert/.style={circle, line width=1pt, draw, fill=red}]
  \node[vert] (A) at (0,0) {\color{white}A};
  \node[vert] (B) [right = 8 of A] {\color{white}B};
  \node[vert] (C) [right = 8 of B] {\color{white}C};
  \draw [->,line width=1pt,arrows={-Stealth}] (A) -- (B);
  \draw [->,line width=1pt,arrows={-Stealth}] (B) -- (C);
  \draw [->,line width=1pt,arrows={-Stealth}] (C.south) .. controls ++(0, -2) .. (B);
  \node at (5,0.6) {3};
  \node at (15,0.6) {3};
  \node at (16,-2.4) {3};
\end{tikzpicture}
\caption{A graph that will result in overflow on a 3-bit machine}
\label{fig:overflow}
\end{figure}

Consider applying Dijkstra's algorithm on a hypothetical 3-bit unsigned machine to 
the graph in figure~\ref{fig:overflow}.  The \texttt{SIZE} of the graph is 3 nodes, and so the na\"ive edge-weight upper bound is $\left\lfloor\frac{\texttt{INT\_MAX}}{\texttt{SIZE}-1}\right\rfloor = \left\lfloor\frac{7}{3-1}\right\rfloor = 3$, exactly as pictured in figure~\ref{fig:overflow}.  Indeed, a glance at the diagram is enough to tell that the true distance from the source~A~to vertices~B~and~C~are~3~and~6, respectively---both of which are representable with 3 bits, and so na\"ively all seems well.  %Unfortunately, Dijkstra's algorithm does not exactly work like that.  
Indeed, after processing vertices~A~and~B, 3~and~6~\textbf{are} the costs reflected in the \texttt{dist} array for~B~and~C, respectively---\emph{but unfortunately vertex~C~is still in the priority queue} \texttt{pq}.  After vertex~C~is popped on line~\ref{code:pop}, we fetch its neighbors in the \texttt{for} loop; vertex~B's cost of~3~will be fetched on line~\ref{code:cost}.  On line~\ref{code:overflow} the currently optimal cost of~B~(3) is compared against the sum of the optimal cost of~C~(6) plus the just-retrieved cost of the edge from~C~to~B~(3).  Since $6+3$ overflows in 3-bit arithmetic, the comparison is not between~3~and~9 but in fact between~3~and~1!  Thus the code decides that a new cheaper path from~A~to~B~exists (in particular, A$\leadsto$B$\leadsto$C$\leadsto$B) and then trashes the \texttt{dist} and \texttt{prev} arrays on line~\ref{code:update}.  %

Our code uses signed \texttt{int} rather than \texttt{unsigned int} so we have undefined behavior rather than defined-but-wrong behavior, but the essence of the overflow is identical.
Our solution is twofold.  First, we restrict the maximum edge cost to $\left\lfloor\frac{\texttt{INT\_MAX}}{\texttt{SIZE}}\right\rfloor$, which in the 3-bit setting just described forces an edge cost of no more than~2.  Second, on line~\ref{code:inf} we choose \texttt{INF} to be $\texttt{INT\_MAX} - \left\lfloor \frac{\texttt{INT\_MAX}}{\texttt{SIZE}} \right\rfloor$, which in the 3-bit setting is 5.  Consider modifying figure~\ref{fig:overflow} to
have edge weights of~2~rather than~3.  After processing vertices~A~and~B, the distances to~B~and~C are no more than~2~and~4, respectively.  When we process vertex~C, the comparison on line~\ref{code:overflow} is thus between the previous best cost to~B~of~2, and the candidate best cost to~B~via~C~of~6; there is no overflow and the code behaves as advertised.

