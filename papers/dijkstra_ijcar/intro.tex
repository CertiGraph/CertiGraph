
The best-known variant of Dijkstra's algorithm [cite Dijk] 
takes a finite directed graph with nonnegative costs 
on the edges, and, for a distinguished \emph{source} vertex, 
finds cost-minimal paths from the source to all the other vertices.

The algorithm is classic and ubiquitous, appearing widely in textbooks
and in real routing protocols. Further, the algorithm has been in 
use for over $60$ years, suggesting, for all practical purposes, 
that its safety and correctness have been verified by decades of application. 

Recent efforts [cite Mizar, cite ACL2, cite Coq] have implemented the algorithm 
in proof assistants and formally proved claims about its behavior.
However, because they operate entirely within idealized formal checkers, 
these works inadvertently gloss over certain classes of issues 
that routinely crop up in real-world settings.

In this paper we verify a~C~implementation of Dijkstra's 
one-to-all shortest path algorithm. We implement 
textbook~C~code [cite CLRS], import it into Coq [cite VST],
and state and prove a correctness claim in Coq [cite VST, cite RamifyCoq].
We expose a subtle overflow issue in the code, and address the issue
via a nontrivial refinement in the precondition of the algorithm.

The paper is organized as follows:
\vspace{-1em}
\begin{itemize}
    \item[\S\ref{sec:overview}] We briefly present and explain 
    our~C~implementation of Dijkstra's algorithm.
    \item[\S\ref{sec:verification}] We present our specification 
    and our most important loop invariant.
    \item[\S\ref{sec:overflow}] We explain the critical overflow issue
    in the code, and propose a fix by refinining the specification.
\end{itemize}