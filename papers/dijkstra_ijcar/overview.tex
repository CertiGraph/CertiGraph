
Figure~\ref{fig:decorated} shows the code and proof
sketch of Dijkstra's algorithm. 
The code is implemented exactly as suggested 
by~\cite{clrs}, and so we elide a discussion 
of the algorithm itself. The heart of the formal verification is in the 
while loop's invariant, which is stated on line~\ref{code:whileinv}
and explained further in Figure~\ref{fig:defns}.
The source vertex $\m{src}$ is taken as a parameter. 
A destination vertex $\m{dst}$ falls into one of three
categories, and subsequently obeys one of three invariants:
\begin{enumerate}
\item $\m{inv\_popped}$: $\m{dst}$ has been fully processed, and has been
popped from the priority queue. 
A globally optimal path from $\m{src}$ 
to $\m{dst}$ exists, the cost of this path is logged in 
the \texttt{dist} array, and all the vertices visited by the path are also popped.
Further, the links of this path are correctly logged in the \texttt{prev} array.
\item $\m{inv\_unpopped}$: $\m{dst}$ is reachable in 
one hop from a ``\emph{mom}'' vertex, which is itself popped. 
This route is locally optimal: we cannot 
improve the cost by going via a \emph{different} popped vertex.
The \texttt{prev} array logs
\emph{mom} as the best-known way to reach $\m{dst}$, and the \texttt{dist}
array logs the cost of the path via \emph{mom} as the best-known cost.
\item $\m{inv\_unseen}$: no path is currently known from $\m{src}$ to $\m{dst}$.
\end{enumerate}

This three-part invariant is trivially true before the while loop. 
On line~\ref{code:pop}, the minimal vertex from the priority queue is popped, 
thus breaking the invariant.

First, we must show that the minimal vertex $\m{u}$ 
obeys $\m{inv\_popped}$. \emph{i.e.}, show that the locally 
optimal path to $\m{u}$ is, in fact, globally optimal. 
This comes from blah blah blah

\newcommand{\s}{11}
\begin{figure}[htbp]
  \centering
  \begin{tikzpicture}[x=0.3cm, y=0.3cm,
      popped/.style={rounded corners=5pt, line width=1pt, draw, fill=MidnightBlue},
      fringe/.style={rounded corners=5pt, line width=1pt, draw, fill=ProcessBlue},
      popping/.style={rounded corners=5pt, line width=1pt, draw, dashed, fill=pink},
      unseen/.style={rounded corners=5pt, line width=1pt, draw}]
    \draw[unseen] (0,0) -- (\s,0) -- (\s,\s) -- (0,\s) -- cycle;
    \draw[fringe] (1.5,1.5) -- (9.5,1.5) -- (9.5,9.5) -- (1.5,9.5) -- cycle;
    \draw[popped] (3,3) -- (8,3) -- (6,6) -- (3,8) -- cycle;
    \node at (1.4,1) {dst$_3$};   
    \node at (2.9,2.5) {dst$_2$};   
    \node at (4.4,4) {\color{white}dst$_1$}; 
    \node at (6.6,7) {u};
    \tikzset{shift={(13.5,0)}}

    \draw[unseen] (0,0) -- (\s,0) -- (\s,\s) -- (0,\s) -- cycle;
    \draw[fringe] (1.5,1.5) -- (9.5,1.5) -- (9.5,9.5) -- (1.5,9.5) -- cycle;
    \draw[popping] (3,3) -- (8,3) -- (8,8) -- (3,8) -- cycle;
    \draw[popped] (3,3) -- (8,3) -- (6,6) -- (3,8) -- cycle;
    \node at (1.4,1) {dst$_3$};   
    \node at (2.9,2.5) {dst$_2$};   
    \node at (4.4,4) {\color{white}dst$_1$}; 
    \node at (6.6,7) {u};     

    \tikzset{shift={(13.5,0)}}

    \draw[unseen] (0,0) -- (\s,0) -- (\s,\s) -- (0,\s) -- cycle;
    \draw[fringe] (1.5,1.5) -- (9.5,1.5) -- (9.5,9.5) -- (1.5,9.5) -- cycle;
    \draw[popped] (3,3) -- (8,3) -- (8,8) -- (3,8) -- cycle;
    \node at (1.4,1) {dst$_3$};   
    \node at (2.9,2.5) {dst$_2$};   
    \node at (4.4,4) {\color{white}dst$_1$}; 
    \node at (6.6,7) {\color{white}u};       
  \end{tikzpicture}
  \caption{Popping $\m{u}$}  
\end{figure}

Next, we must account for the ripple effect that popping 
$\m{u}$ could have had on the other vertices. 
In particular, it is possible that a vertex obeying $\m{inv\_unpopped}$ can
improve its cost via $\m{u}$, and that an unreachable vertex 
obeying $\m{inv\_unseen}$ can now be reached via $\m{u}$. 
The for loop repairs these breakages by 
checking if a path via $\m{u}$ is an improvement for such vertices, and, if so, 
edits both arrays and the priority queue as seen on line~\ref{code:update}.

The for loop's invariant is similar to that of the while loop---$\m{inv\_unseen}$ 
and $\m{inv\_popped}$ are preserved as-is, modulo the popping of 
$\m{u}$ as discussed above. The key edit is in $\m{inv\_unpopped}$. blah blah blah

\colorlet{stash}{red}
\colorlet{red}{maincolor}

\begin{lstlisting}
  void dijkstra (int graph[SIZE][SIZE], int src, 
                           int *dist, int *prev) {
$//$ $\braces{\p{DijkGraph}(\gamma)}$
    int pq[SIZE];
    int i, j, u, cost;
    for (i = 0; i < SIZE; i++) {
      dist[i] = INF; 
      prev[i] = INF; 
      pq[i] = INF;
    }
    dist[src] = 0; 
    pq[src] = 0; 
    prev[src] = src;
$//$ $\braces{\p{DijkGraph}(\gamma) /| \null \\
\m{dijk\_correct}(\gamma,\m{src},\m{prev},\m{dist},\m{priq})}$
    while (!pq_emp(pq)) {
      u = popMin(pq);
      for (i = 0; i < SIZE; i++) {
        cost = graph[u][i]; 
        if (cost < INF) {
          if (dist[i] > dist[u] + cost) {
            dist[i] = dist[u] + cost;
            prev[i] = u; 
            pq[i] = dist[i];
          }
        }  
      }
    }
$//$ $\braces{\p{DijkGraph}(\gamma) /| \null \\ 
\forall \m{dst} \in \m{priq}.~\m{priq}[\m{dst}] = \texttt{INF} /| \null \\ 
\m{dijk\_correct}(\gamma,\m{src},\m{prev},\m{dist},\m{priq})}$
    return;
  }
\end{lstlisting}
\vspace{0.5em}
\begin{equation*}
\begin{split}
\p{list\_rep}(\gamma, \m{i}) &\defeq \texttt{data\_at  array  graph2mat}(\gamma)[\m{i}] \texttt{  list\_addr}(\gamma, \m{i}) \\
\vspace{1em}
\p{graph\_rep}(\gamma) &\defeq \underset{\texttt{vvalid}(\gamma,\m{v})}{\bigstar} \m{v}  \mapsto\p{list\_rep}(\gamma, \m{v})
\end{split}
\end{equation*}

\begin{equation*}
\begin{split}
\m{dijk\_correct}(\gamma, \m{src}, \m{prev}, \m{dist},& \m{priq}) \; \defeq \; \\
\forall \m{dst}.~\m{dst} \in \m{popped}(\m{priq}) \; => \; & \exists \m{path}.~\m{path\_correct}(\gamma, \m{prev}, \m{dist}, \m{path}) /| \null \\
& \m{path\_glob\_optimal}(\gamma, \m{dist}, \m{path}) /| \null \\
& \m{path\_entirely\_in\_popped}(\gamma, \m{prev}, \m{path}) /| \null \\
\m{priq}[\m{dst}] < \ifty \; => \; & \texttt{let }\m{m} \texttt{ := } \m{prev}[\m{dst}] \texttt{ in } \m{m} \in \m{popped}(\m{priq}) /| \null \\
&\forall \m{m'} \in \m{popped}(\m{priq}).~\m{cost}(\m{path2m} +:: (m, dst)) \le \null \\
&\hspace{10em} \m{cost}(\m{path2m'} +:: (m', dst)) /| \null \\
\m{priq}[\m{dst}] = \ifty \; => \; & \forall \m{m} \in \m{popped}(\m{priq}).~\m{cost}(\m{path2m} +:: (m, dst)) = \ifty
\end{split}
\end{equation*}

\colorlet{red}{stash}


