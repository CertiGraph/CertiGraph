
Figure~\ref{fig:decorated} shows the code and proof
sketch of Dijkstra's algorithm. 
The code is implemented exactly as suggested 
by~\cite{clrs}, and so we elide a discussion 
of the algorithm itself. The heart of the formal verification is in the 
while loop's invariant, which is stated on line~\ref{code:whileinv}
and explained further in Figure~\ref{fig:defns}.
The source vertex $\m{src}$ is taken as a parameter. 
A destination vertex $\m{dst}$ falls into one of three
categories, and subsequently obeys one of three invariants:
\begin{enumerate}
\item $\m{inv\_popped}$: $\m{dst}$ has been fully processed, and has been
popped from the priority queue. 
A globally optimal path from $\m{src}$ 
to $\m{dst}$ exists, the cost of this path is logged in 
the \texttt{dist} array, and all the vertices visited by the path are also popped.
Further, the links of this path are correctly logged in the \texttt{prev} array.
\item $\m{inv\_unpopped}$: $\m{dst}$ is reachable in 
one hop from a ``\emph{mom}'' vertex, which is itself popped. 
This route is locally optimal: we cannot 
improve the cost by going via a \emph{different} popped vertex.
The \texttt{prev} array logs
\emph{mom} as the best-known way to reach $\m{dst}$, and the \texttt{dist}
array logs the cost of the path via \emph{mom} as the best-known cost.
\item $\m{inv\_unseen}$: no path is currently known from $\m{src}$ to $\m{dst}$.
\end{enumerate}

This three-part invariant is trivially true before the while loop. 
On line~\ref{code:pop}, the minimal vertex from the priority queue is popped, 
thus breaking the invariant.

First, we must show that the minimal vertex $\m{u}$ 
obeys $\m{inv\_popped}$. \emph{i.e.}, show that the locally 
optimal path to $\m{u}$ is, in fact, globally optimal. 
This comes from blah blah blah

\newcommand{\s}{11}
\begin{figure}[htbp]
  \centering
  \begin{tikzpicture}[x=0.3cm, y=0.3cm,
      popped/.style={rounded corners=5pt, line width=1pt, draw, fill=MidnightBlue},
      fringe/.style={rounded corners=5pt, line width=1pt, draw, fill=ProcessBlue},
      popping/.style={rounded corners=5pt, line width=1pt, draw, dashed, fill=pink},
      unseen/.style={rounded corners=5pt, line width=1pt, draw}]
    \draw[unseen] (0,0) -- (\s,0) -- (\s,\s) -- (0,\s) -- cycle;
    \draw[fringe] (1.5,1.5) -- (9.5,1.5) -- (9.5,9.5) -- (1.5,9.5) -- cycle;
    \draw[popped] (3,3) -- (8,3) -- (6,6) -- (3,8) -- cycle;
    \node at (1.4,1) {dst$_3$};   
    \node at (2.9,2.5) {dst$_2$};   
    \node at (4.4,4) {\color{white}dst$_1$}; 
    \node at (6.6,7) {u};
    \tikzset{shift={(13.5,0)}}

    \draw[unseen] (0,0) -- (\s,0) -- (\s,\s) -- (0,\s) -- cycle;
    \draw[fringe] (1.5,1.5) -- (9.5,1.5) -- (9.5,9.5) -- (1.5,9.5) -- cycle;
    \draw[popping] (3,3) -- (8,3) -- (8,8) -- (3,8) -- cycle;
    \draw[popped] (3,3) -- (8,3) -- (6,6) -- (3,8) -- cycle;
    \node at (1.4,1) {dst$_3$};   
    \node at (2.9,2.5) {dst$_2$};   
    \node at (4.4,4) {\color{white}dst$_1$}; 
    \node at (6.6,7) {u};     

    \tikzset{shift={(13.5,0)}}

    \draw[unseen] (0,0) -- (\s,0) -- (\s,\s) -- (0,\s) -- cycle;
    \draw[fringe] (1.5,1.5) -- (9.5,1.5) -- (9.5,9.5) -- (1.5,9.5) -- cycle;
    \draw[popped] (3,3) -- (8,3) -- (8,8) -- (3,8) -- cycle;
    \node at (1.4,1) {dst$_3$};   
    \node at (2.9,2.5) {dst$_2$};   
    \node at (4.4,4) {\color{white}dst$_1$}; 
    \node at (6.6,7) {\color{white}u};       
  \end{tikzpicture}
  \caption{Popping $\m{u}$}  
\end{figure}

Next, we must account for the ripple effect that popping 
$\m{u}$ could have had on the other vertices. 
In particular, it is possible that a vertex obeying $\m{inv\_unpopped}$ can
improve its cost via $\m{u}$, and that a vertex obeying $\m{inv\_unseen}$ can
be reached via $\m{u}$. The for loop repairs these breakages by 
checking if a path via $\m{u}$ is an improvement for such vertices, and, if so, 
edits both arrays and the priority queue as seen on line~\ref{code:update}.

The for loop's invariant is similar to that of the while loop---$\m{inv\_unseen}$ 
is preserved as-is, and $\m{inv\_popped}$ is preserved modulo the addition of 
$\m{u}$ as discussed above. The key edit is in $\m{inv\_unpopped}$. blah blah blah


\begin{figure}[t]

\begin{lstlisting}[mathescape=true,showlines=true]
void dijkstra (int **g, int src, int *dist,
               int *prev, int size, int inf {
$\color{OliveGreen}//~\braces{\p{AdjMat}(\texttt{g},\gamma) *
\mathsf{array}(\texttt{dist}, \_) * \mathsf{array}(\texttt{prev}, \_)}$
 Item* temp = (Item*) mallocN(sizeof(Item));
 int* keys = mallocN (size * sizeof (int));
 PQ* pq = pq_make(size); int i, j, u, cost;
 for (i = 0; i < size; i++)
 { dist[i] = inf; prev[i] = inf; keys[i] = pq_push(pq,inf,i); } $\label{code:assigninf}$
 dist[src]= 0; prev[src]= src; pq_edit_priority(pq,keys[src],0);
 while (pq_size(pq) > 0) {
$\color{OliveGreen}//~\braces{{\color{red}\exists \m{dist}, \m{prev}, \m{popped}, \m{heap}}.~\p{AdjMat}(\texttt{g},\gamma) * {\color{red}\p{PQ}(\texttt{pq},\m{heap})} *
{\color{red}\mathsf{Item}(\texttt{temp}, \_)} * \null \\
\mathsf{array}(\texttt{dist},{\color{red}\m{dist}}) *
\mathsf{array}(\texttt{prev}, {\color{red}\m{prev}}) *
{\color{red}\mathsf{array}(\texttt{keys}, \m{keys}}) /| \null \\
{\color{red}\m{linked\_correctly}(\gamma, \m{heap}, \m{keys}, \m{dist}, \m{popped})} /| \null \\
{\color{red}\m{dijk\_correct}(\gamma,\texttt{src},\m{popped},\m{prev},\m{dist})}}$ $\label{code:whileinv}$
  pq_pop(pq, temp); u = temp->data; $\label{code:pop}$
  for (i = 0; i < size; i++) {
$\color{OliveGreen}//~\braces{{\color{red}\exists \m{dist'}, \m{prev'}, \m{heap'}}.~\p{AdjMat}(\texttt{g},\gamma) * \p{PQ}(\texttt{pq},{\color{red}\m{heap'}}) * \null \\
\mathsf{array}(\texttt{dist},\m{\color{red}dist'}) *
\mathsf{array}(\texttt{prev}, \m{\color{red}prev'}) *
\mathsf{array}(\texttt{keys}, \m{keys}) * \null \\
\mathsf{Item}(\texttt{temp}, \m{\color{red}(\texttt{keys[u]}, \texttt{dist[u]}, \texttt{u})}) /|
\m{\color{red}min(\texttt{dist[u]}, \m{heap'})} /| \null \\
{\m{linked\_correctly}(\gamma, \m{\color{red}heap'}, \m{keys}, \m{\color{red}dist'},
{\color{red}\m{popped} \uplus \{\texttt{u}\}})} /| \null \\
{\color{red}\m{dijk\_correct\_weak}({\color{OliveGreen}\gamma, \texttt{src}}, \m{popped} \uplus \{\texttt{u}\}, \m{prev'}, \m{dist'}, \texttt{i}, \texttt{u})}}$ $\label{code:forinv}$
   cost = getCell(g, u, i); $\label{code:cost}$
   if (cost < inf) {
    if (dist[i] > dist[u] + cost) { $\label{code:overflow}$
     dist[i] = dist[u] + cost; prev[i] = u; $\label{code:update1}$
     pq_edit_priority(pq, keys[i], dist[i]); $\label{code:update2}$
  }}}} $\color{OliveGreen}//~\braces{{\color{red}\exists \m{dist''}, \m{prev''}}.~\p{AdjMat}(\texttt{g},\gamma) * \p{PQ}(\texttt{pq},\m{\color{red}\emptyset}) * {\mathsf{Item}(\texttt{temp}, {\color{red}\_})} * \null \\
 \mathsf{array}(\texttt{dist},\m{\color{red}dist''}) *
 \mathsf{array}(\texttt{prev}, \m{\color{red}prev''}) *
 \mathsf{array}(\texttt{keys}, \m{keys}) /| \null \\
{\color{red}\forall \m{dst}.~0 \le \m{dst} < \texttt{size} -> \m{inv\_popped}}(\gamma, \m{src}, {\color{red}\m{\gamma.V}, \m{prev''}, \m{dist''}, \m{dst}})}$ $\label{code:end}$
 freeN (temp); pq_free (pq); freeN (keys); return; }
\end{lstlisting}
\vspace{-1em}
\caption{C code and proof sketch for Dijkstra's Algorithm.}
\vspace{-1em}
\label{fig:decorated}
\end{figure}



\hide{
$\color{OliveGreen}//~\braces{{\color{red}\exists \m{dist''}, \m{prev''},\m{heap''}}.~\p{AdjMat}(\texttt{g},\gamma) * \p{PQ}(\texttt{pq},\m{\color{red}heap''}) *
  \mathsf{array}(\texttt{dist},\m{\color{red}dist''}) * \null \\
  \mathsf{array}(\texttt{prev}, \m{\color{red}prev''}) *
  \mathsf{array}(\texttt{keys}, \m{keys}) *
  \mathsf{Item}(\texttt{temp}, \m{(\texttt{keys[u]}, \texttt{dist[u]}, \texttt{u})} /| \null \\
  \m{min(\texttt{dist[u]}, \m{heap'})} /|
  {\color{red}\m{heap''} = \m{heap'} [\texttt{keys[i]} \mapsto (\texttt{dist[i]},\texttt{i})]} /| \null \\
  {\m{\color{red}dijk\_correct}(\gamma, \texttt{src}, \m{popped'}, \m{\color{red}prev''}, \m{\color{red}dist''})}}$ $\label{code:caughtup}$
}

