\documentclass[12pt]{llncs}

\linespread{1.25}

\usepackage{amssymb}
\usepackage{amsmath}
\usepackage{mathtools}
%% The amsthm package provides extended theorem environments

% \usepackage{amsthm}
% AM commented out because it clashes with proof, etc

%% The lineno packages adds line numbers. Start line numbering with
%% \begin{linenumbers}, end it with \end{linenumbers}. Or switch it on
%% for the whole article with \linenumbers.
\usepackage{lineno}

%% Use package enumitem to align enumeration and itemization
\usepackage{enumitem}
\usepackage{listings}
\usepackage{courier}           % for the courier font (optional)
\usepackage{multicol}          % for two equations side by side
\usepackage[justification=centering]{caption}
\usepackage[dvipsnames]{xcolor}
\usepackage{stmaryrd}
\usepackage{hyperref}
\usepackage{hieroglf}
\usepackage{scalerel}
\usepackage{tikz}
\usepackage{pgfplots}
\usepackage[export]{adjustbox} % for subfigures
\usepackage{semantic}          % for mathlig

\usepackage{tkz-berge}			% for graphs

\hypersetup{ % play with these to change the look of hyperlinks
    colorlinks=true,
    linkcolor=black,
    filecolor=magenta,
    urlcolor=blue,
    citecolor=black
}

\newcommand{\coq}{\scalebox{.6}{\textpmhg{\Ha}}}
\newcommand{\p}[1]{\ensuremath{\mathsf{#1}}} % predicate font
\newcommand{\m}[1]{\ensuremath{\mathit{#1}}} % math font
\newcommand{\braces}[1]{\left\{\begin{array}{l@{}} #1 \end{array}\right\}}
\let\ramify\lightning
\newcommand{\sz}{\texttt{SIZE}}
\newcommand{\ifty}{\texttt{INF}}
\newcommand{\defeq}{\mathbin{\stackrel{\Delta}{=}}}
\newcommand{\bigO}{\text{O}}
\usetikzlibrary{shadows}
\usetikzlibrary{arrows.meta, positioning, decorations.pathmorphing, fit, matrix}
\mathlig{/|}{\mathbin{\wedge}} % additive conjunction


% required by LNCS
\renewcommand\UrlFont{\color{blue}\rmfamily}

\colorlet{red}{red!80!black}
\colorlet{green}{green!50!black}

%% NEW COMMANDS =============================================

\lstdefinestyle{myStyle}{
%	language=Coq,
    keywords={Inductive,Require,Import,Definition,Fixpoint,match,with,end,let,in,fix},
	basicstyle=\normalfont\footnotesize\tt,
    keywordstyle=\color{green}, % Blue clashes with the cyan links. Change if you want.
	stepnumber=1,
	tabsize=2,
    numbers=none,
    numberstyle=\tiny,
    numbersep=5pt,
	showspaces=false,
    escapechar=`,
	showstringspaces=false
}
%basicstyle=\fontsize{10}{11}\selectfont\ttfamily,

\lstdefinestyle{myTinyStyle}{
%   language=Coq,
    basicstyle=\normalfont\fontsize{7.0}{7.3}\tt,
    keywordstyle=\color{green}, % Blue clashes with the cyan links. Change if you want.
    stepnumber=1,
    tabsize=2,
    numbers=none,
    numberstyle=\tiny,
    numbersep=5pt,
    showspaces=false,
    showstringspaces=false,
      language=C,
  morecomment=[l][{\color{OliveGreen}}]{//},
  sensitive=true,
  mathescape=true,
  showlines=true,
  escapechar=`,
  basicstyle=\footnotesize\ttfamily,
  keywordstyle=\color{blue}, numbers=left,
  numberstyle=\tiny, numbersep=5pt, boxpos=t
  }

\lstset{style=myTinyStyle}
\makeatletter
\newlength{\@mli}
\newcommand{\mli}[1]{%
  \settowidth{\@mli}{\lstinline/#1/}
  \hspace{-.5ex}\begin{minipage}[t]{\@mli}\lstinline/#1/\end{minipage}}
\makeatother
\newcommand{\li}[1]{\ifmmode\mbox{\mli{#1}}\else\mbox{\lstinline/#1/}\fi}

\newcommand\hide[1]{}

\newenvironment{centermath}
 {\begin{center}$\displaystyle}
 {$\end{center}}

\renewcommand{\note}[2][polish]{{\color{red} #2}{\marginpar{\tiny \color{blue} #1}}}
\renewcommand{\implies}{\Rightarrow}
\renewcommand{\iff}{\Leftrightarrow}

\title{Verification of C implementations of Prim's and Kruskal's Algorithms}
\subtitle{}
\titlerunning{Verification of C implementations of Prim's and Kruskal's Algorithms}
%optional, please use if title is longer than one line

\begin{document}

\author{Qualifying Examination paper by Leow Wei Xiang$^{(\dagger)}$}
\authorrunning{Leow Wei Xiang}

% First names are abbreviated in the running head.
% If there are more than two authors, 'et al.' is used.
%
\institute{($\dagger$) School of Computing}
% \\

\maketitle
%\begin{frontmatter}

%% Title, authors and addresses

%% use the tnoteref command within \title for footnotes;
%% use the tnotetext command for theassociated footnote;
%% use the fnref command within \author or \address for footnotes;
%% use the fntext command for theassociated footnote;
%% use the corref command within \author for corresponding author footnotes;
%% use the cortext command for theassociated footnote;
%% use the ead command for the email address,
%% and the form \ead[url] for the home page:

%% \tnotetext[label1]{}
%% \author{Name\corref{cor1}\fnref{label2}}
%% \ead{email address}
%% \ead[url]{home page}
%% \fntext[label2]{}
%% \cortext[cor1]{}
%% \address{Address\fnref{label3}}
%% \fntext[label3]{}

%% use optional labels to link authors explicitly to addresses:
%% \author[label1,label2]{}
%% \address[label1]{}
%% \address[label2]{}

%\author{}

%\address{}


\begin{abstract}
	\vspace{-1.2em}
	We extend our research group's previous work on verifying C graph implementations
	with verified versions of Dijkstra's, Prim's and Kruskal's classical algorithms.
	We prove functional correctness of C implementations of these algorithms, and expand the previous graph library to reason about undirected graph properties and common spatial representations.
	We observe that a small "change" to Prim's algorithm allows it to return a minimal spanning forest for disconnected graphs in a single run, contrary to the common notion that the input graph must be connected.
	\newline\newline
	This paper, written with intent for the Qualifying Examination, discusses my work on Prim's and Kruskal's algorithms,
	with help from my research partner Anshuman Mohan. Our combined work fits
	into an ongoing exploration of verified graph-manipulating algorithms.
	Outside of the Qualifying Examination, we intend to submit our findings as a conference paper.
	
	\keywords{Prim's algorithm \and Kruskal's algorithm \and minimum spanning trees \and verification \and CompCert \and VST}
\end{abstract} %\and Coq 

%\end{frontmatter}

%% \linenumbers

%% main text

\section{Introduction}
\label{sec:intro}
Over the last fifteen years great strides have been made in automating verifications of programs that manipulate
tree-like data structures using separation logic 
\cite{berdine:smallfoot,chin:hipsleek,jacobs:verifast,chlipala:bedrock,bengtson:charge,appel:programlogics}.  Unfortunately, verifying programs that manipulate graph-like data structures (i.e. structures with \emph{intrinsic sharing}) has been more challenging.  Indeed, verifying such programs was formidable enough that a number of the early landmark results in separation logic devoted substantial effort to verify single examples such as Schorr-Waite~\cite{hongseok:phd} with pen and paper---avoiding the additional challenges inherent in mechanized reasoning.

In recent years, Hobor and Villard introduced the concept of \emph{ramification} as a kind of proof pattern or framework to verify graph-manipulating programs on pen and paper~\cite{hobor:ramification}.  The major focus of this paper is to develop methods to verify realistic graph programs in a mechanized context.  We do so by upgrading the theory of ramification and by developing a general and modular library for graph-related reasoning in separation logic.  We incorporate our approach into two sizeable separation logic-based verification tools: the Floyd system of the Verified Software Toolchain (VST)~\cite{appel:programlogics} and the HIP/SLEEK program verifier~\cite{chin:hipsleek}.  VST and HIP/SLEEK inhabit quite different points in the design space for verification tools, with VST primarily focused on heavily human-guided verifications with an emphasis on end-to-end machine-checked proofs, and HIP/SLEEK focusing on more automation.  Despite these differences, the vast majority of our Coq code base is shared between them,
giving us hope that our work will be applicable to other verification tools.

\marginpar{\color{magenta} computable mathgraphs, null, pregraphs Problem with ``later'' not being precise.}
The structure of our paper is as follows:
%\vspace{-0.25ex}
\begin{itemize}
\item[\S\ref{sec:orientation}] We verify a graph marking algorithm and explain why such algorithms are easier to verify using relations instead of functions.  We introduce \emph{localization blocks} as a new notation for ramification.  We upgrade Hobor and Villard's \infrulestyle{Ramify} rule to handle both modified program variables and existential quantifiers more gracefully.
\vspace{-0.1ex}
\item[\S\ref{sec:mathgraph}] We develop a general mechanization of mathematical graphs powerful enough to support realistic verification. %{\color{magenta} What else can we say here?}
\vspace{-0.1ex}
\item[\S\ref{sec:spacegraph}] We show that the standard Knaster-Tarski fixpoint~\cite{tarski:fixpoint} cannot define a usable separation logic graph predicate.  We propose a better definition for general spatial graphs that still enjoys a ``recursive'' fold/unfold.  We prove general theorems about spatial graphs in a way that can be utilized in multiple flavors of separation logic, such as the logics contained in VST and HIP/SLEEK.
\vspace{-0.1ex}
\item[\S\ref{vst}] We explain how we integrated ramification into VST by developing two new Floyd tactics, \li{localize} and \li{unlocalize}.  We discuss other examples we have verified, including spanning tree and DAG copy.
\vspace{-0.1ex}
\item[\S\ref{sec:hipsleek}] We explain how we modified HIP/SLEEK to introduce ramifications when programs modify data structures with intrinsic sharing and to automatically discharge the associated obligations using Coq-verified external lemmas.
\vspace{-0.1ex}
\item[\S\ref{sec:related}] We discuss related work, future work, and conclude.
\end{itemize}
All of our results are machine checked.



\section{Extending CertiGraph with undirected graph lemmas}
\label{sec:undirected}
\subsection{Graph definition in CertiGraph}

\begin{lstlisting}
Record PreGraph {EV: EqDec Vertex eq} {EE: EqDec Edge eq} := {
	vvalid : Ensemble Vertex;
	evalid : Ensemble Edge;
	src : Edge -> Vertex;
	dst : Edge -> Vertex
}.
\end{lstlisting}
In CertiGraph, every graph has the four following basic functions: $vvalid$ which determines which vertices are valid in the graph; $evalid$ which determines which edges are valid in the graph, and two functions $src$ and $dst$ to map each edge to vertices. For example, an edge $e$ in graph $g$ pointing from vertex $u$ to $v$ will have $src$ $g$ $e$ $=$ $u$ and $dst$ $g$ $e$ $=$ $v$. What these functions actually are, is defined during the actual construction of a graph.

As a result of this definition, the graphs defined in our library are effectively directed - each valid edge in the graph has a clear direction defined by the $src$ and $dst$ functions. This makes sense in most use cases, especially when reasoning about C pointers.

The issue is how then to reason about undirected graphs in a library whose graphs are directed by design. Undirected graphs are generally treated as a separate kind of graph from directed graphs. Thus, the most naive idea is to consider undirected edges as a separate category from directed edges, by introducing a new \texttt{Ensemble} for undirected edges. This has the benefit of reasoning with multigraphs that contains both directed and undirected edges. However, as most of our graphs are either directed or undirected, it will be unwieldy for a graph to have to maintain ensembles of both kinds of edges when it will never use one kind. Furthermore, it requires us to recreate many fundamental operations for manipulating edges, ignoring the already robust suite of lemmas we have for $evalid$, $src$ and $dst$.

Instead, we decided to rely on the observation that \textit{every directed graph can be treated as an undirected graph} by simply ignoring the directions of the edges. That is, we can assert that a graph in our library, directed by default, holds certain undirected graph properties. This is backed by our observation that undirected graph properties are largely mathematical in nature and, depending on the representation, have few to no spatial requirements in separation logic. Furthermore, there is no overlap in their properties - claiming that directed graph holds an undirected graph property by ignoring the direction of its edges, does not affect its directed properties. It also allows us to use our rich set of directed-graph lemmas about the addition and removal of edges.

\subsection{Undirected graph properties}

Here we give a quick explanation of the main undirected definitions and properties we're interested in. Our definitions are largely based on CLRS~\cite{clrs}.

Let $g$ be a graph. A valid edge $e$ in $g$ is an adjacent edge (or \textit{adj\_edge} for short) of valid vertices $u$ and $v$ if those are its $src$ and $dst$. $u$ and $v$ are \textit{adjacent} if such an edge exists in $g$.
\begin{lstlisting}
Definition adj_edge (g: PreGraph V E) (e: E) (u v: V) :=
	strong_evalid g e /\
	((src g e = u /\ dst g e = v) \/ (src g e = v /\ dst g e = u)).

Definition adjacent (g: PreGraph V E) (u v: V) :=
	exists e: E, adj_edge g e u v.
\end{lstlisting}
A similar approach is proposed in Halsbeck~\cite{DBLP:journals/afp/HaslbeckLB19}.

A valid undirected path, or \textit{upath}, in $g$ is a list of vertices such that each vertex in the list is adjacent with the subsequent vertex. An empty path, $nil$, is by default a valid $upath$. A singleton path $[v]$ is also a valid $upath$ if $v$ is a valid vertex.
\begin{lstlisting}
Fixpoint valid_upath (g: PreGraph V E) (p: upath) : Prop :=
match p with
| nil => True
| u :: nil => vvalid g u
| u :: ((v :: _) as p') => adjacent g u v /\ valid_upath g p'
end.
\end{lstlisting}
Two vertices $u$ and $v$ are \textit{connected by p} if $p$ is a valid $upath$ in $g$ with the head vertex $u$ and last vertex $v$. They are \textit{connected} if such a $upath$ exists. By the above, every valid vertex in $g$ is trivially connected to itself.
\begin{lstlisting}
Definition connected_by_path (g: PreGraph V E) (p: upath) (n : V) :=
	fun n' => valid_upath g p /\
		hd_error p = Some n /\ last_error p = Some n'.

Definition connected (g: PreGraph V E) (n : V) :=
	fun n' => exists p, connected_by_path g p n n'.
\end{lstlisting}
An undirected cycle, or $ucycle$, is a $upath$ whose first and last vertices are the same. A $simple$ $upath$ is a valid $upath$ that has no duplicate vertices - no vertex is visited twice. A $simple$ $ucycle$ is a cycle whose tail has no duplicate vertices - the only "duplicates" in the cycle are the first and last vertices.

Note that the definition of path varies between textbooks and papers. For example, \textit{Discrete Mathematics and its Applications}~\cite{rozen} define paths as a sequence of edges with an implicit sequence of vertices, whereas CLRS, which we have followed, defines it as a sequence of vertices with an implicit sequence of edges.

\subsection{Defining forests before trees}

CLRS defines a tree as "a (connected) graph with no simple undirected cycles" - in other words, a connected, acyclic graph. We use the same definition:
\begin{lstlisting}
Definition uforest g:=
	(forall e, evalid g e -> strong_evalid g e) /\
	(forall p l, $\neg$ simple_ucycle g p l).
\end{lstlisting}
Our definition contains two propositions. The first constrains a forest to have no excess, "dangling" edges. Although our graph library needs to reason about such edges, we do not want to tolerate them in our forests. The second is the key "no simple undirected cycles".
\begin{figure}[H]
	\begin{tikzpicture} [auto, node distance =1.5 cm and 1.5cm ,on grid, semithick, state/.style ={ circle}]
		\node[state] (V0) {$V0$};
		\node[state] (V1) [right=of V0] {$V1$};
		\node[state] (V2) [below=of V0] {$V2$};
		\node[state] (?1) [above=of V0] {$?$};
		\node[state] (?2) [right=of V1] {$?$};
		\node[state] (?3) [below right=of ?2] {$?$};
		\path (V0) edge node[above=0.15 cm] {$5$} (V1);
		\path (V0) edge node[left=0.15 cm] {$6$} (V2);
		\path (V0) edge node[left=0.15 cm] {$-1$} (?1);
		\path (?2) edge node[above right=0.15 cm] {$2$} (?3);
	\end{tikzpicture}
	\caption{The above is not a forest, due to the "dangling" edges (? being a placeholder for invalid vertices)}
\end{figure}
We also highlight a difference in our definition compared to mathematical textbooks. Prim's and Kruskal's algorithms are presented as minimum-spanning \textit{tree} algorithms, and often have the implicit assumption that the graph is fully connected. They may or may not discuss forests - CLRS does not, while \textit{Discrete Mathematics} informally defines forests as ``containing no simple circuits that are not necessarily connected [...] and have the property that each of their connected components is a tree." In short, these sources define forests from ``bottom-up" using trees. Our ``top-down" definition instead recognises forests as acyclic graphs, and trees as a special case of forests where every vertex is connected to each other. This definition was also used by Lammich et al~\cite{DBLP:journals/afp/LammichN19}.
\begin{lstlisting}
Definition connected_graph (g: PGraph) :=
	forall u v, vvalid g u -> vvalid g v -> connected g u v.

Definition utree g := uforest g /\ connected_graph g.
\end{lstlisting}

\section{Prim's algorithm}
\label{sec:prim}
Here we discuss our verifications of the classic MST algorithms Prim and Kruskal.  Although our machine-checked proofs are about real~C~code, in this section we take a higher-level approach than we did in \S\ref{sec:dijkstra}, focusing on our key algorithmic findings and overall experience.  Accordingly, we only provide pseudocode for Prim's algorithm rather than a decorated program and do not show any code for Kruskal's.  Our development contains our~C~code and formal proofs~\cite{anonrepo}.

%\vspace*{-0.25em}

\subsection{Prim's Algorithm}
\label{sec:prim}

%\vspace*{-0.25em}

\begin{figure}[t]
\[
\begin{array}{@{}l@{~~}|@{~~}l@{}}
\begin{minipage}{0.475\textwidth}
\begin{lstlisting}
MST-PRIM(G,w,r):
 for each u in G.V
  u.key = INF
  u.parent = NIL $\hide{code:primsetinitparent}$
 r.key = 0 $\label{code:primsetroot}$
 Q = G.V
 while Q $\neq$ $\emptyset$
  u = EXTRACT-MIN(Q) $\hide{code:primextractmin}$
  for each v in G.Adj[u]
   if v $\in$ Q and w(u,v) $<$ v.key
    v.parent = u
    v.key = w(u,v) $\hide{code:primeditpri}$
\end{lstlisting} \end{minipage} &
\begin{minipage}{0.5\textwidth}
\begin{lstlisting}[numbers=none]
MST-NOROOT-PRIM(G,w):
 for each u in G.V
  u.key = INF
  u.parent = NIL

 Q = G.V
 while Q $\neq$ $\emptyset$
  u = EXTRACT-MIN(Q)
  for each v in G.Adj[u]
   if v $\in$ Q and w(u,v) $<$ v.key
    v.parent = u
    v.key = w(u,v)
\end{lstlisting}
\end{minipage}
\end{array}
\]
%\vspace*{-1.25em}
\caption{Left: Prim's algorithm from CLRS~\cite{clrs}. Right: the same omitting line 5.}
%\vspace*{-1.25em}
\label{fig:prims}
\end{figure}

We put the pseudocode for Prim's algorithm in figure~\ref{fig:prims}; the code on the left-hand side is directly from CLRS, whereas the code on the right omits line 5 and will be discussed in~\S\ref{sec:primforest}.  Note that line 12 contains an implicit call to the PQ's \texttt{edit\_priority}.  Since the pseudocode only compares \texttt{key}s (\emph{i.e.}, edge weights) rather than doing arithmetic on them \emph{\`a la} Dijkstra, there are no potential overflows and it is reasonable to set \texttt{INF} to \texttt{INT\_MAX} in~C.

Indeed, our initial verifications of~C~code were largely ``turning the crank'' once we had the definitions and associated lemma support for pure/abstract undirected graphs, forests, \emph{etc.} discussed in \S\ref{sec:newundirected}.  Accordingly, our initial contribution was a demonstration that this new graph machinery was sufficient to verify real code.  We also proved that our extensions to CertiGraph from~\S\ref{sec:extensions} were generic rather than verification-specific by reusing much pure and spatial reasoning that had been originally developed for our verification of Dijkstra.

%\vspace*{-0.25em}

\subsection{Prim's Algorithm handles multiple components out of the box}
\label{sec:primforest}

%\vspace*{-0.25em}

Textbook discussions of Prim's algorithm are usually limited to single-component input graphs (\emph{a.k.a.} connected graphs), producing a minimum spanning tree.  It is widely believed that Prim's is not directly applicable to graphs with multiple components, which should produce a minimum spanning forest.  For example, both Rozen~\cite{rozen} and Sedgewick \emph{et al.}~\cite{sedgewick,DBLP:books/daglib/0029345} leave the extension to multiple components as an formal exercise for the reader, whereas Kepner and Gilbert suggest that multiple-component graphs should be handled by first finding the components and then running Prim on each component~\cite{kepnergilbert}.  This appears to be the standard solution, appearing in numerous lectures and implementations\footnote{Another standard solution is to use Kruskal's \hide{or Boruvka's~\cite{boruuvka1926jistem}} algorithm instead.}. %CLRS does not mention the case of disconnected graphs

After we completed our initial verification, a close examination of our formal invariants showed us that the algorithm \emph{exactly as given by standard textbooks} will properly handle multi-component graphs \textit{in a single run}.  The confusion starts because, in a fully connected graph, any vertex $\texttt{u}$ removed from the PQ on line~8 must have $\texttt{u.key} < \texttt{INF}$; \emph{i.e.}, $\texttt{u}$ must be immediately reachable from the spanning tree that is in the process of being built.  However, nothing in the code relies upon this connectedness fact!  All we need is that $\texttt{u}$ is the ``closest vertex'' to the ``current component.''  If $\texttt{u.key}=\texttt{INF}$ \emph{and} \texttt{u} is a minimum of the PQ, then it simply means that the ``previous component'' is done, and we have started spanning tree construction on a new unconnected component ``rooted'' at \texttt{u}, yielding a forest.  The node $\texttt{u}$'s parent will remain \texttt{NIL}, at it was after the setup loop on line~4, indicating that it is the root of a spanning tree.  Its \texttt{key} will be $\texttt{INF}$ rather than $0$, but the \texttt{key}s are \emph{internal to Prim's algorithm}: clients only get back the spanning forest as encoded in the \texttt{parent} pointers\footnote{The \texttt{key}s simply record the edge-weight connecting a vertex to its candidate parent; recall that line~12 is really a call to the PQ's \texttt{edit\_priority}.  If a client wishes to know this edge weight, it can simply look up the edge in the graph.}.

Having made this discovery, we updated our proofs to support the new weaker precondition, which is what we currently formally verify in Coq~\cite{Coq}.
A little further thought led to the realization that since Prim can handle arbitrary numbers of components, the initialization of the root's \texttt{key} in line~5 is in fact unnecessary.  Accordingly, if we remove this line and the associated function argument \texttt{r} from \texttt{MST-PRIM} (\emph{i.e.}, the code on the right half of figure~\ref{fig:prims}), the algorithm still works correctly.  Moreover, \emph{the program invariants become simpler} because we no longer need to treat a specified vertex (\texttt{r}) in a distinguished manner.  Our formal development verifies this version of the algorithm as well~\cite{anonrepo}.

\subsection{Related work on Prim in algorithms and formal methods}
\label{sec:relworkprim}

Prim's algorithm was in fact first developed by the Czech mathematician Vojt\v{e}ch Jarn\'{i}k in 1930~\cite{prim1:jarnik} before being rediscovered by Robert Prim in 1957~\cite{prim2:prim} and a third time by Edsger~W.~Dijkstra in 1959~\cite{prim3:dijkstra}.  Both Prim's and Dijkstra's treatment explicitly assumes a connected graph; although we cannot read Czech, some time with Google translate suggests that Jarn\'{i}k's treatment probably does the same.  The textbooks we surveyed \cite{kepnergilbert,sedgewick,DBLP:books/daglib/0029345,rozen,DBLP:books/daglib/0022194,clrs,DBLP:books/daglib/0015106} seem to derive from Prim's and/or Dijkstra's treatment.
More casual references such as Wikipedia~\cite{prim:wiki} and innumerable lecture slides are presumably derived from the textbooks cited.  We have not found any references that state that Prim's algorithm \emph{without modification} applies to multi-component graphs, even when executable code is provided: \emph{e.g.}, Heineman \emph{et al.} provide C++ code that aligns closely with our C code~\cite{heineman2008algorithms}, but do not mention that their code works equally well on multi-component graphs.  Indeed, many sources promulgate the false proposition that modifications to the algorithm are needed to handle multi-component graphs (\emph{e.g.},~\cite{kepnergilbert,sedgewick,DBLP:books/daglib/0029345,rozen,prim:wiki}).  Likewise, we have found no reference that removes the initialization step (line~5~in figure~\ref{fig:prims}) from the standard algorithm.

Prim's algorithm has been the focus of a few previous formalization efforts.  Guttman formalised and proved the correctness of Prim's algorithm using Stone-Kleene relation algebras in Isabelle/HOL~\cite{DBLP:conf/ictac/Guttmann16}.  He works in an idealized formal environment that does not require the development of explicit data structures; his code does not appear to be executable.  Lammich \emph{et al.} provided a verification of Prim's algorithm~\cite{DBLP:journals/afp/LammichN19}.  Lammich \emph{et al.} also work within the idealized formal environment of Isabelle/HOL, but in contrast to Guttman develop efficient purely functional data structures and extract them to executable code.  Both Guttman and Lammich explicitly require that the input graph be connected. % and produce a tree.

%axiomatizes
%his second paper that their earlier proof of Prim's assumed "".



%  We make two immediate observations on CLRS's pseudocode: first, that they do not give an explicit definition for

% Note that  does not explicitly define \texttt{EXTRACT-MIN}

%are often limited to the case where the input graph is a connected graph. This is reasonable, as their purpose is to teach the concept of minimum-spanning \textit{tree} algorithms. However, they seldom expand on the subject of spanning forests for disconnected graphs. For example, . \textit{Discrete Mathematics and Its Applications} leaves it as an exercise to the reader, while \textit{Graph Algorithms in the Language of Linear Algebra}~\cite{kepnergilbert} suggests running Prim's on each component of the disconnected graph to obtain a minimum spanning forest. The last appears to be the most common solution, with suggesting this.


\hide{
\begin{figure}[H]
\begin{tikzpicture} [auto, node distance =2 cm and 2cm ,on grid, semithick, state/.style ={circle}]
\node[state] (V0) {$\m{v_0}$};
\node[state] (V1) [right=of V0] {$\m{v_1}$};
\node[state] (V2) [below=of V0] {$\m{v_2}$};
\node[state] (V3) [below right=of V0] {$\m{v_3}$};
\node[color=green] (V4) [right=of V3] {$\m{v_3}$};
\node[color=green] (V5) [right=of V4] {$\m{v_5}$};
\draw[color=red] (V0) edge [ultra thick] node[above=0.15 cm] {$5$} (V1);
\path (V0) edge node[left=0.15 cm] {$6$} (V2);
\path (V0) edge [thin] node[above right=0.05 cm] {$5$} (V3);
\path[color=red] (V1) edge [ultra thick] node[right=0.15 cm] {$5$} (V3);
\path[color=red] (V2) edge [ultra thick] node[below=0.15 cm] {$4$} (V3);
\path (V4) edge node[below=0.15 cm] {$1$} (V5);
\end{tikzpicture}
\caption{Figure of a partial Prim's execution with root $\m{v_0}$. A spanning tree for the left component has been found as indicated in red, while $\m{v_4}$ and $\m{v_5}$ are in the priority queue with weight \texttt{INF}. As our \texttt{EXTRACT-MIN} tolerates \texttt{INF}, our implementation will pop them from the priority queue and proceed as usual, instead of terminating with only the left tree.}
\end{figure}

From the above figure, we demonstrate that our simple priority queue will always pop vertices regardless of its weight. If the popped vertex has weight \texttt{INF}, its parent will also be at its default invalid value, indicating that no edge was added to the graph. We then continue the algorithm as usual, updating the weights of adjacent vertices. As a result, our Prim implementation can return a minimal spanning forest without ``premature termination".

Is our assumption that \texttt{EXTRACT-MIN} can pop \texttt{INF} weight vertices reasonable? We argue that it is, because the abstract algorithm in CLRS makes no statement about \texttt{INF} beyond the initialization. As the algorithm pushes the vertices into queue with weight \texttt{INF}, it is reasonable to say the priority queue can tolerate items with weight \texttt{INF}. Doing so simplifies \texttt{EXTRACT-MIN} to a \texttt{popMin} operation using the priority queue's API, without requiring additional lines of code to further check for specific weights.

Consequently, our code allows a simple implementation of Prim's to return a forest for a disconnected input graph, in a single run of the algorithm without needing to identify the disconnected components beforehand. It is important to note that we do not explicitly define ``components", nor do we tag \texttt{u} as a``new root". It is sufficient to prove that the loop invariant is satisfied whether \texttt{key[u]~< INF} or \texttt{key[u] = INF}.

Note that the priority queue used for this verification is still tied to \texttt{INF}, and thus in the verification our argument is weakened to ``\texttt{INF} is a valid weight in the priority queue" rather than ``membership in the priority queue is independent of \texttt{INF}". Work on a stronger priority queue which completely dissociates from \texttt{INF} was to be delivered by Aquinas Hobor since May 2020.
}

%  If they wish to determine the edge cost from a node to its parent, they can look it up in the graph, with the special case

%We observed that our code is able to return a minimum spanning forest when the input graph is disconnected, . The reason for this is: We treat a vertex of weight \texttt{INF} and a vertex's membership in the priority queue as \textit{separate} matters. A vertex can be in the priority queue with weight \texttt{INF}, represented by \texttt{key[u] = INF}, and this indicates that \texttt{u} is \textit{not} connected to any previously popped vertex; otherwise its weight would have been previously lowered. However, our priority queue does not care about its items having specific weights, only that the item it pops is always the one with the lowest weight. Thus, when a vertex \texttt{u} is returned, it is possible that \texttt{u} has weight \texttt{INF} - the scenario where all vertices remaining in the queue has weight \texttt{INF}, indicating all of them are disconnected from the current forest.

%Rather than start with a distinguished root vertex, we simply start with zero components, and the first vertex we extract (chosen arbitrarily by the PQ) becomes the root of the initial

%A further observation is that if we dissociate \texttt{INF} and the priority queue, then the input root required by Prim's is no longer necessary. The root's key is artificially set to 0 in Prim's, which kickstarts the main loop. However, since our emptiness check and \texttt{EXTRACT-MIN} implementation can return vertices with \texttt{INF} weight, the loop will start as long as there are un-popped vertices. Thus, we suggest it is possible to remove the \texttt{rt} parameter. Doing so simplifies the proof, because we do not have to reason about the artificially weighted \texttt{key[rt]}. The noroot-variant is in Figure 4 at the beginning of this section, while the verified implementation of Prim's without root is in \texttt{noroot\_prim.c}.

%They push their version of \texttt{INF} into the priority queue during the setup, and do not perform any explicit rejections of \texttt{INF}, simply popping from the priority queue at the \texttt{EXTRACT-MIN} step. However, they do not discuss minimum spanning forests, hence this observation is not recorded.


\section{Kruskal's algorithm}
\label{sec:kruskal}
\subsection{Kruskal's Algorithm}
\label{sec:kruskal}

Although Kruskal's algorithm is sometimes presented as taking connected graphs and producing spanning trees, the literature also discusses the more general case of multi-component input graphs and spanning forests.  However, Kruskal has only recently been the focus of formal verification efforts, partly because it relies on the notoriously difficult-to-verify union-find algorithm; fortunately, the CertiGraph project has an existing fully-verified union-find implementation that we can leverage~\cite{DBLP:journals/pacmpl/WangCMH19}.  Kruskal also requires a sorting function; we implemented \texttt{heapsort} as explained in \S\ref{sec:heapsort}.  Kruskal is optimized for compact representations of sparse graphs, so the $O(1)$ space cost of \texttt{heapsort} is a reasonable fit.  %Including the code for union-find and heap sort, the code for Kruskal's algorithm is around 200 lines of~C.

The primary interest of our verification of Kruskal is in our proof engineering.  Kruskal inputs graphs as edge lists rather than adjacency matrices.  In addition to requiring an addition to our spatial graph predicate menu, this means that Kruskal's input graphs can have multiple edges between a given pair of vertices (\emph{i.e.}, a ``multigraph'').  Pleasingly, we can reuse most of the undirected graph definitions (\S\ref{sec:newundirected}), demonstrating that they are generic and reusable.

Another challenge is integrating the pre-existing CertiGraph verification of union-find.  We are pleased to say that no change was required for CertiGraph's existing union-find definitions, lemmas, specifications and verification.  Kruskal actually manipulates two graphs simultaneously: a directed graph with vertex labels (to store parent pointers and ranks) within union-find, and an undirected multigraph with edge labels (for which the algorithm is constructing a spanning forest).  Beyond showing that CertiGraph was capable of this kind of systems-integration challenge, we had to develop additional lemma support to bridge the directed notion of ``reachability,'' used within the directed union-find graph to the undirected notion of ``connectedness,'' used in the MSF graph (\S\ref{sec:newundirected}).

%notions of  as defined and ; with  as defined and }.

%\subsection{Reusing previously verified union-find} %%why is there such an absurdly large space?
%Kruskal's algorithm requires a union-find data structure to keep track of the state of connectedness in the partial forest. Since CertiGraph had published several verified union-find implementations, we decide to make use of them. However, as these implementations had no client using them until now, we discovered that the postconditions of the union-find calls were difficult to use for Kruskal's verification. The union-find implementations were verified prior to our introduction of undirected graph properties, thus were not designed with connectedness in mind.

%To that end, we have extended lemmas about the results of union-find operations as an analog to connectivity. We are pleased to say that no change was required for Wang's existing union-find definitions, lemmas, specifications and verification. Instead, we proved that their existing postconditions mathematically imply an analog to connectivity in undirected graphs. In other words, Wang's verified postconditions were \textit{not} incorrect or poor, they just required mathematical translation into what we wanted in the context of Kruskal's.

%We mention this to emphasise the modularity and buildability of VST and CertiGraph infrastructure - that we were able to use previously, independently proven code in a bigger system later. The internal details and verification of the union-find system are independent from that of Kruskal's, whose proof only required the preconditions and postconditions of whichever union-find implementation we decide to use.

\subsection{Related work on Kruskal in algorithms and formal methods}
\label{sec:relworkkruskal}

Joseph Kruskal published his algorithm in 1956~\cite{kruskal} and it has appeared in numerous textbooks since (\emph{e.g.},~\cite{clrs,DBLP:books/daglib/0022194,sedgewick,DBLP:books/daglib/0015106}).  Kruskal's algorithm is usually preferred over Prim's for sparse graphs, and is sometimes presented as ``the right choice'' when confronted with multi-component graphs under the mistaken assumption that Prim's first requires a component-finding initial step.

Guttman generalised minimum spanning tree algorithms using Stone relation algebras~\cite{DBLP:journals/jlp/Guttmann18}, and provided a proof of Kruskal's algorithm formatted in said algebras.  Like his work on Prim's~\cite{DBLP:conf/ictac/Guttmann16}, Guttmann works within Isabelle/HOL and does not include concrete data structures such as priority-queues and union-find, instead capturing their action as equivalence relations in the underlying algebras. In Guttmann's Kruskal paper, he mentions that his Prim paper axiomatizes the fact that ``every~finite~graph has~a~minimum~spanning~forest,'' which he is then able to prove \emph{using his Kruskal algorithm}.  Interestingly, our Prim verification needs the same fact, but we prove it directly. % that every finite graph has a finite, nonempty list of spanning forests.

In a similar vein, Haslbeck \emph{et al.} verified Kruskal's algorithm~\cite{DBLP:journals/afp/HaslbeckLB19} by building on Lammich \emph{et al.}'s earlier work on Prim~\cite{DBLP:journals/afp/LammichN19}.  Like Lammich \emph{et al.}, Haslbeck \emph{et al.} work within Isabelle/HOL with a focus on purely functional data structures.

One of the stumbling blocks in verifying Kruskal's algorithm is the need to verify union-find.  In addition to CertiGraph, 

%In Lammich et al, their Prim's implementation explicitly expects a connected graph, and they do not reason about the disconnected case.

%Working in an idealized formal environment, they do not require the development of .

%\note{Despite this connectedness requirement, Guttman stated in a subsequent paper~\cite{DBLP:journals/jlp/Guttmann18} that his proof of Prim in~\cite{DBLP:conf/ictac/Guttmann16} asserts that  as an axiom.}

%Guttman later proved this by relying on their proof of Kruskal's. We use the same assertion in our proof of Prim's, but in our case, Kruskal's is defined in the spatial layer of our library, not the mathematical-layer, and it would be unwieldy to bring it back up. Instead, we show in the mathematical layer that  Thus there exists a minimum spanning forest by simply taking the minimal-weight forest in this list. 

\section{Other observations}
\label{sec:other}
\subsection{Speed of loop invariants}
The recommended tactic to solve invariants and ENTAIL goals in VST is \textit{entailer!}. However, we observed that our invariants contained a significiant number of mathematical statements, encoded by VST as PROPs, that were difficult to solve with VST's \textit{entailer!} tactic. As a result, \textit{entailer!} often took a long time to solve or reduce the invariant. To overcome this, we had to pre-solve every PROP in the invariant and preformat the SEP clauses. In the most significant case, an \textit{entailer!} call that took 446 seconds and still returned a large number of unsolved PROPs, was reduced to 36 seconds at the fastest recorded. (Tested on a VirtualBox Lubuntu 18.04 machine assigned 8GB RAM and 2 processors)
\newline\newline
That the tactic could not solve many of our invariant \texttt{PROP}s by itself was expected and reasonable, as they were often complex properties of graphs. However, we were surprised that several solved \texttt{PROP}s took a significant amount of time despite being trivial at first glance. This is especially noticed in PROPs with preconditions of elements in empty lists. Thus, an immediate suggestion to the \textit{entailer!} tactic is to test "\textit{try contradiction}" as early as possible.
\begin{lstlisting}
assert (Hinv_10: forall u v : V,
	In u (nil (A:=V)) -> In v (nil (A:=V)) ->
	connected g u v <-> connected edgeless_graph' u v). {
		intros. contradiction.
}
\end{lstlisting}
The above Coq assertion in our loop precondition, which was easily solved by two basic tactics, cost an additional 50s when left to \textit{entailer!} to prove.
\newline\newline
Given the possibility that invariants carry PROPs on abstract graph properties that we do not expect \textit{entailer!} to solve, a thought is to provide a reduced version of \textit{entailer!} that does not reason about PROPs at all, but leaves them to the user to solve, and focuses on more VST-specific issues such as SEP and LOCAL clauses. This will be useful for verification of functions that implement abstract models.
\subsection{Modularity of library and VST verification}
In previous work by both VST and our team, the C functions and algorithms we have implemented and verified are isolated with little to no dependency on each other. Even the garbage collector verified by Wang et al was its own, independent of the other verified code in the library. This work is the first step our team has taken in verifying code that uses \textit{previously} verified C code, as discussed in Section 4. While a structure of the CertiGraph library has been explained in Section 1, verified functions and programs by both VST and Wang et al are largely isolated and difficult to re-use - even the garbage collector with complex functions is its own isolated system. We found it a necessity to further re-organise the internal hierarchies in our library, for better modularity and re-use of verified code. We have worked to improve this by providing a clearer separation between mathematical lemmas, VST specifications and proofs.
\newline\newline
\includegraphics[scale=0.56]{structure.jpg}
\begin{center}Simplified subset of the dependencies in Dijkstra's, Prim's and Kruskal's verification
\end{center}
Previously, we described CertiGraph as having three layers: The mathematical layer which contains "pure Coq" mathematical models and lemmas; the spatial layer to represent graphs in Verifiable C; and the verification layer, for specifications and verifications of C code, whose ASTs were retrieved from CompCert's \textit{clightgen} utility. We further separate this third layer into specifications and verifications. This allows re-use of a previous specification by another system without being burdened by the verification, as illustrated above. The development and verification of components can then be performed in parallel.
\newline\newline
In addition, we have worked to improve on the modularity of the mathematical layer. While we have highlighted differences between directed and undirected graphs, as well as in the graphs in different representations, we have worked to ensure that the same lemmas can be used by them with little need for repetition, to reduce the code base in CertiGraph. Our new contributions to CertiGraph include: ~2000 lines of lemmas of undirected graph properties; ~200 lines to link unionfind properties with undirected graph properties; ~1500 lines for the symmetric matrix graph and ~900 lines for the edgelist graph; ~1800 lines for the verification of Prim's and Kruskal's algorithms each.
\newline\newline
We illustrate this to highlight the potential of VST for verifying larger, complex programs with modular components.
\iffalse
\paragraph{Shared clightgen issue with libraries.} It is obvious that one must include the dependencies of the program when compiling C code. However, we observe an issue of the opposite nature in VST and CompCert - that when running \textit{clightgen} for a program, one must include all programs that \textit{use} the same dependency to properly regenerate the AST files. Failure to do so results in the other program failing to run due to the dependency's changed AST file.
\newline\newline
For instance, in the above figure, both Dijkstra's and Prim's implementations are dependent on the priority queue. When running clightgen for Prim's, we include the priority queue's C implementation to regenerate the AST tree. However, without including Dijkstra's in the same command, the verification of Dijkstra's will result in an error over priority queue's new AST tree.
\newline\newline
(ok this may not be relevant anymore now that we're operating on two different pqs)
\newline\newline
This is a hindrance to the modular design we aspire to, and is thus worth looking into. As of the writing of this paper, we have not looked into the exact nature of the AST files to suggest a solution to VST and CompCert.
\fi

\section{Related Work, learnings, future work, and ideas for thesis}
\label{sec:relatedwork}
\subsection{Previous work}

In Section 1, we have discussed Wang et al's development and use of CertiGraph to verify graph algorithms and graph-related data structures. We have explained the use of these previous algorithms in our implementation of Kruskal's.

\subsection{Related work}
In 2019, Lammich et al provided a verification of Prim's and Dijkstra's algorithms~\cite{DBLP:journals/afp/LammichN19}. Haslbeck et al, with Lammich in the same team, verified Kruskal's algorithm in a separate paper~\cite{DBLP:journals/afp/HaslbeckLB19}. The main highlights of their papers are writing efficient data structures in Isabelle/HOL and extracting them to executable code. In the Kruskal's paper, they took the same approach as we did in defining forests: That a forest is an acyclic graph, and a tree is a connected forest. In their paper on Prim's and Dijkstra's, the case of forests was not discussed.
\newline\newline
Guttman formalised and proved the correctness of Prim's algorithm using Stone relation and Kleene algebras in Isabelle/HOL~\cite{DBLP:conf/ictac/Guttmann16}. They later generalised minimum spanning tree algorithms using Stone relation algebras~\cite{DBLP:journals/jlp/Guttmann18}, and provided a verified proof of Kruskal's algorithm formatted in their algebra. They do not require the development of explicit data structures such as priority-queues and union-find, as these are captured as equivalence relations in their algebra.
\newline\newline
Guttman observed in their later paper that their earlier proof of Prim's assumed that "every finite graph had a minimum spanning forest". Guttman proved this by relying on their later proof of Kruskal's. We use the same theorem in our proof of Prim's, but in our case, as Kruskal's was defined in the lower layers of our library, it is clunky to extract the minimum spanning forest returned in a VST specification back to the mathematical-layer. Thus, we provide a mathematical proof - by showing that every finite graph has a finite, nonempty list of spanning forests. Using each graph's sum of edge weights as a comparator, we prove that we can obtain a minimal element in the list, thus there exists a minimum spanning forest.
\newline\newline
In terms of graph verification libraries, Krishna et al~\cite{DBLP:conf/esop/KrishnaSW20} has developed a flow algebraic framework to reason about local and global properties of \textit{flow graphs} in the program heap. Their flow algebra, in which a graph is then defined in similar to Guttman's approach, is designed to mainly tackle local reasoning of global graphs in program heaps, tackling similar issues to Wang et al ~\cite{DBLP:journals/pacmpl/WangCMH19}, but in this paper, local reasoning is not required, and our explored graph representations and algorithms are also vastly different. Their flow algebra is said to be compatible with existing separation logics, although actual implementation and integration with SL tools appears to be an ongoing progress.
\newline\newline
We have reviewed both formal and informal sources, the former including textbooks cited in Sections 2 and 3, over their discussion of Prim's and Kruskal's algorithms. As mentioned, they often omit the forest case, and thus did not arrive at the forest-observation of Prim's we have discovered in this work.
\newline\newline
In terms of this paper's contribution, we maintain our specialty of focusing on verified C code, using proof assistants as the verification platform instead of the executable language itself. Yet at the same time, our graph library is capable of proving graph lemmas and theorems entirely in Coq. It is also \textit{not} restricted to ``clean", textbook graphs. We have also verified Prim's and Kruskal's algorithms with minimum spanning \textit{forests} in mind, rather than trees. Lastly, this work is an extension of our graph library, adding Prim's and Kruskal's to our repository of verified graph algorithms, rather than them being standalone proofs.

\subsection{Ongoing and future work.}
Our supervisor remains in the midst of implementing a more versatile, general priority queue, to replace the current array-based queue being used in Dijkstra's and Prim's algorithms.
\newline\newline
We have left the sorting algorithm used by Kruskal's as an axiomatised specification without providing matching C code. This was to be implemented by our supervisor using his aforementioned priority queue. Sorting algorithms in C are also being explored by the VST developers, in response to a C program verification challenge paper~\cite{DBLP:journals/corr/abs-1904-01009}. We have had limited discussion with them on generalizing their current work to sort C-structs, as their primary focus was on floats.
%%Anshuman and I may look into the development and verification of priority queues and sorting algorithms in C in general.
\newline\newline
A graph algorithm we considered verifying is topological sort, specifically the \texttt{tsort} program in coreutils. It is particularly challenging as it handles a graph of \textit{string} vertices, represented in custom, pointer-linked data structures. To begin considering its verification, one must first have verified specifications of string functions. Our search into formalization of string functions in C show that the last work on a verified C-string library was in 2006 by Artem Starostin~\cite{cstring} using Isabelle/HOL, in a different subvariant of ANSI C. The current VST repository has examples of some string functions, but they are not extensive enough for our purposes. Therefore, I think that the verification of a modern, compiler-compatible C-string library is not only an open area for development, but a necessity.
\newline\newline
With the above in mind, I intend to take a step back from graph algorithms, as I am interested in the development and verification of a library of commonly used C functions, effectively building towards a ``verified libc". I believe it is necessary for the development of C programs and systems with safety guarantees. %%Indeed, work towards these systems have recently bore fruit, such as the seL4 microkernel whose resilience to security attacks have been demonstrated in 2018. Thus, I believe this work will contribute to ensuring the safety and security of C programs.%%

\bibliographystyle{splncs}
\bibliography{qebib}

%%\end{thebibliography}

%%\appendix
%%\label{sec:apx}
%%
\appendix

\section{Junk}
{\color{magenta} Universally-quantified metavariables can appear free in the predicates to make further connections.
Assuming that the abstracted pre- and postconditions $A$, $B$, $C$, and $D$ above all use \li{x}, we proceed
as follows.  First we introduce a new fresh metavariable $x$ whose value will be equal to \li{x} after the localization, and then choose $F \stackrel{\Delta}{=} [\li{x} |-> x] (C -* D)$, that is we substitute the program
variable \li{x} for the metavariable $x$.  Since we have substituted away \li{x}, $F$ ignores it and so we satisfy the side condition on \infrulestyle{Solve Ramify-P}.  We then must strengthen $C$ into $C' \stackrel{\Delta}{=} C /| \li{x} = x$ to make the connection at the appropriate program point.  Now we are left with the entailments
\[
\begin{array}{lcl}
\li{x} = 5 /| A & |- & (\li{x} = 5 /| B) * F \\
F & |- & (\li{x} = 6 /| C') -* (x = 6 /| D)
\end{array}
\]
To further relate the earlier and later values of \li{x} in $F$ we can introduce a second fresh $x'$ and use $B' \stackrel{\Delta}{=} B /| \li{x} = x'$.
}


\section{Remaining proof of \infrulestyle{Ramify-PQ}}
\label{apx}

See figure \ref{fig:remainrampq}.

\begin{figure*}[t]
\[
\infrule{}
{
  L_1 |- L_1 \\
  \infrule{}
  {
    \infrule{}
    {
      \infrule{}
      {
        \infrule{}
        {
          \infrule{}
          {
            \infrule{}
            {
              \infrule{}
              {
                \infrule{}
                {
                  \infrule{}
                  {
                    \infrule{}
                    {
                      [x |-> x_0] (L_2 -* G_2) |- [x |-> x_0](L_2 -* G_2)
                    } {
                      \forall x.~ (L_2 -* G_2) |- [x |-> x_0](L_2 -* G_2)
                    } {\forall \mathsf{e}}
                  } {
                    \forall x.~ (L_2 -* G_2) |- ([x |-> x_0]L_2) -* ([x |-> x_0]G_2)
                  } {\textrm{substitute}}
                } {
                  \big(\forall x.~ (L_2 -* G_2)\big) * [x |-> x_0]L_2 |- [x |-> x_0]G_2
                } {(3)}
              } {
                \big(\forall x.~ (L_2 -* G_2)\big) * [x |-> x_0]L_2 |- \exists x.~ G_2
              } {\exists \mathsf{i}}
            } {
            \big(\forall x.~ (L_2 -* G_2)\big) * (\exists x.~ L_2) |- \exists x.~ G_2
            } {\exists \mathsf{e}}
          } {
            \forall x.~ (L_2 -* G_2) |- (\exists x.~ L_2) -* (\exists x.~ G_2)
          } {(3)}
        } {
          |- \big(\forall x.~ (L_2 -* G_2)\big) => \big((\exists x.~ L_2) -* (\exists x.~ G_2)\big)
        } {=> \mathsf{i}}
      } {
        |- \pguards{c}\Big(\big(\forall x.~ (L_2 -* G_2)\big) => \big((\exists x.~ L_2) -* (\exists x.~ G_2)\big)\Big)
      } {\mathsf{N}}
    } {
      |- \Big(\pguards{c}\big(\forall x.~ (L_2 -* G_2)\big) \Big) => \Big( \pguards{c}\big((\exists x.~ L_2) -* (\exists x.~ G_2)\big) \Big)
    } {\mathsf{K}}
  } {
    \pguards{c}\big(\forall x.~ (L_2 -* G_2)\big) |- \pguards{c}\big((\exists x.~ L_2) -* (\exists x.~ G_2)\big)
  } {\mathsf{i} =>}
} {
  L_1 * \pguards{c}\big(\forall x.~ (L_2 -* G_2)\big) |- L_1 * \pguards{c}\big((\exists x.~ L_2) -* (\exists x.~ G_2)\big)
} {* \textrm{ split} }
\]
\caption{Remaining proof of \infrulestyle{Ramify-PQ}}
\label{fig:remainrampq}
\end{figure*}


\end{document}
\endinput
%%